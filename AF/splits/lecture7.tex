%%lecture 7

%% 20.9
\newpage
\section{Hilbert spaces} 
\label{sec:hilbert_spaces}
\begin{beispiel}
	Consider $\mathbb{C}^n = \set[(x_1,x_2,\dots,x_n)]{x_i \in \mathbb{C}}$ and $x,y \in \mathbb{C}^n$ with
	$x= (x_1,\dots,x_n)$, $y = (y_1,\dots,y_n)$. Define the inner product of $x,y$ (scalar product)
	\[
		\skal{x}{y} = \sum^{n}_{i=1}x_i \bar{y}_i \in \mathbb{C}.
	\]
	We have a map
	\begin{align*}
		\mathbb{C}^n \times \mathbb{C}^n &\to \mathbb{C} \\
		(x,y) &\mapsto \skal{x}{y}.
	\end{align*}
	This mapping has properties:
	\begin{itemize}
		\item $x \neq 0$ folgt $\skal{x}{x} = \sum^{n}_{i=1}x_i \bar{x}_i = \sum^{n}_{i=1} \abs{x_i}^2 >0$
		\item $\skal{\lambda x}{y} = \lambda \skal{x}{y}$ for $x,y \in \mathbb{C}^n$, $\lambda \in \mathbb{C}$.
		\item $\skal{x}{y} = \sum^{n}_{i=1} x_i \bar{y}_i = \overline{\sum^{n}_{i=1}y_i \bar{x}_i}$ for $x,y \in \mathbb{C}^n$. \\
		In particular $\skal{x}{\lambda y} = \bar{\lambda} \skal{x}{y}$ for $\lambda \in \mathbb{C}$.
		\item $\skal{x+y}{z} = \skal{x}{z}+ \skal{y}{z}$ for $x,y,z \in \mathbb{C}^n$. 
	\end{itemize}
\end{beispiel}
\begin{definition*}
	An inner product space $V$ is a complex vector space with an inner product which is a map 
	\[
		\skal{.}{.}: V \times V \to \mathbb{C}.
	\]
	Satisfying
	\begin{itemize}
		\item $\skal{\lambda x}{y} = \lambda \skal{x}{y}$ for any $x,y \in V$, $\lambda \in \mathbb{C}$.
		\item $\skal{x+y}{z} = \skal{x}{z}+ \skal{y}{z}$ for any $x,y,z \in V$.
		\item $\skal{x}{y} = \overline{\skal{x}{y}}$ for any $x,y \in V$.
		\item $\skal{x}{x}>0$ for any $x \in V, x \neq 0$.
	\end{itemize}
\end{definition*}
Can we generalize $\mathbb{C}^n$? \\
\[
	\mathbb{C}^{\mathbb{N}} \set[(x_1,x_2, \dots)]{x_i \in \mathbb{C}}
\]
with
\[
	\skal{x}{y} = \sum^{\infty}_{i=1} x_i \bar{y}_i.
\]
This is not necessarily convergent.
\begin{beispiele}
	\begin{enumerate}[(1)]
		\item 	\[
		l^2 = \set[(x_1,x_2, \dots)]{ \sum^{\infty}_{i=1} \abs{x_i}^2 < \infty}.
	\]
	We have with Cauchy Schwarz
	\[
		\sum^{n}_{i=1} \abs{x_i \bar{y}_i} \leq \left( \sum^{n}_{i=1} \abs{x_i}^2 \right)^{\frac{1}{2}} \left( \sum^{n}_{i=1} \abs{y_i}^2 \right)^{\frac{1}{2}}
	\]
	if $x \in l^2$ and $y \in l^2$ we get
	\begin{align*}
		\sum^{n}_{i=1}\abs{x_i \bar{y}_i} \leq \left( \sum^{\infty}_{i=1} \abs{x_i}^2 \right)^{\frac{1}{2}} \left( \sum_{i=1}^{\infty} \abs{y_i}^2. \right)^{\frac{1}{2}} < \infty.
	\end{align*}
	It follows that $\sum_{i=1}^{\infty} x_i \bar{y}_i$ converges absolutely and hence it is convergent. The following 
	\[
		\skal{x}{y} = \sum^{\infty}_{i=1} x_i \bar{y}_i
	\]
	is well-defined for vectors $x,y \in l^2$. Like for $\mathbb{C}^n$ one can easily check that $\skal{.}{.}$ satisfies the axioms for inner products. \\
	$(l^2, \skal{.}{.})$ is an inner product space.
	\item Consider $C([0,1])$ with the inner product
	\[
		\skal{f}{g} = \int_{0}^{1}f(t) \overline{g(t)} \,\mathrm{d}t \qquad \forall\, f,g \in C([0,1]).
	\]
	\begin{itemize}
		\item 	\[
			\skal{\lambda f}{g} = \int_{0}^{1}\lambda f(t) \overline{g(t)} \,\mathrm{d}t = \lambda \int_{0}^{1}f(t) \overline{g(t)} \,\mathrm{d}t = \lambda \skal{f}{g}.
		\]
		\item \[
			\skal{f}{f} = \int_{0}^{1}f(t) \overline{f(t)} \,\mathrm{d}t = \int_{0}^{1} \abs{f(t)}^2 \,\mathrm{d}t >0.
		\]	
		\item $\dots$.
	\end{itemize}
	\end{enumerate}
\end{beispiele}
If we take $\mathbb{R}^3$ with the Eukledian norm on $\mathbb{R}^3$
\[
	\norm{(x_1,x_2,x_3)} = \sqrt{x_1^2 + x_2^2 + x_3^2} = \left( \sum_{i=1}^{3} \abs{x_i}^2 \right)^{\frac{1}{2}} = \skal{x}{x}^{\frac{1}{2}}.
\]
Let $V$ be an inner product space with $\skal{.}{.}$ as the inner product. Let for $x \in V$
\[
	\norm{x} := \skal{x}{x}^{\frac{1}{2}}.
\]
\begin{satz}
	The $x \mapsto  \norm{x}$ with $\norm{.}$ defined above is a norm.
\end{satz}
We are going to prove the norm axioms but first we need another theorem.
	\begin{theorem}[Cauchy-Schwarz inequalitiy]
		For any $x,y \in V$ (inner product space) 
		\[
			\abs{\skal{x}{y}} \leq \skal{x}{x}^{\frac{1}{2}} \skal{y}{y}^{\frac{1}{2}}.
		\]
		The equality holds iff $x,y$ are linearly dependent.
	\end{theorem}
	\begin{beweis}
		Assume $x,y$ linearly dependent. We can assume that $x= \lambda y$ for some $\lambda \in \mathbb{C} $.
		\[
			\abs{\skal{x}{y}} = \abs{ \skal{\lambda y}{y}} = \abs{\lambda} \skal{y}{y}
		\]
		and
		\begin{align*}
					\skal{x}{x}^{\frac{1}{2}} \skal{y}{y}^{\frac{1}{2}} &= \skal{\lambda y}{\lambda y}^{\frac{1}{2}} \skal{y}{y}^{\frac{1}{2}} \\
					&= \abs{\lambda} \skal{y}{y}^{\frac{1}{2}} \skal{y}{y}^{\frac{1}{2}} \\
					&= \abs{\lambda} \skal{y}{y}.
		\end{align*}
		Hence \[
			\abs{\skal{x}{y}} = \skal{x}{x}^{\frac{1}{2}} \skal{y}{y}^{\frac{1}{2}}.
		\]
		Assume $x,y$ are linearly independent. Hence $x + \lambda y \neq 0$ for any $\lambda \in \mathbb{C}$. By an axiom for inner product we get
		\[
			0< \skal{x+ \lambda y}{x + \lambda y} = \skal{x}{x} + \lambda \skal{y}{x} + \bar{\lambda} \skal{x}{y} + \abs{\lambda}^2 \skal{y}{y}.
		\]
		Pick now
		\[
			\lambda = - \frac{\skal{x}{y}}{\skal{y}{y}}.
		\]
		(Note that $y \neq 0$ as $x,y$ linearly independent.)
		We have \begin{align*}
						0 &< \skal{x}{x} - \frac{\overset{= \abs{\skal{x}{y}}^2}{\overbrace{\skal{x}{y}\skal{y}{x}}}}{\skal{y}{y}} - \frac{\overset{= \abs{\skal{x}{y}}^2}{\overbrace{\overline{\skal{x}{y}}\skal{x}{y}}}}{\skal{y}{y}}+ \frac{\abs{\skal{x}{y}}^2}{\skal{y}{y}^2} \skal{y}{y} \\
						&= \skal{x}{x} - \frac{\abs{\skal{x}{y}}^2}{\skal{y}{y}}.
		\end{align*}
		This gives
		\[
			\frac{\abs{\skal{x}{y}}^2}{\skal{y}{y}} < \skal{x}{x}
		\]
		and it follows
		\[
			\abs{\skal{x}{y}}^2 < \skal{x}{x} \skal{y}{y}.
		\]
	\end{beweis}
Now we can use this inequality to proof the statement above:
\begin{beweis}
	\begin{enumerate}[(i)]
		\item $\norm{x} >0$ for all $x \neq 0$ in $V$ (Exercise).
		\item $\norm{\lambda x} = \abs{\lambda} \norm{x}$ for all $x \in V$, $\lambda \in \mathbb{C}$ (Exercise).
		\item Let $x,y \in V$. Then 
		\begin{align*}
			\norm{x+y}^2 &= \skal{x+y}{x+y} \\ &= \skal{x}{x}+ \skal{x}{y}+ \skal{y}{x} + \skal{y}{y} \\
			&= \skal{x}{x} + 2 \re( \skal{x}{y}) + \skal{y}{y} \\
			&\leq  \skal{x}{x}+ 2 \abs{\skal{x}{y}} + \skal{y}{y} \\
			&\leq  \skal{x}{x} + 2 \skal{x}{x}^{\frac{1}{2}}\skal{y}{y}^{\frac{1}{2}} + \skal{y}{y} \\
			&= \left( \skal{x}{x}^{\frac{1}{2}} + \skal{y}{y}^{\frac{1}{2}} \right)^2.
		\end{align*} 
		So
		\[
			\norm{x+y}^2 \leq \left( \norm{x} + \norm{y} \right)^2.
		\]
	\end{enumerate}
\end{beweis}
\begin{theorem}[The Parallelogram Law]
	Let $(V, \skal{.}{.})$ be an inner product space. Let $\norm{x} = \skal{x}{x}^{\frac{1}{2}}$. Then
	\[
		\norm{x+y}^2 + \norm{x-y}^2 = 2 (\norm{x}^2 + \norm{y}^2) \qquad \forall\, x,y \in V.
	\]
\end{theorem}
\begin{satz}
	$l^p$ has inner product $\skal{.}{.}_{l^p}$ such that
	\[
		\norm{x}_p = \sqrt{\skal{x}{x}_{l^p}}
	\]
	iff $p =2$.
\end{satz}
\begin{beweis}
	Enough to show that $\norm{.}_p$-norm does not satisfy the parallelogram law for some $x,y \in l^p$ if $p \neq 2$. Take for example $x = (1,0,0, \dots)$
	and $y= (0,1,0, \dots)$. Note that $\norm{x}_{l^p} = \norm{y}_{l^p} = 1$
	\begin{align*}
		\norm{x+y}^2_{l^p} &= \norm{(1,1,0, \dots)}_{l^p} = 2^{\frac{2}{p}} \\
		\norm{x-y}^2_{l^p} &= \norm{(1,-1,0,\dots)}_{l^p} = 2^{\frac{2}{p}} \\
		\norm{x+y}^2_{l^p} + \norm{x-y}_{l^p}^2 &= 2 \cdot 2^{\frac{2}{p}} = 2( \norm{x}^2_{l^p}+ \norm{y}^2_{l^p}) = 2 \cdot 2.
	\end{align*}
\end{beweis}
All $l^p$ with $p \neq 2$ are not inner product spaces. 
\minisec{Exercise:}Show that $(C([0,1]),\norm{.}_{\infty})$ is not an inner product space.
\begin{bemerkung}
	Whenever a norm satisfies the parallelogram law then there exists an inner product on $V$ such that
	\[
		\norm{x} = \skal{x}{x}^{\frac{1}{2}}.
	\]
\end{bemerkung}
\begin{theorem}[The Polarization Identity]
	Let $(V,\skal{.}{.})$ be an inner product space. Then 
	\[
		4 \skal{x}{y}  = \norm{x+y}^2- \norm{x-y}^2 + i \norm{x+ iy}^2 - i \norm{x - iy}^2.
	\]
\end{theorem}
\begin{definition}
	Let $(V, \skal{.}{.})$ be an inner product space. We say that $x,y$ in $V$ are orthogonal if $\skal{x}{y} = 0$ (We write $x \perp y$). Let $M \subseteq V$
	Define the orthogonal complement
	\[
		M^{\perp} = \set[x \in V]{x \perp y \text{ for any }y \in M}.
	\]
\end{definition}
\begin{proposition}
	If $M \subseteq V$ then $M^{\perp}$ is a subspace of $V$.
\end{proposition}
\begin{theorem}[Pythagorean formula]
	$x,y \in V$ (inner product space). Then
	\[
		x \perp y \qquad \text{iff} \qquad \norm{x+y}^2 = \norm{x}^2 + \norm{y}^2.
	\]
\end{theorem}

\subsection{Orthogonal Systems} 
\label{sub:orthogonal_systems}

Let $(V, \skal{.}{.})$ be an inner product space $\set{u_n} \subseteq V$ is called orthogonal system (with $n$ finite or infinite) if $u_n \perp u_m$ for all $n \neq m$. It is an orthonormal system if in addition $\norm{u_n}=1$. 

\begin{beispiele}
	\begin{enumerate}[1)]
		\item $\set{e_k}_{k=1}^{\infty} \subseteq l^2$ with 
		\[
			\skal{x}{y} = \sum^{\infty}_{i=1} x_i \bar{y}_i
		\]
		with 
		\[
			e_k = (0,\dots,1,0,\dots).
		\]
		$\Rightarrow$ $\set{e_k}$ is an ON-system. 
		\item $C([-\pi,\pi])$ with
		\[
			\skal{f}{g} = \int_{- \pi}^{\pi} f(t) \overline{g(t)} \,\mathrm{d}t.
		\]
		\[
			\set[\frac{1}{\sqrt{2 \pi}} e^{-int}]{n \in \mathbb{Z}}
		\]
		is an orthonormal system.
	\end{enumerate}
\end{beispiele}

\begin{definition}
	Let $\set[a_n]{n \in \mathbb{N}}$ be an orthonormal system in $V$. The formal series 
	\[
		\sum_{n=1}^{\infty} \skal{x}{a_n}a_n
	\]
	is called a fourier series of $x$ corresponding $\set[a_n]{n \in \mathbb{N}}$ and $\skal{x}{a_n}$ are called fourier coefficients of $x$ corresponding to $\set[a_n]{n \in \mathbb{N}}$. 
\end{definition}

\begin{theorem}[Bessel's Equality and Inequality]
	If $\set{a_n}$ orthonormal system in an inner product space $V$, then for all $x \in V$
	\[
		\norm{x- \sum_{k=1}^{n} \skal{x}{a_k}a_k}^2 = \norm{x}^2 - \sum_{k=1}^{n} \abs{\skal{x}{a_k}}^2
	\]
	and 
	\[
		\sum_{k=1}^{\infty} \abs{\skal{x}{a_k}}^2 \leq \norm{x}^2.
	\]
\end{theorem}
\begin{beweis}
	\begin{align*}
		\norm{x- \sum_{k=1}^{n} \skal{x}{a_k}a_k}^2 &= \skal{x- \sum_{k=1}^{n} \skal{x}{a_k}a_k }{x - \sum_{k=1}^{n} \skal{x}{a_k}a_k} \\
		&= \skal{x}{x} - \sum_{k=1}^{n} \overline{\skal{x}{a_k}} \skal{x}{a_k} - \sum_{k=1}^{n} \skal{x}{a_k}\skal{a_k}{x} 
		\\ & \qquad \qquad + \skal{\sum_{k=1}^{n}\skal{x}{a_k}a_k}{\sum_{k=1}^{n}\skal{x}{a_k}a_k} \\
		& = \norm{x}^2 - \sum_{k=1}^{n} \abs{\skal{x}{a_k}}^2 - \sum_{k=1}^{n} \abs{\skal{x}{a_k}}^2 + \sum_{k=1}^{n}\abs{\skal{x}{a_k}}^2  \\
		& = \norm{x}^2 - \sum_{k=1}^{n} \abs{\skal{x}{a_k}}^2.
	\end{align*}
	This gives also:
	\[
		\sum_{k=1}^{n}\abs{\skal{x}{a_k}}^2 = \norm{x}^2 - \norm{x - \sum_{k=1}^{n} \skal{x}{a_k}a_k} \leq \norm{x}^2 
	\]
	for all $n \in \mathbb{N}$. Hence
	\[
		\sum_{k=1}^{\infty} \abs{\skal{x}{a_k}}^2 \leq \norm{x}^2.
	\]
\end{beweis}

\begin{definition}[Hilbert space]
	A Hilbert space is an inner product space which is complete w.r.t. the norm is defined through the inner product.
\end{definition}

\begin{beispiele}
	\begin{itemize}
		\item $\mathbb{C}^n$ is an inner product space and complete w.r.t the Eukledean norm. Hence $\mathbb{C}^n$ is a Hilbert space.
		\item $l^2$ is a Banach space w.r.t. 
		\[
			\norm{x}_{l^2} = \left( \sum_{i=1}^{\infty} \abs{x_i}^2 \right)^{\frac{1}{2}}
		\]
		and
		\[
			\norm{x}_{l^2} = \skal{x}{x}^{\frac{1}{2}},
		\]
		where
		\[
			\skal{x}{y} = \sum_{i=1}^{\infty} x_i \bar{y}_i.
		\]
		\item $(C([0,1]), \norm{.}_{\infty})$ is a Banach space but not an inner product space. Hence it is no Hilbert space.
		\item $(C([0,1]),\skal{.}{.})$ is an inner product space $f,g \in C([0,1])$ with
		\[
			\skal{f}{g} = \int_{0}^{1} f(t) \overline{g(t)} \,\mathrm{d}t
		\]
		and the corresponding
		\[
			\norm{f}_2 = \skal{f}{f} = \int_{0}^{1} \abs{f(t)}^2 \,\mathrm{d}t.
		\]
	\end{itemize}
\end{beispiele}
\begin{bemerkung}
	Other $l^p$ spaces are not Hilbert spaces!!!! They are not inner product spaces.
\end{bemerkung}

\begin{satz}
	$(C([0,1]), \skal{.}{.})$ is not a Hilbert space since $(C([0,1]), \norm{.}_2)$ is not complete.
\end{satz}

\begin{beweis}
	Sketch: Show that $f_n(t)$, which is defined as a piecewise continuous function for example
	\[
		f_n(x)= \begin{cases}
			1, &\text{ if }x \in [0,\frac{1}{2}]\\
			0, &\text{ if }x \in [\frac{1}{2} + \frac{1}{n}] \\
			\text{continuous}, & \text{elsewhere} 
		\end{cases}
	\] is a Cauchy sequence w.r.t $\norm{.}_2$ but has no limit in $C([0,1])$.
\end{beweis}

Consider
\[
	C_F = \set[(x_1,x_2,\dots)]{\text{only finite }x_i \neq 0} 
\]
with \[
	\skal{x}{y} = \sum_{i=1}^{\infty} x_i \bar{y}_i.
\]
Show that $(C_F, \skal{.}{.})$ is not a Hilbert space.

\begin{definition}[strongly and weakly convergent]
	A sequence $\set{x_n} \subseteq H$, where $H$ is a Hilbert space, is called strongly convergent $(x_n \to x \in H)$ if 
	\[
		\norm{x_n -x} \to 0, \qquad n \to  \infty.
	\]
	(Norm induced by an inner product) \\ We say that $x_n$ is weakly convergent ($x_n \rightharpoonup x$) if
	\[
		\skal{x_n}{y} \to \skal{x}{y}, \qquad \forall\, y \in H.
	\]
\end{definition}

\begin{satz}
	$x_n \to x$$ \qquad \Rightarrow \qquad $ $x_n \rightharpoonup x$.
\end{satz}

\begin{beweis}
	Assume strong convergence for $(x_n)_{n \in \mathbb{N}}$. Then
	\begin{align*}
		\abs{\skal{x_n}{y}- \skal{x}{y}} &= \abs{\skal{x_n-x}{y}} \\
		&\leq \underset{=\norm{x_n-x}}{\underbrace{\skal{x_n-x}{x_n-x}^{\frac{1}{2}}}} \underset{= \norm{y}}{\underbrace{\skal{y}{y}^{\frac{1}{2}}}} \\
		&= \underset{ \to 0}{\underbrace{x_n-x}} \norm{y} \to 0, \qquad n \to \infty.
	\end{align*}
	Hence $\skal{x_n}{y} \to \skal{x}{y}$.
\end{beweis}

\begin{bemerkung}
	The converse is not true in general: \\
	Take $H=l^2$ and 
	\begin{align*}
		x_n &= e_n = (0, \dots,1,0,\dots) \\
		y &= (y_1,y_2,\dots) \in l^2.
	\end{align*}
	We have for all $y \in H$
	\[
		\skal{e_n}{y} = y_n \to 0, \qquad n \to \infty
	\]
	as
	\[
		\norm{e_n -0}_{l^2} = \norm{e_n}_{l^2} =1.
	\]
\end{bemerkung}

\begin{satz}
	$x_n \to x$ and $y_n \to y$ yields
	\[
		\skal{x_n}{y_n} \to \skal{x}{y}.
	\]
	In particular
	\[
		x_n \to x \qquad \Rightarrow \qquad \norm{x_n} \to \norm{x}.
	\]
\end{satz}
\begin{beweis}
	\begin{align*}
		\abs{\skal{x_n}{y_n}-\skal{x}{y}} &= \abs{\skal{x_n}{y_n}-\skal{x}{y_n} + \skal{x}{y_n} - \skal{x}{y}} \\
		&= \abs{\skal{x_n -x}{y_n} + \skal{x}{y_n-y}} \\
		&\leq \abs{\skal{x_n-x}{y_n}} + \abs{\skal{x}{y_n-y}} \\
		&\leq \underset{\to 0}{\underbrace{\norm{x_n-x}}} \underset{< \infty}{\underbrace{\norm{y_n}}} + \underset{< \infty}{\underbrace{\norm{x}}} \underset{\to 0}{\underbrace{\norm{y_n-y}}} \to 0, \qquad  n \to \infty.
	\end{align*}
	Check $\set{\norm{y_n}}$ is bounded
	\[
		\norm{y_n} = \norm{y_n -y +y} \leq \underset{\to 0}{\underbrace{\norm{y_n -y}}}+ \underset{< \infty}{\underbrace{\norm{y}}} \to 0, \qquad n \to \infty.
	\]
\end{beweis}

\begin{satz}
	$x_n \rightharpoonup x$ and $\norm{x_n} \to \norm{x}$ yields
	\[
		x_n \to x.
	\]
\end{satz}
\begin{beweis}
	\begin{align*}
		\norm{x_n-x}^2 &= \skal{x_n -x}{x_n-x} \\
		&= \underset{= \norm{x_n}^2}{\underbrace{\skal{x_n}{x_n}}} - \skal{x}{x_n}- \skal{x_n}{x} + \skal{x}{x} \\
		&= \norm{x_n}^2 - \overline{\skal{x_n}{x}} - \skal{x_n}{x} + \norm{x}^2 \\
		&\to \norm{x}^2 - \norm{x}^2 - \norm{x}^2 + \norm{x}^2 = 0.
	\end{align*}
\end{beweis}

We have proved 
\[
	x_n \to x \qquad \Rightarrow \qquad \set{\norm{x_n}} \text{ is bounded}.
\]

\begin{theorem}
	\[
		x_n \rightharpoonup x \qquad \Rightarrow \qquad \sup_{n \in \mathbb{N}}\norm{x_n} < \infty.
	\]
\end{theorem}
\begin{beweis}
	Let $x_n \rightharpoonup x$. Consider $f_n: H \to \mathbb{C}$ where
	\[
		f_n(y) = \skal{y}{x_n}, \qquad y \in H.
	\]
	\begin{itemize}
		\item $f_n$ is a linear functional for every $n \in \mathbb{N}$.
		\item $\forall\, n \in \mathbb{N}$ $f_n$ is a bounded ($\Leftrightarrow$ continuous) linear functional as if 
		\[
			y_k \stackrel{k \to \infty}{\to }y \qquad \Rightarrow \qquad f_n(y_k) = \skal{y_k}{x_n} \to \skal{y}{x_n} = f_n(y), \qquad k \to \infty.
		\]
		\item $f_n(y) \to \skal{y}{x}$. \\
		$\set{f_n(y)}_n$ is a convergent sequence in $\mathbb{C}$ and hence bounded for all $y \in H$. \\
		Hence it exists $M_y$ such that 
		\[
			\abs{f_n(y)} \leq M_y.
		\]
		By Banach-Steinhaus-Theorem it holds
		\[
			\norm{f_n} \leq M \text{ for some }M >0.
		\]
		We are done if we proof that $\norm{f_n} = \norm{x_n}$.
		\[
			\abs{f_n(y)}= \abs{\skal{y}{x_n}} \leq \norm{y} \norm{x_n}, \qquad \forall\, y \in H.
		\]
		Hence 
		\[
			\norm{f_n} \leq \norm{x_n} \qquad \qquad (1).
		\]
		On the other Hand we have
		\[
			f_n(x_n) = \skal{x_n}{x_n}= \norm{x_n}^2
		\]
		and thus
		\[
			\norm{f_n} = \sup_{x \in H} \frac{\abs{f_n(x)}}{\norm{x}} \geq \frac{\abs{f_n(x_n)}}{\norm{x_n}} = \norm{x_n} \qquad \qquad (2)
		\]
		With (1) and (2) we are finished.
	\end{itemize}
\end{beweis}

\subsection{Orthogonal decomposition in Hilbert spaces} 
\label{sub:orthogonal_decomposition_in_hilber_spaces}

Remember Linear Algebra. Take $\mathbb{R}^n$ and a subspace $M \subseteq \mathbb{R}^n$ 
\[
	\Rightarrow \qquad \forall\, x \in \mathbb{R}^n \qquad x = z + y, \qquad \text{where }z \in M, y \in M^{\perp}.
\]
This can be done in a unique way
\begin{align*}
	M &= \spn \set{e_z} \\ 
	M^{\perp} &= \spn\set{e_y}
\end{align*}
and
\[
	z = \proj_{M^{\perp}}x, \qquad \qquad \norm{x - \proj_Mx} = \min_{y \in M}\norm{x-y}.
\]

\minisec{General Hilbert space case}

\begin{proposition}
	$M \subseteq H$, then $M^{\perp}$ is a closed subspace and
	\[
		\left( M^{\perp} \right)^{\perp} = \overline{\spn M}.
	\]
\end{proposition}

\begin{satz}
	$H$ Hilbert space and $M$-closed subspace of $H$ and $x \in H$. Then there exists a unique $z \in M$ such that
	\[
		\norm{x-z} = \dist(x,M) := \inf_{y \in M} \norm{x-y}.
	\]
	($z$ analog of the $\text{proj}_Mx$ in the other case).
\end{satz}

\begin{proposition}
	Taking $z \in M$ from the previous proposition. We have $x - z \in M^{\perp}$, i.e.
	\[
		x = \underset{\in M}{\underbrace{z}} + \underset{\in M^{\perp}}{\underbrace{(x - z)}}.
	\]
\end{proposition}