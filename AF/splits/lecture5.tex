%lecture 5

\subsection{Fixed point theory} 
\label{sub:fixed_point_theory}
\begin{beispiel}
	Consider
	\[
		f(x)+ 5 \int_{0}^{1-x} \min(x,y)f(y) \,\mathrm{d}y = g(x), \qquad x \in [0,1] \qquad (*)
	\]
	where $g \in C([0,1])$. \\ 
	\textbf{Claim:} \text{    }     There exists an unique solution $f \in C([0,1])$ that $(*)$. \\
	Idea:
	\[
		f(x) = f(x) - 5 \int_{0}^{1-x} \min(x,y)f(y) \,\mathrm{d}y, \qquad x \in [0,1].
	\]
	Set for $x \in [0,1]$ 
	\[
			\tilde T(f)(x) = RHS(x).
	\]
	To find a solution to $(*)$ is the same finding $f \in C([0,1])$ such that 
	\[
		f = \tilde T(f).
	\]
	Clearly $ \tilde T : C([0,1]) \to C([0,1])$. (continual later).
\end{beispiel}

\begin{theorem}[Banach's fixed point theorem]
	$(E, \norm{.})$ Banach space. $T: E \to E$ (no assumption on linearity) is a contraction on $E$, i.e. there exists $c<1$ such that
	\[
		\norm{T(x)-T(\tilde x)} \leq c \norm{x- \tilde x} \qquad \text{for all }x,\tilde x \in E.
	\]
	Then there exists a unique $ \bar{x} \in E$ such that 
	\[
		\bar{x} = T( \bar{ x}).
	\]
	($\bar{x}$ is a fixed point)
\end{theorem}
\begin{beweis}
	\begin{description}
		\item[Uniqueness:]Assume $T( \bar{x}) = \bar{x}$ and $T ( \tilde x) = \tilde x$. Then
		\[
			\underset{\geq 0}{\underbrace{\norm{ \bar{ x} - \tilde x }}} = \norm{T( \bar{x})- T( \tilde x)} \leq \underset{< 1}{\underbrace{c}} \norm{ \bar{x}- \tilde x}.
		\] 
		Thus $\norm{\bar{x}- \tilde x} = 0$, i.e. $\bar{x} = \tilde x$.
		\item[Existence:] Pick an arbitrary $x_0 \in E$. Set
		\[
			x_{n+1} = T(x_{n}), \qquad n=0,1,2,\dots.
		\]
		\textbf{Claim:} \text{    }     $(x_n)_{n=1}^{\infty}$ is a Cauchy sequence in $(E,\norm{.})$.
		Note:
		\begin{align*}
			\norm{x_{n+1}-x_n}  &= \norm{T(x_n)-T(x_{n-1})} \\
			&\leq  c \norm{x_n - x_{n-1}} \\
			&\leq \dots \\
			&\leq  c^n \norm{x_1-x_0}, \qquad n=1,2,\dots.
		\end{align*}
		For $n>m$
		\begin{align*}
			\norm{x_n-x_m} &= \norm{x_n - x_{n-1}+ x_{n-1}- \dots + x_{m+1}- x_m} \\
			&\leq \norm{x_n - x_{n-1}} + \norm{x_{n-1} - x_{n-2}} + \dots + \norm{x_{m+1}- x_m} \\
			&\leq (c^{n-1}+ c^{n-2} + \dots c^m) \norm{x_1-x_0} \\
			&\leq \frac{c^m}{1-c} \norm{x_1 - x_0} \to 0 \qquad \text{as }n,m \to \infty.
		\end{align*}
		Hence $(x_n)_{n=1}^{\infty}$ is a Cauchy sequence in $(E,\norm{.})$. $(E, \norm{.})$ is a Banach space. So $(x_n)_{n=1}^{\infty}$ converges in $(E,\norm{.})$. Call the limit $\bar{x}$. \\
		\textbf{Claim:} \text{    }     $\bar{x}$ is a fixed point for $T$. 
		\begin{align*}
			\norm{\bar{ x}- T(\bar{x})} & = \norm{\bar{x}-x_{n+1}+ x_{n+1} - T(\bar{x})} \\
			&\leq \norm{\bar{x}-x_{n+1}} + \norm{\underset{T(x_n)}{\underbrace{x_{n+1}}} - T(\bar{x})} \\
			&\leq \underset{\to 0}{\underbrace{\norm{\bar{x}- x_{n+1}}}} + c \underset{\to 0}{\underbrace{\norm{x_n - \bar{x}}}} \to 0, \qquad n \to \infty
		\end{align*}
	\end{description}
\end{beweis}
\begin{bemerkung}
	\begin{enumerate}[(1)]
		\item $x_n \to \bar{x}$ for $n \to \infty$ independend of the choice of $x_0$
		\item Fix $z \in E$
		\begin{align*}
			\norm{\bar{x}-z} &= \norm{T(\bar{x})- T(z) + T(z) -z} \\
			&\leq \norm{T(\bar{x})-T(z)} + \norm{T(z) -z} \\
			&\leq c \norm{\bar{x}-z} + \norm{T(z) - z}.
		\end{align*}
		Hence 
		\[
			\norm{\bar{x}-z} \leq \frac{1}{1-c}\norm{T(z)-z}.
		\]
	\end{enumerate}
\end{bemerkung}
\begin{beispiel}
	Consider now the example from above: $(C([0,1]), \norm{.})$ with $\norm{f} = \max_{x \in [0,1]}\abs{f(x)}$ is a Banach space! To apply Banach's fixed point theorem we need $\tilde T$ to be a contraction. \\
	Fix $f_1,f_2 \in C([0,1])$ and get for $x \in [0,1]$
	\begin{align*}
		\abs{(\tilde T(f_1)- \tilde T(f_2))(x)} & = \abs{5 \int_{0}^{1-x} \min(x,y)f_2(y) \,\mathrm{d}y - 5 \int_{0}^{1-x}\min(x,y)f_1(y) \,\mathrm{d}y} \\
		&= \abs{5 \int_{0}^{1-x} \min(x,y)(f_2(y)-f_1(y)) \,\mathrm{d}y} \\
		&\leq 5 \int_{0}^{1-x} \min(x,y) \underset{\leq \norm{f_2-f_1}}{\underbrace{\abs{f_2(y)-f_1(y)}}} \,\mathrm{d}y \\
		&\leq 5 \underset{0 \leq  \dots \leq \frac{1}{6}}{\underbrace{\int_{0}^{1-x} \min(x,y)\,\mathrm{d}y}} \norm{f_2-f_1} \\
		&\leq \frac{5}{6} \norm{f_2-f_1}.
	\end{align*}
	Hence \[
		\norm{\tilde T(f_1)- \tilde T(f_2)} \leq \frac{5}{6} \norm{f_1-f_2}.
	\]
	We conclude that $\tilde T$ is a contraction. We can take $c = \frac{5}{6}$. By Banach's fixed point theorem $\tilde T$ has a unique fixed point. Finally $(*)$
	has a unique solution $f \in C([0,1])$ which is the fixed point. 
\end{beispiel}
\begin{theorem}[Banach's fixed point theorem (generalization)]
	$(E, \norm{.})$ Banach space. $T: F \to F$ where $F$ is a closed set in $E$. $N$ positive integer. Assume $T^N = \underset{N-\text{times}}{\underbrace{T \circ T \circ \dots \circ T}}$ is a contraction on $F$, i.e. there exists $c > 1$ such that
	\[
		\norm{T^N(x)- T^N(\tilde x)} \leq c \norm{x-\tilde x}, \qquad \text{for all }x, \tilde x \in F.
	\]
	Then $T$ has unique fixed point $\bar{ x}$, i.e.
	\[
		\bar{x} = T(\bar{x}) \in F.
	\]
\end{theorem}
\begin{beweis}
	\begin{description}
		\item[$N=1$:] Fix $x_0 \in F$ and consider $(x_n)_{n=1}^{\infty}$ where $x_{n+1} = T(x_n)$ for $n=0,1,2, \dots$. There $(x_n)_{n=1}^{\infty}$ is a Cauchy sequence and hence this converges in $E$ since this is a Banach space. Call the limit $\bar{x}$. Note
		\[
			\underset{\in F}{\underbrace{x_n}} \to \bar{x} \text{ in }E \text{ and $F$ is closed}
		\] 
		implies $\bar{x} \in F$. The rest of the argument is the same as before.
		\item[$N>1$:] By previous result we know that $T^N$ has a unique fixed point $\bar{x} \in F$, i.e. $\bar{x} = T^N(\bar{x})$. \\
		\textbf{Claim:} \text{    }     $\bar{x}$ is a fixed point for $T$.
		\begin{align*}
			\norm{T(\bar{x})-\bar{x}} &= \norm{T(T^N(\bar{x}))- T^N(\bar{x})} \\
			&= \norm{T^N(T(\bar{x}))- T^N(\bar{x})} \\
			&\leq c \norm{T(\bar{x})-\bar{x}}.
 		\end{align*}
		This gives 
		\[
			\norm{T(\bar{x}-\bar{x})} = 0, \qquad \text{i.e. }\bar{x} = T(\bar{x}).
		\]
		Existence of a fixed point for $T$ done. For the uniqueness assume $\bar{x} = T(\bar{x})$ and $\tilde x = T( \tilde x)$. Then
	\begin{align*}
		\bar{x} &= T( \bar{x}) = T^2(\bar{x}) = \dots = T^N(\bar{x}) \\
		\tilde x &= T(\tilde x) = T^2(\tilde x) = \dots = T^N(\tilde x).
	\end{align*}
	But $T^N$ has a unique fixed point so 
	\[
		\bar{x} = \tilde x.
	\]
	\end{description}
\end{beweis}
\begin{bemerkung}
	\begin{enumerate}[(1)]
		\item $T: (0,1] \to (0,1]$ where $T(x) = \frac{x}{2}$. Clearly $T$ is a contraction on $(0,1]$ but has no fixed point. Note that $(0,1]$ is not a closed intervall. 
		\item $T: [0,\infty) \to [0,\infty)$, where $T(x) = x + \frac{1}{x}$. Clearly $[0,\infty)$ is a closed intervall in $\mathbb{R}$ but $T$ has no fixed point. \\
		\textbf{Claim:} \text{    }     $T$ is not a contraction but 'close' to be a contraction. \\
		\begin{align*}
			\abs{T(x)-T(\tilde x)} < \abs{x- \tilde x} \qquad \text{for }x, \tilde x \in [1, \infty), x \neq \tilde x
		\end{align*}
		Note \[
			\abs{ T(x)- T( \tilde x)} = \abs{\underset{\substack{(1- \frac{1}{t})\leq 1 \\ \text{for }t \in [1,\infty)}}{\underbrace{T'(x)}}}\abs{x- \tilde x}
		\] for some $t$ betweeen $x$ and $\tilde x$.
	\end{enumerate}
\end{bemerkung}
\begin{beispiel}
	$(E,\norm{.})$ Banach space. $K$ compact set in $E$ and $T : K \to K$ where
	\[
		\norm{T(x)- T( \bar{x})} < \norm{x - \bar{x}} \qquad \text{for all }x, \bar{x} \in K, x \neq \bar{x}.
	\]
	Show: $T$ has a unique fixed point in $K$.
	\begin{description}
		\item[Uniqueness:] Assume $\bar{x} = T(\bar{x})$ and $\tilde x = T( \tilde x)$ and $\bar{x} \neq \tilde x$ for $ \bar{x}, \tilde x \in K$. Then
		\[
			\norm{\bar{x} - \tilde x} = \norm{ T( \bar{ x})- \tilde x} < \norm{ \bar{x}- \tilde x}.
		\]
		Contradiction because then $\bar{x} = \tilde x$.
		\item[Existence:] To show: There exists $x \in K$ such that $x = T(x)$, i.e.
		\[
			\norm{T(x)- x} = 0.
		\]
		Set $d := \inf_{x \in K} \norm{T(x)-x}$. Let $(x_n)_{n=1}^{\infty}$ be a sequence in $K$ such that 
		\[
			\norm{T(x_n)-x_n} \to d, \qquad \text{as }n \to \infty.
		\]
		$K$ compact implies that there exists a subsequence $(\tilde x_n)_{n=1}^{\infty}$ of $(x_n)_{n=1}^{\infty}$ such that $(\tilde x_n)_{n=1}^{\infty}$ converges in $K$. Call the limit element $\bar{x} \in K$. We know
		\[
			\tilde x_n \to  \bar{x} \qquad \text{in }K
		\]
		and	 
		\[
			\norm{T( \tilde x_n)- \tilde x_n} \to d.
		\]
		Question: \[
			T(\tilde x_n) \to T(\bar{x}) \qquad \text{ in }K?
		\]
		But since
		\[
			\norm{T(x)- T( \tilde x)} \leq  \norm{x- \tilde x} \qquad \text{ for all }x, \tilde x \in K
		\]
		we have 
		\[
			\tilde x_n \to \bar{x} \qquad \text{ in }K
		\]
		which implies
		\[
			T( \tilde x_n) \to T( \bar{ x}) \text{ in }K.
		\]
		Hence: 
		\[
			\norm{T( \bar{ x})- \bar{x}} \leftarrow \norm{T(\tilde x_n)- \tilde x_n} \to d, \qquad  n \to  \infty.
		\]
		We obtain
		\[
			\norm{T(\bar{x})- \bar{x}} = d.
		\]
		Question: Is $d=0$? \\
		If $d>0$ then $\bar{x} \neq  T( \bar{x})$, $\bar{x}, T( \bar{x}) \in K$
		\[
			\norm{T(\bar{x})- T(T(\bar{x}))} < \norm{\bar{x}- T(\bar{x})} = d = \inf_{x \in K} \norm{x- T(x)}.
		\]
		This is a contradiction which gives $d=0$ and so $\bar{x} = T(\bar{x})$.
	\end{description}
\end{beispiel}
\begin{beispiel}
	Consider
	\[
		f(x) = \int_{0}^{x}k(x,y)h(y,f(y)) \,\mathrm{d}y + g(x), \qquad x \in [0,1] \qquad (*),
	\]
	where $g \in C([0,1])$, $k \in C([0,1] \times [0,1])$ and $h: [0,1] \times \mathbb{R} \to \mathbb{R}$ continuous and satisfies: \\
	There exists $M>0$ such that
	\[
		\abs{h(x,z_1)-h(x,z_2)} \leq M \abs{z_1- z_2} \qquad \text{for all }x \in [0,1],\,z_1,z_2 \in \mathbb{R}.
	\]
	\textbf{Claim:} \text{    }     $(*)$ has a unique solution $f \in C([0,1])$. \\ For $f \in C([0,1])$ set
	\[
		T(f)(x) = \int_{0}^{x}k(x,y)h(y,f(y)) \,\mathrm{d}y + g(x) \qquad x \in [0,1].
	\]
	Here $T(f)(x) \in C([0,1])$. \\ Want to show: $T: C([0,1]) \to C([0,1])$ has a unique fixed point. \\
	Start with the Banach space $(C([0,1]), \text{max-norm})$. Check if $T$ is a contraction in $C([0,1])$. Fix $f_1,f2 \in C([0,1])$
	\[
		T(f_1)(x)- T(f_2)(x) = \int_{0}^{x} k(x,y)(h(y,f_1(y))-h(y,f_2(y))) \,\mathrm{d}y.
	\] 
	$k$ is continuous on the compact set $[0,1] \times [0,1]$ so 
	\[
		\sup\limits_{(x,y) \in [0,1]\times[0,1]} \abs{k(x,y)} =: N < \infty.
	\]
	We obtain 
	\begin{align*}
		\abs{(T(f_1)-T(f_2))(x)} &\leq \int_{0}^{x} \underset{\leq N}{\underbrace{\abs{k(x,y)}}}\underset{\leq M \underset{\leq \norm{f_1-f_2}}{\underbrace{f_1(y)-f_2(y)}}}{\underbrace{h(y,f_1(y))-h(y,f_2(y))}} \,\mathrm{d}y \\ & \leq \int_{0}^{x}NM \,\mathrm{d}y \norm{f_1-f_2} \\& \leq NM \norm{f_1-f_2}.
		\end{align*}
	This yields
	\[
		\norm{T(f_1)-T(f_2)} \leq NM \norm{f_1-f_2}.
	\]
	\begin{Large}
		\underline{IF:}
	\end{Large} \,$NM<1$ Then $T$ is a contaction. \\ Trick: For $a >0$ set 
	\[
		\norm{f}_a = \max\limits_{x \in [0,1]} e^{-ax}\abs{f(x)}
	\] 
	for $f \in C([0,1])$. \\
	\textbf{Claim:} \text{    }     $\norm{.}_a$ defines a norm on $C([0,1])$. This is easy to check. \\
	\textbf{Claim:} \text{    }     $\norm{.}$ and $\norm{.}_a$ are equivalent. \\
	This follows from
	\[
		e^{-a} \norm{f} \leq \norm{f}_a \leq \norm{f}
	\]
	for all $f \in C([0,1])$ (note that $\norm{.}$ is the max-norm).\\
	\textbf{Claim:} \text{    }     $(C([0,1]), \norm{.}_a)$ is a Banach space. \\
	This follows from the fact that $\norm{.}$ und $\norm{.}_a$ are equivalent and $(C([0,1]),\norm{.})$ is a Banach space. \\
	\textbf{Claim:} \text{    }     $T$ is a contraction on $(C([0,1]),\norm{.}_a)$ for $a >0$ large enough. \\
	For $f_1,f_2 \in C([0,1])$ and $x \in [0,1]$ we have
	\begin{align*}
		\abs{(T(f_1)-T(f_2))(x)} &\leq \int_{0}^{x} NM \abs{(f_1-f_2)(y)} \,\mathrm{d}y \\
		&= \int_{0}^{x}NM e^{ay} \cdot \underset{\leq \norm{f_1-f_2}_a}{\underbrace{e^{-ay} \abs{(f_1-f_2)(x)}}} \,\mathrm{d}y \\
		&\leq NM \underset{\frac{1}{a}(e^{ax}-1)}{\underbrace{\int_{0}^{x} e^{ay} \,\mathrm{d}y}} \norm{f_1-f_2}_a.
	\end{align*}
	So
	\[
		e^{-ax} \abs{(T(f_1)-T(f_2))(x)} \leq \frac{NM}{a}(1-e^{-ax})\norm{f_1-f_2}_a
	\]
	and
	\[
		\norm{T(f_1)-T(f_2)}_a \leq  \frac{NM}{a} \norm{f_1-f_2}_a
	\]
	For $a > NM$ is $T$ a contraction on $(C([0,1]),\norm{.}_a)$. Banach fixed point theorem implies that there is a unique $f \in C([0,1])$ that solves $(*)$.
\end{beispiel}
