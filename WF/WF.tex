%!TEX TS-program = xelatex

\newcommand{\Semester}{WiSe 2016/2017, Term 2}
\newcommand{\fach}{Fourier and wavelet analysis}
\newcommand{\prof}{Prof.\ Mohammad Asadzadeh}

%!TEX root = ../../PDGL1_SS16/pdeskript.tex
%!TEX TS-program = xelatex
%\documentclass[a4paper, twoside, headsepline, index=totoc,toc=listof, fontsize=11pt, cleardoublepage=empty, headinclude, DIV=12, BCOR=5mm, titlepage]{scrartcl}
\documentclass[a4paper, twoside, headsepline, index=totoc,toc=listof,toc=bibliography,toc=index, fontsize=11pt, cleardoublepage=empty, headinclude, DIV=12, BCOR=5mm, titlepage,draft]{scrartcl}
\usepackage{scrtime} % KOMA, Uhrzeit ermoeglicht
\usepackage{scrpage2} % wie fancyhdr, nur optimiert auf KOMA-Skript, leicht andere Syntax
\usepackage{etoolbox}
\usepackage{letltxmacro}
\usepackage{ifthen}


%--Farbdefinitionen
%-- muss vor tikz geladen werden
\usepackage[usenames, table, x11names]{xcolor}
\definecolor{dark_gray}{gray}{0.45}
\definecolor{light_gray}{gray}{0.6}
\definecolor{fb10_blue}{cmyk}{0.8,0.4,0.13,0.07}
\usepackage[final]{graphicx}

%--Zum Zeichnen
%-- muss vor polyglossia bzw. babel geladen werden
\usepackage{tikz}
\usepackage{tikz-cd}
\usetikzlibrary{external}
\tikzset{>=latex}
\usetikzlibrary{shapes,arrows.meta,intersections}
\usetikzlibrary{calc,3d}
\usetikzlibrary{decorations.pathreplacing,decorations.markings, decorations.pathmorphing}
\usetikzlibrary{angles}

%-- Konfiguration von tikzexternalize
\tikzexternalize[prefix=tikz/,up to date check=diff]
\pgfkeys{/pgf/images/include external/.code=\includegraphics{#1}}
\tikzset{external/system call={xelatex \tikzexternalcheckshellescape -halt-on-error -interaction=batchmode --shell-escape -jobname "\image" "\texsource"}}


%-- tikzexternalize fuer tikzcd deaktivieren, da inkompatibel
\AtBeginEnvironment{tikzcd}{\tikzexternaldisable}
\AtEndEnvironment{tikzcd}{\tikzexternalenable}

%-- um Inkompatibilitaeten von quotes und polyglossia bzw. babel zu vermeiden
\tikzset{
  every picture/.append style={
    execute at begin picture={\shorthandoff{"}},
    execute at end picture={\shorthandon{"}}
  }
}
\usetikzlibrary{quotes}
\usepackage{pgfplots}
\usepgfplotslibrary{colormaps}
% \pgfplotsset{compat=1.12}


%-- Mathesymbole etc.
\usepackage{mathtools} % beinhaltet amsmath
\mathtoolsset{showonlyrefs, centercolon}
\newtagform{brackets}[\textbf]{[}{]}
\usetagform{brackets}
\usepackage{wasysym} % zusätzliche Symbole
\usepackage{amssymb} % zusätzliche Symbole
\usepackage{latexsym} % zusätzliche Symbole
\usepackage{stmaryrd} % für Blitz
\usepackage{nicefrac} % schräge Brüche
\usepackage{cancel} % Befehle zum Durchstreichen
\usepackage{mathdots} % Verbesserung von Punkten wie zB \ldots
\usepackage{mathrsfs} % das X wurde benötigt
\usepackage{esint}

%-- einzelne Symbole aus dem mathx-package
\DeclareFontFamily{U}{mathx}{\hyphenchar\font45}
\DeclareFontShape{U}{mathx}{m}{n}{<-> mathx10}{}
\DeclareSymbolFont{mathx}{U}{mathx}{m}{n}
\DeclareMathSymbol{\bigplus}{1}{mathx}{"90}

%-- 
\DeclareSymbolFont{bbold}{U}{bbold}{m}{n}
\DeclareSymbolFontAlphabet{\mathbbold}{bbold}
\newcommand{\mathds}[1]{\mathbb{#1}} % Um Kompatibilitaet mit frueheren Benutzung von dsfont herzustellen
\newcommand{\ind}{\mathbbold{1}} % charakteristische-Funktion-Eins

\def\mathul#1#2{\color{#1}\underline{{\color{black}#2}}\color{black}} %farbiges Untersteichen im Mathe-Modus
\renewcommand{\le}{\leqslant}
\renewcommand{\ge}{\geqslant}

%-- Underbrace als Befehl in LaTeX-Syntax (und ohne Spacing probleme mit nachfolgenden Operatoren...)
\newcommand{\Underbrace}[2]{{\underbrace{#1}_{#2}}}

%-- Alles was mit Schrift und XeTeX zu tun hat
\usepackage[no-math]{fontspec}
\usepackage{polyglossia} % moderner babel-ersatz
%\setmainlanguage[spelling=new,babelshorthands=true]{english}
%\shorthandoff{"}
\setmainlanguage{english}
%\setotherlanguage{english}
% \newcommand\glqq{"}
% \newcommand\grqq{"}
\defaultfontfeatures{Mapping=tex-text, WordSpace={1.4}} %
\setmainfont{SourceSansPro}[Scale=MatchUppercase,Extension=.otf,UprightFont=*-Regular,BoldFont=*-Semibold,ItalicFont=*-LightIt,BoldItalicFont=*-SemiboldIt]
\setsansfont{SourceSansPro}[Scale=MatchUppercase,Extension=.otf,UprightFont=*-Regular,BoldFont=*-Semibold,ItalicFont=*-LightIt,BoldItalicFont=*-SemiboldIt]

\setmonofont{Inconsolatazi4}[Scale=MatchLowercase,Extension=.otf,UprightFont=*-Regular,BoldFont=*-Bold,StylisticSet=1]
\usepackage{xltxtra}
% \usepackage{fontawesome}
\usepackage{microtype}



%--Mixed
\usepackage[neverdecrease]{paralist}
\usepackage[german=quotes]{csquotes}
\usepackage{makeidx}
\usepackage{booktabs}
\usepackage{wrapfig}
\usepackage{float}
\usepackage[margin=10pt, font=small, labelfont={sf, bf}, format=plain, indention=1em]{caption}
\captionsetup[wrapfigure]{name=Abb. }
\usepackage{stackrel}
\usepackage{multicol}

\flushbottom


%--Unterstreichung
\usepackage[normalem]{ulem}
\setlength{\ULdepth}{1.8pt}

%--Indexverarbeitung
\newcommand{\bet}[1]{\textbf{#1}}
\newcommand{\Index}[1]{\textbf{#1}\index{#1}}
\makeindex
\setindexpreamble{{\noindent \itshape Die \emph{Seitenzahlen} sind mit Hyperlinks zu den entsprechenden Seiten versehen, also anklickbar}\par \bigskip}
\renewcommand{\indexpagestyle}{scrheadings}


%--Marginnote & todonotes
\deffootnote[1.5em]{1.5em}{1.5em}{\textsuperscript{\thefootnotemark}\ }
\usepackage[fulladjust]{marginnote}
\renewcommand*{\marginfont}{\color{dark_gray} \itshape \footnotesize}
\usepackage[textsize=small]{todonotes}
\usepackage{ragged2e}
\renewcommand*{\raggedleftmarginnote}{\RaggedLeft}
\renewcommand*{\raggedrightmarginnote}{\RaggedRight}

%--Todonotes mit tikz/externalize kompatibel machen
\LetLtxMacro{\oldtodo}{\todo}
\renewcommand{\todo}[2][]{\tikzexternaldisable\oldtodo[#1]{#2}\tikzexternalenable}
\LetLtxMacro{\oldmissingfigure}{\missingfigure}
\renewcommand{\missingfigure}[2][]{\tikzexternaldisable\oldmissingfigure[{#1}]{#2}\tikzexternalenable}

%--Konfiguration von Hyperref pdfstartview=FitH, 
\usepackage[hidelinks, pdfpagelabels,  bookmarksopen=true, bookmarksnumbered=true, linkcolor=black, urlcolor=SkyBlue2, plainpages=false,pagebackref, citecolor=black, hypertexnames=true, pdfauthor={Tim Keil}, pdfborderstyle={/S/U}, linkbordercolor=SkyBlue2, colorlinks=false,final,backref=false]{hyperref}

\usepackage[nameinlink,noabbrev]{cleveref}

% \newcommand{\appendLink}[1]{#1\,\faExternalLink}
% \newcommand{\hrefsym}[2]{\href{#1}{\texttt{\appendLink{#2}}}}
% \newcommand{\hrefsymX}[2]{\href{#1}{\appendLink{#2}}}
% \newcommand{\hrefsymmail}[2]{\href{#1}{\texttt{\faEnvelopeO\,#2}}}
% \renewcommand{\url}[1]{\hrefsym{#1}{\nolinkurl{#1}}}

% -- QR-Codes
% \usepackage{qrcode} %-- hinter hyperref laden!

%--Römische Zahlen
\newcommand{\RM}[1]{\MakeUppercase{\romannumeral #1{}}}
% \renewcommand{\thesection}{\Roman{section}} % für römische section nummerierung

%%--Abkürzungen etc.
\newcommand{\light}{\text{\Large $\lightning$}}

%-- Definitionen von weiteren Mathe-Befehlen
\DeclareMathOperator{\re}{Re} %Realteil
\DeclareMathOperator{\diver}{div} %Realteil
\DeclareMathOperator{\im}{Im} %Imaginaerteil
\DeclareMathOperator{\id}{id} %identische Abbildung
\DeclareMathOperator{\Hom}{Hom} 
\DeclareMathOperator{\Tr}{Tr} 
\DeclareMathOperator{\dist}{dist} 
\DeclareMathOperator{\Mat}{Mat}
\DeclareMathOperator{\Eig}{Eig}
\DeclareMathOperator{\GL}{GL}
\DeclareMathOperator{\pr}{pr}
\DeclareMathOperator{\supp}{supp}
\DeclareMathOperator{\sign}{sign}
\DeclareMathOperator{\esssup}{esssup}
\DeclareMathOperator{\spn}{span}
\newcommand{\mathd}{\ensuremath{\mathrm{d}\mkern-1.0mu}}



%--Skalarprodukt
\DeclarePairedDelimiterX\sprod[2]{\langle}{\rangle}{#1\,\delimsize\vert\,#2}
\DeclarePairedDelimiterX\skal[2]{\langle}{\rangle}{#1\,,\,#2}

%--Betrag, Gaußklammer
\DeclarePairedDelimiter{\abs}{\lvert}{\rvert}
\DeclarePairedDelimiter{\floor}{\lfloor}{\rfloor}
\DeclarePairedDelimiter{\ceil}{\lceil}{\rceil}

%--Norm
\DeclarePairedDelimiter\doppelstrich{\Vert}{\Vert}
\newcommand{\norm}[2][\relax]{
\ifx#1\relax \ensuremath{\doppelstrich*{#2}}
\else \ensuremath{\doppelstrich*{#2}_{#1}}
\fi}


%--Umklammern
\DeclarePairedDelimiter\enbrace{(}{)}
\DeclarePairedDelimiter\benbrace{[}{]}
\DeclarePairedDelimiter\lenbrace{<}{>}
\newcommand{\ssbrace}[1]{{\scriptscriptstyle\enbrace{#1}}}

%--Mengen
\DeclarePairedDelimiterX\mengenA[1]{\lbrace}{\rbrace}{#1}
\DeclarePairedDelimiterX\mengenB[2]{\lbrace}{\rbrace}{#1\, \delimsize\vert \, #2}

\makeatletter
\newcommand{\set}[2][\relax]{
\ifx#1\relax \ensuremath{
\mengenA*{#2}}
\else \ensuremath{%
  \mengenB*{#1}{#2}}
\fi}
\makeatother

\DeclareRobustCommand{\minwidthbox}[2]{%
  \ifmmode
    \expandafter\mathmakebox
  \else
    \expandafter\makebox
  \fi
  [\ifdim#2<\width\width\else#2\fi]{#1}%
}


%--offen und abgeschlossen
\newcommand{\off}{\! \stackrel[\text{offen}]{}{\subset} \!}
\newcommand{\abg}{\! \stackrel[\text{abg.}]{}{\subset} \!}

%--Differential
\newcommand{\diff}[2]{\ensuremath{\frac{{\partial #1}}{{\partial #2}} }}
\newcommand{\diffd}[2]{\ensuremath{\frac{\mathd #1}{\mathd #2} }}


%thrm declare

% -- theorem packages
\usepackage{amsthm}
\usepackage{thmtools}
\usepackage{mdframed}
\usepackage{blindtext}
\renewcommand{\listtheoremname}{Übersicht aller Aussagen}

% -- Theoreme als PDF-Lesezeichen
\usepackage{bookmark}
\bookmarksetup{open,numbered}
\makeatletter
\newcommand*{\theorembookmark}{%
  \bookmark[
    dest=\@currentHref,
    rellevel=1,
    keeplevel,
  ]{%
    \thmt@thmname\space\csname the\thmt@envname\endcsname
    \ifx\thmt@shortoptarg\@empty
    \else
      \space(\thmt@shortoptarg)%
    \fi
  }%
}   
\makeatother

% -- Definition der einzelnen Umgebungen
\declaretheoremstyle[%
	headfont=\sffamily\bfseries,
	notefont=\normalfont\sffamily,
	bodyfont=\normalfont,
	headformat=\NAME\ \NUMBER\NOTE,
	%headpunct=.,
	postheadspace=1em,
	spaceabove=\parsep,spacebelow=\parsep,
	%shaded={bgcolor=gray!20},
	postheadhook=\theorembookmark,
    mdframed={
        backgroundcolor=gray!20, 
            linecolor=gray!20, 
            innertopmargin=6pt,
            roundcorner=5pt, 
            innerbottommargin=6pt, 
            skipbelow=\parsep, 
            skipbelow=\parsep }
	]%
{mainstyle}
\declaretheoremstyle[%
	headfont=\sffamily\bfseries,
	notefont=\normalfont\sffamily,
	bodyfont=\normalfont,
	headformat=\NAME\ \NUMBER\NOTE,
	%headpunct=.,
	postheadspace=1em,
	spaceabove=\parsep,spacebelow=\parsep,
	%shaded={bgcolor=fb10_blue!20},
	postheadhook=\theorembookmark,
    mdframed={
        backgroundcolor=fb10_blue!20, 
            linecolor=fb10_blue!20, 
            innertopmargin=6pt,
            roundcorner=5pt, 
            innerbottommargin=6pt, 
            skipbelow=\parsep, 
            skipbelow=\parsep }
]{mainstyle_blue}
\declaretheoremstyle[%
	headfont=\sffamily\bfseries,
	notefont=\normalfont\sffamily,
	bodyfont=\normalfont,
	headformat=\NAME\ \NUMBER\NOTE,
	%headpunct=.,
	postheadspace=1em,
	spaceabove=15pt,spacebelow=10pt,
	postheadhook=\theorembookmark]%
{mainstyle_unshaded}
\declaretheoremstyle[%
	headfont=\sffamily\bfseries,
	notefont=\normalfont\sffamily,
	bodyfont=\normalfont,
	headformat=\NUMBER\NAME\NOTE,
	%headpunct=.,
	postheadspace=1em,
	spaceabove=15pt,spacebelow=10pt,
	% shaded={bgcolor=gray!20},
	postheadhook=\theorembookmark]%
{mainstyle_unnumbered}
\declaretheoremstyle[%
	headfont=\sffamily\bfseries,
	notefont=\normalfont\sffamily,
	bodyfont=\normalfont,
	headformat=swapnumber,
	%headpunct=.,
	postheadspace=1em,
	spaceabove=15pt,spacebelow=10pt,
	shaded={bgcolor=gray!20},
	postheadhook=\theorembookmark,
	qed=\qedsymbol]%
{mainstyleB}

\declaretheorem[name=Definition,parent=section,style=mainstyle_blue]{definition}
\declaretheorem[name=Definition,numbered=no,style=mainstyle_blue]{definition*}
\declaretheorem[name=Theorem,sharenumber=definition,style=mainstyle]{theorem}
\declaretheorem[name=Theorem,numbered=no,style=mainstyle]{theorem*}
\declaretheorem[name=Proposition,sharenumber=definition,style=mainstyle]{proposition}
\declaretheorem[name=Proposition,numbered=no,style=mainstyle_blue]{proposition*}
\declaretheorem[name=Lemma,sharenumber=definition,style=mainstyle]{lemma}
\declaretheorem[name=Lemma,numbered=no,style=mainstyle_blue]{lemma*}
\declaretheorem[name=Statement,sharenumber=definition,style=mainstyle]{satz}
\declaretheorem[name=Statement,sharenumber=definition,style=mainstyle_unshaded]{satzUnshaded}
\declaretheorem[name=Definition,sharenumber=definition,style=mainstyle_unshaded]{definitionUnshaded}
\declaretheorem[name=Satz,numbered=no,style=mainstyle_unnumbered]{satz*}
\declaretheorem[name=Korollar,sharenumber=definition,style=mainstyle]{korollar}
\declaretheorem[name=Korollar,sharenumber=definition,style=mainstyleB]{korollarB}

\declaretheorem[name=Notation,numbered=no,style=mainstyle_unnumbered]{notation}
\declaretheorem[name=Remark,numbered=no,style=mainstyle_unnumbered]{bemerkung}
\declaretheorem[name=Nebenbedingung,numbered=no,style=mainstyle_unnumbered]{nb}
\declaretheorem[name=Example,numbered=no,style=mainstyle_unnumbered]{beispiel}
\declaretheorem[name=Examples,numbered=no,style=mainstyle_unnumbered]{beispiele}


% \declaretheorem[name=Notation,sharenumber=definition,style=mainstyle_unshaded]{notation}
% \declaretheorem[name=Bemerkung,sharenumber=definition,style=mainstyle_unshaded]{bemerkung}
% \declaretheorem[name=Nebenbedingung,sharenumber=definition,style=mainstyle_unshaded]{nb}
% \declaretheorem[name=Beispiel,sharenumber=definition,style=mainstyle_unshaded]{beispiel}
% \declaretheorem[name=Beispiele,sharenumber=definition,style=mainstyle_unshaded]{beispiele}

% -- Beweise
\declaretheoremstyle[headfont=\bfseries,bodyfont=\normalfont,headpunct=.,postheadspace=1em,spacebelow=12pt,spaceabove=2pt,qed=\qedsymbol]{beweise}
\declaretheoremstyle[headfont=\sffamily\bfseries,bodyfont=\normalfont,headpunct=:,postheadspace=1em,spacebelow=10pt,spaceabove=10pt]{bemerkungen}
\declaretheorem[name=proof,numbered=no,style=beweise]{beweis}

% %für schönere abhebungen
% \usepackage[framemethod=TikZ]{mdframed}
% \usepackage{amsthm}
% \usepackage{thmtools}
%
% \declaretheoremstyle[%
% 	headfont=\sffamily\bfseries,
% 	notefont=\normalfont\sffamily,
% 	bodyfont=\normalfont,
% 	headformat=\NUMBER\ \NAME\NOTE,
% 	headpunct={},
% 	postheadspace=1ex,
% 	spaceabove=15pt,spacebelow=30pt,]%
% {mainstyle}
%
% \declaretheoremstyle[%
% 	headfont=\sffamily\bfseries,
% 	notefont=\normalfont\sffamily,
% 	bodyfont=\normalfont,
% 	headformat=\NUMBER\ \NAME\NOTE,
% 	headpunct={},
% 	postheadspace=1ex,
% 	backgroundcolor = blue!10,
% 	align = center , % align the environment itself (left, center, rigth)
% 	nobreak = true, % prevent a frame from splitting
% 	hidealllines = true , %
% 	topline = true , bottomline = true , %
% 	spaceabove=15pt,spacebelow=30pt,]%
% {unterlegt}
%
% \declaretheoremstyle[%
% 	headfont=\bfseries\scshape,
% 	bodyfont=\normalfont,
% 	headpunct=:,
% 	postheadspace=1ex,
% 	spacebelow=12pt,spaceabove=2pt,
% 	qed=\qedsymbol]%
% {beweise}
%
% \declaretheorem[name=Definition,parent=section,style=unterlegt]{definition}
% \declaretheorem[name=Satz,sharenumber=definition,style=unterlegt]{satz}
% \declaretheorem[name=Korollar,sharenumber=definition,style=mainstyle]{korollar}
% \declaretheorem[name=Lemma,sharenumber=definition,style=mainstyle]{lemma}
% \declaretheorem[name=Proposition,sharenumber=definition,style=mainstyle]{proposition}
%
% \declaretheorem[name=Beweis,numbered=no,style=beweise]{beweis}
% \declaretheorem[name=Bemerkung,numbered=no,style=mainstyle]{bemerkung}
% \declaretheorem[name=Notation,numbered=no,style=mainstyle]{notation}
% \declaretheorem[name=Nebenbedingung,numbered=no,style=mainstyle]{nb}
% \declaretheorem[name=Beispiel,numbered=no,style=mainstyle]{beispiel}
% \declaretheorem[name=Beispiele,numbered=no,style=mainstyle]{beispiele}

\declaretheoremstyle[
    headfont=\bfseries, 
    notebraces={[}{]},
    bodyfont=\normalfont\itshape,
    headpunct={},
    postheadspace=\newline,
    spacebelow=\parsep,
    spaceabove=\parsep,
    mdframed={
        backgroundcolor=red!20, 
            linecolor=red!30, 
            innertopmargin=6pt,
            roundcorner=5pt, 
            innerbottommargin=6pt, 
            skipbelow=\parsep, 
            skipbelow=\parsep } 
]{myexamplestyle}

% example environment - thmtools
\declaretheorem[
    style=myexamplestyle,
    name=Example,
    numberwithin=section
]{example}

%%% Titelseite
\providecommand{\verfasser}{Tim Keil}

%--Konfiguration von scrheadings
\setheadsepline{1pt}[\color{light_gray}]
\pagestyle{scrheadings}
\clearscrheadfoot

\providecommand{\shortFach}{\fach}
\lehead{ \includegraphics[height=1.0 cm,keepaspectratio]{!config/Bilder/GU}}
\rehead{\rule{0cm}{0.6cm}\footnotesize \sffamily \color{light_gray} \verfasser{} -- Script \shortFach}
\lohead{\rule{0cm}{0.6cm} \footnotesize \sffamily \color{light_gray} Effective: \today \; \thistime[:]}
\rohead{\includegraphics[height=1.0 cm,keepaspectratio]{!config/Bilder/CH}}


\ofoot[{ \color{dark_gray} \LARGE \sffamily \thepage}]{{ \color{dark_gray} \LARGE \sffamily \thepage}} %hier wir auch der plain Stil bearbeitet!
\automark{section}
\ifoot{ \color{dark_gray} \small \leftmark}


%--Inhaltsverzeichnis
\usepackage[tocindentauto]{tocstyle}
\usetocstyle{KOMAlike}	
%\shorthandon{"}

%--Punkte (als letztes laden)
\usepackage{ellipsis}

%--Metadaten
\providecommand{\mail}{keil.menden@web.de}
\author{\verfasser}
\titlehead{\includegraphics[height=2.5cm, keepaspectratio]{!config/Bilder/GU}%
\hfill \includegraphics[height=2.3cm, keepaspectratio]{!config/Bilder/CH}}
\title{\fach}
\subtitle{Script of \enquote{\fach} by \prof}





\begin{document}

\maketitle
\cleardoubleoddemptypage

\pagenumbering{Alph}
\section*{foreword --- cooperation}
This document is a transcript of the lecture \enquote{\fach, \Semester}, by \prof.
It mainly contains the written content of the lecture. I will not assume any responsibility for the correctness of the content! For questions, remarks and mistakes please write an email to \href{mailto:keil.menden@web.de}{\nolinkurl{keil.menden@web.de}}. I'm grateful for every email. 
\newpage

\newpage

\tableofcontents
\cleardoubleoddemptypage
\pagenumbering{arabic}
\setcounter{page}{1}

\section{Introduction} 
\label{sec:introduction}
\begin{definition*}[Fouriertransforms]
	We remember the Fouriertransformation for $\mathbb{R}^n$
	\[
		\hat{f}(\xi) = \int_{\mathbb{R}^n}^{}e^{-2 \pi i x \cdot \xi} f(x) \,\mathrm{d}x.
	\]
	where
	\[
		x \cdot \xi = \sum^{n}_{i=1}x_i \xi_j
	\]
\end{definition*}
The course will be about
\begin{itemize}
	\item The Fourier transforms
	\item Distribution theory
	\item Transforms related to Fouriertransforms. (Radon, Hankel z-transforms)
	\item The Wavelet transforms
	\item Multiresolution analysis
	\item Discrete Fouriertransforms, Sampling
	\item Applications (3 Lab obligus)
\end{itemize}
\begin{description}
	\item[Notation] 
	\[
		\begin{cases}
			\hat{f}(\xi) &= (\mathcal{F}f)(\xi) \\
			f(x) &\stackrel{\mathcal{F}}{\supset} \hat{f}(\xi)
		\end{cases}
	\]
\item[Basic properties]
\begin{description}
	\item[Linearity] \begin{align*}
		f + g &\supset \hat{f} + \hat{g} \\
		\alpha f &\supset \alpha \hat{f} \qquad (\alpha \in \mathbb{C}, \alpha \in \mathbb{R}) 
	\end{align*} 
	\item[Scaling] \[
		f \supset \hat{f} \qquad \Leftrightarrow \qquad \frac{1}{a}f ( \frac{x}{a}) \stackrel{\mathcal{F}}{\supset} \hat{f}(a \xi)
	\]
\end{description}
If $f$ is "resonably regular/smoth" so is $\hat{f}$.
\item[ouriertransform in $L_2(\mathbb{R})$]
\[
	L_2(\mathbb{R}):= \set[f]{f \text{ is measurable and } \int_{\mathbb{R}}^{}\abs{f(x)}^2 \,\mathrm{d}x < \infty}
\]
\item[Scalar product]
With $f,g :\mathbb{R} \to \mathbb{C}$ we have
\[
	\skal{f}{g} = \int_{\mathbb{R}}^{} f(x) \overline{g(x)} \,\mathrm{d}x.
\]
\item[Parsevals formula]
\[
	\int_{\mathbb{R}}^{}\abs{f(x)}^2 \,\mathrm{d}x = \int_{\mathbb{R}}^{} \abs{\hat{f}(\xi)}^2 \,\mathrm{d}\xi.
\]
\end{description}
By "regular/smoth" we will mean Schwartz class $S$. \\
Often we will have that $f(x)$ is a signal where $f(t)= sin(\omega t)$.
\begin{definition*}[Hevyside]
	We call
	\[
		f(t) = H(x) := \begin{cases}
			1, &x \geq 0 \\
			0, &x <0
		\end{cases}
	\]
	the Hevyside-function. \\
	We have also 
	\[
		f(t) = \delta(t)
	\]
	which is called the Diracs "$\delta $-function" and it is the derivative of the Hevyside function. The Dirac functions are not in $L_2$ (nor in $S$).
\end{definition*}
\begin{fundamental}[The Fourier inversion forumula]
	\[
		\hat{f}(\xi) = \int_{\mathbb{R}}^{} e^{-2 \pi i x \xi} f(x) \,\mathrm{d}x \qquad \Leftrightarrow \qquad f(x) = \int_{\mathbb{R}}^{}e^{2 \pi i x \xi}\hat{f}(\xi) \,\mathrm{d}\xi.
	\]
\end{fundamental}
\begin{beispiel}
	Consider
	\[
		\mathcal{F}(e^{-\pi x^2}) = e^{- \pi \xi^2}.
	\]
	We want to scale with $a=2$. Then we get
	\[
		\frac{1}{2} e^{- \pi \left( \frac{x}{2} \right)^2} \stackrel{\mathcal{F}}{\supset } e^{-\pi (2 \xi)^2}.
	\]
\end{beispiel}
\minisec{Other transforms:}
\begin{description}
	\item[Hankel:]
	\[
		\hat{f}(\xi_1,\xi_2) = \int_{\mathbb{R}^2}^{} e^{- 2 \pi i (x_1 \xi_1 + x_2 \xi_2)} f(x_1,x_2) \,\mathrm{d}x_1 \,\mathrm{d}x_2
	\] 
	Let 
	\[
		f(x_1,x_2) = F( \sqrt{x_1^2+ x_2^2}) = F(r), \qquad \begin{cases}
			x_1 &= r \cos \theta, \\	
			x_2 &= r \sin \theta		
		\end{cases}
	\]
	Then
	\[
		\hat{f}(\xi_1,\xi_2) = \tilde F(\sqrt{\xi_1^2+ \xi_2^2}) = \tilde F(\rho), \qquad \begin{cases}
			% \xi_1 &= r \cos \theta, \\
% 			\xi_2 &= r \sin \theta		
		\end{cases}
	\]
	\[
		F(r) \mapsto \tilde F( \rho)
	\] 
	\item[Raden] IF $f(x_1,x_2)$ is the density of "the head" at point $(x_1,x_2)$. Then
	\[
		\int_{L_{r,\omega}}^{} f(x_1,x_2) \,\mathrm{d}t
	\]
	gives a measure of damping along $L_{r,\omega}$ (which can be measured).
	\[
		\mathcal{R}_{\omega} = \int_{L_{r,\omega}}^{} f(x_1,x_2) \,\mathrm{d}t
	\]
	is the Radon transform. It has an application in computer tomography.
\end{description}
\subsection{Functions with Fouriertransform} 
\label{sub:functions_with_fouriertransform}
What functions do have a Fouriertransform?
\begin{align*}
	L_2(\mathbb{R}) = \\
	\mathcal{S} \qquad \text{(smooth functions with rapid dacay)} \\
	L_1(\mathbb{R}) = \set[f]{\int_{\mathbb{R}}^{}\abs{f(x)} \,\mathrm{d}x < \infty}
\end{align*}
Note that if
\[
	f \in L_1(\mathbb{R}) 
\]
we get
\begin{align*}
	\abs{\hat{f}f(\xi)} = \abs{\int_{\mathbb{R}}^{} e^{-2 \pi x \xi} f(x) \,\mathrm{d}x} \\
	\leq \int_{\mathbb{R}}^{} \underset{= \abs{f(x)}}{\underbrace{\abs{e^{-2 \pi i x \xi}f(x)}}} \,\mathrm{d}x < \infty
\end{align*}
So we have in general
\[
	f \in L_1(\mathbb{R}) \qquad \Rightarrow \qquad \hat{f} \in L_{\infty}(\mathbb{R})
\]
And more general we have it for $L_p$ and $L_q$ for $\frac{1}{p}+ \frac{1}{q} =1$.
\begin{theorem}[Hausdorff-Young's inequality]
	For $1 \leq p \leq 2$ with $\frac{1}{p} + \frac{1}{q} = 1$ we have
	\[
		f \in L_p(\mathbb{R}) \qquad \Rightarrow \qquad \hat{f} \in L_q(\mathbb{R})
	\]
	This is a variation of Hölder's inequality.
\end{theorem}
\begin{theorem}[Convolution Theorem]
	Let \[
		f(x) \stackrel{\mathcal{F}}{\supset} \hat{f}(\xi) \text{ and } g(x) \stackrel{\mathcal{F}}{\supset} \hat{g}(\xi)
	\]
	Then we define
	\[
		(f * g)(x) = \int_{\mathbb{R}}^{} f(x-y)g(y) \,\mathrm{d}y \qquad \Leftrightarrow \qquad \supset \hat{f}(\xi)\hat{g}(\xi)
	\]
	with the properties
	\begin{description}
		\item[commutative:] $f * g= g * f$
		\item[associative] $f *( g * h) = (f*g) * h$
		\item[distributive] $f * (g + h) = f *g + f *h$ 
	\end{description}
\end{theorem}
\minisec{Translation:}
Let $\tau_a$ be a translation such that $\tau_a: f(.) \mapsto f(. -a)$. Then we have
\begin{align*}
	\mathcal{F}(\tau_a f)(\xi) &= \int_{\mathbb{R}}^{}e ^{-2 \pi i \xi x}f(\underset{x-a}{\underbrace{=y}}) \,\mathrm{d}x \\
	&= \int_{\mathbb{R}}^{} e^{-2 \pi i x \xi(x+y)} f(y) \,\mathrm{d}y \\
	&= e^{-2 \pi i x \xi a} \int_{\mathbb{R}}^{} e^{-2 \pi i \xi y} f(y) \,\mathrm{d}y 
\end{align*}
and also
\begin{align*}
	\tau_a \hat{f}(\xi) &= \int_{\mathbb{R}}^{} e^{-2 \pi i x ( \xi - a )} f(x) \,\mathrm{d}x \\
	&= \int_{\mathbb{R}}^{} e^{-2 \pi i x \xi} \left( e^{2 \pi i x a }f(x) \right) \,\mathrm{d}x
\end{align*}
\minisec{Derivation:}
We have
\[
	f \stackrel{\mathcal{F}}{\supset} \hat{f} \qquad \Rightarrow \qquad \begin{cases}
		Df &\stackrel{\mathcal{F}}{\supset} 2 \pi i \xi \hat{f}(\xi) \\
		- 2 \pi i (.)f &\stackrel{\mathcal{F}}{\supset} D \hat{f}
	\end{cases}
\]
\begin{beweis}
	We have
	\[
		\hat{f}(\xi) = \int_{\mathbb{R}}^{} f(x) e^{-2 \pi i x \xi} \,\mathrm{d}x
	\]
	Then
	\[
		D \hat{f}(\xi) = \int_{\mathbb{R}}^{} \left( -2 \pi i x f(x) \right) e^{-2 pi i x \xi} \,\mathrm{d}x.
	\]
	And also with partial differentiation
	\begin{align*}
		\mathcal{F}(Df) &= \int_{\mathbb{R}}^{} e^{-2 \pi i x \xi} f'(x) \,\mathrm{d}x \\
		&= - \int_{\mathbb{R}}^{} - 2 \pi i \xi e^{-2 \pi i x \xi}f(x) \,\mathrm{d}x.
	\end{align*}
\end{beweis}
\subsection{The schwartz classes $\mathcal{S}$ and $\mathcal{S}'$} 
\label{sub:the_classes_mathcal_s_and_mathcal_s}
[Distributions (generalized functions)]

\begin{definition}[Schwartz-class]
	The function class $\mathcal{S}$ are complex-valued functions $f$ of a real variable such that $f: \mathbb{R} \to \mathbb{C}$ satisfies
	\[
		\sup_{x \in \mathbb{R}} \abs{ \abs{x}^{\alpha} D^{\beta}f(x)} < \infty.
	\]   
	for any choice of $\alpha \geq 0$ and $\beta \geq 0$.
\end{definition}
\begin{beispiele}
	\begin{itemize}
		\item For $a >0$ \[
		f(x) = e^{- a x^2}
	\]
	What if $a = i$ or $a <0$?
	\item \[
		f(x) = \begin{cases}
			e^{- \frac{1}{(x-2)^2} - \frac{1}{(x-b)^2}}, &a < x < b \\
			0, &\text{else}
		\end{cases}
	\]
	$f$ has compact support $f \in C^{\infty}$ "but not! real analytic" (i.e. its power series is not convergent)
	\end{itemize}
\end{beispiele}

%% missing

We will now state properties of $\mathcal{S}$.
\begin{lemma}[properties of $\mathcal{S}$]
	\begin{enumerate}
		\item if $f \in \mathcal{S}$ and if for $\alpha, \beta \in \mathbb{Z}^+$ $g(x) = x^{\alpha} D^{\beta}f(x)$ then $g \in \mathcal{S}$
		\item $f \in  \mathcal{S} \qquad \Rightarrow \qquad \hat{f} \in S$.
	\end{enumerate}
\end{lemma}

\cleardoubleoddemptypage
\pagenumbering{Alph}
\setcounter{page}{1}

\end{document}