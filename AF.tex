%!TEX TS-program = xelatex

\newcommand{\Semester}{WiSe 2016/2017, Term 1}
\newcommand{\fach}{Applied Functionalanalysis}
\newcommand{\prof}{Prof.\ Peter Kumlin}

%!TEX root = ../../PDGL1_SS16/pdeskript.tex
%!TEX TS-program = xelatex
%\documentclass[a4paper, twoside, headsepline, index=totoc,toc=listof, fontsize=11pt, cleardoublepage=empty, headinclude, DIV=12, BCOR=5mm, titlepage]{scrartcl}
\documentclass[a4paper, twoside, headsepline, index=totoc,toc=listof,toc=bibliography,toc=index, fontsize=11pt, cleardoublepage=empty, headinclude, DIV=12, BCOR=5mm, titlepage,draft]{scrartcl}
\usepackage{scrtime} % KOMA, Uhrzeit ermoeglicht
\usepackage{scrpage2} % wie fancyhdr, nur optimiert auf KOMA-Skript, leicht andere Syntax
\usepackage{etoolbox}
\usepackage{letltxmacro}
\usepackage{ifthen}


%--Farbdefinitionen
%-- muss vor tikz geladen werden
\usepackage[usenames, table, x11names]{xcolor}
\definecolor{dark_gray}{gray}{0.45}
\definecolor{light_gray}{gray}{0.6}
\definecolor{fb10_blue}{cmyk}{0.8,0.4,0.13,0.07}
\usepackage[final]{graphicx}

%--Zum Zeichnen
%-- muss vor polyglossia bzw. babel geladen werden
\usepackage{tikz}
\usepackage{tikz-cd}
\usetikzlibrary{external}
\tikzset{>=latex}
\usetikzlibrary{shapes,arrows.meta,intersections}
\usetikzlibrary{calc,3d}
\usetikzlibrary{decorations.pathreplacing,decorations.markings, decorations.pathmorphing}
\usetikzlibrary{angles}

%-- Konfiguration von tikzexternalize
\tikzexternalize[prefix=tikz/,up to date check=diff]
\pgfkeys{/pgf/images/include external/.code=\includegraphics{#1}}
\tikzset{external/system call={xelatex \tikzexternalcheckshellescape -halt-on-error -interaction=batchmode --shell-escape -jobname "\image" "\texsource"}}


%-- tikzexternalize fuer tikzcd deaktivieren, da inkompatibel
\AtBeginEnvironment{tikzcd}{\tikzexternaldisable}
\AtEndEnvironment{tikzcd}{\tikzexternalenable}

%-- um Inkompatibilitaeten von quotes und polyglossia bzw. babel zu vermeiden
\tikzset{
  every picture/.append style={
    execute at begin picture={\shorthandoff{"}},
    execute at end picture={\shorthandon{"}}
  }
}
\usetikzlibrary{quotes}
\usepackage{pgfplots}
\usepgfplotslibrary{colormaps}
% \pgfplotsset{compat=1.12}


%-- Mathesymbole etc.
\usepackage{mathtools} % beinhaltet amsmath
\mathtoolsset{showonlyrefs, centercolon}
\newtagform{brackets}[\textbf]{[}{]}
\usetagform{brackets}
\usepackage{wasysym} % zusätzliche Symbole
\usepackage{amssymb} % zusätzliche Symbole
\usepackage{latexsym} % zusätzliche Symbole
\usepackage{stmaryrd} % für Blitz
\usepackage{nicefrac} % schräge Brüche
\usepackage{cancel} % Befehle zum Durchstreichen
\usepackage{mathdots} % Verbesserung von Punkten wie zB \ldots
\usepackage{mathrsfs} % das X wurde benötigt
\usepackage{esint}

%-- einzelne Symbole aus dem mathx-package
\DeclareFontFamily{U}{mathx}{\hyphenchar\font45}
\DeclareFontShape{U}{mathx}{m}{n}{<-> mathx10}{}
\DeclareSymbolFont{mathx}{U}{mathx}{m}{n}
\DeclareMathSymbol{\bigplus}{1}{mathx}{"90}

%-- 
\DeclareSymbolFont{bbold}{U}{bbold}{m}{n}
\DeclareSymbolFontAlphabet{\mathbbold}{bbold}
\newcommand{\mathds}[1]{\mathbb{#1}} % Um Kompatibilitaet mit frueheren Benutzung von dsfont herzustellen
\newcommand{\ind}{\mathbbold{1}} % charakteristische-Funktion-Eins

\def\mathul#1#2{\color{#1}\underline{{\color{black}#2}}\color{black}} %farbiges Untersteichen im Mathe-Modus
\renewcommand{\le}{\leqslant}
\renewcommand{\ge}{\geqslant}

%-- Underbrace als Befehl in LaTeX-Syntax (und ohne Spacing probleme mit nachfolgenden Operatoren...)
\newcommand{\Underbrace}[2]{{\underbrace{#1}_{#2}}}

%-- Alles was mit Schrift und XeTeX zu tun hat
\usepackage[no-math]{fontspec}
\usepackage{polyglossia} % moderner babel-ersatz
%\setmainlanguage[spelling=new,babelshorthands=true]{english}
%\shorthandoff{"}
\setmainlanguage{english}
%\setotherlanguage{english}
% \newcommand\glqq{"}
% \newcommand\grqq{"}
\defaultfontfeatures{Mapping=tex-text, WordSpace={1.4}} %
\setmainfont{SourceSansPro}[Scale=MatchUppercase,Extension=.otf,UprightFont=*-Regular,BoldFont=*-Semibold,ItalicFont=*-LightIt,BoldItalicFont=*-SemiboldIt]
\setsansfont{SourceSansPro}[Scale=MatchUppercase,Extension=.otf,UprightFont=*-Regular,BoldFont=*-Semibold,ItalicFont=*-LightIt,BoldItalicFont=*-SemiboldIt]

\setmonofont{Inconsolatazi4}[Scale=MatchLowercase,Extension=.otf,UprightFont=*-Regular,BoldFont=*-Bold,StylisticSet=1]
\usepackage{xltxtra}
% \usepackage{fontawesome}
\usepackage{microtype}



%--Mixed
\usepackage[neverdecrease]{paralist}
\usepackage[german=quotes]{csquotes}
\usepackage{makeidx}
\usepackage{booktabs}
\usepackage{wrapfig}
\usepackage{float}
\usepackage[margin=10pt, font=small, labelfont={sf, bf}, format=plain, indention=1em]{caption}
\captionsetup[wrapfigure]{name=Abb. }
\usepackage{stackrel}
\usepackage{multicol}

\flushbottom


%--Unterstreichung
\usepackage[normalem]{ulem}
\setlength{\ULdepth}{1.8pt}

%--Indexverarbeitung
\newcommand{\bet}[1]{\textbf{#1}}
\newcommand{\Index}[1]{\textbf{#1}\index{#1}}
\makeindex
\setindexpreamble{{\noindent \itshape Die \emph{Seitenzahlen} sind mit Hyperlinks zu den entsprechenden Seiten versehen, also anklickbar}\par \bigskip}
\renewcommand{\indexpagestyle}{scrheadings}


%--Marginnote & todonotes
\deffootnote[1.5em]{1.5em}{1.5em}{\textsuperscript{\thefootnotemark}\ }
\usepackage[fulladjust]{marginnote}
\renewcommand*{\marginfont}{\color{dark_gray} \itshape \footnotesize}
\usepackage[textsize=small]{todonotes}
\usepackage{ragged2e}
\renewcommand*{\raggedleftmarginnote}{\RaggedLeft}
\renewcommand*{\raggedrightmarginnote}{\RaggedRight}

%--Todonotes mit tikz/externalize kompatibel machen
\LetLtxMacro{\oldtodo}{\todo}
\renewcommand{\todo}[2][]{\tikzexternaldisable\oldtodo[#1]{#2}\tikzexternalenable}
\LetLtxMacro{\oldmissingfigure}{\missingfigure}
\renewcommand{\missingfigure}[2][]{\tikzexternaldisable\oldmissingfigure[{#1}]{#2}\tikzexternalenable}

%--Konfiguration von Hyperref pdfstartview=FitH, 
\usepackage[hidelinks, pdfpagelabels,  bookmarksopen=true, bookmarksnumbered=true, linkcolor=black, urlcolor=SkyBlue2, plainpages=false,pagebackref, citecolor=black, hypertexnames=true, pdfauthor={Tim Keil}, pdfborderstyle={/S/U}, linkbordercolor=SkyBlue2, colorlinks=false,final,backref=false]{hyperref}

\usepackage[nameinlink,noabbrev]{cleveref}

% \newcommand{\appendLink}[1]{#1\,\faExternalLink}
% \newcommand{\hrefsym}[2]{\href{#1}{\texttt{\appendLink{#2}}}}
% \newcommand{\hrefsymX}[2]{\href{#1}{\appendLink{#2}}}
% \newcommand{\hrefsymmail}[2]{\href{#1}{\texttt{\faEnvelopeO\,#2}}}
% \renewcommand{\url}[1]{\hrefsym{#1}{\nolinkurl{#1}}}

% -- QR-Codes
% \usepackage{qrcode} %-- hinter hyperref laden!

%--Römische Zahlen
\newcommand{\RM}[1]{\MakeUppercase{\romannumeral #1{}}}
% \renewcommand{\thesection}{\Roman{section}} % für römische section nummerierung

%%--Abkürzungen etc.
\newcommand{\light}{\text{\Large $\lightning$}}

%-- Definitionen von weiteren Mathe-Befehlen
\DeclareMathOperator{\re}{Re} %Realteil
\DeclareMathOperator{\diver}{div} %Realteil
\DeclareMathOperator{\im}{Im} %Imaginaerteil
\DeclareMathOperator{\id}{id} %identische Abbildung
\DeclareMathOperator{\Hom}{Hom} 
\DeclareMathOperator{\Tr}{Tr} 
\DeclareMathOperator{\dist}{dist} 
\DeclareMathOperator{\Mat}{Mat}
\DeclareMathOperator{\Eig}{Eig}
\DeclareMathOperator{\GL}{GL}
\DeclareMathOperator{\pr}{pr}
\DeclareMathOperator{\supp}{supp}
\DeclareMathOperator{\sign}{sign}
\DeclareMathOperator{\esssup}{esssup}
\DeclareMathOperator{\spn}{span}
\newcommand{\mathd}{\ensuremath{\mathrm{d}\mkern-1.0mu}}



%--Skalarprodukt
\DeclarePairedDelimiterX\sprod[2]{\langle}{\rangle}{#1\,\delimsize\vert\,#2}
\DeclarePairedDelimiterX\skal[2]{\langle}{\rangle}{#1\,,\,#2}

%--Betrag, Gaußklammer
\DeclarePairedDelimiter{\abs}{\lvert}{\rvert}
\DeclarePairedDelimiter{\floor}{\lfloor}{\rfloor}
\DeclarePairedDelimiter{\ceil}{\lceil}{\rceil}

%--Norm
\DeclarePairedDelimiter\doppelstrich{\Vert}{\Vert}
\newcommand{\norm}[2][\relax]{
\ifx#1\relax \ensuremath{\doppelstrich*{#2}}
\else \ensuremath{\doppelstrich*{#2}_{#1}}
\fi}


%--Umklammern
\DeclarePairedDelimiter\enbrace{(}{)}
\DeclarePairedDelimiter\benbrace{[}{]}
\DeclarePairedDelimiter\lenbrace{<}{>}
\newcommand{\ssbrace}[1]{{\scriptscriptstyle\enbrace{#1}}}

%--Mengen
\DeclarePairedDelimiterX\mengenA[1]{\lbrace}{\rbrace}{#1}
\DeclarePairedDelimiterX\mengenB[2]{\lbrace}{\rbrace}{#1\, \delimsize\vert \, #2}

\makeatletter
\newcommand{\set}[2][\relax]{
\ifx#1\relax \ensuremath{
\mengenA*{#2}}
\else \ensuremath{%
  \mengenB*{#1}{#2}}
\fi}
\makeatother

\DeclareRobustCommand{\minwidthbox}[2]{%
  \ifmmode
    \expandafter\mathmakebox
  \else
    \expandafter\makebox
  \fi
  [\ifdim#2<\width\width\else#2\fi]{#1}%
}


%--offen und abgeschlossen
\newcommand{\off}{\! \stackrel[\text{offen}]{}{\subset} \!}
\newcommand{\abg}{\! \stackrel[\text{abg.}]{}{\subset} \!}

%--Differential
\newcommand{\diff}[2]{\ensuremath{\frac{{\partial #1}}{{\partial #2}} }}
\newcommand{\diffd}[2]{\ensuremath{\frac{\mathd #1}{\mathd #2} }}


%thrm declare

% -- theorem packages
\usepackage{amsthm}
\usepackage{thmtools}
\usepackage{mdframed}
\usepackage{blindtext}
\renewcommand{\listtheoremname}{Übersicht aller Aussagen}

% -- Theoreme als PDF-Lesezeichen
\usepackage{bookmark}
\bookmarksetup{open,numbered}
\makeatletter
\newcommand*{\theorembookmark}{%
  \bookmark[
    dest=\@currentHref,
    rellevel=1,
    keeplevel,
  ]{%
    \thmt@thmname\space\csname the\thmt@envname\endcsname
    \ifx\thmt@shortoptarg\@empty
    \else
      \space(\thmt@shortoptarg)%
    \fi
  }%
}   
\makeatother

% -- Definition der einzelnen Umgebungen
\declaretheoremstyle[%
	headfont=\sffamily\bfseries,
	notefont=\normalfont\sffamily,
	bodyfont=\normalfont,
	headformat=\NAME\ \NUMBER\NOTE,
	%headpunct=.,
	postheadspace=1em,
	spaceabove=\parsep,spacebelow=\parsep,
	%shaded={bgcolor=gray!20},
	postheadhook=\theorembookmark,
    mdframed={
        backgroundcolor=gray!20, 
            linecolor=gray!20, 
            innertopmargin=6pt,
            roundcorner=5pt, 
            innerbottommargin=6pt, 
            skipbelow=\parsep, 
            skipbelow=\parsep }
	]%
{mainstyle}
\declaretheoremstyle[%
	headfont=\sffamily\bfseries,
	notefont=\normalfont\sffamily,
	bodyfont=\normalfont,
	headformat=\NAME\ \NUMBER\NOTE,
	%headpunct=.,
	postheadspace=1em,
	spaceabove=\parsep,spacebelow=\parsep,
	%shaded={bgcolor=fb10_blue!20},
	postheadhook=\theorembookmark,
    mdframed={
        backgroundcolor=fb10_blue!20, 
            linecolor=fb10_blue!20, 
            innertopmargin=6pt,
            roundcorner=5pt, 
            innerbottommargin=6pt, 
            skipbelow=\parsep, 
            skipbelow=\parsep }
]{mainstyle_blue}
\declaretheoremstyle[%
	headfont=\sffamily\bfseries,
	notefont=\normalfont\sffamily,
	bodyfont=\normalfont,
	headformat=\NAME\ \NUMBER\NOTE,
	%headpunct=.,
	postheadspace=1em,
	spaceabove=15pt,spacebelow=10pt,
	postheadhook=\theorembookmark]%
{mainstyle_unshaded}
\declaretheoremstyle[%
	headfont=\sffamily\bfseries,
	notefont=\normalfont\sffamily,
	bodyfont=\normalfont,
	headformat=\NUMBER\NAME\NOTE,
	%headpunct=.,
	postheadspace=1em,
	spaceabove=15pt,spacebelow=10pt,
	% shaded={bgcolor=gray!20},
	postheadhook=\theorembookmark]%
{mainstyle_unnumbered}
\declaretheoremstyle[%
	headfont=\sffamily\bfseries,
	notefont=\normalfont\sffamily,
	bodyfont=\normalfont,
	headformat=swapnumber,
	%headpunct=.,
	postheadspace=1em,
	spaceabove=15pt,spacebelow=10pt,
	shaded={bgcolor=gray!20},
	postheadhook=\theorembookmark,
	qed=\qedsymbol]%
{mainstyleB}

\declaretheorem[name=Definition,parent=section,style=mainstyle_blue]{definition}
\declaretheorem[name=Definition,numbered=no,style=mainstyle_blue]{definition*}
\declaretheorem[name=Theorem,sharenumber=definition,style=mainstyle]{theorem}
\declaretheorem[name=Theorem,numbered=no,style=mainstyle]{theorem*}
\declaretheorem[name=Proposition,sharenumber=definition,style=mainstyle]{proposition}
\declaretheorem[name=Proposition,numbered=no,style=mainstyle_blue]{proposition*}
\declaretheorem[name=Lemma,sharenumber=definition,style=mainstyle]{lemma}
\declaretheorem[name=Lemma,numbered=no,style=mainstyle_blue]{lemma*}
\declaretheorem[name=Statement,sharenumber=definition,style=mainstyle]{satz}
\declaretheorem[name=Statement,sharenumber=definition,style=mainstyle_unshaded]{satzUnshaded}
\declaretheorem[name=Definition,sharenumber=definition,style=mainstyle_unshaded]{definitionUnshaded}
\declaretheorem[name=Satz,numbered=no,style=mainstyle_unnumbered]{satz*}
\declaretheorem[name=Korollar,sharenumber=definition,style=mainstyle]{korollar}
\declaretheorem[name=Korollar,sharenumber=definition,style=mainstyleB]{korollarB}

\declaretheorem[name=Notation,numbered=no,style=mainstyle_unnumbered]{notation}
\declaretheorem[name=Remark,numbered=no,style=mainstyle_unnumbered]{bemerkung}
\declaretheorem[name=Nebenbedingung,numbered=no,style=mainstyle_unnumbered]{nb}
\declaretheorem[name=Example,numbered=no,style=mainstyle_unnumbered]{beispiel}
\declaretheorem[name=Examples,numbered=no,style=mainstyle_unnumbered]{beispiele}


% \declaretheorem[name=Notation,sharenumber=definition,style=mainstyle_unshaded]{notation}
% \declaretheorem[name=Bemerkung,sharenumber=definition,style=mainstyle_unshaded]{bemerkung}
% \declaretheorem[name=Nebenbedingung,sharenumber=definition,style=mainstyle_unshaded]{nb}
% \declaretheorem[name=Beispiel,sharenumber=definition,style=mainstyle_unshaded]{beispiel}
% \declaretheorem[name=Beispiele,sharenumber=definition,style=mainstyle_unshaded]{beispiele}

% -- Beweise
\declaretheoremstyle[headfont=\bfseries,bodyfont=\normalfont,headpunct=.,postheadspace=1em,spacebelow=12pt,spaceabove=2pt,qed=\qedsymbol]{beweise}
\declaretheoremstyle[headfont=\sffamily\bfseries,bodyfont=\normalfont,headpunct=:,postheadspace=1em,spacebelow=10pt,spaceabove=10pt]{bemerkungen}
\declaretheorem[name=proof,numbered=no,style=beweise]{beweis}

% %für schönere abhebungen
% \usepackage[framemethod=TikZ]{mdframed}
% \usepackage{amsthm}
% \usepackage{thmtools}
%
% \declaretheoremstyle[%
% 	headfont=\sffamily\bfseries,
% 	notefont=\normalfont\sffamily,
% 	bodyfont=\normalfont,
% 	headformat=\NUMBER\ \NAME\NOTE,
% 	headpunct={},
% 	postheadspace=1ex,
% 	spaceabove=15pt,spacebelow=30pt,]%
% {mainstyle}
%
% \declaretheoremstyle[%
% 	headfont=\sffamily\bfseries,
% 	notefont=\normalfont\sffamily,
% 	bodyfont=\normalfont,
% 	headformat=\NUMBER\ \NAME\NOTE,
% 	headpunct={},
% 	postheadspace=1ex,
% 	backgroundcolor = blue!10,
% 	align = center , % align the environment itself (left, center, rigth)
% 	nobreak = true, % prevent a frame from splitting
% 	hidealllines = true , %
% 	topline = true , bottomline = true , %
% 	spaceabove=15pt,spacebelow=30pt,]%
% {unterlegt}
%
% \declaretheoremstyle[%
% 	headfont=\bfseries\scshape,
% 	bodyfont=\normalfont,
% 	headpunct=:,
% 	postheadspace=1ex,
% 	spacebelow=12pt,spaceabove=2pt,
% 	qed=\qedsymbol]%
% {beweise}
%
% \declaretheorem[name=Definition,parent=section,style=unterlegt]{definition}
% \declaretheorem[name=Satz,sharenumber=definition,style=unterlegt]{satz}
% \declaretheorem[name=Korollar,sharenumber=definition,style=mainstyle]{korollar}
% \declaretheorem[name=Lemma,sharenumber=definition,style=mainstyle]{lemma}
% \declaretheorem[name=Proposition,sharenumber=definition,style=mainstyle]{proposition}
%
% \declaretheorem[name=Beweis,numbered=no,style=beweise]{beweis}
% \declaretheorem[name=Bemerkung,numbered=no,style=mainstyle]{bemerkung}
% \declaretheorem[name=Notation,numbered=no,style=mainstyle]{notation}
% \declaretheorem[name=Nebenbedingung,numbered=no,style=mainstyle]{nb}
% \declaretheorem[name=Beispiel,numbered=no,style=mainstyle]{beispiel}
% \declaretheorem[name=Beispiele,numbered=no,style=mainstyle]{beispiele}

\declaretheoremstyle[
    headfont=\bfseries, 
    notebraces={[}{]},
    bodyfont=\normalfont\itshape,
    headpunct={},
    postheadspace=\newline,
    spacebelow=\parsep,
    spaceabove=\parsep,
    mdframed={
        backgroundcolor=red!20, 
            linecolor=red!30, 
            innertopmargin=6pt,
            roundcorner=5pt, 
            innerbottommargin=6pt, 
            skipbelow=\parsep, 
            skipbelow=\parsep } 
]{myexamplestyle}

% example environment - thmtools
\declaretheorem[
    style=myexamplestyle,
    name=Example,
    numberwithin=section
]{example}

%%% Titelseite
\providecommand{\verfasser}{Tim Keil}

%--Konfiguration von scrheadings
\setheadsepline{1pt}[\color{light_gray}]
\pagestyle{scrheadings}
\clearscrheadfoot

\providecommand{\shortFach}{\fach}
\lehead{ \includegraphics[height=1.0 cm,keepaspectratio]{!config/Bilder/GU}}
\rehead{\rule{0cm}{0.6cm}\footnotesize \sffamily \color{light_gray} \verfasser{} -- Script \shortFach}
\lohead{\rule{0cm}{0.6cm} \footnotesize \sffamily \color{light_gray} Effective: \today \; \thistime[:]}
\rohead{\includegraphics[height=1.0 cm,keepaspectratio]{!config/Bilder/CH}}


\ofoot[{ \color{dark_gray} \LARGE \sffamily \thepage}]{{ \color{dark_gray} \LARGE \sffamily \thepage}} %hier wir auch der plain Stil bearbeitet!
\automark{section}
\ifoot{ \color{dark_gray} \small \leftmark}


%--Inhaltsverzeichnis
\usepackage[tocindentauto]{tocstyle}
\usetocstyle{KOMAlike}	
%\shorthandon{"}

%--Punkte (als letztes laden)
\usepackage{ellipsis}

%--Metadaten
\providecommand{\mail}{keil.menden@web.de}
\author{\verfasser}
\titlehead{\includegraphics[height=2.5cm, keepaspectratio]{!config/Bilder/GU}%
\hfill \includegraphics[height=2.3cm, keepaspectratio]{!config/Bilder/CH}}
\title{\fach}
\subtitle{Script of \enquote{\fach} by \prof}




\numberwithin{equation}{section}
\numberwithin{figure}{section}

\begin{document}

\maketitle
\cleardoubleoddemptypage

\pagenumbering{Alph}
\section*{foreword --- cooperation}
This document is a transcript of the lecture \enquote{\fach, \Semester}, by \prof.
It mainly contains the written content of the lecture. I will not assume any responsibility for the correctness of the content! For questions, remarks and mistakes please write an email to \href{mailto:keil.menden@web.de}{\nolinkurl{keil.menden@web.de}}. I'm grateful for every email. 
\newpage

\newpage

\tableofcontents
\cleardoubleoddemptypage
\pagenumbering{arabic}
\setcounter{page}{1}

%%% lecture 1
%%% lecture 1

\section{Introduction}
\subsection{Introduction example} 
\label{sub:introduction_example}
We have
\[
	\begin{cases}
		f''+f =g, &\text{ in }I = [0,1]\\
		f(0)=1, \,f'(0)=1
	\end{cases}
\]
where $g$ is a known continous function on $I$. We will now consider different cases:

\begin{enumerate}[1.]
	\item $g=0$
	\[
		\Rightarrow \,f(x) = A \cos(x) + B \sin(x), x \in I
	\]
	where $A,B \in \mathbb{R}$.
	\item $g$ arbitrary. We will now introduce the Method of variation of constants. Set
	\[
		f(x)=A(x) \cos(x)+ B(x) \sin(x)
	\]
	Differentiate
	\[
		f'(x) = A'(x) \cos(x) + B'(x) \sin(x) - A(x) \sin(x) + B(x) \cos(x)
	\]
	Aussume (This is part of the method)
	\[
		A'(x)\cos(x) + B'(x) \sin(x) = 0, \qquad x \in I
	\]
	Differentiate $f'(x)$ and get
	\[
		f''(x)=\underset{= -f(x)}{\underbrace{-A(x) \cos(x) - B(x) \sin(x)}} - A'(x) \sin(x) + B'(x) \cos(x)
	\]
	We get
	\[
		g(x) = f''(x)+f(x) = -A'(x) \sin(x) + B'(x) \cos(x).
	\]
	Now:
	\[
		\begin{cases}
			A'(x)\cos(x) + B'(x) \sin(x) = 0, & x \in I\\
			- A'(x) \sin(x)+ B'(x) \cos(x) = g(x), & x \in I \\
			A(0)=1, \qquad B(0)=0 &
		\end{cases}
	\]
	We get
	\begin{align*}
		A'(x) &= - g(x)\sin(x) \\
		A(0) &= 1 \\
		B'(x) &= g(x) \cos(x) \\
		B(0) &=0
	\end{align*}
	This implies
	\begin{align*}
		A(x) &= A(0) + \int_{0}^{x} A'(t) \,\mathrm{d}t = 1 - \int_{0}^{x} g(t) \sin(t) \,\mathrm{d}t \\
		B(x) &= B(0) + \int_{0}^{x}B'(t) \,\mathrm{d}t = 0 + \int_{0}^{x}g(t)\cos(t) \,\mathrm{d}t
	\end{align*}
	Hence
	\begin{align*}
		f(x) &= \cos(x) - \int_{0}^{x} g(t) \sin(t) \,\mathrm{d}t \cos(x) + \int_{0}^{x} g(t) \cos(t) \,\mathrm{d}t \sin(x) \\
		&= \cos(x) + \int_{0}^{x} (\underset{=\sin(x-t)}{\underbrace{\sin(x)\cos(t)- \sin(t)\cos(x)}})g(t) \,\mathrm{d}t \\
		&= \cos(x) + \int_{0}^{x}\sin(x-t)g(t) \,\mathrm{d}t \qquad (*)
	\end{align*}
	Check that $f(x)$ in $(*)$ satisfies the PDE.
	\minisec{special case:}
	Assume for $x \in I$
	\[
		g(x) = k(x)f(x)
	\]
	Here $k$ is a known continous function on $I$. Insert this in $(*)$. We obtain
	\[
		f(x) = \cos(x) + \int_{0}^{x} \sin(x-t)k(t)f(t) \,\mathrm{d}t, \qquad x \in I \qquad (**)
	\]
	Observe that $f$ appears both in LHS and RHS. $(**)$ is a reformulation of the PDE with $g=kf$. Pick a $\underset{\in C(I)}{\underbrace{\text{continous function in $I$}}}$. call it $f_0$. Set
	\begin{align*}
		f_1(x) &= \cos(x) + \int_{0}^{x}\sin(x-t)k(t)f_0(t) \,\mathrm{d}t \\
		f_2(x) &= \cos(x) + \int_{0}^{x}\sin(x-t)k(t)f_1(t) \,\mathrm{d}t \\
		\vdots &\qquad \qquad  \vdots \\
		f_{n+1}(x) &= \cos(x) + \int_{0}^{x}\sin(x-t)k(t)f_n(t) \,\mathrm{d}t, \qquad n=1,2,3, \dots \\
	\end{align*}
	\minisec{Hope:} $f_n$ tends to some continous function $f$ on $I$, denoted $f_n \to f$. 'Tends to' has to be more precis! 
	\begin{align*}
		f_{n+1}(x) &= \cos(x) + \int_{0}^{x} \sin(x-t)k(t)f_n(t) \,\mathrm{d}t \\
		\downarrow & \qquad \qquad \downarrow \\
		f(x) &= \cos(x) + \int_{0}^{x} \sin(x-t)k(t)f(t) \,\mathrm{d}t
	\end{align*}
	for $x \in I$. Simplify notation set for $v \in C(I)$
	\[
		\begin{cases}
			u(x)&=\cos(x)\\
			kv(x)&= \int_{0}^{x} \sin(x-t)k(t)v(t) \,\mathrm{d}t
		\end{cases}
	\]
	We have $f_0 \in C(I)$, $f_{n+1}=u + k f_n$ for $n=0,1,2, \dots$ (!) \\
	Facts from previous calculus classes:
	\begin{definition*}[Sequenze of continous functions]
		\[
			v_n \in C(I), \qquad n=1,2,\dots
		\]
		We say that $(v_n)_{n=1}^{\infty}$ converges uniformly in $I$ if
		\[
			\max_{x \in I} \abs{v_n(x)-v_m(x)} \to 0 , \qquad n,m \to \infty
		\]
		i.e.
		\[
			\forall\, \varepsilon >0 \exists\, N: \forall\, n,m \geq N: \, \max_{x \in I}\abs{v_n(x)-v_m(x)}< \varepsilon
		\]
	\end{definition*}
	\begin{lemma*}
		Suppose that $(v_n)_{n=1}^{\infty}$ converges uniformly on $I$. then there exists $v \in C(I)$ such that
		\[
			 \max_{x \in I}\abs{v_m(x)-v(x)} \to 0 \qquad \text{as }m \to \infty
		\]
	\end{lemma*}
	Back to (!): \\
	\minisec{More Notation:}
	\[
		k(kv) = k^2 v, \qquad v \in C(I)
	\]
	and
	\[
		k^{n+1}v = k(k^nv), \qquad n=1,2,\dots
	\]
	We have 
	\begin{align*}
		f_0 & \in C(I) \\ f_1 &=u+kf_0 \\ \text{ and }  
				f_2 &= u + kf_1 = u + k(u+kf_0)
	\end{align*}
	and so on. Note that
	\[
		k(v+w)=kv+kw
	\]
	Then 
	\begin{align*}
		f_2 &= u +k (u+kf_0) = k + ku + k(kf_0) = u + ku +k^2f_0 \\
		f_3 &= u + kf_2 = u + ku + k^2u + k^3f_0
	\end{align*}
	and in general for $n=1,2,\dots$
	\[
		f_n = ku + \dots + k^{n-1}u + k^n f_0, \qquad n=1,2,\dots
	\]
	Assume $n>m$ then
	\[
		f_n-f_m = k^mu + \dots + k^{n-1}u + k^nf_0 - k^mf_0
	\]
	Set for $v \in C(I)$
	\[
		\norm{v} = \max_{x \in I}\abs{v(x)}
	\]
	Note
	\[
		\norm{v+w} \leq \norm{v} + \norm{w} \qquad \text{for }v,w \in C(I)
	\]
	and
	\[
		\norm{-v}=\norm{v}.
	\]
	We have
	\begin{align*}
		\norm{f_n-f_m} &= \norm{k^mu + \dots + k^{n-1}u + k^nf_0 - k^m f_0} \\
		&\leq \norm{k^mu} + \dots + \norm{ k^{n-1}u} + \norm{k^nf_0} + \norm{- k^mf_0}.
	\end{align*}
	Assumption:
	\[
		\sum_{l=1}^{\infty} \norm{k^lv} < \infty \qquad \text{for all }v \in C(I) \qquad (***).
	\]
	Under this assumption
	\[
		\norm{f_n-f_m} \to 0 \qquad \text{as }n,m \to \infty
	\]
	since
	\begin{align*}
		\sum_{l=1}^{\infty}\norm{k^lu} &< \infty \qquad \qquad (u(x)=\cos(x)) \\
		\sum_{l=1}^{\infty}\norm{k^lf_0} &< \infty \qquad \qquad (f_0 \in C(I))
	\end{align*}
	conclusion: $(f_n)_{n=1}^{\infty}$ converges uniformly on $I$. By lemma above there exists $f \in C(I)$ such that
	\[
		\max_{x \in I}\abs{f_n(x)-f(x)} \to 0, \qquad n \to \infty
	\]
	i.e.
	\[
		\norm{f_n -f} \to 0, \qquad n \to \infty
	\]
	'Back hope':
	$f_n$ tends to $f$, denoted $f_n \to f$ shall be interpretated as
	\[
		\norm{f_n -f} \to 0, \qquad n \to \infty
	\]
	Remember
	\[
		f_{n+1}(x) = u(x) + k f_n(x) \to ?
	\]
	For $x \in I$ there is
	\begin{align*}
		\abs{k f_n(x)- kf(x)} &= \abs{ \int_{0}^{x} \sin(x-t)k(t)f_n(t) \,\mathrm{d}t- \int_{0}^{x}\sin(x-t)k(t)f(t) \,\mathrm{d}t} \\
		&\leq \int_{0}^{x}\abs{\sin(x-t)k(t)}\underset{\leq \norm{f_n-f}}{\underbrace{\abs{f_n(t)-f(t)}}} \,\mathrm{d}t \\
		&\leq \int_{0}^{x}\abs{\sin(x-t)k(t)} \,\mathrm{d}t \norm{f_n-f}
	\end{align*}
	In particular
	\begin{align*}
		\norm{k f_n- kf} &\leq \max_{x \in I}\int_{0}^{x} \underset{\leq 1}{\underbrace{\abs{\sin(x-t)}}} \underset{\max_{t \in I}\abs{k(t)}< \infty}{\underbrace{\abs{k(t)}}} \,\mathrm{d}t \norm{f_n -f} \\
		&\leq \norm{k} \norm{f_n-f}
	\end{align*}
	We have, provided $(***)$ holds, shown
	\begin{align*}
		f_{n+1} &= u + k f_n \\
		\downarrow & \\
		f &= u + kf
	\end{align*}
	Let us try to prove $(***)$. For $v \in C(I)$ arbitrary and for $x \in I$
	\begin{align*}
		\norm{kv(x)} &= \abs{\int_{0}^{x}\sin(x-t)k(t)v(t) \,\mathrm{d}t} \\
		&\leq \int_{0}^{x}\underset{\leq 1}{\underbrace{\abs{\sin(x-t)}}}\underset{\leq \norm{k}}{\underbrace{\abs{k(t)}}}\abs{v(t)} \,\mathrm{d}t \\
		&\leq \int_{0}^{x}\underset{\leq \norm{v}}{\underbrace{\abs{v(t)}}} \,\mathrm{d}t \norm{k} \\
		&\leq \norm{k} \norm{v}x
	\end{align*}
	In particular
	\[
		\norm{kv} \leq \norm{k}\norm{v}
	\]
	and
	\begin{align*}
		\abs{k^2v(x)} &\leq \int_{0}^{x} \abs{kv(t)} \,\mathrm{d}t \norm{k} \\
		&\leq \int_{0}^{x}\norm{k}\norm{v}t \,\mathrm{d}t \cdot \norm{k} \\
		&= \norm{k}^2 \norm{v} \frac{x^2}{2}
	\end{align*}
	In particular
	\[
		\norm{k^2v} \leq \norm{k}^2 \norm{v} \frac{1}{2}
	\]
	By induction we get
	\begin{align*}
		\abs{k^n v(x)} &\leq \norm{k}^n \norm{v} \frac{x^m}{m!} \qquad x \in I \\
		\norm{k^n v} &\leq  \norm{k}^n \norm{v} \frac{1}{n!}
	\end{align*}
	So 
	\begin{align*}
		\sum_{l=1}^{\infty}\norm{k^lv} &\leq \sum_{l=1}^{\infty}\norm{k}^l \norm{v} \frac{1}{l!} \\
		&= \norm{v} \sum_{l=1}^{\infty} \frac{\norm{k}^l}{l!} \\
		&\leq \norm{v} e^{\norm{k}} < \infty
	\end{align*}
	consider Taylor expansion.
	$\Rightarrow $ $(***)$ holds true. \\
	We have now shown that $f = u+kf$ where $u(x) = \cos(x)$ and
	\[
		kv = \int_{0}^{x}\sin(x-t)k(t)v(t) \,\mathrm{d}t
	\]
	$x \in I$ for $v \in C(I)$, has a solution $f \in C(I)$. \\
	\minisec{Question:} Is the solution unique? \\
	Assume $f,\tilde f \in C(I)$ such that $f = u + k f$ and $\tilde f = u+ k \tilde f$. Set 
	\[
		v = f- \tilde f \in C(I)
	\]
	\begin{align*}
		\Rightarrow v &= (u+kf) - (u+ k \tilde f) \\ &= kf - k \tilde f \\ &= k(f- \tilde f) \\ &= kv
	\end{align*}
	We have $v = kv$, implies that $kv = k(kv) = k^2v$. So for $n=1,2,\dots$
	\[
		v = kv = k^2v = \dots = k^nv.
	\]
	We know 
	\[
		\sum_{n=1}^{\infty}\norm{k^n \hat{v}} < \infty \qquad \text{for all }\hat{v} \in C(I).
	\]
	Apply this to $\hat{v}=v$:
	\[
		\sum_{n=1}^{\infty}\underset{=\norm{v}}{\underbrace{\norm{k^nv}}} < \infty .
	\]
	So $\norm{v}=0$ with implies $v(x)=0$ for all $x \in I$.
	So we have $f(x)=\tilde f(x)$ for $x \in I$. \\
	$\Rightarrow $ Answer to the question above: YES ! 
\end{enumerate}
We have more or less proved the following theorem:
\begin{theorem}
	Set $I=[0,1]$. Suppose $u \in C(I)$ and $k \in C(I \times I)$. Consider 
	\[
		f(x) = u(x)+ \int_{0}^{x} k(x,t) f(t) \,\mathrm{d}t, \qquad x \in I \qquad \qquad (1)
	\]
	Then $(1)$ has a unique solution $f \in C(I)$
\end{theorem}
With the same technology we can prove:
\begin{theorem}
	Set $I = [0,1]$. Suppose $u \in C(I)$, $k \in C(I \times I)$ and $\max\limits_{(x,t) \in I \times I} \abs{k(x,t)} <1$. Consider \[
		f(x) = u(x) + \int_{0}^{1}k(x,t)f(t) \,\mathrm{d}t, \qquad x \in I \qquad \qquad (2).
	\]
	Then $(2)$ has a unique solution $f \in C(I)$.
\end{theorem}
Different notions: see introductional example.
\begin{definition*}[vector space]
	$C(I)$ with the operations for $x \in I$
	\begin{description}
		\item[addition] $v,w \in C(I)$: $\qquad (v+w)(x) = v(x)+ w(x)$ 
		\item[mult. by scalar] $v \in C(I)$, $ \lambda \in \mathbb{R}$: $\qquad (\lambda v)(x) = \lambda v(x)$ 
	\end{description}
	Note that $v+w, \lambda v \in C(I)$.
\end{definition*}
\begin{definition*}[norm]
	norm on $C(I)$ for instance 
	\[
		\norm{v} = \max_{x \in I} \abs{v(x)}
	\]
	with norm given we can talk about convergence and continuity.
\end{definition*}
\begin{definition*}[Cauchy sequence]
	In our example a sequence $(f_n)_{n=1}^{\infty}$ is called Cauchy sequence if $\norm{f_n-f_m} \to 0$ for $n,m \to \infty$.
\end{definition*}
\begin{definition*}
	$C(I)$ with the max-norm. Lemma above says that every Cauchy sequence converges i.e.
	\[
		\norm{v_n-v_m} \to 0, \qquad n,m \to \infty
	\]
	This applies
	\[
		\exists\, v \in C(I): \norm{v_n-v} \to 0, \qquad n \to \infty
	\]
	This is the defining property of a Banach space. \\
	$K$ linear mapping $C(I) \to C(I)$ with
	\begin{align*}
		K(v+w) &= K(v) + K(w) \\
		K(\lambda v) &= \lambda K(v)
	\end{align*}
	for $v,w \in  C(I)$, $\lambda \in \mathbb{R}$. \\
	$K$ bounded linear:
	\[
		\norm{Kv} \leq M \norm{v} \qquad \forall\, v \in C(I)
	\]
	where $M >0$ independent of $v$.	
\end{definition*}
\begin{definition*}[operator norm]
	Define
	\[
		\norm{K}:= \inf \set[M>0]{\norm{Kv} \leq M \norm{v} \text{ for all }v \in C(I)}.
	\]
\end{definition*}
\minisec{fixed point results:}
Our example: $f=u+kf =: T(f)$ and $f_0 \in C(I)$ fixed. \\
Form sequence of iterants $(f_n)_{n=1}^{\infty}$, $f_n = T(f_{n-1})$, $n=1,2,\dots$ if
\[
	\norm{T(v)-T(w)} \leq c \norm{v-w}
\]
for all $v,w \in C(I)$ for some $c<1$. Then there is a unique $v \in C(I)$ such that $v = T(v)$. \\
This is \underline{Banach's fixed point theorem}.
\begin{definition*}[Green's function]
	Our example: 
	\[
		L = \left( \diffd{}{x} \right)^2 + 1 
	\]
	differential operator. Boundary conditions 
	\[
		f(0) = f'(0) = 0.
	\]
	Then 
	\[
		f(x) = \int_{0}^{1} g(x,t)h(t) \,\mathrm{d}t 
	\]
	is a solution to
	\[
		\begin{cases}
			f''+f &= h, \\
			f(0) = f'(0)& = 0	
		\end{cases}
	\]
\end{definition*}
\begin{definition*}[real vector space]
	We say that $E$ is a real vector space  if it is a non-empty set with the operations 
	\begin{description}
		\item[addition] $E \times E \to E$, $\qquad (x,y) \mapsto x+y$
		\item[mult. with scalar] $\mathbb{R} \times E \to E$, $ \qquad (\lambda,x) \mapsto \lambda x$ 
	\end{description}
	satisfying the axioms:
	\begin{enumerate}[(1)]
		\item $x+y = y+x, \qquad$ for all $x,y \in E$
		\item $x+(y+z)= (x+y)+z, \qquad $ for all $x,y,z \in E$
		\item For all $x,y \in E$ there exists $z \in E$ such that $x+z = y$
		\item $\alpha (\beta x) = (\alpha \cdot \beta)x, \qquad $ for all $\alpha,\beta \in \mathbb{R}, x \in E$
		\item $\alpha(x+y) = \alpha x+ \alpha y, \qquad $ for all $\alpha \in \mathbb{R}, x,y \in E$
		\item $(\alpha + \beta) x = \alpha x + \beta x, \qquad $ for all $\alpha, \beta \in \mathbb{R}, x \in E$
		\item $1 \cdot x = x, \qquad $ for all $x \in E$.  
	\end{enumerate}
\end{definition*}
\begin{bemerkung}
	$E$ is a complex vector space if all $\mathbb{R}$ in the definition above are replaced by $\mathbb{C}$.
\end{bemerkung}
\begin{bemerkung}
	\begin{enumerate}[(1)]
		\item \[
			\exists\,! 0 \in E: \qquad x + 0 = x \qquad \text{for all }x \in E.
		\]
		since: Fix $x \in E$, by $(3)$, $\exists\, 0_x$ such that $0_x + x =x$. \\
		Fix $y \in E$. We want to show that $y + 0_y = y$. By $(3)$, there exists $z \in E$ such that $x+z = y$. So
		\begin{align*}
			y + 0_x & \stackrel{\hphantom{(1)}}{=} (x+z)+ 0_x \\
			&\stackrel{(1)}{=} (z+x)+ 0_x \\
			&\stackrel{(2)}{=} z + (x + 0_x) \\
			&\stackrel{\hphantom{(1)}}{=} z+x \\
			&\stackrel{(1)}{=}x+z \\
			&\stackrel{\hphantom{(1)}}{=} y.
		\end{align*}
		Assume $x+ 0_1 = x$, $x+ 0_2 =x$ for all $x \in E$. We want to show $0_1 = 0_2$:
		\[
			0_1 = 0_1 + 0_2 = 0_2 + 0_1 = 0_2
		\]
		\item 
		\[
			\forall\, x \in E: \, \exists\,! \,-x \in E: \,x+(-x)=0
		\]
		proof: exercise.
		\item \begin{align*}
			0x &=0 \qquad \text{for all }x \in E \\
			(-1)x &= -x \qquad \text{for all }x \in E
		\end{align*}
	\end{enumerate}
\end{bemerkung}
\begin{beispiele}[Examples of real vector spaces]
	\begin{enumerate}[1)]
		\item $\mathbb{R}$ with standard addition and mult. by scalar.
		\item $\mathbb{R}^n$, $n=2,3, \dots$
		\begin{description}
			\item[addition] $(x_1,x_2,\dots) + (y_1,y_2, \dots) = (x_1+y_1,x_2+y_2, \dots)$ 
			\item[mult.] $ \lambda (x_1,x_2,\dots) = (\lambda x_1, \lambda x_2, \dots)$
		\end{description} 
		\item $\mathbb{R}^{\infty} = \set[(x_1,\dots,x_n,\dots)]{x_n \in \mathbb{R}, n=1,2,\dots}$
		\item $1 \leq p < \infty$, 
		\[
			l^p = \set[(x_1,\dots,x_n, \dots) \in \mathbb{R}^{\infty}]{\left( \sum_{n=1}^{\infty} \abs{x_n}^p \right)^{\frac{1}{p}} < \infty}
		\]
		with the same addition and mult. by scalar as in $\mathbb{R}^{\infty}$. We have to check:
		\begin{enumerate}[(1)]
			\item $x,y \in l^p \qquad \Rightarrow \qquad x+y \in l^p$
			\item $x \in l^p, \lambda \in \mathbb{R} \qquad \Rightarrow \qquad \lambda x \in l^p$ 
		\end{enumerate}
		For $(1)$ we assume $x = (x_1, \dots, x_n, \dots)$ and $y = (y_1, \dots, y_n, \dots)$.
		\begin{align*}
			x \in l^p \qquad &\Rightarrow \qquad \sum_{n=1}^{\infty}\abs{x_n}^p < \infty \\
			y \in l^p \qquad &\Rightarrow \qquad \sum_{n=1}^{\infty}\abs{y_n}^p < \infty
		\end{align*}
		\[
			\Rightarrow \qquad  x+y = (x_1+y_1, \dots) \stackrel{?}{\in } l^p?
		\]
		\begin{align*}
			\Rightarrow \sum_{n=1}^{\infty}\abs{x_n+y_n}^p & \leq \set{\abs{x_n+y_n} \leq \abs{x_n}+ \abs{y_n} \leq 2 \max \set{\abs{x_n},\abs{y_n}}} \\
			& \,\set{\abs{x_n+y_n}^p \leq 2^p \left( \abs{x_n}^p + \abs{y_n}^p \right)} \\
			&\leq \sum_{n=1}^{\infty}2^p (\abs{x_n}^p + \abs{y_n}^p) \\
			&= 2^p \underset{< \infty}{\underbrace{\sum_{}^{}\abs{x_n}^p}}+ 2^p \underset{< \infty}{\underbrace{\sum_{}^{}\abs{y_n}^p}} < \infty
		\end{align*}
		and \[
			\sum_{n=1}^{\infty} \abs{\lambda x_n}^p = \sum_{n=1}^{\infty} \abs{\lambda}^p \cdot \abs{x_n}^p = \abs{\lambda}^p \sum_{n=1}^{\infty}\abs{x_n}^p < \infty
		\]
		\item function spaces, say real-valued functions on $I$.
		\begin{description}
			\item[addition:] $(f+g)(x) = f(x)+ g(x), \qquad x \in I$
			\item[mult. by scalar:] $(\lambda f)(x)= \lambda f(x) \qquad $ for functions $f$ and $g$ 
		\end{description}
		\item $C(I):$ addition and mult. by scalar as in $(5)$. \\ $f,g$ continuous in $I$ implies that $f+g$ is continuous in $I$. \\
		Also if $f$ is continuous and $\lambda \in \mathbb{R}$ then $(\lambda f)$ is continuous in $I$.
		\item $P(I)= \,$ polynomials in $I$.
		\item $P_k(I)= \,$ polynomials of degree at most $k$ in $I$.
	\end{enumerate}
\end{beispiele}


%%% lecture 2
%01.09.2016
\begin{theorem}[Hölder's inequality]
	Assume $1<p<\infty$ and $\frac{1}{p}+ \frac{1}{q}=1$. \\ Let $(x_1, \dots, x_n, \dots)$ and $(y_1,y_2, \dots, y_n, \dots)$ be sequences of complex numbers. Then
	\[
		\sum_{n=1}^{\infty}\abs{x_ny_n} 
		\leq \left( \sum_{n=1}^{\infty}\abs{x_n}^p \right)^{\frac{1}{p}} \cdot \left( \sum_{n=1}^{\infty}\abs{y_n}^q \right)^{\frac{1}{q}}
	\]
	Remark there the LHS can be infinity, but the RHS can also be infinity.
\end{theorem}
\begin{beweis}
	\begin{description}
		\item[Step 1] We're going to proof 
		\[
			ab \leq \frac{a^p}{p}+ \frac{b^q}{q}, \qquad \text{for all }a,b >0.
		\] 
		\[
			\int_{0}^{a} x^{p-1} \,\mathrm{d}x = \frac{a^p}{p}
		\]
		Note $y = x^{p-1}$ gives \[
			x  = y ^{\frac{1}{p-1}} = y^{\frac{1}{\frac{1}{1-\frac{1}{q}}-1}}= y ^{\frac{1}{\frac{q}{q-1}-1}} = y^{q-1}
		\] 
		so
		\[
			\int_{0}^{b}y^{q-1} \,\mathrm{d}y = \frac{b^q}{q}
		\]
		We get
		\[
			ab \leq \frac{a^p}{p}+ \frac{b^q}{q}
		\]
		(You also get condition for $=$)
		\item[Step 2] It is enough to consider the cases LHS $>0$ and RHS $< \infty$. There exists an integer $N$ such that
		\[
			0 < \sum_{n=1}^{N}\abs{x_n}^p, \, \sum_{n=1}^{N}\abs{y_n}^q < \infty.
		\]
		Set 
		\begin{align*}
			a &= \frac{\abs{x_k}}{\left( \sum_{n=1}^{N}\abs{x_n}^p \right)^{\frac{1}{p}}}, \qquad k = 1,2, \dots,N, \\
			b &= \frac{\abs{y_k}}{\left( \sum_{n=1}^{N}\abs{y_n}^q \right)^{\frac{1}{q}}}, \qquad k = 1,2, \dots,N.
		\end{align*}
		Insert into
		\[
			ab \leq \frac{a^p}{p}+ \frac{b^q}{q}.
		\]
		\[
			\frac{\abs{x_ky_k}}{\left( \sum_{n=1}^{N}\abs{x_n}^p \right)^{\frac{1}{p}}\left( \sum_{n=1}^{N}\abs{y_n}^q \right)^{\frac{1}{q}}} 
			\leq \frac{\abs{x_k}^p}{p \sum_{n=1}^{N}\abs{x_n}^p} + \frac{\abs{y_k}^q}{q \sum_{n=1}^{N}\abs{y_n}^q}, \qquad k = 1,2, \dots, N.
		\]
		We sum over $k$ from $1$ to $N$.
		\[
			\sum_{k=1}^{N}\abs{x_ky_k} \leq  \left( \sum_{n=1}^{N}\abs{x_n}^p \right)^{\frac{1}{p}} \cdot \left( \sum_{n=1}^{N}\abs{y_n}^q \right)^{\frac{1}{q}}
		\]
		Let $N \to \infty$. First in RHS and then in LHS. 
	\end{description}
\end{beweis}
\begin{theorem}[Minkowski's inequality]
	Assume $1 \leq p < \infty$. and $X,Y \in l^p$. Then
	\[
		\norm{X+Y}_{l^p} \leq \norm{X}_{l^p} + \norm{Y}_{l^p}.
	\]
\end{theorem}
\begin{beweis}
	\begin{description}
		\item[$p=1$:] 
		\begin{align*}
			\norm{X+Y}_{l^1} &= \norm{(x_1,x_2, \dots,x_n, \dots)+ (y_1,y_2, \dots,y_n, \dots)}_{l^1} \\
			&= \norm{(x_1+y_1, \dots,x_n + y_n, \dots)}_{l^1} \\
			&= \sum_{n=1}^{\infty} \abs{x_n+y_n} \\
			&\leq \sum_{n=1}^{\infty} (\abs{x_n}+\abs{y_n}) \\
			&= \sum_{n=1}^{\infty}\abs{x_n}+ \sum_{n=1}^{\infty}\abs{y_n} \\
			&= \norm{X}_{l^1}+ \norm{Y}_{l^1}
		\end{align*} 
		\item[$1 < p < \infty$:] 
		\begin{align*}
					\norm{X+Y}_{l^p}^p &= \sum_{n=1}^{\infty}\abs{x_n+y_n}^p \\
					&= \sum_{n=1}^{\infty}\abs{x_n+y_n}\abs{x_n+y_n}^{p-1} \\
					&\leq \sum_{n=1}^{\infty}\abs{x_n}\abs{x_n+y_n}^{p-1} + \sum_{n=1}^{\infty}\abs{y_n}\abs{x_n+y_n}^{p-1}.
		\end{align*}
		Use Hölder to get
		\begin{align*}
			\sum_{n=1}^{\infty}\abs{x_n}\abs{x_n+y_n}^{p-1} &\leq
			 \underset{=\norm{X}_{l^p}}{\underbrace{\left( \sum_{n=1}^{\infty}\abs{x_n}^p \right)^{\frac{1}{p}}}} \cdot \left( \sum_{n=1}^{\infty}\abs{x_n+y_n}^{(p-1)q} \right)^{\frac{1}{q}} \\
			 &= \set{(p-1)q = (p-1)\frac{1}{1-\frac{1}{p}}=p} \\
			 &= \norm{X}_{l^p}  \left( \sum_{n=1}^{\infty}\abs{x_n+y_n}^p \right)^{\frac{1}{q}}.
		\end{align*}
		We have
		\[
			\norm{X+Y}_{l^p}^p \leq \left( \norm{X}_{l^p} + \norm{Y}_{l^p} \right) \norm{X+Y}_{l^p}^{\frac{p}{q}}.
		\]
		If $\norm{X+Y}_{l^p} \neq 0$ then
		\[
			\norm{X+Y}_{l^p}^{p-\frac{p}{q}} \leq \norm{X}_{l^p} + \norm{Y}_{l^p}
		\]
		there
		\[
			p- \frac{p}{q} = p (1- \frac{1}{q}) = p \frac{1}{p} = 1.
		\]
	\end{description}
\end{beweis}
\begin{bemerkung}
	$f \in C([0,1])$ then for $1 \leq p < \infty$
	\[
		\norm{f}_{L^p} = \left( \int_{0}^{1} \abs{f(t)}^p \,\mathrm{d}t \right)^{\frac{1}{p}}.
	\]
	\textbf{Claim:} \text{    }     
	\begin{align*}
		\norm{fq}_{L^1} = \int_{0}^{1} \abs{f(t)\cdot g(t)} \,\mathrm{d}t \leq \norm{f}_{L^p} \cdot \norm{g}_{L^q}
	\end{align*}
	where $\frac{1}{p}+ \frac{1}{q}= 1$. Also we have
	\[
		\norm{f+q}_{L^p} \leq \norm{f}_{L^p}+ \norm{g}_{L^p}
	\]
	This is proven with the same technique as we used for $l^p$. $\sum_{n=1}^{\infty}$ is replaced by $\int_{0}^{1} \,\mathrm{d}t$. \\
	$E$ real/complex vector space. $x_1, \dots,x_n \in E$, $\lambda_1, \dots, \lambda_n$ scalar. We say that 
	\[
		\lambda_1 x_1, \dots, \lambda_n x_n
	\]
	is a linear combination of $x_1,\dots,x_n$. We say that $x_1,\dots,x_n$ are linear independent if 
	\[
		\alpha_1 x_1 + \dots + \alpha_n x_n = 0 \qquad \Rightarrow \qquad \alpha_1 = \dots = \alpha_n = 0.
	\]
	If $A \subset E$, we say that $A$ is linear independant if every linear combination of vectors in $A$ is linear independent.
\end{bemerkung}
	\begin{beispiele}
		\begin{enumerate}[(1)]
			\item 
		Set $E = P([0,1])$ and $A = \set[p_k]{p_k(x) = x^k, x \in [0,1], k= 0,1, \dots}$. A is linear independant since: \\ consider
		\[
			\alpha_0 p_0 + \alpha_1 p_1 + \dots + \alpha_np_n = 0
		\]
		i.e. 
		\[
			\alpha_0 p_0(x) + \alpha_1 p_1(x) + \dots + \alpha_n p_n(x) = 0(x), \qquad x \in [0,1] 
		\]
		i.e.
		\[
			\alpha_0 + \alpha_1 x + \dots + \alpha_n x^n = 0, \qquad x \in [0,1]
		\]
		If $x = 0$ then $\alpha_0 = 0$
		\[
			\alpha_1 x + \dots + \alpha_n x^n = 0, \qquad x \in [0,1].
		\]
		Differentiate
		\[
			\alpha_1 + 2 \alpha_2 x + \dots + n \alpha_n x^{n-1} = 0
		\]
		gives $\alpha_1 = 0$. Continue and get
		\[
			\alpha_0 = \alpha_1 = \dots = \alpha_n = 0.
		\]
		Set $B \subset E$ where
		\begin{align*}
			\text{span } B &= \set{\text{set of all linear combinations of elements in B}} \\
			&= \set[\sum_{k=1}^{n}lambda_k x_k]{x_k \in B, \lambda_k \in \mathbb{R}, k=1,2,\dots,n \text{ where n is a positive integer}}
		\end{align*}
		\begin{bemerkung}
			\[
				\sum_{k=1}^{n}\lambda_k x_k \in E
			\]
			\[
				\sum_{k=1}^{\infty} \lambda_k x_k \text{    has no meaning}
			\]
		\end{bemerkung}
		$C \subset E$ is called a basis for E if
		\begin{enumerate}[1)]
			\item $C$ linear independent.
			\item $ \text{span } C = E$
		\end{enumerate}
		continue of the example above: \\
		\textbf{Claim:} \text{    }     $A$ is a basis for E.
		\item Set $E = l^2$ and
		\[
			A = \set[X_k]{k =1,2,\dots}
		\]
		\[
			X_k = (0,0,\dots,0,1,0,0,\dots)
		\]
		\textbf{Claim:} \text{    }     A is linear independent since
		\[
			\alpha_1 X_1 + \alpha_2 X_2 + \dots + \alpha_n X_n = 0
		\]
		Here 
		\[
			\alpha_1 X_1 = (\alpha_1,0,0,\dots), \qquad etc
		\]
		and
		\[
			0 = (0,0, \dots)
		\]
		So
		\[
			(\alpha_1,\alpha_2, \dots, \alpha_n,0, \dots) = (0,0,\dots)
		\]
		So $\alpha_1= \alpha_2 = \dots = \alpha_n = 0$. \\
		Question: Is $A$ a basis for $l^2$? \\
		We note: If $X \in \text{span }A$ then
		\[
			X = (x_1,x_2, \dots,x_n,0,0,\dots)
		\]
		for some positive integer $n$, i.e. $X$ has only finitely many nonzero positions. \\
		Cosider:
		\[
			X := (1, \frac{1}{2}, \dots, \frac{1}{n}, \dots)
		\]
		\[
			\norm{X}_{l^2} = \left( \sum_{n=1}^{\infty} \frac{1}{n^2} \right)^{\frac{1}{2}} < \infty
		\]
		So $X \in l^2 \setminus \text{span }A$.
		\end{enumerate}
		\begin{bemerkung}
			Every vector space has a basis (if we are allowed to use Axiom of Choice/ zorns lemma). \\ Basis = vector space basis = Hamel basis
		\end{bemerkung}
		Assume $x_1, \dots,x_n$ is a basis for $E$. Then every basis for $E$ must contain $n$ different elements. 
		\[
			n = \dim E
		\]
		is well-defined. (System of linear equations, homogeneous with more unknowns than equations. Then there exists a nontrivial solution.)
	\end{beispiele}
\begin{definition*}[norm]
	$E$ vector space. We say that $\norm{.}: E \to [0,\infty)$ is a norm on $E$ if
	\begin{enumerate}[1)]
		\item $\norm{x}=0 \qquad \Rightarrow x =0$
		\item $\norm{\lambda x} = \abs{\lambda} \norm{x} \qquad $ for all $x \in E, \lambda \in \mathbb{R}$
		\item $\norm{x+y} \leq \norm{x} + \norm{y} \qquad $ for all $x,y \in E$
	\end{enumerate}
	
\end{definition*}
\begin{bemerkung}
	\[
		\norm{0} = \norm{0 \cdot 0} = \underset{=0}{\underbrace{\abs{0}}} \norm{0} = 0
	\]
\end{bemerkung}
\begin{beispiele}
	\begin{enumerate}[(1)]
		\item $1 < p < \infty$ and 
	\[
		\norm{X}_{l^p} = \left( \sum_{n=1}^{\infty} \abs{x_n}^p \right)^{\frac{1}{p}}
	\]
	is a norm on $l^p$. Check $1)$,$2)$ and $3)$ above:
	\begin{enumerate}[1)]
		\item \phantom{1} \[
			0 = \norm{X}_{l^p} = \left( \sum_{n=1}^{\infty} \abs{x_n}^p \right)^{\frac{1}{p}} 
		\]
		It follows
		\[
			x_n=0, \qquad n=1,2,\dots
		\]
		\[
			\Rightarrow \qquad X = (x_1,x_2, \dots) = (0,0,\dots) = 0
		\]
		\item \phantom{1}\[
			\norm{\lambda X}_{l^p} = \left( \sum_{n=1}^{\infty} \abs{\lambda x_n}^p \right)^{\frac{1}{p}} 
			= \left( \abs{\lambda}^p \sum_{n=1}^{\infty} \abs{x_n}^p \right)^{\frac{1}{p}} = \abs{\lambda} \norm{X}_{l^p}
		\]
		\item \phantom{1}\[
			\norm{X+Y}_{l^p} \leq \set{\text{Minkowski's inequality}} \leq \norm{X}_{l^p} + \norm{Y}_{l^p}
		\]
	\end{enumerate}
	\item $E = C([0,1])$ and $f \in E$
	\[
		\norm{f} = \max\limits_{t \in [0,1]} \abs{f(t)} \in [0,\infty)
	\]
	Check the axioms above
	\begin{enumerate}[1)]
		\item If $\norm{f} = 0$ it follows
		\[
			\abs{f(t)} = 0 \,\text{ for all }t \in [0,1], \qquad \Rightarrow \qquad f=0
		\]
		\item \[
			\norm{\lambda f} = \max\limits_{t \in [0,1]} \underset{\abs{\lambda}\abs{f(t)}}{\underbrace{\abs{\underset{\lambda f(t)}{\underbrace{(\lambda f)(t)}}}}}
			= \abs{\lambda} \max\limits_{t \in [0,1]} \abs{f(t)} = \abs{\lambda} \norm{f}
		\]
		\item 
		\[
			\norm{f+g} = \max\limits_{t \in [0,1]} \abs{\underset{f(t)+g(t)}{\underbrace{(f+g)(t)}}} = \max\limits_{t \in [0,1]}  \left( \abs{f(t)} + \abs{g(t)} \right)
			\leq \max\limits_{t \in [0,1]} \abs{f(t)} + \max\limits_{t \in [0,1]} \abs{g(t)} = \norm{f} + \norm{g}
		\]
	\end{enumerate}
	\item $E = C([0,1])$ and $f \in E$.
	\[
		\norm{f}_{L^1} = \int_{0}^{1} \abs{f(t)} \,\mathrm{d}t 
	\]
	defines also a norm on $E$.
	\begin{description}
		\item[3)]
		\begin{align*}
			\norm{f+g}_{L^1} &= \int_{0}^{1} \abs{\underset{f(t)+g(t)}{\underbrace{(f+g)(t)}}} \,\mathrm{d}t \\
			&\leq \int_{0}^{1}(\abs{f(t)}+ \abs{g(t)}) \,\mathrm{d}t \\
			&= \int_{0}^{1}\abs{f(t)} \,\mathrm{d}t + \int_{0}^{1}\abs{g(t)} \,\mathrm{d}t \\
			&= \norm{f}_{L^1} + \norm{g}_{L^1}
		\end{align*}
		\item[2)] \[
			\norm{\lambda f} = \int_{0}^{1} \underset{= \abs{\lambda}\abs{f(t)}}{\underbrace{\abs{(\lambda f)(t)}}} \,\mathrm{d}t = \abs{\lambda} \norm{f}_{L^1}
		\]
		\item[1)] \[
			0 = \norm{f}_{L^1} = \int_{0}^{1}\abs{f(t)} \,\mathrm{d}t
		\]
		This implies $f(t)=0$ for $t \in [0,1]$ since f is continuous! i.e. $f=0$
	\end{description}
	\end{enumerate}
\end{beispiele}
\begin{theorem}[equivalent norm]
	$E$ vector space with norms $\norm{.}$ and $\norm{.}_{*}$. We say that $\norm{.}$ and $\norm{.}_{*}$ are equivalent if there exists $\alpha, \beta >0$ such that
	\[
		\alpha \norm{x}_{*} \leq \norm{x} \leq \beta \norm{x}_{*} \qquad \text{for all }x \in E.
	\]
\end{theorem}
\begin{beispiel}
		\item $E = C([0,1])$. Choose $y = f(t)$ and $y = \abs{f(t)}$
		\[
			\norm{f} = \max\limits_{t \in [0,1]} \abs{f(t)}, \qquad \norm{f}_{*} = \norm{f}_{L^1} = \text{area}.
		\]
		Question: Are these norms equivalent? \\
		\textbf{Claim:} \text{    }   $f \in C([0,1])$ 
		\[
			\norm{f}_{*} = \int_{0}^{1} \underset{\leq \norm{f}}{\underbrace{\abs{f(t)}}} \,\mathrm{d}t \leq \norm{f}
		\]
		Choose $f_n(t)$ such that
		\[
			\norm{f_n} = 1, \qquad \norm{f_n}_{*} = \frac{1}{2n}
		\]
		So 
		\[
			\frac{\norm{f_n}_{*}}{\norm{f_n}} = \frac{1}{2n} \to 0 \qquad n \to \infty
		\]
		The norms are not equivalent! Answer: NO ! 
	\end{beispiel}
\begin{theorem}
	$E$ vector space with $\dim E < \infty$.  \\
	$\Rightarrow $ All norms on $E$ are equivalent.
\end{theorem}
\begin{beweis}
	Assume $n = \dim E$ with a positive integer $n$. Let $x_1,x_2, \dots , x_n$ be a basis for $E$. For every $x \in E$
	\[
		x = \alpha_1(x)x_1 + \dots + \alpha_n(x)x_n
	\]
	where $\alpha_1(x), \dots, \alpha_n(x)$ unique. Set 
	\[
		\norm{x}_{*} = \abs{\alpha_1(x)}+ \dots + \abs{\alpha_n(x)}, \qquad x \in E
	\]
	\textbf{Claim:} \text{    }     $\norm{.}_{*}$ defines a norm on $E$ (easy proof) \\
	Fix an arbitrary norm $\norm{.}$ on $E$. \\
	\textbf{Claim:} \text{    }     $\norm{.}_{*}$ and $\norm{.}$ are equivalent. \\
	Note for $x \in E$
	\begin{align*}
		\norm{x} &= \norm{\alpha_1(x)x_1 + \dots + \alpha_n(x)x_n}  \\
		&\leq \abs{\alpha_1(x)}\norm{x_1} + \dots + \abs{\alpha_n(x)} \norm{x_n} \\
		&\leq \max\limits_{k=1,2,\dots,n} \norm{x_k} ( \underset{= \norm{x}_{*}}{\underbrace{\abs{\alpha_1(x)}+ \dots + \abs{\alpha_n(x)}}}) 
	\end{align*}
	Set $\beta = \max\limits_{k=1,2,\dots,n} \norm{x_k}$.
	Then
	\[
		\norm{x} \leq \beta \norm{x}_{*} \qquad \text{for all }x \in E.
	\]
	Remains to prove: There exists $\alpha >0$ such that
	\[
		\alpha \norm{x}_{*} \leq \norm{x} \qquad \text{for all } x \in E \qquad (*)
	\]
	Let $E$ be a vector space with norm $\norm{.}$ and $(v_m)_{m=1}^{\infty}$ a sequence in $E$. We say that $(v_m)_{m=1}^{\infty}$ converges in $(E,\norm{.})$ if there exists $v \in E$ such that $\norm{v_m-v} \to 0$ for $n \to \infty$. \\
	Notation: $v_m \to v$ in $(E, \norm{.})$. \\
	Note: If we have $\norm{.}$ and $\norm{.}_{*}$ are equivalent, then
	\[
		v_n \to v \text{ in }(E,\norm{.}) \qquad  \Leftrightarrow \qquad v_n \to v \text{ in }(E,\norm{.}_{*})
	\] 
	Back to $(*)$: Argue by contradiction. \\
	Assume there is no $\alpha >0$ such that
	\[
		\alpha \norm{x}_{*} \leq \norm{x} \qquad \text{for all } x \in E
	\]
	For $k=1,2,3,\dots$ there are $y_k \in E$ such that
	\[
		\frac{1}{k}\norm{y_k}_{*} > \norm{y_k}. \qquad (**)
	\]
	We have 
	\[
		y_k = \alpha_1^{(k)} x_1 + \dots + \alpha_n^{(k)} x_n
	\]
	where $\alpha_1^{(k)}, \dots, \alpha_n^{(k)}$ are unique scalars and $k = 1,2, \dots$. \\
	$(**)$ implies that
	\[
		k \norm{y_k} < \abs{\alpha_1^{(k)}} + \dots + \abs{\alpha_n^{(k)}}
	\]
	WLOG we can assume $\abs{\alpha_1^{(k)}}+ \dots + \abs{\alpha_n^{(k)}} = 1$. ( If not consider 
	\begin{align*}
		\lambda z &= \lambda ( \alpha_1(z)x_1+ \dots+ \alpha_n(z)x_n)  \\
		&= (\lambda \alpha_1(z))x_1 + \dots + (\lambda \alpha_n (z))x_n \\
		&= \alpha_1(\lambda z)x_1 + \dots + \alpha_n(\lambda z)x_n
	\end{align*}
	We have \[
		\alpha_k(\lambda z) = \lambda \alpha_k(z), \qquad k=1,2,\dots,n)
	\]
	We have 
	\[
		k \norm{y_k} < 1 \qquad k=1,2,\dots
	\]
	which implies $y_k \to 0$ in $(E,\norm{.})$. \\
	\begin{Large}
		\underline{IF:}
	\end{Large} \begin{align*}
		\alpha_1^{(k)} &\to \bar{\alpha_1} \\
		\alpha_2^{(k)} &\to \bar{\alpha_2} \\
		&\vdots \\
		\alpha_n^{(k)} &\to \bar{\alpha_n}
	\end{align*}
	for $k \to \infty$. Then set
	\[
		\bar{y} = \bar{\alpha_1}x_1 + \dots + \bar{\alpha_n}x_n
	\]
	and get
	\begin{align*}
		\norm{y_k-\bar{y}} &= \norm{(\alpha_1^{(k)}-\bar{\alpha_1})x_1 + \dots+ (\alpha_n^{(k)}-\bar{\alpha_n})x_n} \\
		&\leq \underset{\to 0}{\underbrace{\abs{\alpha_1^{(k)}-\bar{\alpha_1}}}}\underset{< \infty}{\underbrace{\norm{x_1}}} 
		+ \dots + \underset{\to 0}{\underbrace{\abs{\alpha_n^{(k)}-\bar{\alpha_n}}}}\underset{< \infty}{\underbrace{\norm{x_n}}} \to 0, \qquad k \to \infty
	\end{align*}
	\[
		\norm{\bar{y}} = \norm{\bar{y}-y_k + y_k} \leq \underset{\to 0}{\underbrace{\bar{y}-y_k}} + \underset{\to 0}{\underbrace{\norm{y_k}}} \to 0, \qquad k \to \infty
	\]
	So $\norm{\bar{y}} = 0$ hence $\bar{y}=0$. But
	\[
		\abs{\bar{\alpha_1}} + \abs{\bar{\alpha_2}} + \dots + \abs{\bar{\alpha_n}} = 1.
	\]
	This contradicts $x_1, \dots,x_n$ is a basis. \\
	We have for $k= 1,2, \dots$ the vector $(\alpha_1^{(k)},\alpha_2^{(k)},\dots,\alpha_n^{(k)})$ where
	\[
		\abs{\alpha_1^{(k)}}+ \dots + \abs{\alpha_n^{(k)}} = 1
	\]
	We focus on the first one and we have
	\[
		\abs{\alpha_1^{(k)}} \leq 1, \qquad k =1,2,\dots
	\]
	By Bolzano-Weierstraß then there exists a converging subsequence $(\alpha_{1,1}^{(k)})_{k=1}^{\infty}$ of $(\alpha_{1}^{(k)})_{k=1}^{\infty}$. Set
	\[
		\bar{\alpha_1} = \lim_{k \to \infty} \alpha_{1,1}^{(k)}
	\]
	consider
	\[
		(\alpha_{1,1}^{(k)},\alpha_{2,1}^{(k)}, \dots ,\alpha_{n,1}^{(k)}), \qquad k=1,2,\dots
	\]
	We have 
	\[
		\abs{\alpha_{2,1}^{(k)}} \leq 1, \qquad k=1,2,\dots
	\]
	Bolzano-Weierstraß implies that there exists a converging subsequenz $(\alpha_{2,2}^{(k)})_{k=1}^{\infty}$ of $(\alpha_{2,1}^{(k)})_{k=1}^{\infty}$. 
	Set 
	\[
		\bar{\alpha_2} = \lim_{k \to \infty} \alpha_{2,2}^{(k)}
	\]
\end{beweis}

%%% lecture 3
%%% lecture3

\begin{definition*}[normed space]
	Let $E$ be a vector space over $\mathbb{R}$ or $\mathbb{C}$. $\norm{.}:E \to \mathbb{R}$ a norm on $E$ if
	\begin{enumerate}[(i)]
		\item $\norm{x} >0 \qquad $ for any $x \in E \setminus \set{0}$
		\item $\norm{\lambda x} = \abs{\lambda x} \qquad $ for any $\lambda \in \mathbb{C},x \in E$.
		\item $\norm{x+y} \leq \norm{x} + \norm{y} \qquad$ for any $x,y \in E$.
	\end{enumerate}
	Obs. $\norm{x}=0$ if $x =0$. $(E, \norm{.})$ is called a normed space. A norm generates a distance function (metric)
	\[
		L(x,y):= \norm{x-y} \qquad \text{ for any }x,y \in E.
	\]
\end{definition*}
\begin{beispiele}
	\begin{itemize}
		\item $\mathbb{R}^n$ with $\norm{x}_2 = \sqrt{\sum^{n}_{i=1}\abs{x_i}^2}$ is the eukledian norm.
		\item $C([0,1])$ continuous functions in $[0,1]$ with
		\[
			L(f,g)=\norm{f-g}_{\infty}:= \max_{x \in [0,1]}\abs{f(x)-g(x)}
		\]
	\end{itemize}
\end{beispiele}
\begin{definition*}[balls]
	 Let $x \in E$, $r >0$. Define
	\begin{align*}
		B(x,r) &:= \set[y \in E]{\norm{x-y}<r} \qquad \text{open ball} \\
		\bar{B}(x,r) &:= \set[y \in E]{\norm{x-y}\leq r} \qquad \text{closed ball}
	\end{align*}
\end{definition*}
\begin{definition*}[open/closed]
	A subset $A \subset E$ of a normed space $(E,\norm{.})$ is called \underline{open} of any point $x$ of $A$ is inner, i.e 
	\[
		\exists\,r>0 \,:\, B(x,r) \subset A.
	\]
	It is called \underline{closed} if the complement $E \setminus A$ is open.
\end{definition*}
\begin{bemerkung}
	\begin{itemize}
		\item open balls are open sets.
		\item closed balls are closed.
		\item $(C([0,1]),\norm{.}_{\infty})$ with $\norm{f}_{\infty}= \max_{x \in [0,1]}\abs{f(x)}.$ 
		\[
			A := \set[g \in C([0,1])]{f(x) < g(x),\,\forall\, x \in [0,1]}
		\]
		is an open set $C([0,1])$.
		\[
			B := \set[g \in C([0,1])]{f(x) \leq g(x),\,\forall\, x \in [0,1]}
 		\]
		is a closed set.
	\end{itemize}
	\minisec{Properties}
	\begin{itemize}
		\item Any union of open sets is an open set.
		\item Any \underline{finite} intersection of open sets is open.
		\item $\emptyset,E$ are both closed and open.
		\item Normed spaces are topological spaces.
	\end{itemize}
\end{bemerkung}
\begin{definition*}[convergence in normed spaces]
	Let $(E,\norm{.})$ be a normed space $\set{x_n}_n \subset E$. We say that $x_n$ converges to $x \in E$ if 
	\[
		\norm{x_n-x} \to 0, \qquad n \to \infty
	\]
\end{definition*}
One can define open and closed using the definition of convergence:
\begin{satz} % statement
	$A \subseteq E$ is closed if any convergent sequence in $A$ has a limit in $A$, i.e
	\[
		\substack{x_n \to x \\ \text{for }n \to \infty \\ x_n \in A} \Rightarrow x \in A
	\]
\end{satz}
\begin{beweis}
	\begin{description}
		\item[$\Rightarrow$:]Assume that $A$ is closed and $x_n \to x$. $x_n \in A$, but $x_n \not \in A$. (try to get a contradiction). \\
	 $A$ is closed $\Rightarrow $ $E \setminus A$ is open and hence $\exists\, r >0$ such that
	 \[
	 	B(x,r) \subset E \setminus A.
	 \]
	 Hence $\norm{x_n -x} \geq r$ for any $n$. This is a contradiction because in that case $x_n \not \to x$
	 \item[$\Leftarrow $:] Assume that for any sequence $\set{x_n} \subset A$ such that $x_n \to x$ we have $x \in A$. We try to get a contradiction and assume that $A$ is not closed. Hence $E \setminus A$ is not open and therefore $\exists\, x \in E \setminus A$ which is not inner.
	 \[
	 	\Rightarrow \qquad  \forall\, B(x,\frac{1}{n}) \text{ containts points outside }E \setminus A 
	 \]
	 i.e.
	 \[
	 	\exists\, x_n \in B(x, \frac{1}{n}), \, x_n \in A.
	 \]
	 We get a sequence $\set{x_n} \subset A$ such that
	 \[
	 	\norm{x_n-x} < \frac{1}{n} \qquad \Rightarrow \qquad x_n \to x
	 \]
	 This is a contradiction
	\end{description}
\end{beweis}
\begin{definition*}[closure]
	$A \subset E$. The closure of $A$ is the minimal closed subset containing $A$. We write $\bar{A}$.
\end{definition*}
\begin{proposition}
	$\bar{A}$ is the set of all limit points of $A$ which means
	\[
		\bar{A} := \set[x \in E]{\text{there exists $\set{x_n} \subseteq A$ such that $x_n \to x$}}
	\]
\end{proposition}
\begin{beweis}
	exercise.
\end{beweis}
\begin{definition*}[dense]
	$A \subset E$ is dense in $E$ if 
	\[
		\bar{A} = E.
	\]
\end{definition*}
\begin{bemerkung}
	This definition of dense is equivalent to the following definition:
	\[
		\forall\, x \in E,\,\forall\, \varepsilon>0 \,\exists\,y \in A \text{ such that } \norm{x-y} < \varepsilon.
	\]
\end{bemerkung}
\begin{beispiele}
	\begin{enumerate}[1)]
		\item $\mathbb{Q} \subseteq \mathbb{R}$ with $\abs{.}$ usual absolut value function. $\mathbb{Q}$ is dense in $\mathbb{R}$.
		\item $C([a,b])$. The \underline{Weierstraß-Theorem} says that the set of all polynomials are dense in $(C([a,b],\norm{.}_{\infty}))$:
		\[
			\forall\, f \in C([a,b]),\,\forall\, \varepsilon>0 \,\exists\,p-\text{polynomial such that }\max_{x \in [a,b]}\abs{f(x)-p(x)} < \varepsilon.
		\]
	\end{enumerate}
\end{beispiele}
Another example is $(C_0, \norm{.}_{\infty})$ where
		\[
			C_0 = \set[x = (x_1,x_2,\dots)]{x_k \to 0 \text{ as }k \to \infty}
		\]
		\[
			\norm{x}_{\infty}= \sup_{i}\abs{x_i}
		\]
		$(C_0,\norm{.}_{\infty})$ is a normed space. 
		\[
			C_F = \set[x = (x_1,x_2,\dots)]{\text{only a finite number of }x_i \neq 0} \subset C_0
		\]
\begin{satz}
	$C_F$ is dense in $C_0$
\end{satz}
\begin{beweis}
	\[
		\forall\,  x \in C_0 \,\forall\, \varepsilon>0 \text{ must find }y \in C_F \text{ such that }\norm{y-x}_{\infty} < \varepsilon.
	\]
	\[
		x \in C_0 \qquad \Rightarrow \qquad x_k \to 0 \text{ for }k \to \infty 
	\]
	\[
		\Rightarrow \qquad \forall\, \varepsilon >0 \,\exists\,K \,\text{ such that } \abs{x_k} < \varepsilon \, \forall\, k \geq K
	\]
	Let now $y = (x_1,x_2, \dots,x_K, 0, \dots) \in C_F$. Then 
	\[
		\norm{x-y}_{\infty} = \norm{(0,0,\dots,0,x_{K+1},x_{K+2},\dots)}_{\infty} = \sup_{k >K}\abs{x_k} < \varepsilon
	\]
\end{beweis}
\begin{definition*}[separable]
	A normed space $(E,\norm{.})$ is called \underline{separable} if it contains a countable dense subset.
\end{definition*}
\begin{beispiele}
	\begin{itemize}
		\item $(\mathbb{R},\abs{.})$ is separable as $\mathbb{Q}$ is countable and dense in $\mathbb{R}$.
		\item $(\mathbb{R}^n,\norm{.}_2)$ is separable, $\mathbb{Q}^n$ is countable and dense in $\mathbb{R}$.
	\end{itemize}
\end{beispiele}
\begin{definition*}[compact set]
	For a normed space $(E,\norm{.})$ is $A \subset E$ a compact set if any sequence $\set{x_n} \subset A$ has a subsequence convergent to an element $x \in A$.
\end{definition*}
\begin{beispiel}
	 Any bounded and closed subset in $\mathbb{R}, \mathbb{R}^n, \mathbb{C}^n$ is compact. A sequence $\set{x_n}$ of a bounded set is bounded. From real Analysis one knows it has a subsequence that is convergent. If the subset is closed then the limit point is inside the set. 
\end{beispiel}
\begin{lemma*}
	$S \subset$ compact in $(E, \norm{.})$ implies that $S$ is closed and bounded. (Bounded means that $S \subset B(0,R)$ for some $R>0$)
\end{lemma*}
\begin{beweis}
	Let $S$ be a compact subset of $E$. Assume that $S$ is not bounded. Hence for any $n >0$ there exists points in $S$ which are outside $B(0,n)$, i.e. 
	\[
		\exists\, x_n \in S \,: \norm{x_n} >n.
	\]
	Then $\set{x_n}$ can not have a convergent subsequence as if $x_{n_k} \to x$ then
	\[
		n_k < \norm{x_{n_k}} = \norm{x_{n_k}-x + x} \leq \norm{x_{n_k}-x} + \norm{x} \to \norm{x}
	\]
	but $n_k \to \infty$. This is a contradiction, hence $S$ must be bounded. \\ $S$ must be closed, because if $x_n \to x$ then any subsequence converges to $x$. From the definition of compactness and uniqueness of the limit we have $x \in S$. \\
\end{beweis}
\begin{bemerkung}
	In general, $S$ bounded and closed doesn't imply that $S$ is compact. \\
For instance let $E= C([0,1])$. Then $S=\set[g \in C([0,1])]{\norm{g}_{\infty}\leq 1}$ is closed and bounded, but not compact. \\
Take $x_n(t):=t^n$. Then $x_n \in S$. $\set{x_n}$ does not have a subsequence convergent to a continuous function.
\end{bemerkung}
\begin{theorem}
	$(E,\norm{.})$ normed space and $\dim E < \infty$ \\ iff \[
		\forall\, A \subset E,\,A \text{ compact } \Leftrightarrow A \text{ is closed and bounded}
	\]
\end{theorem}
\begin{beweis}
	\begin{description}
		\item[$\Rightarrow$:]If $\dim E < \infty$ then $A$ is compact iff $A$ is bounded and closed (exsercise)
		\item[$\Leftarrow$:]Enough to prove the following: \\
		If $\dim E = \infty$ then the unit ball $S = \set[x \in E]{\norm{x}\leq 1}$ is not compact.
	\end{description}
\begin{lemma}[Riesz's lemma]
	If $X$ is a proper closed subspace of a normed space $(E,\norm{.})$ then for every $\varepsilon \in (0,1)$ there exists an $x_{\varepsilon} \in E$ with $\norm{x_\varepsilon}=1$ such that
	\[
		\norm{x_{\varepsilon}-x} \geq \varepsilon \qquad \forall\, x \in X.
	\]
\end{lemma}
\begin{beweis}
	Let $z \in E \setminus X$ ($X$ proper and hence $E \setminus X \neq \emptyset$). Set 
	\[
		d:= \inf_{x \in X}\norm{z-x}
	\] As $X$ is closed, $d >0$, otherwise $z$ is a limit point in $E \setminus X$. Fix $\varepsilon \in (0,1)$. Then there exists $x_0 \in X$ such that
	\[
		d \leq \norm{z-x_0} < \frac{d}{\varepsilon}.
	\]
	Let $x_\varepsilon := \frac{z-x_0}{\norm{z-x_0}}$; We have $\norm{x_\varepsilon}=1$ and
	\begin{align*}
		\norm{x- x_\varepsilon} &= \norm{x - \frac{z-x_0}{\norm{z-x_0}}} \\
		&= \frac{\norm{x \norm{z-x_0}-z + x_0}}{\norm{z-x_0}} \\
		&= \frac{\norm{\overset{\in X}{\overbrace{x \norm{z-x_0}+ x_0 }}- z}}{\norm{z-x_0}} \\
		&\geq \frac{d}{d}\varepsilon = \varepsilon
	\end{align*}
\end{beweis}
Continue now proof of the theorem above: \\
Let $x_1 \in S$. Consider $X = \text{span} \set{x_1}$ which is a proper closed subspace of $E$. Hence by Riesz's lemma exists $x_2$ with $\norm{x_2}=1$ such that
\[
	\norm{x_2-x_1} \geq \frac{1}{2} 
\]
and
\[
	\norm{x_2-x} \geq \frac{1}{2} \qquad \forall\, x \in X.
\]
Now consider $\text{span} \set{x_1,x_2}$ which is a proper closed subspace of $E$. By Riesz's lemma follows
\[
	\exists\,x_3 \in E,\, \norm{x_3}=1: \, \norm{x_3-x_1}\geq \frac{1}{2}, \norm{x_3-x_2} \geq \frac{1}{2}.
\]
Continuing in the same fashion we get $\set{x_n}$, $\norm{x_n}=1$ such that 
\[
	\norm{x_n-x_m} \geq \frac{1}{2} \qquad \forall\, n,m,\,n \neq m.
\]
Clearly $\set{x_n} \subset S$ has no convergent subsequence. Hence $S$ is not compact.
\end{beweis}
\begin{definition*}[Cauchy sequence]
	$(E, \norm{.})$ normed space. $\set{x_n} \subseteq E$ is called Cauchy if
	\[
		\forall\, \varepsilon >0 \,\exists\,N: \,\norm{x_n-x_m}< \varepsilon \,\text{ for any }n,m \geq N.
	\]
\end{definition*}
\begin{beispiel}
	$(C_F,\norm{.}_{\infty})$, $\norm{x}_{\infty}= \sup_{k \in \mathbb{N}}\abs{x_k}$ where $x = (x_1,x_2,\dots)$. Define
	\[
		x_n = (1, \frac{1}{2}, \frac{1}{3}, \dots, \frac{1}{n}, 0, \dots)
	\]
	Then $\set{x_n}$ is Cauchy, as for $n >m$ 
	\begin{align*}
		\norm{x_n-x_m}_{\infty} &= \norm{(0, \dots,0, \frac{1}{m+1}, \dots, \frac{1}{n},0,\dots)}_{\infty} \\
		&= \frac{1}{m+1}
	\end{align*}
	Observe that $x_n$ is convergent in $(C_0,\norm{.}_{\infty})$
	\[
		\underset{\in C_F}{\underbrace{x_n}} \to (1, \frac{1}{2}, \frac{1}{3}, \dots, \frac{1}{n}, \dots) \in C_0 \setminus C_F
	\]
\end{beispiel}
\begin{satz}
	A convergent sequence is always a Cauchy sequence.
\end{satz}
\begin{definition*}[complete space]
	A normed vector space $(E, \norm{.})$ is called \underline{complete} if any Cauchy sequence in $E$ is convergent in $E$.
\end{definition*}
$(C_F,\norm{.}_{\infty})$ is not complete.
\begin{definition*}[Banach space]
	A complete normed space is called \underline{Banach space}.
\end{definition*}
\begin{beispiele}
	\begin{itemize}
		\item $(\mathbb{R},\abs{.})$ is a Banach space.
		\item $(\mathbb{C},\abs{.})$ is a Banach space.
		\item $(l^2,\norm{.}_2)$ where
		\[
			l^2= \set[(x_1,x_2,\dots)]{\sum^{\infty}_{i=1} \abs{x_i}^2 < \infty, x_i \in \mathbb{C}}
		\]
		and 
		\[
			\norm{(x_1,x_2,\dots)}_2 = \left( \sum_{i=1}^{\infty} \abs{x_i}^2 \right)^{\frac{1}{2}}
		\]
		$(l^2,\norm{.}_2)$ is complete.
		\begin{beweis}
			Let $x_n = (x_1^n,x_2^n,\dots)$ be a Cauchy sequence in $l^2$. We must show that it has a limit in $l^2$. We will do it in a few steps:
			\begin{enumerate}[Step 1:]
				\item Find a candidate for a limit $a$
				\item Show that $a \in l^2$.
				\item $\norm{x_n - a }_2 \to 0$ as $n \to \infty$.
			\end{enumerate}
			\begin{description}
				\item[Step 1:] Let
				\begin{align*}
					x_1 &= (x_1^1,x_2^1, \dots) \\
					x_2 &= (x_1^2,x_2^2, \dots) \\
					\vdots & \qquad \vdots \\
					x_n &= (x_1^n,x_2^n, \dots)
				\end{align*}
				For each $k$ consider sequence $\set{x_k^n} \subset \mathbb{C}$ ($k$-th coordinates in each $x_n$). \\
				Each sequence is Cauchy, as for all $n,m \geq N$
				\[
					\abs{x_k^n-x_k^m} < \left( \sum_{k=1}^{\infty} \abs{x_k^n-x_k^m}^2 \right)^{\frac{1}{2}} = \norm{x_n-x_m}_2 < \varepsilon
				\]
				As $(\mathbb{C}, \abs{.})$ is complete, $\set{x_k^n}_n$ has a limit $a_k \in \mathbb{C}$. Candidate for limit of $x_n$ is 
				\[
					a= (a_1,a_2, \dots, a_k, \dots).
				\]
				\item[Step 2:] Write 
				\[
					a = \underset{\in l^2}{\underbrace{x_n}} - (x_n - a)
				\]
				In order to show that $a \in l^2$ it is enough to see that $x_n - a \in l^2$ for some $n$. \\
				$\set{x_n}$ Cauchy implies
				\[
					\forall\, \varepsilon>0 \,\exists\,N: \,\forall\, n,m \geq N: \,\norm{x_n-x_m}_2 < \varepsilon.
				\]
				Consider for some $u>0$
				\begin{align*}
					\sum^{u}_{i=1} \abs{x_i^n-x_i^m}^2 \leq \sum_{i=1}^{\infty}\abs{x_i^n-x_i^m}^2 = \norm{x_n-x_m}^2_2 < \varepsilon^2
				\end{align*}
				Let $m \to \infty$. We get 
				\[
					\sum^{m}_{i=1} \abs{x_i^n-a_i}^2 \leq  \varepsilon^2
				\]
				This holds for any $u \in \mathbb{N}$. Hence for any $n \geq \mathbb{N}$
				\[
					\underset{= \norm{x_n-a}_2^2}{\underbrace{\sum_{i=1}^{\infty}\abs{x_i^n-a_i}^2}} \leq \varepsilon^2.
				\]
				Hence $x_n - a \in l^2$ and moreover $\norm{x_n - a} \to 0$ as $n \to \infty$.
			\end{description}
		\end{beweis}
		\item $(C([a,b]),\norm{.}_{\infty})$ is a Banach space.
		\item $(l^p,\norm{.}_{l^p})$ for $1 \leq p < \infty$ are all Banach spaces.
		\item $(C([a,b]),\norm{.}_2)$ with
		\[
			\norm{f}_2 = \left( \int_{}^{} \abs{f(t)}^2 \,\mathrm{d}t \right)^{\frac{1}{2}}
		\]
		One can prove that $(C([a,b]),\norm{.}_2)$ is not a Banach space.
		\minisec{Exercise:} $[a,b] = [0,1]$ and \[
			f_n(t)= \begin{cases}
				0, &\text{ falls }t < \frac{1}{2} - \frac{1}{n}\\
				1, &\text{ falls }t > \frac{1}{2}\\
				\text{continuous linear function}
			\end{cases}.
		\]
		Show that $\set{f_n}$ is Cauchy in $C([0,1],\norm{.}_2)$ but $f_n \not \to f \in C([0,1])$.
	\end{itemize}
\end{beispiele}
\begin{definition*}[Convergent and absolutely convergent series]
	A series $\sum_{n=1}^{\infty}x_n$ in $E$ is called \underline{convergent} if $\set{\sum_{n=1}^{m}x_n}_m$, a sequence of partial sums, is convergent in $E$. If $\sum_{n=1}^{\infty}\norm{x_n} < \infty$ then we say that $\sum_{n=1}^{\infty}x_n$ converges absolutely.
\end{definition*}
\begin{theorem}
	A normed space $E$ is complete iff every absolutely convergent series converges in $E$.
\end{theorem}
\begin{beweis}
	\begin{description}
		\item[$\Rightarrow$:] Suppose $X$ is complete and $\sum_{n=1}^{\infty}\norm{x_n} < \infty$. Let 
		\[
			S_N := \sum_{n=1}^{N}x_n \in E.
		\] 
		For $M >N$:
		\begin{align*}
			\norm{S_N-S_M} &= \norm{\sum_{n=N+1}^{M}x_n} \\
			&\leq \sum_{n=N+1}^{M} \norm{x_n} \\
			&\leq \sum_{n=N+1}^{\infty} \norm{x_n} \to 0 \qquad \text{as }N \to \infty
		\end{align*}
		Hence $\set{S_N}$ is Cauchy. As $E$ is complete, $S_N$ has a limit in $E$ i.e. $\sum_{n=1}^{\infty}x_n$ converges in $E$.
		\item[$\Leftarrow$:] Assume that every absolut convergent series is convergent in $E$. We want to see that $E$ is complete. \\
		Let $\set{x_n}$ be a Cauchy sequence. We want to prove that $\set{x_n}$ has a limit in $E$. We know that
		\[
			\forall\, k \,\exists\,n_k: \, \norm{x_n-x_m}< \frac{1}{2^k} \qquad \forall\, n,m \geq n_k.
		\]
		We can assume that $\set{n_k}$ is an increasing sequence. Write
		\[
			x_{n_k} = (x_{n_k}-x_{n_{k-1}})+ (x_{n_{k-1}}-x_{n_{k-2}}) + \dots +(x_{n_1}-\underset{=0}{\underbrace{x_{n_0}}}) = \sum_{l=1}^{k}(x_{n_l}-x_{n_{l-1}}).
		\]
		\[
			\sum_{l=1}^{\infty}\norm{x_{n_l}-x_{n_{l-1}}} \leq \sum_{l=1}^{\infty}\frac{1}{2^l} < \infty
		\]
		Hence $\sum_{l=1}^{\infty}(x_{n_l}-x_{n_{l-1}})$ is absolutely convergent. By assumption 
		\[
			\sum_{l=1}^{\infty}(x_{n_l}-x_{n_{l-1}}) 
		\]
		is convergent in $E$. Hence the partial sums is convergent. Subsequence is convergent. $\set{x_{n_k}}$ is convergent to some $x \in E$.
		\minisec{Exercise:} Show that the whole $\set{x_n} \to x$.
	\end{description}
\end{beweis}

%%% lecture 4
%%% lecture 4


%%% 8.9.2016
\minisec{Recall:}
converging squences $(x_n)_{n=1}^{\infty}$ in $(E,\norm{.})$. $\norm{x_n-x} \to 0$ for $n \to \infty$ for some $x \in E$. (Notation: $x_n \to x$ in $(E,\norm{.})$)
\begin{bemerkung}
	Assume $x_n \to x$ in $(E,\norm{.})$. Then
	\begin{enumerate}[1)]
		\item $\norm{x_n} \to \norm{x}$ in $(E,\norm{.})$. 
		\item $\sup_{n} \norm{x_n} < \infty$.
	\end{enumerate}
	because
	\begin{enumerate}[1)]
		\item \[
			\norm{x_n} \leq \norm{x_n-x} + \norm{x},
		\]
		so
		\[
			\norm{x_n} - \norm{x} \leq \norm{x_n -x}.
		\]
		It follows
		\[
			-(\norm{x_n}-\norm{x}) \leq \norm{x_n -x}.
		\]
		So
		\[
			\abs{\norm{x_n}-\norm{x}} \leq \norm{x_n -x} \to 0, \qquad \text{for } n \to \infty.
		\]
		Cauchy sequence in $(x_n)_{n=1}^{\infty}$ in $(E,\norm{.})$ if $\norm{x_n-x_m} \to 0$ for $n,m \to \infty$. \\
		We obtain:
	 	$(x_n)_{n=1}^{\infty}$ converges in $(E,\norm{.})$  $\qquad \Rightarrow \qquad$  $(x_n)_{n=1}^{\infty}$ Cauchy sequence in $(E,\norm{.})$. ($\not \Leftarrow $ in general).
		If $\Leftarrow$ then we call $(E,\norm{.})$ a Banach space. 
	\end{enumerate}
	$\sum_{n=1}^{\infty}x_m$ converges in $(E,\norm{.})$ if $\left( \sum_{n=1}^{k}x_n \right)_{k=1}^{\infty}$ converges in $(E,\norm{.})$. \\
	$\sum_{n=1}^{\infty}x_m$ converges absolutely in $(E,\norm{.})$ if $\sum_{n=1}^{\infty}\norm{x_n}$ converges $(\mathbb{R},\norm{.})$. \\
\end{bemerkung}

\subsection{Mappings between normed spaces} 
\label{sub:mappings_between_normed_spaces}
\begin{definition*}
	Let $(E_1,\norm{.}_1)$, $(E_2,\norm{.}_2)$ be normed spaces. $T: E_1 \to E_2$ (not necessarily linear) is called continuous at $x_0 \in E_1$, if 
	\[
		x_n \to x_0 \text{ in } (E_1,\norm{.}_1) \qquad \Rightarrow \qquad T(x_n) \to T(x_0) \text{ in } (E_2,\norm{.}_2).
	\]
	$T$ is called \underline{continuous} if it is continuous at $x_0 \in E_1$ for all $x_0 \in E_1$. We say that $T: E_1 \to E_2$ is \underline{linear} if 
	\[
		T(\lambda_1 x_1 + \lambda_2 x_2) = \lambda_1 T(x_1) + \lambda_2 T(x_2)
	\]
	for all scalars $\lambda_1$, $\lambda_2$ and $x_1,x_2 \in E_1$. \\
	$T: E_1 \to E_2$ linear is called \underline{bounded} if there exists $M>0$ such that
	\[
		\norm{T(x)}_2 \leq M \norm{x}_1 \qquad \text{for all }x \in E_1.
	\]If $T$ is bounded linear $E_1 \to E_2$ define
	\[
		\norm{T} = \norm{T}_{E_1 \to E_2} := \inf \set[M \geq 0]{\norm{T(x)}_2 \leq M \norm{x}_1 \text{ for all }x \in E_1}.
	\]
\end{definition*}
\begin{lemma*}
	\[
		\norm{T} = \sup\limits_{\substack{x \in E_1 \\ x \neq 0}} \frac{\norm{T(x)}_2}{\norm{x}_1} = \sup\limits_{\substack{x \in E_1 \\ \norm{x}_1=1}} \norm{T(x)}_2.
	\]
\end{lemma*}
\begin{proposition}
	Assume $T: E_1 \to E_2$ linear. Then all the following statements are equivalent:
	\begin{enumerate}[(1)]
		\item $T$ continuous at $0 \in E_1$.
		\item $T$ continuous at $x_0 \in E_1$ for some $x_0 \in E_1$.
		\item $T$ continuous at $x_0 \in E_1$ for all $x_0 \in E_1$.
		\item $T$ is bounded.
	\end{enumerate}
\end{proposition}
\begin{beweis}
	\begin{description}
		\item[$(1) \Rightarrow (4)$:] Assume $T$ is continuous at $0 \in E_1$, i.e. 
		\[
			x_n \to 0 \text{ in }(E_1, \norm{.}_1) \qquad \Rightarrow \qquad T(x_n) \to T(\underset{\in E_1}{\underbrace{0}}) = \underset{\in E_2}{\underbrace{0}} \text{ in }(E_2,\norm{.}_2).
		\] 
		We want to prove that $T$ is bounded. We search a $M>0$ such that
		\[
			\norm{T(x)}_2 \leq  M \norm{x}_1.
		\]
		We assume that this doesn't hold true. \\
		For $n=1,2,\dots$ there exists $x_n \in E_1$ such that 
		\[
			\norm{T(x_n)}_2 > n \norm{x_n}_1.
		\]
		Set for $n=1,2,\dots$
		\[
			z_n := \frac{1}{n \norm{x_n}_1}x_n.
		\]
		(Note that $\norm{x_n}_1 >0$. Otherwise we would get a contradiction.) \\
		Note
		\begin{align*}
		\norm{z_n}_1 &= \norm{\frac{1}{n \norm{x_n}_1}}_1 = \frac{1}{n \norm{x_n}_1} \norm{x_n}_1 = \frac{1}{n} \to 0, \qquad \text{for }n \to \infty.
		\end{align*}
		We have $z_n \to 0$ in $(E_1,\norm{.}_1)$. But 
		\[
			\norm{T(z_n)}_2 = \norm{\frac{1}{n \norm{x_n}_1}T(x_n)}_2 = \frac{1}{n \norm{x_n}_1} \norm{T(x_n)}_2 > 1 \qquad \text{ for all }n.
		\]
		Hence
		\[
			T(z_n) \not \to 0 \qquad \text{ in }(E_2, \norm{.}_2).
		\]
		This is a contradiction.
		\item[$(1) \Leftarrow (4)$:] Assume $T$ is bounded. For some $M > 0$ 
		\[
			\norm{T(x)}_2 \leq M \norm{x}_1, \qquad \text{ for all }x \in E_1.
		\] 
		We need to show that $T$ is continuous at $0 \in E_1$, i.e.
		\[
			x_n \to 0 \text{ in } (E_1, \norm{.}_1) \qquad \Rightarrow \qquad T(x_n) \to T(0) = 0 \text{ in } (E_2, \norm{.}_2).
		\]
		From \[
			\norm{T(x_n)}_2 \leq  M \norm{x_n}_1 \to 0
		\]
		so
		\[
			T(x_n) \to \underset{=T(0)}{\underbrace{0}} \text{ in }(E_2, \norm{.}_2).
		\]
	\end{description}
\end{beweis}
\begin{beispiele}
	\begin{enumerate}[(A)]
	\item 
	$E_1 = E_2 = C([0,1])$, $\norm{.}_1 = \norm{.}_2 = \norm{.}_{\infty}=: \norm{.}$, i.e.
	\[
		\norm{f} := \max \limits _{x \in [0,1]} \abs{f(x)}.
	\]
	\[
		T(f)(x) = \int_{0}^{1-x} \min(x,y)f(y) \,\mathrm{d}y, \qquad \text{for }f \in C([0,1]), x \in [0,1].
	\]
	\begin{enumerate}[(1)]
		\item $T(f) \in C([0,1])$ for $ f \in C([0,1])$,
		\item $T$ linear,
		\item $T$ bounded,
		\item Calculate $\norm{T}$.
	\end{enumerate}
	\begin{beweis}
		\begin{enumerate}[(1)]
			\item Fix $f \in C([0,1])$ arbitrary and fix $x \in [0,1]$. Show that $T(f)$ is continuous at $x$. Consider a sequence $(x_n)_{n=1}^{\infty}$ in $[0,1]$ such that $x_n \to x$ in $(\mathbb{R},\abs{.})$. \\
			To show $T(f)(x_n) \to T(f)(x)$ in $(\mathbb{R},\abs{.})$.
			\begin{align*}
				\abs{T(f)(x_n)- T(f)(x)} &= \set{\text{assume that }x_n \leq x} \\
				&= \abs{\int_{0}^{1-x_n}\min(x_n,y)f(y) \,\mathrm{d}y - \int_{0}^{1-x} \min (x,y)f(y) \,\mathrm{d}y} \\
				&\leq  \abs{\int_{0}^{1-x}(\min (x_n,y) - \min(x,y) )f(y) \,\mathrm{d}y} \\ 
				&\qquad  + \abs{\int_{1-x}^{1-x_n}\min(x_n,y)f(y) \,\mathrm{d}y} \\
				&\leq \underset{\leq \abs{x_n-x} \norm{f}}{\underbrace{\int_{0}^{1-x} \underset{\leq \abs{x_n-x }}{\underbrace{\abs{\min (x_n,y) - \min(x,y)}}}
				 \underset{\leq \norm{f}}{\underbrace{\abs{f(y)}}} \,\mathrm{d}y}} \\ 
				& \qquad + \underset{0 \leq \dots \leq \abs{x_n-x}\cdot \norm{f}}{\underbrace{\int_{1-x}^{1-x_n}\underset{\leq
					 	1}{\underbrace{\min(x_n,y)}}\underset{\leq \norm{f}}{\underbrace{\abs{f(y)}}} \,\mathrm{d}y }} \to 0, \qquad \text{ as }n \to \infty
			\end{align*}
			If $x_n > x$ we get a similar calculation. Conclusion: 
			\[
				T(f)(x_n) \to T(f)(x) \text{ in }(\mathbb{R},\abs{.}) \text{ as } n \to \infty.
			\]
			\item Fix $f_1,f_2 \in C([0,1])$ and $\lambda_1, \lambda_2$ scalars. Then
			\begin{align*}
				T(\lambda_1 f_1 + \lambda_2 f_2)(x) &= \int_{0}^{1-x} \min(x,y)\underset{= \lambda_1 f_1(y)+\lambda_2 f_2(y)}{\underbrace{(\lambda_1 f_1 + \lambda_2 f_2)(y)}} \,\mathrm{d}y \\
				&= \lambda_1 \int_{0}^{1-x}\min(x,y)f_1(y) \,\mathrm{d}y + \lambda_2 \int_{0}^{1-x} \min(x,y)f_2(y) \,\mathrm{d}y \\
				&= \lambda_1 T(f_1)(x) + \lambda_2 T(f_2)(x) \qquad \text{ for }x \in [0,1]
			\end{align*}
			\item Fix $f \in C([0,1])$. For $x \in [0,1]$
			\begin{align*}
				\abs{T(f)(x)} &= \abs{ \int_{0}^{1-x} \underset{\geq 0}{\underbrace{\min(x,y)}}f(y) \,\mathrm{d}y }\\
				&\stackrel{(*_1)}{\leq} \int_{0}^{1-x}\min(x,y)\underset{\leq \norm{f}}{\underbrace{\abs{f(y)}}} \,\mathrm{d}y \\
				&\stackrel{(*_2)}{\leq} \int_{0}^{1-x} \min(x,y) \,\mathrm{d}y \norm{f}.
			\end{align*}
			Clearly 
			\[
				\max\limits_{x \in [0,1]}\int_{0}^{1-x} \min(x,y) \,\mathrm{d}y \leq 1.
			\]
			This gives:
			\[
				\norm{T(f)} = \max\limits_{x \in [0,1]}\abs{T(f)(x)} \leq 1 \cdot \norm{f}, \qquad \text{for all }f \in C([0,1]).
			\]
			Conclusion: $T$ is bounded with ($M=1$)
			\item Consider the unequality above. $(*_1)$ is an equality if $f$ has a constant sign. $(*_2)$ is an equality if $f$ is a constant function. So we have to calculate 
			\[
				\int_{0}^{1-x} \min(x,y) \,\mathrm{d}y \qquad \text{for }x \in [0,1].
			\]
			\begin{description}
				\item[case 1:]$1-x \leq x$ i.e. $ \frac{1}{2} \leq x$ and we get
				\begin{align*}
					\int_{0}^{1-x} \underset{=y}{\underbrace{\min(x,y)}} \,\mathrm{d}y &= \left[ \frac{1}{2} y^2 \right]_0^{1-x}  \\ &= \frac{1}{2} (1-x)^2.
				\end{align*}
				\item[case 2:] $x < 1-x$ i.e. $ x < \frac{1}{2}$ and we get
				\begin{align*}
					\int_{0}^{1-x} \min(x,y) \,\mathrm{d}y &= \int_{0}^{x} y  \,\mathrm{d}y + \int_{x}^{1-x}x \,\mathrm{d}y  \\ &= \frac{1}{2}x^2 + x(1-2x) 
					\\ & = x - \frac{3}{2}x^2.
				\end{align*}
				\textbf{Claim:} \text{    }   
				\[
					\norm{T} = \max \left( \max\limits_{x \in [\frac{1}{2},1]} \frac{1}{2}(1-x)^2 , \max\limits_{x \in [0,\frac{1}{2}]}\left( x - \frac{3}{2}x^2
					 \right) \right) = 
					\dots = \frac{1}{6}.
				\]
				Note 
				\begin{itemize}
					\item $\norm{T(f)}\leq \norm{T} \cdot \norm{f}$ for all $f \in C([0,1])$,
					\item $\norm{T(1)} = \norm{T} \cdot \norm{1}$ where $1(x)=1$ for $x \in [0,1]$.
				\end{itemize}
			\end{description}
		\end{enumerate}
	\end{beweis}
	\item $E_1 = C([0,1])$ with maximumnorm, $E_2 = \mathbb{R}$ with absolut value. $T: E_1 \to E_2$ with
	\[
		T(f) = \int_{0}^{\frac{1}{2}}f(y) \,\mathrm{d}y - \int_{\frac{1}{2}}^{1} f(y) \,\mathrm{d}y \qquad \text{for }f \in E_1
	\]
	\begin{align*}
		\abs{T(f)} &= \abs{\int_{0}^{\frac{1}{2}}f(y) \,\mathrm{d}y - \int_{\frac{1}{2}}^{1} f(y) \,\mathrm{d}y} \\
		&\leq \abs{\int_{0}^{\frac{1}{2}}f(y) \,\mathrm{d}y} + \abs{\int_{\frac{1}{2}}^{1} f(y) \,\mathrm{d}y} \\
		&\leq \int_{0}^{\frac{1}{2}}\underset{\leq \norm{f}}{\underbrace{\abs{f(y)}}} \,\mathrm{d}y 
		+ \int_{\frac{1}{2}}^{1} \underset{\leq \norm{f}}{\underbrace{\abs{f(y)}}} \,\mathrm{d}y \\
		&\leq 1 \norm{f}.
	\end{align*}
	Hence $T$ is bounded and $\norm{T}\leq 1$.
	\[
		T(f) = \int_{0}^{1}k(y)f(y) \,\mathrm{d}y, 
	\]
	where 
	\[
		T(f_n)= \begin{cases}
			\text{to be completed}, &\text{ falls }case\\
		\end{cases}.
	\]
	\[
		T(f_n) \leq 1 \left( \frac{1}{2} - \frac{1}{2n} + \frac{1}{2} - \frac{1}{2n} \right) = 1 - \frac{1}{n}, \qquad  n=1,2,\dots.
	\]
	Note
	\[
		k(y)f_n(y) \geq 0 \qquad \text{ for } y \in [0,1].
	\]
	Hence $\norm{T} \leq 1- \frac{1}{n}$ for $n =1,2,\dots$. Note $\norm{f_n}=1$ for all $n$. Conclusion $\norm{T}=1$. \\
	Here
	\[
		\abs{T(f)} \leq \underset{\leq 1}{\underbrace{\norm{T}}} \norm{f} \text{ for all }f \in C([0,1])
	\]
	but
	\[
		\abs{T(f)} < \norm{T} \norm{f} \qquad \text{ for all }f \in C([0,1]).
	\]
	\end{enumerate}
\end{beispiele}
\begin{satz}
	$T_1,T_2$ bounded linear mappings $(E_1,\norm{.}_1) \to (E_2,\norm{.}_2)$ and $\lambda$ scalar. Set
	\begin{align*}
		(T_1+T_2)(x) &= T_1(x) + T_2(x) \qquad x \in E_1 \\
		( \lambda T_1)(x) &= \lambda T_1(x) \qquad x \in E_1.
	\end{align*}
	\textbf{Claim:} \text{    }     
	\begin{enumerate}[(1)]
		\item $T_1 + T_2$ and $\lambda T_1$ are both linear mappings $(E_1,\norm{.}_1) \to (E_2,\norm{.}_2)$,
		\item $T_1 + T_2$ and $\lambda T_1$ are both bounded mappings $(E_1,\norm{.}_1) \to (E_2,\norm{.}_2)$. \\
		$B(E_1,E_2)$ denote the vector space of all bounded linear mappings $(E_1,\norm{.}_1) \to (E_2,\norm{.}_2)$.
		\item \[
			\norm{T}_{E_1 \to E_2} := \inf \set[M > 0]{\norm{T(x)}_2 \leq M \norm{x}_1 \text{ for all }x \in E_1}
		\]
		defines a norm in $B(E_1,E_2)$.
	\end{enumerate}
\end{satz}
\begin{beweis}
	\begin{enumerate}[(1)]
		\item $\norm{T}=0$ implies that $\norm{T(x)}_2 = 0$ for all $x \in E_1$ $ \,\, \Rightarrow \,\, $ $T(x) = 0 \in E_2$.
		\[
			T = 0 \in B(E_1,E_2)
		\]
		\item $T \in B(E_1,E_2)$ and $\lambda$ scalar. 
		\begin{align*}
			\norm{\lambda T} &= \inf \set[M >0]{\norm{(\lambda T)(x)}_2 \leq M \norm{x}_1 \text{ for all }x \in E_1} \\
			&= \inf \set[M>0]{\abs{\lambda} \norm{T(x)}_2 \leq M \norm{x}_1 \text{ for all }x \in E_1} \\
			&= \set{\text{if }\lambda \neq 0} \\
			&= \inf \set[\underset{= \abs{\lambda} \tilde M}{\underbrace{M}}>0]{\norm{T(x)}_2 \leq \underset{= \tilde M}{\underbrace{\frac{M}{\abs{\lambda}}}}\norm{x}_1 \text{ for all }x \in E_1} \\
			&= \abs{\lambda} \inf \set[\tilde M >0]{ \norm{T(x)}_2 \leq \tilde M \norm{x}_1 \text{ for all }x \in E_1} \\
			&= \abs{\lambda} \norm{T}
		\end{align*}
		\item Set $T_1,T_2 \in B(E_1,E_2)$.
		\begin{align*}
			\norm{T_1+ T_2} &= \inf \set[M>0]{\norm{(T_1+T_2)(x)}_2 \leq M \norm{x}_1 \text{ for all }x \in E_1} \\
			&\leq \inf \set[M_1 + M_2>0]{\norm{T_1(x)}_2 \leq M_1 \norm{x}_1, \, \norm{T_2(x)}_2 \leq M_2 \norm{x}_1 \text{ for all }x \in E_1} \\
			&= \norm{T_1} + \norm{T_2}
		\end{align*}
	\end{enumerate}
\end{beweis}
Conclusion: $(B(E_1,B_2),\norm{.}_{E_1 \to E_2})$ is a normed space. 
\begin{satz}
	 $(B(E_1,B_2),\norm{.}_{E_1 \to E_2})$ is a Banach space if $(E_2,\norm{.}_2)$ is a Banach space.
\end{satz}
\begin{beweis}
	Assume $(T_n)_{n=1}^{\infty}$ is a Cauchy sequence in $(B(E_1,B_2),\norm{.}_{E_1 \to E_2})$ where $(E_2, \norm{.}_2)$ is a Banach space. Fix $x \in E_1$
	\begin{align*}
		\norm{T_n(x)-T_m(x)}_2 &= \norm{(T_n-T_m)(x)}_2  \\
		&\leq \underset{\substack{\to 0 \\ n,m \to \infty}}{\underbrace{\norm{T_n-T_m}_{E_1 \to E_2}}} \cdot \norm{x}_1 \to 0, \qquad n,m \to \infty.
	\end{align*}
	Hence $(T_n(x))_{n=1}^{\infty}$ is a Cauchy sequence in $(E_2, \norm{.}_2)$. This is a Banach space which implies that $(T_n(x))_{n=1}^{\infty}$ converges in 
	$(E_2,\norm{.}_2)$. Call the limit $T(x) \in E_2$ for all $x \in E_1$. Show now 
	\begin{enumerate}[(1)]
		\item $T: E_1 \to E_2$ is linear,
		\item $T$ is bounded,
		\item $\norm{T_n - T}_{E_1 \to E_2} \to 0$ for $n \to \infty$.
	\end{enumerate}
	\begin{enumerate}[(1)]
		\item Observe 
		\begin{align*}
		T(\lambda_1 x_1 + \lambda_2 x_2)\leftarrow T_n(\lambda_1 x_1 + \lambda_2 x_2) = \set{T \text{ linear}} = \underset{\to \lambda_1 T(x_1) + \lambda_2 T(x_2)}{\underbrace{\underset{\to \lambda_1 T(x_1)}{\underbrace{\lambda_1 \underset{\to T(x_1)}{\underbrace{T_n(x_1)}}}} + \underset{\to \lambda_2 T(x_2)}{\underbrace{\lambda_2 \underset{\to T(x_2)}{\underbrace{T_n(x_2)}}}}}}.
		\end{align*}
		So for $n \to  \infty$ it is
		\[
			T(\lambda_1 x_1 + \lambda_2 + x_2) = \lambda_1 T(x_1) + \lambda_2 T(x_2) \qquad \text{ in }(E_2, \norm{.}_2).
		\]
		\item Fix $\varepsilon >0$. Then there exists $N$ such that:
		\[
			\norm{T_n- T_m}_{E_1 \to E_2} < \varepsilon \qquad \text{for }n,m \geq N
		\]
		So for $x \in E_1$
		\[
			\norm{T_n(x)-T_m(x)}_2 \leq \norm{T_n - T_m}_{E_1 \to E_2} \norm{x}_1 < \varepsilon \norm{x}_1 \qquad \text{for }n,m \geq N.
		\]
		Let $m \to \infty$.
		\[
			\norm{T_n(x)- T(x)}_2 \leq \varepsilon \norm{x}_1 \qquad \text{for }n \geq N
 		\]
		So
		\begin{align*}
			\norm{T(x)}_2 &\leq  \norm{T(x)- T_N(x)}_2 + \norm{T_N(x)}_2 \\
			&\leq \varepsilon \norm{x}_1 + \norm{T_N}_{E_1 \to E_2} \cdot \norm{x}_1 \\
			&= \left( \varepsilon +  \norm{T_N}_{E_1 \to E_2} \right) \norm{x}_1 \qquad \text{for }x \in E_1.
		\end{align*}
		\item Look above and get
		\[
			\norm{T_n - T}_{E_1 \to E_2} \to 0, \qquad  n \to \infty.
		\]
	\end{enumerate}
\end{beweis}
\begin{theorem}[Banach-Steinhaus Theorem (uniform boundedness principle)]
	Set \\ $(E_1,\norm{.}_1)$ Banach space, $(E_2,\norm{.}_2)$ normed space and $\mathcal{F} \subset B(E_1,E_2)$. Assume
	\[
		\sup\limits_{T \in \mathcal{F}}\norm{T(x)}_2 < \infty \qquad \text{for all } x \in E_1
	\]
	then
	\[
		\sup\limits_{T \in \mathcal{F}}\norm{T}_{E_1 \to E_2} < \infty.
	\]
	\end{theorem}
	\begin{bemerkung}
		The implication $\Leftarrow$ is easy to prove. If $\mathcal{F}$ is a finite set, the theorem is trivial.
	\end{bemerkung}
	\begin{beweis}
		\begin{enumerate}[Step 1:]
			\item Assume 
			\[
				\exists\,x_0 \in E_1\, \exists\, r >0 \,\exists\, M>0: \,\forall\, x \in \overline{B(x_0,r)} \, \forall\,  T \in \mathcal{F}: \,\norm{T(x)}_2 \leq M.
			\]
			We have to show that 
			\[
				\sup\limits_{T \in \mathcal{F}} \norm{T}_{E_1 \to E_2} < \infty.
			\]
			Fix $T \in \mathcal{F}$. For $\norm{x}_1 \leq r$
			\[
				\norm{T(x_0+x)}_2 \leq M.
			\]
			Note that $x_0+x \in \overline{B(x_0,r)}$.
			\begin{align*}
				\norm{T(x)}_2 &= \norm{T(x_0+x-x_0)}_2 \\ &= \set{T \text{ linear}} \\ &= \norm{T(x_0 + x)-T(x_0)}_2 \\ &\leq \norm{T(x_0+x)}_2 +
				\norm{T(x_0)}_2 \\ &\leq 2M.
			\end{align*}
			For $0 \neq x \in E_1$ 
			\[
				\norm{T \left( \frac{r}{\norm{x}_1} x \right)}_2 \leq 2M.
			\]
			$\frac{r}{\norm{x}_1} x$ has the $\norm{.}_1$-norm equal to $r$. This implies, since T linear,
			\[
				\frac{r}{\norm{x}_1} \norm{T(x)}_2 \leq 2M,
			\]
			i.e.
			\[
				\norm{T(x)}_2 \leq \frac{2M}{r}\norm{x}_1 \qquad \text{for all }0 \neq x \in E_1.
			\]
			We have
			\[
				\norm{T}_{E_1 \to E_2} \leq \underset{\substack{\text{independant}\\ \text{of }T}}{\underbrace{\frac{2M}{r}}} < \infty
			\]
			\[
				\sup\limits_{T \in \mathcal{F}}\norm{T}_{E_1 \to E_2} \leq \frac{2M}{r} < \infty.
			\]
			\item Justify the assumption in step 1. This assumption is equivalent to
			\[
			\exists\,x_0 \in E_1\, \exists\, r >0 \,\exists\, M>0: \,\forall\, x \in B(x_0,r) \, \forall\,  T \in \mathcal{F}: \,\norm{T(x)}_2 \leq M.	
			\]
			(Note $\overline{B(x_0,r_1)} \subset B(x_0,r) \subset B(x_0,r_2)$ for $0 < r_1 < r < r_2$). \\
			Argue by contradiction. Assume that the assumption is false. Then it holds
			\[
				\forall\, x_0 \in E_1 \, \forall\, r >0 \,\forall\,  M>0: \,\exists\, x \in B(x_0,r) \,\exists\, T \in \mathcal{F} : \, \norm{T(x)}_2 > M.
			\]
			Idea: Find a converging sequence $x_n \in E_1$, $x_n \to x$ in $(E_1,\norm{.}_1)$ and a sequence $(T_n)_{n=1}^{\infty} \subset \mathcal{F}$ such that
			\[
				\norm{T_n(x_n)}_2 > n \qquad \text{for all }n, \qquad \text{and} \qquad \norm{T_n(x)}_2 > n \qquad \text{ for all }n.
			\]
			We have from above $x_1 \in B(0,1)$ and $T_1 \in  \mathcal{F}$ such that \[
				\norm{T_1(x_1)}_2 > 1.
			\] $T_1$ is bounded linear, hence continuous. This implies that there exists $0<r_1 < \frac{1}{2}$ such that
			\[
				\norm{T_1(x)}_2 >1 \qquad \text{for }x \in B(x_1,r_1)
			\]
			and \[
				\overline{B(x_1,r_1)}\subset B(0,1).
			\]
		\end{enumerate}
	\end{beweis}

%%% lecture 5
%lecture 5

\subsection{Fixed point theory} 
\label{sub:fixed_point_theory}
\begin{beispiel}
	Consider
	\[
		f(x)+ 5 \int_{0}^{1-x} \min(x,y)f(y) \,\mathrm{d}y = g(x), \qquad x \in [0,1] \qquad (*)
	\]
	where $g \in C([0,1])$. \\ 
	\textbf{Claim:} \text{    }     There exists an unique solution $f \in C([0,1])$ that $(*)$. \\
	Idea:
	\[
		f(x) = f(x) - 5 \int_{0}^{1-x} \min(x,y)f(y) \,\mathrm{d}y, \qquad x \in [0,1].
	\]
	Set for $x \in [0,1]$ 
	\[
			\tilde T(f)(x) = RHS(x).
	\]
	To find a solution to $(*)$ is the same finding $f \in C([0,1])$ such that 
	\[
		f = \tilde T(f).
	\]
	Clearly $ \tilde T : C([0,1]) \to C([0,1])$. (continual later).
\end{beispiel}

\begin{theorem}[Banach's fixed point theorem]
	$(E, \norm{.})$ Banach space. $T: E \to E$ (no assumption on linearity) is a contraction on $E$, i.e. there exists $c<1$ such that
	\[
		\norm{T(x)-T(\tilde x)} \leq c \norm{x- \tilde x} \qquad \text{for all }x,\tilde x \in E.
	\]
	Then there exists a unique $ \bar{x} \in E$ such that 
	\[
		\bar{x} = T( \bar{ x}).
	\]
	($\bar{x}$ is a fixed point)
\end{theorem}
\begin{beweis}
	\begin{description}
		\item[Uniqueness:]Assume $T( \bar{x}) = \bar{x}$ and $T ( \tilde x) = \tilde x$. Then
		\[
			\underset{\geq 0}{\underbrace{\norm{ \bar{ x} - \tilde x }}} = \norm{T( \bar{x})- T( \tilde x)} \leq \underset{< 1}{\underbrace{c}} \norm{ \bar{x}- \tilde x}.
		\] 
		Thus $\norm{\bar{x}- \tilde x} = 0$, i.e. $\bar{x} = \tilde x$.
		\item[Existence:] Pick an arbitrary $x_0 \in E$. Set
		\[
			x_{n+1} = T(x_{n}), \qquad n=0,1,2,\dots.
		\]
		\textbf{Claim:} \text{    }     $(x_n)_{n=1}^{\infty}$ is a Cauchy sequence in $(E,\norm{.})$.
		Note:
		\begin{align*}
			\norm{x_{n+1}-x_n}  &= \norm{T(x_n)-T(x_{n-1})} \\
			&\leq  c \norm{x_n - x_{n-1}} \\
			&\leq \dots \\
			&\leq  c^n \norm{x_1-x_0}, \qquad n=1,2,\dots.
		\end{align*}
		For $n>m$
		\begin{align*}
			\norm{x_n-x_m} &= \norm{x_n - x_{n-1}+ x_{n-1}- \dots + x_{m+1}- x_m} \\
			&\leq \norm{x_n - x_{n-1}} + \norm{x_{n-1} - x_{n-2}} + \dots + \norm{x_{m+1}- x_m} \\
			&\leq (c^{n-1}+ c^{n-2} + \dots c^m) \norm{x_1-x_0} \\
			&\leq \frac{c^m}{1-c} \norm{x_1 - x_0} \to 0 \qquad \text{as }n,m \to \infty.
		\end{align*}
		Hence $(x_n)_{n=1}^{\infty}$ is a Cauchy sequence in $(E,\norm{.})$. $(E, \norm{.})$ is a Banach space. So $(x_n)_{n=1}^{\infty}$ converges in $(E,\norm{.})$. Call the limit $\bar{x}$. \\
		\textbf{Claim:} \text{    }     $\bar{x}$ is a fixed point for $T$. 
		\begin{align*}
			\norm{\bar{ x}- T(\bar{x})} & = \norm{\bar{x}-x_{n+1}+ x_{n+1} - T(\bar{x})} \\
			&\leq \norm{\bar{x}-x_{n+1}} + \norm{\underset{T(x_n)}{\underbrace{x_{n+1}}} - T(\bar{x})} \\
			&\leq \underset{\to 0}{\underbrace{\norm{\bar{x}- x_{n+1}}}} + c \underset{\to 0}{\underbrace{\norm{x_n - \bar{x}}}} \to 0, \qquad n \to \infty
		\end{align*}
	\end{description}
\end{beweis}
\begin{bemerkung}
	\begin{enumerate}[(1)]
		\item $x_n \to \bar{x}$ for $n \to \infty$ independend of the choice of $x_0$
		\item Fix $z \in E$
		\begin{align*}
			\norm{\bar{x}-z} &= \norm{T(\bar{x})- T(z) + T(z) -z} \\
			&\leq \norm{T(\bar{x})-T(z)} + \norm{T(z) -z} \\
			&\leq c \norm{\bar{x}-z} + \norm{T(z) - z}.
		\end{align*}
		Hence 
		\[
			\norm{\bar{x}-z} \leq \frac{1}{1-c}\norm{T(z)-z}.
		\]
	\end{enumerate}
\end{bemerkung}
\begin{beispiel}
	Consider now the example from above: $(C([0,1]), \norm{.})$ with $\norm{f} = \max_{x \in [0,1]}\abs{f(x)}$ is a Banach space! To apply Banach's fixed point theorem we need $\tilde T$ to be a contraction. \\
	Fix $f_1,f_2 \in C([0,1])$ and get for $x \in [0,1]$
	\begin{align*}
		\abs{(\tilde T(f_1)- \tilde T(f_2))(x)} & = \abs{5 \int_{0}^{1-x} \min(x,y)f_2(y) \,\mathrm{d}y - 5 \int_{0}^{1-x}\min(x,y)f_1(y) \,\mathrm{d}y} \\
		&= \abs{5 \int_{0}^{1-x} \min(x,y)(f_2(y)-f_1(y)) \,\mathrm{d}y} \\
		&\leq 5 \int_{0}^{1-x} \min(x,y) \underset{\leq \norm{f_2-f_1}}{\underbrace{\abs{f_2(y)-f_1(y)}}} \,\mathrm{d}y \\
		&\leq 5 \underset{0 \leq  \dots \leq \frac{1}{6}}{\underbrace{\int_{0}^{1-x} \min(x,y)\,\mathrm{d}y}} \norm{f_2-f_1} \\
		&\leq \frac{5}{6} \norm{f_2-f_1}.
	\end{align*}
	Hence \[
		\norm{\tilde T(f_1)- \tilde T(f_2)} \leq \frac{5}{6} \norm{f_1-f_2}.
	\]
	We conclude that $\tilde T$ is a contraction. We can take $c = \frac{5}{6}$. By Banach's fixed point theorem $\tilde T$ has a unique fixed point. Finally $(*)$
	has a unique solution $f \in C([0,1])$ which is the fixed point. 
\end{beispiel}
\begin{theorem}[Banach's fixed point theorem (generalization)]
	$(E, \norm{.})$ Banach space. $T: F \to F$ where $F$ is a closed set in $E$. $N$ positive integer. Assume $T^N = \underset{N-\text{times}}{\underbrace{T \circ T \circ \dots \circ T}}$ is a contraction on $F$, i.e. there exists $c > 1$ such that
	\[
		\norm{T^N(x)- T^N(\tilde x)} \leq c \norm{x-\tilde x}, \qquad \text{for all }x, \tilde x \in F.
	\]
	Then $T$ has unique fixed point $\bar{ x}$, i.e.
	\[
		\bar{x} = T(\bar{x}) \in F.
	\]
\end{theorem}
\begin{beweis}
	\begin{description}
		\item[$N=1$:] Fix $x_0 \in F$ and consider $(x_n)_{n=1}^{\infty}$ where $x_{n+1} = T(x_n)$ for $n=0,1,2, \dots$. There $(x_n)_{n=1}^{\infty}$ is a Cauchy sequence and hence this converges in $E$ since this is a Banach space. Call the limit $\bar{x}$. Note
		\[
			\underset{\in F}{\underbrace{x_n}} \to \bar{x} \text{ in }E \text{ and $F$ is closed}
		\] 
		implies $\bar{x} \in F$. The rest of the argument is the same as before.
		\item[$N>1$:] By previous result we know that $T^N$ has a unique fixed point $\bar{x} \in F$, i.e. $\bar{x} = T^N(\bar{x})$. \\
		\textbf{Claim:} \text{    }     $\bar{x}$ is a fixed point for $T$.
		\begin{align*}
			\norm{T(\bar{x})-\bar{x}} &= \norm{T(T^N(\bar{x}))- T^N(\bar{x})} \\
			&= \norm{T^N(T(\bar{x}))- T^N(\bar{x})} \\
			&\leq c \norm{T(\bar{x})-\bar{x}}.
 		\end{align*}
		This gives 
		\[
			\norm{T(\bar{x}-\bar{x})} = 0, \qquad \text{i.e. }\bar{x} = T(\bar{x}).
		\]
		Existence of a fixed point for $T$ done. For the uniqueness assume $\bar{x} = T(\bar{x})$ and $\tilde x = T( \tilde x)$. Then
	\begin{align*}
		\bar{x} &= T( \bar{x}) = T^2(\bar{x}) = \dots = T^N(\bar{x}) \\
		\tilde x &= T(\tilde x) = T^2(\tilde x) = \dots = T^N(\tilde x).
	\end{align*}
	But $T^N$ has a unique fixed point so 
	\[
		\bar{x} = \tilde x.
	\]
	\end{description}
\end{beweis}
\begin{bemerkung}
	\begin{enumerate}[(1)]
		\item $T: (0,1] \to (0,1]$ where $T(x) = \frac{x}{2}$. Clearly $T$ is a contraction on $(0,1]$ but has no fixed point. Note that $(0,1]$ is not a closed intervall. 
		\item $T: [0,\infty) \to [0,\infty)$, where $T(x) = x + \frac{1}{x}$. Clearly $[0,\infty)$ is a closed intervall in $\mathbb{R}$ but $T$ has no fixed point. \\
		\textbf{Claim:} \text{    }     $T$ is not a contraction but 'close' to be a contraction. \\
		\begin{align*}
			\abs{T(x)-T(\tilde x)} < \abs{x- \tilde x} \qquad \text{for }x, \tilde x \in [1, \infty), x \neq \tilde x
		\end{align*}
		Note \[
			\abs{ T(x)- T( \tilde x)} = \abs{\underset{\substack{(1- \frac{1}{t})\leq 1 \\ \text{for }t \in [1,\infty)}}{\underbrace{T'(x)}}}\abs{x- \tilde x}
		\] for some $t$ betweeen $x$ and $\tilde x$.
	\end{enumerate}
\end{bemerkung}
\begin{beispiel}
	$(E,\norm{.})$ Banach space. $K$ compact set in $E$ and $T : K \to K$ where
	\[
		\norm{T(x)- T( \bar{x})} < \norm{x - \bar{x}} \qquad \text{for all }x, \bar{x} \in K, x \neq \bar{x}.
	\]
	Show: $T$ has a unique fixed point in $K$.
	\begin{description}
		\item[Uniqueness:] Assume $\bar{x} = T(\bar{x})$ and $\tilde x = T( \tilde x)$ and $\bar{x} \neq \tilde x$ for $ \bar{x}, \tilde x \in K$. Then
		\[
			\norm{\bar{x} - \tilde x} = \norm{ T( \bar{ x})- \tilde x} < \norm{ \bar{x}- \tilde x}.
		\]
		Contradiction because then $\bar{x} = \tilde x$.
		\item[Existence:] To show: There exists $x \in K$ such that $x = T(x)$, i.e.
		\[
			\norm{T(x)- x} = 0.
		\]
		Set $d := \inf_{x \in K} \norm{T(x)-x}$. Let $(x_n)_{n=1}^{\infty}$ be a sequence in $K$ such that 
		\[
			\norm{T(x_n)-x_n} \to d, \qquad \text{as }n \to \infty.
		\]
		$K$ compact implies that there exists a subsequence $(\tilde x_n)_{n=1}^{\infty}$ of $(x_n)_{n=1}^{\infty}$ such that $(\tilde x_n)_{n=1}^{\infty}$ converges in $K$. Call the limit element $\bar{x} \in K$. We know
		\[
			\tilde x_n \to  \bar{x} \qquad \text{in }K
		\]
		and	 
		\[
			\norm{T( \tilde x_n)- \tilde x_n} \to d.
		\]
		Question: \[
			T(\tilde x_n) \to T(\bar{x}) \qquad \text{ in }K?
		\]
		But since
		\[
			\norm{T(x)- T( \tilde x)} \leq  \norm{x- \tilde x} \qquad \text{ for all }x, \tilde x \in K
		\]
		we have 
		\[
			\tilde x_n \to \bar{x} \qquad \text{ in }K
		\]
		which implies
		\[
			T( \tilde x_n) \to T( \bar{ x}) \text{ in }K.
		\]
		Hence: 
		\[
			\norm{T( \bar{ x})- \bar{x}} \leftarrow \norm{T(\tilde x_n)- \tilde x_n} \to d, \qquad  n \to  \infty.
		\]
		We obtain
		\[
			\norm{T(\bar{x})- \bar{x}} = d.
		\]
		Question: Is $d=0$? \\
		If $d>0$ then $\bar{x} \neq  T( \bar{x})$, $\bar{x}, T( \bar{x}) \in K$
		\[
			\norm{T(\bar{x})- T(T(\bar{x}))} < \norm{\bar{x}- T(\bar{x})} = d = \inf_{x \in K} \norm{x- T(x)}.
		\]
		This is a contradiction which gives $d=0$ and so $\bar{x} = T(\bar{x})$.
	\end{description}
\end{beispiel}
\begin{beispiel}
	Consider
	\[
		f(x) = \int_{0}^{x}k(x,y)h(y,f(y)) \,\mathrm{d}y + g(x), \qquad x \in [0,1] \qquad (*),
	\]
	where $g \in C([0,1])$, $k \in C([0,1] \times [0,1])$ and $h: [0,1] \times \mathbb{R} \to \mathbb{R}$ continuous and satisfies: \\
	There exists $M>0$ such that
	\[
		\abs{h(x,z_1)-h(x,z_2)} \leq M \abs{z_1- z_2} \qquad \text{for all }x \in [0,1],\,z_1,z_2 \in \mathbb{R}.
	\]
	\textbf{Claim:} \text{    }     $(*)$ has a unique solution $f \in C([0,1])$. \\ For $f \in C([0,1])$ set
	\[
		T(f)(x) = \int_{0}^{x}k(x,y)h(y,f(y)) \,\mathrm{d}y + g(x) \qquad x \in [0,1].
	\]
	Here $T(f)(x) \in C([0,1])$. \\ Want to show: $T: C([0,1]) \to C([0,1])$ has a unique fixed point. \\
	Start with the Banach space $(C([0,1]), \text{max-norm})$. Check if $T$ is a contraction in $C([0,1])$. Fix $f_1,f2 \in C([0,1])$
	\[
		T(f_1)(x)- T(f_2)(x) = \int_{0}^{x} k(x,y)(h(y,f_1(y))-h(y,f_2(y))) \,\mathrm{d}y.
	\] 
	$k$ is continuous on the compact set $[0,1] \times [0,1]$ so 
	\[
		\sup\limits_{(x,y) \in [0,1]\times[0,1]} \abs{k(x,y)} =: N < \infty.
	\]
	We obtain 
	\begin{align*}
		\abs{(T(f_1)-T(f_2))(x)} &\leq \int_{0}^{x} \underset{\leq N}{\underbrace{\abs{k(x,y)}}}\underset{\leq M \underset{\leq \norm{f_1-f_2}}{\underbrace{f_1(y)-f_2(y)}}}{\underbrace{h(y,f_1(y))-h(y,f_2(y))}} \,\mathrm{d}y \\ & \leq \int_{0}^{x}NM \,\mathrm{d}y \norm{f_1-f_2} \\& \leq NM \norm{f_1-f_2}.
		\end{align*}
	This yields
	\[
		\norm{T(f_1)-T(f_2)} \leq NM \norm{f_1-f_2}.
	\]
	\begin{Large}
		\underline{IF:}
	\end{Large} \,$NM<1$ Then $T$ is a contaction. \\ Trick: For $a >0$ set 
	\[
		\norm{f}_a = \max\limits_{x \in [0,1]} e^{-ax}\abs{f(x)}
	\] 
	for $f \in C([0,1])$. \\
	\textbf{Claim:} \text{    }     $\norm{.}_a$ defines a norm on $C([0,1])$. This is easy to check. \\
	\textbf{Claim:} \text{    }     $\norm{.}$ and $\norm{.}_a$ are equivalent. \\
	This follows from
	\[
		e^{-a} \norm{f} \leq \norm{f}_a \leq \norm{f}
	\]
	for all $f \in C([0,1])$ (note that $\norm{.}$ is the max-norm).\\
	\textbf{Claim:} \text{    }     $(C([0,1]), \norm{.}_a)$ is a Banach space. \\
	This follows from the fact that $\norm{.}$ und $\norm{.}_a$ are equivalent and $(C([0,1]),\norm{.})$ is a Banach space. \\
	\textbf{Claim:} \text{    }     $T$ is a contraction on $(C([0,1]),\norm{.}_a)$ for $a >0$ large enough. \\
	For $f_1,f_2 \in C([0,1])$ and $x \in [0,1]$ we have
	\begin{align*}
		\abs{(T(f_1)-T(f_2))(x)} &\leq \int_{0}^{x} NM \abs{(f_1-f_2)(y)} \,\mathrm{d}y \\
		&= \int_{0}^{x}NM e^{ay} \cdot \underset{\leq \norm{f_1-f_2}_a}{\underbrace{e^{-ay} \abs{(f_1-f_2)(x)}}} \,\mathrm{d}y \\
		&\leq NM \underset{\frac{1}{a}(e^{ax}-1)}{\underbrace{\int_{0}^{x} e^{ay} \,\mathrm{d}y}} \norm{f_1-f_2}_a.
	\end{align*}
	So
	\[
		e^{-ax} \abs{(T(f_1)-T(f_2))(x)} \leq \frac{NM}{a}(1-e^{-ax})\norm{f_1-f_2}_a
	\]
	and
	\[
		\norm{T(f_1)-T(f_2)}_a \leq  \frac{NM}{a} \norm{f_1-f_2}_a
	\]
	For $a > NM$ is $T$ a contraction on $(C([0,1]),\norm{.}_a)$. Banach fixed point theorem implies that there is a unique $f \in C([0,1])$ that solves $(*)$.
\end{beispiel}


%%% lecture 6
%% lecture 6



%%% 15.9.2016
\begin{theorem}
	$(E,\norm{.})$ Banach space, $(Y,\norm{.})$ normed space. $T: E \times Y \to E$ where
	\begin{enumerate}[(1)]
		\item There exists a $C > 1$ such that
		\[
			\norm{T(x,y)-T(\tilde x,y)} \leq C \norm{x - \tilde x} \qquad \text{for all }x, \tilde x \in E,\, y \in Y.
		\]
		\item $T_x: Y \to E$ where $T_x(y)= T(x,y)$ is continuous for all $x \in E$.
	\end{enumerate}
	$\Rightarrow $ For every $y \in Y$ there exists a unique $g(y) \in E$ such that \[
		g(y)= T(g(y),y)
	\] and $g: Y \to E$ is continuous.
\end{theorem}
\begin{beweis}
	The existence of a unique element $g(y) \in E$ for every $y \in Y$ follows from Banach's fixed point theorem. \\
	Assume $y_n \to \tilde y$ in $(Y, \norm{.}_{*})$, i.e. \[
		\norm{y_n - \tilde y}_* \to 0, \qquad n \to \infty.
	\]
	Remains to show
	\[
		g(y_m) \to g(\tilde y) \qquad \text{ in }(E(,\norm{.})).
	\]
	\begin{align*}
		\norm{g(y_n)- g( \tilde y)} &= \norm{T(g(y_n),y_n) - T(g(\tilde y), \tilde y)} \\
		&\leq  \underset{ \stackrel{(1)}{\leq } c \norm{g(y_n)-g(\tilde y)}}{\underbrace{\norm{T(g(y_n),y_n) - T(g( \tilde y),y_n)}}} + 
		\underset{^{(2)}\to 0, \, n \to \infty}{\underbrace{\norm{T(g(\tilde y),y_n)- T(g(\tilde y),\tilde y)}}}
	\end{align*}
	We obtain
	\[
		\norm{g(y_n)-g(\tilde y)} \leq \frac{1}{1-c} \norm{T(g(\tilde y), y_n) - T(g(\tilde y),\tilde y)} \to 0, \qquad n \to \infty.
	\]
\end{beweis}

\begin{theorem}[Brouwer's fixed point theorem]
	$K$ compact ($=$ closed and bounded) convex subset of $\mathbb{R}^n$ and $T: K \to K$ continuous. Then $T$ has a fixed point, i.e. there exists $\bar{x} \in K$ with
	\[
		T(\bar{x}) = \bar{x}.
	\]
\end{theorem}
\begin{bemerkung}
	\begin{itemize}
		\item No uniqueness! Consider the case $T= \id_K$.
		\item Set $K \subseteq \mathbb{R}^n$ (in general) is convex if
		\[
			x, \tilde x \in K \text{ and } \lambda \in [0,1] \qquad \Rightarrow \qquad \lambda x + ( 1- \lambda)\tilde x \in K.
		\]
	\end{itemize}
\end{bemerkung}

\begin{theorem}[Perron's theorem]
	$A$ real-valued $n \times n$-Matrix with positive entries. $A = [a_{ij}]_{i,j=1, \dots,n}$ all $a_{ij}>0$. \\
	$\Rightarrow $ The mapping for $x \in \mathbb{R}^n$ 
	\[
		x \mapsto Ax
	\]
	has an eigenvalue $>0$ with an eigenvecto with positive entries, i.e. there exists $\lambda >0$ and $\tilde x \in \mathbb{R}^n$ with $A \tilde x = \lambda \tilde x$and all entries in $\tilde x$ are positive.
\end{theorem}

\begin{beweis}
We use Brouwer's fixed point theorem. Set \[
	K := \set[(x_1,x_2,\dots,x_n) \in \mathbb{R}^n]{x_k \geq 0, \, \sum^{n}_{i=1} x_i = 1}.
\]	
\textbf{Claim:} \text{    }     $K$ is closed, bounded and a convex set in $\mathbb{R}^n$. Thus $K$ is compact (since $K \subseteq \mathbb{R}^n$). Set
\[
	T(x_1,\dots,x_n) = \underset{\in K}{\underbrace{\frac{1}{\norm{Ax}_{l^1}}A \cdot \begin{pmatrix}
		x_1 \\ \vdots \\ x_n
	\end{pmatrix}}} \qquad \text{for all }(x_1,\dots,x_n) \in K
\]
\textbf{Claim:} \text{    }     $T: K \to K$ is continuous. \\
Since
\[
	x_k \to x \qquad \text{ in } K \text{ w.r.t. }l^1-\text{norm}.
\]
To show:
\[
	T(x_k) \to T(x) \qquad \text{ in } K \text{ w.r.t. }l^1-\text{norm}.
\]
Set
\begin{align*}
	x &= (x_1,x_2, \dots,x_n) \\
	x_k &= (x_1^{(k)},x_2^{(k)}, \dots, x_n^{(k)}) \qquad k = 1,2,\dots.
\end{align*}
Consider
\begin{align*}
	\norm{T(x_k)-T(x)}_{l^1} &= \norm{ \frac{1}{\norm{Ax_k}_{l^1}} A x_k - \frac{1}{\norm{Ax}_{l^1}} A x }_{l^1} \\
	&\leq  \norm{\frac{1}{\norm{Ax_k}_{l^1}} A x_k - \frac{1}{\norm{Ax}_{l^1}} A x_k}_{l^1} + \norm{\frac{1}{\norm{Ax}_{l^1}} A x_k - \frac{1}{\norm{Ax}_{l^1}} A x}_{l^1} \\
	&= \abs{\frac{1}{\norm{Ax_k}_{l^1}} - \frac{1}{\norm{Ax}_{l^1}}} \norm{A x_k}_{l^1} + \frac{1}{\norm{Ax}_{l^1}} \norm{A(x -x_k)}_{l^1}
\end{align*}
and
\begin{align*}
	\norm{A(x-x_k)}_{l^1} &= \sum^{n}_{i=1} \abs{ \sum^{n}_{j=1} a_{ij}(x_j-x_j^{(k)})} \\
	&\leq  \sum^{n}_{i=1} \sum^{n}_{j=1} a_{ij} \abs{x_j-x_j^{(k)}} \\
	&\leq \underset{< \infty}{\underbrace{n \cdot \max\limits_{i,j} a_{ij}}} \underset{\to 0}{\underbrace{\norm{x-x_k}_{l^1}}} \to 0, \qquad k \to \infty.
\end{align*}
So \[
	Ax_k \to Ax \qquad \text{in }l^1.
\]
This implies
\[
	\norm{Ax_k}_{l^1} \to \norm{Ax}_{l^1} \qquad \text{ in }\mathbb{R}.
\]
Brouwer's fixed point theorem implies that $T$ has a fixed point $\bar{x} \in K$.
\begin{align*}
	\bar{x} &= (\bar{x}_1,\bar{x}_2,\dots,\bar{x}_n) \\
	\bar{x} &= T(\bar{x}) = \frac{1}{\norm{A \bar{x}}_{l^1}} A \bar{x} 
\end{align*}
Hence
$A \bar{x} = \norm{ A \bar{x}}_{l^1} \bar{x}$ where $\abs{A \bar{x}}_l^1 >0$ and $\bar{x}$ has all entries $>0$.
\end{beweis}

\begin{theorem}[Schauder's fixed point theorem]
	$(E, \norm{.})$ Banach space. $K$ compact, convex set in $E$. $T: K \to K$ continuous. \\
	$\Rightarrow $ $T$ has a fixed point in $K$.
\end{theorem}
\begin{beispiel}
	\[
		S = \set[f \in C((0,1))]{f(0) = 0,\, f(1)=1,\, \norm{f} = \max\limits_{x \in [0,1]} \abs{f(x)} \leq 1}
	\]
	$T: S \to S$ defined by
	\[
		T(f)(x) = f(x^2), \qquad x \in [0,1].
	\]
	$C([0,1])$ is equipped with the max-norm. \\
	\textbf{Claim:} \text{    }     \begin{itemize}
		\item $S$ is closed, bounded and convex in $C([0,1])$.
		\item $T: S \to S$ is continuous.
		\item $T$ has no fixed point in $S$.
	\end{itemize}
	\begin{itemize}
		\item 
	$S$ bounded: $f \in S$ implies $\norm{f}\leq 1$. 
	\item $S$ closed: $f_n \to f$ in $(C([0,1]),\norm{.})$. \\
	To show: $f \in S$. \\
	Note \[
		\max_{x \in [0,1]}\abs{f_n(x)-f(x)} \to 0, \qquad n \to \infty.
	\]
	This implies
	\[
		\abs{f(0)} = \abs{f_n(0)-f(0)} \to 0, \qquad n \to \infty.
	\]
	So $f(0)=0$.
	\[
		\abs{1-f(1)} = \norm{f_n(1)-f(1)} \to 0, \qquad n \to \infty.
	\]
	So $f(1)=1$. For $x \in [0,1]$ we get
	\begin{align*}
		\abs{f(x)} &\leq \norm{f(x)-f_n(x)} + \abs{f_n(x)} \\
		&\leq \underset{\to 0}{\underbrace{\norm{f-f_n}}} + \underset{\leq 1}{\underbrace{\norm{f_n}}}. 
	\end{align*}
	Conclusion $f \in S$
	\[
		\norm{f} = \max_{x \in [0,1]}\abs{f(x)} \leq 1.
	\]
	\item $f,\tilde f \in S$ and $\lambda \in [0,1]$. \\
	To show: 
	\[
		\lambda f + (1- \lambda) \tilde f \in S.
	\]
	Trivial since
	\[
		(\lambda f + (1-\lambda) \tilde f)(0) = 0 
	\]
	\[
		(\lambda f + (1- \lambda) \tilde f)(1) = \lambda f(1)+ (1- \lambda)\tilde f(1)= 1
	\]
	and	
	\[
		\norm{\lambda f + (1-\lambda) \tilde f} \leq \abs{\lambda} \norm{f} + \abs{1- \lambda} \norm{ \tilde f} \leq 1.
	\]
	\end{itemize}
	We want to show that $T: S \to S$ is continuous. (obvious that $T(S) \subseteq S$)\\ Assume $f_n \to f$ in $S$ in max-norm, i.e.
	\[
		\max_{x \in [0,1]} \abs{f_n(x)-f(x)} \to 0, \qquad n \to \infty.
	\]
	To show: $T(f_n) \to T(f)$ in $S$ in max-norm.
	\begin{align*}
		\norm{T(f_n)-T(f)} &= \max_{x \in [0,1]}\abs{T(f_n)(x)- T(f)(x)} \\ & = \max_{x \in [0,1]}\abs{f_n(x^2)-f(x^2)}  \\ &= \norm{f_n -f} \to 0, \qquad n \to \infty.
	\end{align*}
	$T: S \to S $ has no fixed point. \\
	If $f \in S$ is a fixed point for $T$ then 
	\[
		f(x^2) = T(f)(x) = f(x), \qquad x \in [0,1].
	\]
	To show: there can be no such $f \in S$. \\
	Set $a = \inf \set[x \in [0,1]]{f(x) = \frac{1}{2}} \neq \emptyset \text{ since $f$ is continuous}$. $a \in (0,1)$ since if $a = 0$ then there exists a sequence
	\[
		a_n \in \set[x \in [0,1]]{f(x)= \frac{1}{2}} 
	\]
	such that $a_n \to a$ in $\mathbb{R}$ as $n \to \infty$. Contradiction since 
	\[
		\frac{1}{2} = f(a_n) \to f(a) = f(0) = 0
	\]
	since $f$ is continuous. \\
	But $0 < a^2 < a$ and $f(a^2) = f(a) = \frac{1}{2}$. This is a contradiction. \\
	If we believe in Schauder then we can conclude that $S \subseteq C([0,1])$ is not compact.
\end{beispiel}
\begin{theorem}[Arzela-Ascoli theorem]
	Assume $K$ is a compact set in $\mathbb{R}^n$ (e.g. $K = [0,1]$ in $\mathbb{R}^n$ $n=1$) and $S \subseteq C(K)$ where $C(K)$ is equipped with the max-norm. \\
	$\Rightarrow $ S is relatively compact in $C(K)$ iff
	\begin{enumerate}[(1)]
		\item $S$ uniformly bounded.
		\item $S$ is equicontinuous.
	\end{enumerate}
\end{theorem}
\begin{definition*}
	\begin{enumerate}[(i)]
		\item $S$ is uniformly bounded if
		\[
			\sup\limits_{f \in S} \norm{f} < \infty.
		\]
		\item $S$ is equicontinuous if: for every $\varepsilon >0$ there exists $\delta >0$ such that
		\[
			\abs{x- \tilde x} < \delta, \, x, \tilde x \in K \qquad \Rightarrow \qquad \abs{f(x)-f(\tilde x)}< \varepsilon.
		\]
		$\delta = \delta (\varepsilon)$ must not depend on $f$. \\
	\end{enumerate}
\end{definition*}
$S$ is relatively compact in $C(K)$ if for every sequence $(f_n)_{n=1}^{\infty}$ in $S$ there exists a converging subsequence in $C(K)$. \\
To show:
$S$ is relatively compact in $C(K)$ iff the closure $\bar{S}$ is compact in $C(K)$. 
\minisec{Things to do:}
\begin{enumerate}[(1)]
	\item Proof of Schander's theorem.
	\item Proof of Arzela-Ascoli theorem.
	\item Application with Schander.
	\item Proof of Brouwer's thereom (special case).
	\item Completion of normed spaces.
\end{enumerate}	
For (4) wie consider the following lemma.
\begin{lemma}[Sperner's lemma]
	Big triangle $T$ 
	\[
		T = \bigcup_{a \in A} T_a.
	\]
	$\set{T_a}_{a \in A}$ is triangle of $T$, i.e. for any pair $T_a$, $T_{\tilde a}$ in the triangulation
	\[
		T_a \cup T_{\tilde a}= \set{\emptyset \text{ or common vertrex or common side or }T_a = T_{\tilde a}}.
	\]
	$\Rightarrow $ There must exists a triangle $T_a$ with all vertices colored differently. MISSING FIGURE!
\end{lemma}

\begin{description}
	\item[Proof of Schander's fixed point theorem:]
	To prove: $(E, \norm{.})$ Banach space, $K$ compact \\ convex set in $E$ and $T: K \to K$ continuous. \\
	\textbf{Claim:} \text{    }     $T$ has a fixed point. \\
	\begin{beweis}
		\begin{lemma*}
			Assume $(x_n)_{n=1}^{\infty}$ sequence in $K$ such that
			\[
				\norm{T(x_n)-x_n} \to 0, \qquad  n \to \infty.
			\]
			T has a fixed point in $K$.
		\end{lemma*}
		\begin{beweis}
			Consider $(T(x_n))_{n=1}^{\infty}$ in $K$. $K$ compact implies that there exists a $z \in K$ and a subsequence $(T(\tilde x_n))_{n=1}^{\infty}$ of $(T(x_n))_{n=1}^{\infty}$ such that 
			\[
				T(\tilde x_n) \to  z \qquad  \text{in $K$ as }n \to \infty.
			\]
			Then
			\[
				\norm{\underset{\to z}{\underbrace{T(\tilde x_n)}}- \tilde x_n} \to 0, \qquad \text{as }n \to \infty.
			\]
			So $\tilde x_n \to z$ for $n \to \infty$. But $T$ continuous implies 
			\[
				z \leftarrow T( \tilde x_n) \to T(z), \qquad  n \to \infty.
			\]
			Conclusion: $z = T(z)$ so $z$ is a fixed point.
		\end{beweis}
		\begin{lemma*}
			$K$ compact set in $E$. Let $\varepsilon >0$. Then there exists a finite set $x_1,\dots,x_n \in K$ such that for all $x \in K$
			\[
				\min\limits_{k = 1, \dots, N} \norm{x- x_k} < \varepsilon.
			\] 
		\end{lemma*}
		\begin{beweis}
			Assume there is no finite sequence $x_1, \dots, x_N$. Then there exists a sequence $(x_n)_{n=1}^{\infty}$ such that
			\[
				\norm{x_k-x_l} \geq \varepsilon, \qquad \text{for }k \neq l.
			\]
			Clearly $(x_n)_{n=1}^{\infty}$ has no converging subsequence. This contradicts $K$ beeing compact.
		\end{beweis}
		Fix positive integer $n$. Apply previous lemma with $\varepsilon = \frac{1}{\varepsilon}$. then there exists a finite set $x_1,\dots,x_N$ such that
		\[
			K \subset \bigcup_{k=1}^N B \left(x_k, \frac{1}{n} \right).
		\]
		Set 
		\begin{align*}
			K_n  &= \set{\text{set of all convex combinations of $x_1, \dots,x_N$}} \\
			&= \set[\sum_{k=1}^{N} \lambda_k x_k]{\lambda_k \geq 0 \text{ for all }k,\, \sum_{k=1}^{N}\lambda_k = 1}.
		\end{align*}
		This set is a closed and bounded set in $\text{span}(K_n)$ finite dimensional. Also $K_n$ is convex.  \\
		(want $T_n: K_n \to K_n$ where $T_n$ close to $T$). \\
		Set $f_k(x)= \max \left(0, \frac{1}{n}- \norm{x-x_k}\right)$ for $x \in K$ and $k=1,2, \dots,N$. \\
		For each $x \in K$ there exists a $k$ such that $f_k(x)>0$. Set
		\[
			P_n(x) = \frac{f_1(x)x_1+f_2(x_2)+ \dots+ f_N(x_N)}{f_1(x)+f_2(x)+ \dots+ f_N(x)}, \qquad x \in K.
		\]
		$P_n$ is a convex combination of $x_1,\dots,x_N$ for every $x \in K$. So $P_n(x) \in K_n$ for every $x \in K$. \\
		\textbf{Claim:} \text{    }     $\norm{P_n(x)-x} < \frac{1}{n}$ for all $x \in K$.Set $T_n$ to be defined like
		\[
			T_n := P_n T : K_n \to K_n.
		\]
		Here $T_n$ is continuous since $T$ and $P_n$ are continuous. $K_n$ is compact and convex in a finite dimensional space. Brouwer's fixed point theorem implies that $T_n$ has a fixed point in $K_n$,i.e. there exists $x_n \in K_n$ such that
		\[
			x_n = T_n(x_n)= P_n(x_n).
		\]
		But then
		\[
			\norm{x_n - T(x_n)} \leq  \underset{=0}{\underbrace{\norm{x_n - \underset{=T_n}{\underbrace{P_nT(x_n)}}}}} + \underset{< \frac{1}{n}}{\underbrace{\norm{ P_nT(x_n)- T(x_n)}}}.
		\]
		The first lemma above gives that $T$ has a fixed point in $K$.
	\end{beweis} 
\end{description}

%%% lecture 8a
%%% lecture 8a

\begin{beispiel}
	Assume $k(x,y)$ continuous on $[0,1] \times [0,1]$ and $h(y,z)$ continuous on $[0,1]\times \mathbb{R}$ and 
	\[
		\sup\limits_{(y,z) \in [0,1] \times \mathbb{R}} \abs{h(y,z)} \equiv B < \infty.
	\]
	Then there exists a solution $f \in C([0,1])$ to 
	\[
		f(x) = \int_{0}^{1} k(x,y)h(y,f(y)) \,\mathrm{d}y, \qquad x \in [0,1].
	\]
	Method: Set $f \in C([0,1])$ and
	\[
		T(f)(x) = \int_{0}^{1}k(x,y)h(y,f(y)) \,\mathrm{d}y, \qquad x \in [0,1] \qquad (*).
	\]
	We want to apply (a generalized version of) Schauder's fixed point theorem. Assume $(E, \norm{.})$ is a Banach space and $F$ closed convex subset of $E$. Moreover assume $T: E \to E$ continuos and $T(F)$ relatively compact in $(E,\norm{.})$. Then $T$ has a fixed point in $F$. \\
	\begin{description}
		\item[Step 1:] $T$ as in $(*)$. \\
		\textbf{Claim:} \text{    }     $T(C([0,1])) \subseteq C([0,1])$. \\
		To proof this we note that $k$ is continuous on $[0,1] \times [0,1]$ whicht is compact in $\mathbb{R}^2$. This implies that $k$
 is uniformly continuous on $[0,1]\times [0,1]$. Fix now $\varepsilon >0$. \\
 Then there exists $\delta = \delta (\varepsilon) >0$ such that
 \[
 	\abs{k(x_1,y_1)- k(x_2,y_2)} < \frac{\varepsilon}{B}
 \]
 for $\abs{(x_1,y_1)- (x_2,y_2)}< \delta $. \\
 Fix $f \in C([0,1])$
 	\begin{align*}
 		\abs{T(f)(x_1)-T(f)(x_2)} & = \abs{ \int_{0}^{1}(k(x_1,y)-k(x_2,y))h(y,f(y)) \,\mathrm{d}y} \\
		&\leq \int_{0}^{1}\underset{< \frac{\varepsilon}{B} \text{ if } \abs{x_1-x_2}< \delta }{\underbrace{\abs{k(x_1,y)-k(x_2,y)}}}\underset{\leq B}{\underbrace{\abs{h(y,f(y))}}} \,\mathrm{d}y < \varepsilon, \qquad \text{provided }\abs{x_1-x_2}< \delta
 	\end{align*}
	Conclusion: $T(f) \in C([0,1])$ for $f \in C([0,1])$
	\item[Step 2:] Choose $F$. \\
	$k$ is a continuous function on a compact set $[0,1] \times [0,1]$ implies
	\[
		\sup\limits_{(x,y) \in [0,1] \times [0,1]} \abs{k(x,y)} \equiv A < \infty.
	\]
	Hence 
	\[
		\abs{T(f)(x)} \leq  AB \qquad \text{ for all }f \in C([0,1]).
	\]
	Set 
	\[
		F:= \set[f \in C([0,1])]{\norm{f} = \max_{x \in [0,1]}\abs{f(x)} \leq AB}.
	\]
	Clearly $F$ is closed convex in $(C([0,1]),\norm{.})$ which is a Banach space.
	\item[Step 3:] \textbf{Claim:} \text{    }     $T(F)$ is relatively compact. \\
	To prove this we use the Arzela-Ascoli Theorem. \\
	$\phantom{...}$ \\
	Let $K$ be a compact set in $\mathbb{R}^n$. Let $\mathcal{S} \subset C(K)$ (realvalued continuous functions on $K$). \\
	Then $\mathcal{S}$ is relatively compact in $(C(K),\norm{.}_{\infty})$ if 
	\begin{enumerate}[(1)]
		\item $\mathcal{S}$ uniformly bounded, i.e.
		\[
			\sup_{f \in \mathcal{S}} \norm{f} < \infty.
		\]
		\item Equicontinuity of $f \in \mathcal{S}$, i.e.
		\begin{align*}
			\forall\, \varepsilon>0 \,\exists\, \delta = \delta (\varepsilon) >0: \,\forall\,  f \in \mathcal{S}: & \\
			\abs{x_1-x_2} < \delta, \, x_1,x_2 \in K \qquad \Rightarrow & \qquad \abs{f(x_2)-f(x_1)}< \varepsilon.
		\end{align*}
	\end{enumerate}
	In our example it is $\mathcal{S} = F$, $K = [0,1]$ in $\mathbb{R}$. Check that (1) and (2) in AA-Theorem are satisfied. \\
	\begin{enumerate}[(1)]
		\item $F$ is uniformly bounded since
	\[
		\sup_{f \in F}\norm{f} \leq AB < \infty.
	\]
	\item Equicontinuity follows from calculations in Step 1. \\
	\end{enumerate}	
	Conclusion: $T(F)$ is relatively compact.
	\item[Step 4:] \textbf{Claim:} \text{    }     $T: F \to F$ continuous \\
	In step 1 we had $f \in F$ and $x_n \to x$ in $[0,1]$. We have shown that $T(f)(x_n) \to T(f)(x)$ in $\mathbb{R}$. So $T(f)$ is a continuous function. \\
	Now we want to show that for $f_n \to f$ in $F$ we've got $T(f_n) \to T(f)$ in $C([0,1])$. \\
	Note that $h: [0,1] \times [-AB,AB] \to \mathbb{R}$ is continuous and $[0,1] \times [-AB,AB]$ is compact set in $\mathbb{R}^2$. So $h: [0,1] \times [-AB,AB] \to \mathbb{R}$ is uniformly continuous. \\
	Fix $\varepsilon >0$. Then there exists a $\delta = \delta (\varepsilon) >0$ such that
	\[
		\abs{h(y_1,z_1)-h(y_2,z_2)} < \frac{\varepsilon}{A}
	\] 
	for $\abs{(y_1,z_1)-(y_2,z_2)} < \delta $. For $f_1,f_2 \in F$ with
	\[
		\norm{f_1-f_2} < \delta.
	\]
	We have 
	\begin{align*}
		\abs{T(f_1)(x)-T(f_2)(x)} &= \abs{\int_{0}^{1}k(x,y)(h(y,f_1(y))-h(y,f_2(y))) \,\mathrm{d}y} \\
		&\leq  \int_{0}^{1}\underset{\leq A}{\underbrace{\abs{k(x,y)}}}\underset{< \frac{\varepsilon}{A}}{\underbrace{\abs{h(y,f_1(y))-h(y,f_2(y))}}} \,\mathrm{d}y < \varepsilon.
	\end{align*}
	Conclusion: $T: F \to F$ is continuous. 
	\item[Step 5:] Apply Schauder's fixed point theorem.
	\end{description}
\end{beispiel}
\subsection{Completion of normed spaces} 
\label{sub:completion_of_normed_spaces}
$(E,\norm{.})$ normed spaces. We say that $(\tilde E, \norm{.}_*)$ is a completion of $(E,\norm{.})$ if $(\tilde E, \norm{.}_*)$ is a normed space such that
\begin{enumerate}[(1)]
	\item $\exists\, \Phi: E \to \tilde E$ injective and linear.
	\item $\norm{x} = \norm{\Phi(x)}_*$ for all $x \in E$.
	\item $\Phi(E)$ is dense in $\tilde E$.
	\item $(\tilde E, \norm{.}_*)$ is a Banach space.
\end{enumerate}
\minisec{Construction:}
Let $(x_n)_{n=1}^{\infty}$ and $(y_n)_{n=1}^{\infty}$ be Cauchy sequences in $(E,\norm{.})$. We say that $(x_n)_{n=1}^{\infty}$ and $(y_n)_{n=1}^{\infty}$ are equivalent, denoted by $(x_n) \sim (y_n)$, if 
\[
	\norm{x_n-y_n} \to 0, \qquad n \to \infty.
\]
Set \[
	\tilde E= \set[ \left((x_n) \right)_N]{(x_n)_{n=1}^{\infty} \text{ Cauchy sequence in } (E,\norm{.})}.
\]
Vector space structure:
\[
	\begin{cases}
		[(x_n)]_N + [(\tilde x_n)]_N &= [(x_n + \tilde x_n)]_N \\
		\lambda [(x_n)]_N &= [(\lambda x)_n]_N.
	\end{cases}
\]
Show that these definitions are well-defined, i.e. independent of the choice of representative norm
\[
	\norm{ [(x_n)]_N}_* = \lim_{n \to \infty} \norm{x_n}.
\]
Note \[
	(x_n) \sim (y_n)
\]
implies
\[
	\lim_{n \to \infty}\norm{x_n} = \lim_{n \to \infty} \norm{y_n}.
\]
Since
\[
	\abs{\norm{x_n}- \norm{y_n}} \leq \norm{x_n-y_n} \to 0, \qquad n \to \infty
\]
Check that the axioms for being a norm are satisfied. \\
Now we have $(\tilde E,\norm{.}_*)$ is a normed space. \\
Define $\Phi$: For $x \in E$ set $\Phi(x) = \left[ (x)_{n=1}^{\infty} \right]_N$ where 
\[
	(x)_{n=1}^{\infty} = (x,x,x, \dots).
\]
\begin{description}
\item[Claim 1 \& 2:] easy to prove. 
\item[Claim 3:] $\Phi(E)$ dense in $(\tilde E,\norm{.}_*)$. Fix $\left[ (x_n) \right]_N \in \tilde E$. Consider $\Phi(x_k)$ where $x_k$ is the element in the $k$-th position in the sequence $(x_1,x_2, \dots,x_n, \dots)$.
\begin{align*}
	\norm{\left[ (x_n) \right]_N - \Phi(x_k)}_* = \lim_{n \to \infty}\norm{x_n - x_k} \to 0 \qquad k \to \infty.
\end{align*}
Since $(x_n)_{n=1}^{\infty}$ is a Cauchy sequence. \\
\item[Claim 4:] $(\tilde E, \norm{.}_*)$ is a Banach space.\\
Consider a Cauchy sequence $z_n \in \tilde E$ such that $\norm{z_n - z} \to 0$ as $n \to \infty$. \\
To show: There exists $z \in \tilde E$ such that 
\[
	\norm{z_n - z} \to 0, \qquad n \to \infty.
\]
By 3 we have that $\Phi(E)$ is dense in $ \tilde E$ so for $n=1,2,\dots$ there exists $x_n \in E$, $n=1,2,\dots$ such that
\[
	\norm{z_n - \Phi(z_n)} < \frac{1}{n}, \qquad  n=1,2,\dots.
\]
Set $z=: \left[ (x_n) \right]_N$. \\
Need to show that $(x_n)_{n=1}^{\infty}$ is a Cauchy sequence
\begin{align*}
	\norm{x_n - x_m} &= \norm{\Phi(x_n)-\Phi(x_m)}_* \\
	& \leq  \norm{\Phi(x_n)- z_n}_* + \norm{z_n-z_m}_* + \norm{z_m - \Phi(x_m)}_* \\
	&< \frac{1}{n} + \norm{z_n-z_m} + \frac{1}{m} \to 0, \qquad n,m \to \infty.
\end{align*}
Conclusion: $(x_n)_{n=1}^{\infty}$ is a Cauchy sequence in $(E, \norm{.})$. Remains to show:
\[
	\norm{z_n-z}_* \to 0, \qquad n \to \infty
\]
\[
	\norm{z_n - z}_* \leq \underset{< \frac{1}{n}}{\underbrace{\norm{z_n - \Phi(x_n)}_*}} + \underset{= \lim_{n \to \infty}\norm{x_n-x_m}}{\underbrace{\norm{\Phi(x_n)-z}_*}} \to 0, \qquad n \to \infty.
\]
\end{description}

Consider $ f \in C([0,1])$
\begin{itemize}
	\item max-norm: $\norm{f} = \max_{x \in [0,1]}\abs{f(x)}$. Then $(C([0,1]),\norm{.})$ is a Banach space.
	\item $p \geq 1:$
	\[
		\norm{f}_{L^p} = \left( \int_{0}^{1}\abs{f(x)}^p \,\mathrm{d}x \right)^{\frac{1}{p}} 
	\]
	defines a norm for $C([0,1])$.
\end{itemize}
\begin{bemerkung}
	\begin{itemize}
		\item Consider piecewise linear $f_n \in C([0,1])$ for $n =1,2, \dots$
		\[
			f_n(x) = \begin{cases}
				1, &\text{ if } \frac{1}{2} \leq x \leq 1 \\
				0, &\text{ if } x \leq \frac{1}{2} - \frac{1}{2n}
			\end{cases}
		\]
		with
		\[
			\norm{f_n-f_m}_{L^1} \leq \frac{1}{2} \frac{1}{\min(m,n)} \to 0, \qquad n,m \to \infty.
		\]
		So $(f_n)_{n=1}^{\infty}$ is a Cauchy sequence in $(C([0,1]),\norm{.}_{L^1})$ but $(f_n)_{n=1}^{\infty}$ does not converge in $(C([0,1]),\norm{.}_{L^1})$ since
		if $\norm{f_n - f}_{L^1} \to 0$ as $n \to \infty$ and $f \in C([0,1])$ then
		\[
			f(x) = \begin{cases}
				0, &\text{ if }x \in [0,\frac{1}{2})\\
				1, &\text{ if }x \in [\frac{1}{2},1]
			\end{cases}.
		\]
		Conclusion: $(C([0,1]), \norm{.}_{L^1})$ is not a Banach space.
		\item Consider:
		\[
			f(x) = \begin{cases}
				1, &\text{ if }x = \frac{1}{2}\\
				0, &\text{ if }x \in [0,1] \setminus \set{\frac{1}{2}}
			\end{cases}.
		\]
		Then
		\[
			\norm{f}_{L^1} = 0 = \norm{0}_{L^1}.
		\]
		Compare this with the first axiom for a norm function.
		\item Replace $[0,1]$ with $\mathbb{R}$. For $f : \mathbb{R} \to \mathbb{R}$ set \[
			\supp(f) = \set[x \in \mathbb{R}]{f(x) \neq 0}.
		\]
		Set 
		\[
			C_0(\mathbb{R}) = \set[f \in C(\mathbb{R})]{ \supp(f) \text{ is compact in }\mathbb{R}}.
		\]
		\textbf{Claim:} \text{    }      $C_0(\mathbb{R})$ forms a vector space and for every $p \geq 1$ and $f \in C_0(\mathbb{R})$
		\[
			\norm{f}_{L^p} = \left( \int_{\mathbb{R}}^{} \abs{f(x)}^p \,\mathrm{d}x \right)^{\frac{1}{p}}
		\] defines a norm on $C_0(\mathbb{R})$. \\
		Problem: $(C_0(\mathbb{R}), \norm{.}_{L^p})$ for $p \geq 1$ are not Banach spaces. \\
		$(L^1(\mathbb{R}),\norm{.}_{L^1})$ is a completion of $(C_0(\mathbb{R}),\norm{.}_{L^1})$. \\
		Note $A \subset \mathbb{R}$ and $A$ bounded. Define
		\[
			f_A(x) \begin{cases}
				1, & x \in A\\
				0, &\text{elsewhere}
			\end{cases}.
		\]
		Lebesguesmeasure of $A = \norm{f_A}_{L^1} = \mu(f_A)$. $A \subset \mathbb{R}$ and $A$ unbounded
		\[
			\mu(A) = \lim_{n \to \infty} \mu ( A \cap [-n,n]).
		\]
		We say that $A \subset \mathbb{R}$ is a $0$- set if for all $\varepsilon >0$ there exist open intervals $I_n$, $n=1,2, \dots$ such that
		\begin{enumerate}[(1)]
			\item $ A \subseteq \bigcup_{n=1}^{\infty}I_n$,
			\item $\sum_{n=1}^{\infty}$ lenghts of $I_m < \varepsilon$.
		\end{enumerate} 
		In particular
		\[
			A = \mathbb{Q} = \set[r_n]{n=1,2,\dots}\qquad \text{is a $0$-set}.	
		\]
	\end{itemize}
\end{bemerkung}

%%% lecture 7
%%lecture 7

%% 20.9
\newpage
\section{Hilbert spaces} 
\label{sec:hilbert_spaces}
\begin{beispiel}
	Consider $\mathbb{C}^n = \set[(x_1,x_2,\dots,x_n)]{x_i \in \mathbb{C}}$ and $x,y \in \mathbb{C}^n$ with
	$x= (x_1,\dots,x_n)$, $y = (y_1,\dots,y_n)$. Define the inner product of $x,y$ (scalar product)
	\[
		\skal{x}{y} = \sum^{n}_{i=1}x_i \bar{y}_i \in \mathbb{C}
	\]
	We have a map
	\begin{align*}
		\mathbb{C}^n \times \mathbb{C}^n &\to \mathbb{C} \\
		(x,y) &\mapsto \skal{x}{y}
	\end{align*}
	This mapping has properties:
	\begin{itemize}
		\item $x \neq 0$ folgt $\skal{x}{x} = \sum^{n}_{i=1}x_i \bar{x}_i = \sum^{n}_{i=1} \abs{x_i}^2 >0$
		\item $\skal{\lambda x}{y} = \lambda \skal{x}{y}$ for $x,y \in \mathbb{C}^n$, $\lambda \in \mathbb{C}$.
		\item $\skal{x}{y} = \sum^{n}_{i=1} x_i \bar{y}_i = \overline{\sum^{n}_{i=1}y_i \bar{x}_i}$ for $x,y \in \mathbb{C}^n$. \\
		In particular $\skal{x}{\lambda y} = \bar{\lambda} \skal{x}{y}$ for $\lambda \in \mathbb{C}$.
		\item $\skal{x+y}{z} = \skal{x}{z}+ \skal{y}{z}$ for $x,y,z \in \mathbb{C}^n$. 
	\end{itemize}
\end{beispiel}
\begin{definition*}
	An inner product space $V$ is a complex vector space with an inner product which is a map 
	\[
		\skal{.}{.}: V \times V \to \mathbb{C}
	\]
	satisfying
	\begin{itemize}
		\item $\skal{\lambda x}{y} = \lambda \skal{x}{y}$ for any $x,y \in V$, $\lambda \in \mathbb{C}$
		\item $\skal{x+y}{z} = \skal{x}{z}+ \skal{y}{z}$ for any $x,y,z \in V$
		\item $\skal{x}{y} = \overline{\skal{x}{y}}$ for any $x,y \in V$
		\item $\skal{x}{x}>0$ for any $x \in V, x \neq 0$
	\end{itemize}
\end{definition*}
Can we generalize $\mathbb{C}^n$? \\
\[
	\mathbb{C}^{\mathbb{N}} \set[(x_1,x_2, \dots)]{x_i \in \mathbb{C}}
\]
with
\[
	\skal{x}{y} = \sum^{\infty}_{i=1} x_i \bar{y}_i 
\]
This is not necessarily convergent.
\begin{beispiele}
	\begin{enumerate}[(1)]
		\item 	\[
		l^2 = \set[(x_1,x_2, \dots)]{ \sum^{\infty}_{i=1} \abs{x_i}^2 < \infty}.
	\]
	We have with Cauchy Schwarz
	\[
		\sum^{n}_{i=1} \abs{x_i \bar{y}_i} \leq \left( \sum^{n}_{i=1} \abs{x_i}^2 \right)^{\frac{1}{2}} \left( \sum^{n}_{i=1} \abs{y_i}^2 \right)^{\frac{1}{2}}
	\]
	if $x \in l^2$ and $y \in l^2$ we get
	\begin{align*}
		\sum^{n}_{i=1}\abs{x_i \bar{y}_i} \leq \left( \sum^{\infty}_{i=1} \abs{x_i}^2 \right)^{\frac{1}{2}} \left( \sum_{i=1}^{\infty} \abs{y_i}^2 \right)^{\frac{1}{2}} < \infty.
	\end{align*}
	It follows that $\sum_{i=1}^{\infty} x_i \bar{y}_i$ converges absolutely and hence it is convergent. The following 
	\[
		\skal{x}{y} = \sum^{\infty}_{i=1} x_i \bar{y}_i
	\]
	is well-defined for vectors $x,y \in l^2$. Like for $\mathbb{C}^n$ one can easily check that $\skal{.}{.}$ satisfies the axioms for inner products. \\
	$(l^2, \skal{.}{.})$ is an inner product space.
	\item Consider $C([0,1])$ with the inner product
	\[
		\skal{f}{g} = \int_{0}^{1}f(t) \overline{g(t)} \,\mathrm{d}t \qquad \forall\, f,g \in C([0,1])
	\]
	\begin{itemize}
		\item 	\[
			\skal{\lambda f}{g} = \int_{0}^{1}\lambda f(t) \overline{g(t)} \,\mathrm{d}t = \lambda \int_{0}^{1}f(t) \overline{g(t)} \,\mathrm{d}t = \lambda \skal{f}{g}
		\]
		\item \[
			\skal{f}{f} = \int_{0}^{1}f(t) \overline{f(t)} \,\mathrm{d}t = \int_{0}^{1} \abs{f(t)}^2 \,\mathrm{d}t >0
		\]	
		\item $\dots$
	\end{itemize}
	\end{enumerate}
\end{beispiele}
If we take $\mathbb{R}^3$ with the Eukledian norm on $\mathbb{R}^3$
\[
	\norm{(x_1,x_2,x_3)} = \sqrt{x_1^2 + x_2^2 + x_3^2} = \left( \sum_{i=1}^{3} \abs{x_i}^2 \right)^{\frac{1}{2}} = \skal{x}{x}^{\frac{1}{2}}
\]
Let $V$ be an inner product space with $\skal{.}{.}$ as the inner product. Let for $x \in V$
\[
	\norm{x} := \skal{x}{x}^{\frac{1}{2}}
\]
\begin{satz}
	The $x \mapsto  \norm{x}$ with $\norm{.}$ defined above is a norm.
\end{satz}
We are going to prove the norm axioms but first we need another theorem
	\begin{theorem}[Cauchy-Schwarz inequalitiy]
		For any $x,y \in V$ (inner product space) 
		\[
			\abs{\skal{x}{y}} \leq \skal{x}{x}^{\frac{1}{2}} \skal{y}{y}^{\frac{1}{2}}
		\]
		The equality holds iff $x,y$ are linearly dependent.
	\end{theorem}
	\begin{beweis}
		Assume $x,y$ linearly dependent. We can assume that $x= \lambda y$ for some $\lambda \in \mathbb{C} $.
		\[
			\abs{\skal{x}{y}} = \abs{ \skal{\lambda y}{y}} = \abs{\lambda} \skal{y}{y}
		\]
		and
		\begin{align*}
					\skal{x}{x}^{\frac{1}{2}} \skal{y}{y}^{\frac{1}{2}} &= \skal{\lambda y}{\lambda y}^{\frac{1}{2}} \skal{y}{y}^{\frac{1}{2}} \\
					&= \abs{\lambda} \skal{y}{y}^{\frac{1}{2}} \skal{y}{y}^{\frac{1}{2}} \\
					&= \abs{\lambda} \skal{y}{y}
		\end{align*}
		Hence \[
			\abs{\skal{x}{y}} = \skal{x}{x}^{\frac{1}{2}} \skal{y}{y}^{\frac{1}{2}}.
		\]
		Assume $x,y$ are linearly independent. Hence $x + \lambda y \neq 0$ for any $\lambda \in \mathbb{C}$. By an axiom for inner product we get
		\[
			0< \skal{x+ \lambda y}{x + \lambda y} = \skal{x}{x} + \lambda \skal{y}{x} + \bar{\lambda} \skal{x}{y} + \abs{\lambda}^2 \skal{y}{y}
		\]
		Pick now
		\[
			\lambda = - \frac{\skal{x}{y}}{\skal{y}{y}}
		\]
		(Note that $y \neq 0$ as $x,y$ linearly independent.)
		We have \begin{align*}
						0 &< \skal{x}{x} - \frac{\overset{= \abs{\skal{x}{y}}^2}{\overbrace{\skal{x}{y}\skal{y}{x}}}}{\skal{y}{y}} - \frac{\overset{= \abs{\skal{x}{y}}^2}{\overbrace{\overline{\skal{x}{y}}\skal{x}{y}}}}{\skal{y}{y}}+ \frac{\abs{\skal{x}{y}}^2}{\skal{y}{y}^2} \skal{y}{y} \\
						&= \skal{x}{x} - \frac{\abs{\skal{x}{y}}^2}{\skal{y}{y}}
		\end{align*}
		This gives
		\[
			\frac{\abs{\skal{x}{y}}^2}{\skal{y}{y}} < \skal{x}{x}
		\]
		and it follows
		\[
			\abs{\skal{x}{y}}^2 < \skal{x}{x} \skal{y}{y}
		\]
	\end{beweis}
Now we can use this inequality to proof the statement above:
\begin{beweis}
	\begin{enumerate}[(i)]
		\item $\norm{x} >0$ for all $x \neq 0$ in $V$ (Exercise)
		\item $\norm{\lambda x} = \abs{\lambda} \norm{x}$ for all $x \in V$, $\lambda \in \mathbb{C}$ (Exercise)
		\item Let $x,y \in V$. Then 
		\begin{align*}
			\norm{x+y}^2 &= \skal{x+y}{x+y} \\ &= \skal{x}{x}+ \skal{x}{y}+ \skal{y}{x} + \skal{y}{y} \\
			&= \skal{x}{x} + 2 \text{Re}( \skal{x}{y}) + \skal{y}{y} \\
			&\leq  \skal{x}{x}+ 2 \abs{\skal{x}{y}} + \skal{y}{y} \\
			&\leq  \skal{x}{x} + 2 \skal{x}{x}^{\frac{1}{2}}\skal{y}{y}^{\frac{1}{2}} + \skal{y}{y} \\
			&= \left( \skal{x}{x}^{\frac{1}{2}} + \skal{y}{y}^{\frac{1}{2}} \right)^2 
		\end{align*} 
		So
		\[
			\norm{x+y}^2 \leq \left( \norm{x} + \norm{y} \right)^2
		\]
	\end{enumerate}
\end{beweis}
\begin{theorem}[The Parallelogram Law]
	Let $(V, \skal{.}{.})$ be an inner product space. Let $\norm{x} = \skal{x}{x}^{\frac{1}{2}}$. Then
	\[
		\norm{x+y}^2 + \norm{x-y}^2 = 2 (\norm{x}^2 + \norm{y}^2) \qquad \forall\, x,y \in V.
	\]
\end{theorem}
\begin{satz}
	$l^p$ has inner product $\skal{.}{.}_{l^p}$ such that
	\[
		\norm{x}_p = \sqrt{\skal{x}{x}_{l^p}}
	\]
	iff $p =2$.
\end{satz}
\begin{beweis}
	Enough to show that $\norm{.}_p$-norm does not satisfy the parallelogram law for some $x,y \in l^p$ if $p \neq 2$. Take for example $x = (1,0,0, \dots)$
	and $y= (0,1,0, \dots)$. Note that $\norm{x}_{l^p} = \norm{y}_{l^p} = 1$
	\begin{align*}
		\norm{x+y}^2_{l^p} &= \norm{(1,1,0, \dots)}_{l^p} = 2^{\frac{2}{p}} \\
		\norm{x-y}^2_{l^p} &= \norm{(1,-1,0,\dots)}_{l^p} = 2^{\frac{2}{p}} \\
		\norm{x+y}^2_{l^p} + \norm{x-y}_{l^p}^2 &= 2 \cdot 2^{\frac{2}{p}} = 2( \norm{x}^2_{l^p}+ \norm{y}^2_{l^p}) = 2 \cdot 2
	\end{align*}
\end{beweis}
All $l^p$ with $p \neq 2$ are not inner product spaces. 
\minisec{Exercise:}Show that $(C([0,1]),\norm{.}_{\infty})$ is not an inner product space.
\begin{bemerkung}
	Whenever a norm satisfies the parallelogram law then there exists an inner product on $V$ such that
	\[
		\norm{x} = \skal{x}{x}^{\frac{1}{2}}
	\]
\end{bemerkung}
\begin{theorem}[The Polarization Identity]
	Let $(V,\skal{.}{.})$ be an inner product space. Then 
	\[
		4 \skal{x}{y}  = \norm{x+y}^2- \norm{x-y}^2 + i \norm{x+ iy}^2 - i \norm{x - iy}^2
	\]
\end{theorem}
\begin{definition}
	Let $(V, \skal{.}{.})$ be an inner product space. We say that $x,y$ in $V$ are othogonal if $\skal{x}{y} = 0$ (We write $x \perp y$). Let $M \subseteq V$
	Define the orthogonal complement
	\[
		M^{\perp} = \set[x \in V]{x \perp y \text{ for any }y \in M}
	\]
\end{definition}
\begin{proposition}
	If $M \subseteq V$ then $M^{\perp}$ is a subspace of $V$
\end{proposition}
\begin{theorem}[Pythagorean formula]
	$x,y \in V$ (inner product space). Then
	\[
		x \perp y \qquad \text{iff} \qquad \norm{x+y}^2 = \norm{x}^2 + \norm{y}^2.
	\]
\end{theorem}

\subsection{Orthogonal Systems} 
\label{sub:orthogonal_systems}

Let $(V, \skal{.}{.})$ be an inner product space $\set{u_n} \subseteq V$ is called orthogonal system (with $n$ finite or infinite) if $u_n \perp u_m$ for all $n \neq m$. It is an orthonormal system if in addition $\norm{u_n}=1$. 

\begin{beispiele}
	\begin{enumerate}[1)]
		\item $\set{e_k}_{k=1}^{\infty} \subseteq l^2$ with 
		\[
			\skal{x}{y} = \sum^{\infty}_{i=1} x_i \bar{y}_i
		\]
		with 
		\[
			e_k = (0,\dots,1,0,\dots)
		\]
		$\Rightarrow$ $\set{e_k}$ is an ON-system. 
		\item $C([-\pi,\pi])$ with
		\[
			\skal{f}{g} = \int_{- \pi}^{\pi} f(t) \overline{g(t)} \,\mathrm{d}t.
		\]
		\[
			\set[\frac{1}{\sqrt{2 \pi}} e^{-int}]{n \in \mathbb{Z}}
		\]
		is an orthonormal system.
	\end{enumerate}
\end{beispiele}

\begin{definition}
	Let $\set[a_n]{n \in \mathbb{N}}$ be an orthonormal system in $V$. The formal series 
	\[
		\sum_{n=1}^{\infty} \skal{x}{a_n}a_n
	\]
	is called a fourier series of $x$ corresponding $\set[a_n]{n \in \mathbb{N}}$ and $\skal{x}{a_n}$ are called fourier coefficients of $x$ corresponding to $\set[a_n]{n \in \mathbb{N}}$. 
\end{definition}

\begin{theorem}[Bessel's Equality and Inequality]
	If $\set{u_n}$ orthonormal system in an inner product space $V$, then for all $x \in V$
	\[
		\norm{x- \sum_{k=1}^{n} \skal{x}{a_k}a_k}^2 = \norm{x}^2 - \sum_{k=1}^{n} \abs{\skal{x}{a_k}}^2
	\]
	and 
	\[
		\sum_{k=1}^{\infty} \abs{\skal{x}{a_k}}^2 \leq \norm{x}^2
	\]
\end{theorem}
\begin{beweis}
	\begin{align*}
		\norm{x- \sum_{k=1}^{n} \skal{x}{a_k}a_k}^2 &= \skal{x- \sum_{k=1}^{n} \skal{x}{a_k}a_k }{x - \sum_{k=1}^{n} \skal{x}{a_k}a_k} \\
		&= \skal{x}{x} - \sum_{k=1}^{n} \overline{\skal{x}{a_k}} \skal{x}{a_k} - \sum_{k=1}^{n} \skal{x}{a_k}\skal{a_k}{x} 
		\\ & \qquad \qquad + \skal{\sum_{k=1}^{n}\skal{x}{a_k}a_k}{\sum_{k=1}^{n}\skal{x}{a_k}a_k} \\
		& = \norm{x}^2 - \sum_{k=1}^{n} \abs{\skal{x}{a_k}}^2 - \sum_{k=1}^{n} \abs{\skal{x}{a_k}}^2 + \sum_{k=1}^{n}\abs{\skal{x}{a_k}}^2  \\
		& = \norm{x}^2 - \sum_{k=1}^{n} \abs{\skal{x}{a_k}}^2
	\end{align*}
	This gives also:
	\[
		\sum_{k=1}^{n}\abs{\skal{x}{a_k}}^2 = \norm{x}^2 - \norm{x - \sum_{k=1}^{n} \skal{x}{a_k}a_k} \leq \norm{x}^2 
	\]
	for all $n \in \mathbb{N}$. Hence
	\[
		\sum_{k=1}^{\infty} \abs{\skal{x}{a_k}}^2 \leq \norm{x}^2
	\]
\end{beweis}

\begin{definition}[Hilbert space]
	A Hilbert space is an inner product space which is complete w.r.t. the norm is defined through the inner product.
\end{definition}

\begin{beispiele}
	\begin{itemize}
		\item $\mathbb{C}^n$ is an inner product space and complete w.r.t the Eukledean norm. Hence $\mathbb{C}^n$ is a Hilbert space.
		\item $l^2$ is a Banach space w.r.t. 
		\[
			\norm{x}_{l^2} = \left( \sum_{i=1}^{\infty} \abs{x_i}^2 \right)^{\frac{1}{2}}
		\]
		and
		\[
			\norm{x}_{l^2} = \skal{x}{x}^{\frac{1}{2}}
		\]
		where
		\[
			\skal{x}{y} = \sum_{i=1}^{\infty} x_i \bar{y}_i
		\]
		\item $(C([0,1]), \norm{.}_{\infty})$ is a Banach space but not an inner product space. Hence it is no Hilbert space.
		\item $(C([0,1]),\skal{.}{.})$ is an inner product space $f,g \in C([0,1])$ with
		\[
			\skal{f}{g} = \int_{0}^{1} f(t) \overline{g(t)} \,\mathrm{d}t
		\]
		and the corresponding
		\[
			\norm{f}_2 = \skal{f}{f} = \int_{0}^{1} \abs{f(t)}^2 \,\mathrm{d}t.
		\]
	\end{itemize}
\end{beispiele}
\begin{bemerkung}
	Other $l^p$ spaces are not Hilbert spaces!!!! They are not inner product spaces.
\end{bemerkung}

\begin{satz}
	$(C([0,1]), \skal{.}{.})$ is not a Hilbert space since $(C([0,1]), \norm{.}_2)$ is not complete.
\end{satz}

\begin{beweis}
	Sketch: Show that $f_n(t)$, which is defined as a piecewise continuous function for example
	\[
		f_n(x)= \begin{cases}
			1, &\text{ if }x \in [0,\frac{1}{2}]\\
			0, &\text{ if }x \in [\frac{1}{2} + \frac{1}{n}] \\
			\text{continuous}, & \text{else} 
		\end{cases}
	\] is a Cauchy sequence w.r.t $\norm{.}_2$ but has no limit in $C([0,1])$.
\end{beweis}

Consider
\[
	C_F = \set[(x_1,x_2,\dots)]{\text{only finite }x_i \neq 0} 
\]
with \[
	\skal{x}{y} = \sum_{i=1}^{\infty} x_i \bar{y}_i
\]
Show that $(C_F, \skal{.}{.})$ is not a Hilbert space.

\begin{definition}[strongly and weakly convergent]
	A sequence $\set{x_n} \subseteq H$, where $H$ is a Hilbert space, is called strongly convergent $(x_n \to x \in H)$ if 
	\[
		\norm{x_n -x} \to 0, \qquad n \to  \infty.
	\]
	(Norm induced by an inner product) \\ We say that $x_n$ is weakly convergent ($x_n \rightharpoonup x$) if
	\[
		\skal{x_n}{y} \to \skal{x}{y}, \qquad \forall\, y \in H.
	\]
\end{definition}

\begin{satz}
	$x_n \to x$$ \qquad \Rightarrow \qquad $ $x_n \rightharpoonup x$.
\end{satz}

\begin{beweis}
	Assume strong convergence for $(x_n)_{n \in \mathbb{N}}$. Then
	\begin{align*}
		\abs{\skal{x_n}{y}- \skal{x}{y}} &= \abs{\skal{x_n-x}{y}} \\
		&\leq \underset{=\norm{x_n-x}}{\underbrace{\skal{x_n-x}{x_n-x}^{\frac{1}{2}}}} \underset{= \norm{y}}{\underbrace{\skal{y}{y}^{\frac{1}{2}}}} \\
		&= \underset{ \to 0}{\underbrace{x_n-x}} \norm{y} \to 0, \qquad n \to \infty
	\end{align*}
	Hence $\skal{x_n}{y} \to \skal{x}{y}$.
\end{beweis}

\begin{bemerkung}
	The converse is not true in general: \\
	Take $H=l^2$ and 
	\begin{align*}
		x_n &= e_n = (0, \dots,1,0,\dots) \\
		y &= (y_1,y_2,\dots) \in l^2 
	\end{align*}
	We have for all $y \in H$
	\[
		\skal{e_n}{y} = y_n \to 0, \qquad n \to \infty
	\]
	as
	\[
		\norm{e_n -0}_{l^2} = \norm{e_n}_{l^2} =1.
	\]
\end{bemerkung}

\begin{satz}
	$x_n \to x$ and $y_n \to y$ yields
	\[
		\skal{x_n}{y_n} \to \skal{x}{y}.
	\]
	In particular
	\[
		x_n \to x \qquad \Rightarrow \qquad \norm{x_n} \to \norm{x}.
	\]
\end{satz}
\begin{beweis}
	\begin{align*}
		\abs{\skal{x_n}{y_n}-\skal{x}{y}} &= \abs{\skal{x_n}{y_n}-\skal{x}{y_n} + \skal{x}{y_n} - \skal{x}{y}} \\
		&= \abs{\skal{x_n -x}{y_n} + \skal{x}{y_n-y}} \\
		&\leq \abs{\skal{x_n-x}{y}} + \abs{\skal{x}{y_n-y}} \\
		&\leq \underset{\to 0}{\underbrace{\norm{x_n-x}}} \underset{< \infty}{\underbrace{\norm{y_n}}} + \underset{< \infty}{\underbrace{\norm{x}}} \underset{\to 0}{\underbrace{\norm{y_n-y}}} \to 0, \qquad  n \to \infty
	\end{align*}
	Check $\set{\norm{y_n}}$ is bounded
	\[
		\norm{y_n} = \norm{y_n -y +y} \leq \underset{\norm{y_n -y}}{\underbrace{\to 0}}+ \underset{< \infty}{\underbrace{\norm{y}}} \to 0, \qquad n \to \infty
	\]
\end{beweis}

\begin{satz}
	$x_n \rightharpoonup x$ and $\norm{x_n} \to \norm{x}$ yields
	\[
		x_n \to x.
	\]
\end{satz}
\begin{beweis}
	\begin{align*}
		\norm{x_n-x}^2 &= \skal{x_n -x}{x_n-x} \\
		&= \underset{= \norm{x_n}^2}{\underbrace{\skal{x_n}{x_n}}} - \skal{x}{x_n}- \skal{x_n}{x} + \skal{x}{x} \\
		&= \norm{x_n}^2 - \overline{\skal{x_n}{x}} - \skal{x_n}{x} + \norm{x}^2 \\
		&\to \norm{x}^2 - \norm{x}^2 - \norm{x}^2 + \norm{x}^2 = 0
	\end{align*}
\end{beweis}

We have proved 
\[
	x_n \to x \qquad \Rightarrow \qquad \set{\norm{x_n}} \text{ is bounded}
\]

\begin{theorem}
	\[
		x_n \rightharpoonup x \qquad \Rightarrow \qquad \sup_{n \in \mathbb{N}}\norm{x_n} < \infty
	\]
\end{theorem}
\begin{beweis}
	Let $x_n \rightharpoonup x$. Consider $f_n: H \to \mathbb{C}$ where
	\[
		f_n(y) = \skal{y}{x_n}, \qquad y \in H.
	\]
	\begin{itemize}
		\item $f_n$ is a linear functional for every $n \in \mathbb{N}$.
		\item $\forall\, n \in \mathbb{N}$ $f_n$ is a bounded ($\Leftrightarrow$ continuous) linear functional as if 
		\[
			y_k \stackrel{k \to \infty}{\to }y \qquad \Rightarrow \qquad f_n(y_k) = \skal{y_k}{x_n} \to \skal{y}{x_n} = f_n(y), \qquad k \to \infty
		\]
		\item $f_n(y) \to \skal{y}{x}$. \\
		$\set{f_n(y)}_n$ is a convergent sequence in $\mathbb{C}$ and hence bounded for all $y \in H$. \\
		Hence it exists $M_y$ such that 
		\[
			\abs{f_n(y)} \leq M_y
		\]
		By Banach-Steinhaus-Theorem it holds
		\[
			\norm{f_n} \leq M \text{ for some }M >0.
		\]
		We are done if we proof that $\norm{f_n} = \norm{x_n}$.
		\[
			\abs{f_n(y)}= \abs{\skal{y}{x_n}} \leq \norm{y} \norm{x_n}, \qquad \forall\, y \in H
		\]
		Hence 
		\[
			\norm{f_n} \leq \norm{x_n} \qquad \qquad (1)
		\]
		On the other Hand we have
		\[
			f_n(x_n) = \skal{x_n}{x_n}= \norm{x_n}^2
		\]
		and thus
		\[
			\norm{f_n} = \sup_{x \in H} \frac{\abs{f_n(x)}}{\norm{x}} \geq \frac{\abs{f_n(x_n)}}{\norm{x_n}} = \norm{x_n} \qquad \qquad (2)
		\]
		With (1) and (2) we are finished.
	\end{itemize}
\end{beweis}

\subsection{Orthogonal decomposition in Hilbert spaces} 
\label{sub:orthogonal_decomposition_in_hilber_spaces}

Remember Linear Algebra. Take $\mathbb{R}^n$ and a subspace $M \subseteq \mathbb{R}^n$ 
\[
	\Rightarrow \qquad \forall\, x \in \mathbb{R}^n \qquad x = z + y, \qquad \text{where }z \in M, y \in M^{\perp}
\]
This can be done in a unique way
\begin{align*}
	M &= \spn \set{e_z} \\ 
	M^{\perp} &= \spn\set{e_y}
\end{align*}
and
\[
	z = \text{proj}_{M^{\perp}}x, \qquad \qquad \norm{x - \text{proj}_Mx} = \min_{y \in M}\norm{x-y}
\]

\minisec{General Hilbert space case}

\begin{proposition}
	$M \subseteq H$, then $M^{\perp}$ is a closed subspace and
	\[
		\left( M^{\perp} \right)^{\perp} = \overline{\spn M}
	\]
\end{proposition}

\begin{satz}
	$H$ Hilbert space and $M$-closed subspace of $H$ and $x \in H$. Then there exists a unique $z \in M$ such that
	\[
		\norm{x-z} = \dist(x,M) := \inf_{y \in M} \norm{x-y}
	\]
	($z$ analog of the $\text{proj}_Mx$ in the other case)
\end{satz}

\begin{proposition}
	Taking $z \in M$ from the previous proposition. We have $x - z \in M^{\perp}$, i.e.
	\[
		x = \underset{\in M}{\underbrace{z}} + \underset{\in M^{\perp}}{\underbrace{(x - z)}}
	\]
\end{proposition}

%%% lecture 8b
%%% lecture 8b

\begin{theorem}[Orthogonal Decompostion Theorem]
	Let $(E, \skal{.}{.})$ be a Hilbert space and $S$ be a closed subspace of $E$. 
	\[
		\Rightarrow \qquad E = S \oplus S^{\perp}
	\]
	which means that for every $x \in E$ there exists an unique decomposition
	\[
		x = y + z
	\]
	with $y \in S$ and $z \in S^{\perp}$.
\end{theorem}
\begin{beispiel}
	Let $A \subseteq E$ where $E$ is a Hilbert space. It follows
	\[
		\overline{\spn  A} = \left( A^{\perp} \right)^{\perp}.
	\]
	Note 
	\[
		A \subseteq \underset{\text{subspace of $E$}}{\underbrace{ \left( A^{\perp} \right)^{\perp}}} \qquad \Rightarrow \qquad \spn  A \subseteq 
		\underset{\text{closed}}{\underbrace{ \left( A^{\perp} \right)^{\perp}}} \qquad \Rightarrow \qquad \overline{\spn  A} \subseteq 
		\left( A^{\perp} \right)^{\perp}
	\]
	\[
		A \subseteq \overline{\spn  A} \qquad \Rightarrow \qquad \overline{\spn  A}^{\perp} \subseteq A^{\perp} \qquad \Rightarrow \qquad 
		\left( A^{\perp} \right)^{\perp} \subseteq \left( \overline{\spn  A}^{\perp} \right)^{\perp}.
	\]
	Hence
	\[
		\overline{\spn  A} \subseteq \left( A^{\perp} \right)^{\perp} \subseteq \left( \overline{\spn  A}^{\perp} \right)^{\perp}.
	\]
	By the Orthogonal Decomposition Theorem we get
	\[
		E = \overline{\spn  A} \oplus \overline{\spn  A}^{\perp} = \overline{\spn  A}^{\perp} \oplus \left( \overline{\spn  A}^{\perp} \right)^{\perp}, 
	\]
	which implies
	\[
		\overline{\spn  A} = \left( \overline{\spn  A}^{\perp} \right)^{\perp},
	\]
	\[
		\Rightarrow \qquad \left( A^{\perp} \right)^{\perp} = \overline{\spn  A}.
	\]
\end{beispiel}

Now we are going to prove the Orthogonal Decomposition Theorem.

\begin{beweis}
	\begin{description}
		\item[Step 1:] $S$ is a closed convex set in a Hilbert space $E$. This implies that 
		\[
			\forall\,  x \in E \, \exists\,! \,y \in S: \qquad \norm{x-y} \leq \norm{x- \tilde y} \qquad \forall\, \tilde y \in S.
		\] 
		which means
		\[
			\norm{x-y} = \inf_{\tilde y \in S}\norm{x-\tilde y}.
		\]
		Fix $x \not \in S$ with
		\[
			\inf_{ \tilde y \in S} \norm{x- \tilde y} = d > 0.
		\]
		Take a sequence $(y_n)_{n=1}^{\infty}$ in $S$ such that 
		\[
			\norm{x-y_n} \to d, \qquad n \to \infty.
		\]
		\textbf{Claim:} \text{    } This is a Cauchy sequence. \\
		(use Parallelogram-law for $\norm{.}$)
		\item[Step 2:] $S$ as in ODT. \\
		Note: $S$ must be convex. \\
		Fix $x \in E$, choose $y \in S$ with
		\[
			\norm{x-y} \leq  \norm{x- \tilde y}, \qquad \forall\,  \tilde y \in S.
		\]
		Set
		\[
			\underset{\in E}{\underbrace{x}} = \underset{\in S}{\underbrace{y}} + (x-y).
		\]
		To show: $x-y \in S^{\perp}$. A variational argument of this is 
		\[
			\skal{x-y}{v}= 0, \qquad \forall\, v \in S.
		\]
		We know
		\begin{align*}
			\norm{x-y}^2 &\leq \norm{x-y + \alpha v}^2 \qquad \forall\, \text{scalars }\alpha \\
			\norm{x-y}^2 &\leq \skal{x-y+\alpha v}{x-y+ \alpha v} \\
			&= \norm{x-y}^2+ \alpha \skal{v}{x-y} + \bar{\alpha} \skal{x-y}{v} + \abs{\alpha}^2 \norm{v}^2
		\end{align*}
		and
		\[
			0 \leq 2 \re( \alpha \skal{x-y}{v}) + \abs{\alpha}^2 \norm{v}^2.
		\]
		Set 
		\[
			\alpha = t \overline{\skal{x-y}{v}}, \qquad t \in \mathbb{R},
		\]
		\[
			\Rightarrow \qquad 0 \leq  2 t \abs{\skal{x-y}{v}}^2 + t^2 \abs{\skal{x-y}{v}}^2 \norm{v}^2.
		\]
		Assume $\skal{x-y}{v} \neq 0$: \\
		We have 
		\begin{align*}
			0 &\leq 2t + t^2 \norm{v}^2 \qquad \forall\, t \in \mathbb{R} \\
			\Rightarrow \qquad -2t &\leq  t^2 \norm{v}^2, \qquad \text{Let }t <0 \\
			\Leftrightarrow \qquad 2 &\leq -t \norm{v}^2, \qquad t<0.
		\end{align*}
		Let $t \to 0$, then
		\[
			2 \leq 0
		\]
		which is a contradiction.
	\end{description}
\end{beweis}

%%% lecture 9
%% lecture 9


\subsection{Bounded linear functionals on Hilbert spaces} 
\label{sub:bounded_linear_functionals_on_hilbert_spaces}

Consider $(H, \skal{.}{.})$- Hilbert space (inner product space which is complete w.r.t. to a norm $\norm{x}= \sqrt{\skal{x}{x}}$). \\
Let $M$ be a closed subspace of $H$. 
\[
	\mathcal{M}^{\perp} = \set[y \in H]{\skal{x}{y}= 0, \, \forall\,  x \in M}.
\]
Then we know $H = M + M^{\perp}$, i.e. for any $x \in H$ there exists a unique $y \in M$ and $z \in M^{\perp}$ such that
\[
	x = y + z.
\]
\begin{theorem}[Riesz-Frechét represantation theorem]
	Let $(H, \skal{.}{.})$ be a Hilbertspace. Let $f$ be a bounded linear functionall on $H$. Then there exists a unique $x_f \in H$ such that
	\[
		f(x) = \skal{x}{x_f}, \qquad \forall\,  x \in H.
	\]
	Moreover \[
		\norm{f} = \norm{x_f}_H
	\]
\end{theorem}
\begin{bemerkung}
	If $f: H \to \mathbb{C}$ is of the form
	\[
		f(x) = \skal{x}{y}, \qquad \text{for all } x \in H \text{ and some }y \in H.
	\]
	Then $f$ is bounded and linear (easy with Cauchy-Schwarz and properties of the scalarproduct).
\end{bemerkung}
\begin{beweis}
	\begin{description}
		\item[Existence of $x_f$:] If $f$ is a zero linear functional, i.e. $f(x)= 0$ for all $x \in H$ take $x_f = 0$. Assume now that $f$ is not the zero functional. Consider \[
			N(f) := \ker{f} = \set[x \in H]{f(x)= 0}.
		\] 
		Then $N(f)$ is a closed subspace of $H$: \\
		For $x_1,x_2 \in N(f), \,\alpha, \beta \in \mathbb{C}$ it holds
		\[
			f( \alpha x_1, \beta x_2) \stackrel{\text{lin}}{=} \alpha f(x_1) + \beta f(x_2).
		\]
		Hence $\alpha x_1 + \beta x_2 \in N(f)$ and $N(f)$ is a subspace. $N(f)$ is closed since if $x_n \in N(f)$ with $x_n \to x$ strongly. Then 
		\[
			f(x_n) \to f(x)
		\] 
		because of bounded and hence continuous. But we know that $f(x_n) = 0$ so the limit has to be $f(x)=0$, i.e $x \in N(f)$. $N(f)$ is a proper closed subspace. ($N(f) \neq H$). Consider now $N(f)^{\perp}$ which is non-zero. \begin{itemize}
			\item $\dim N(f)^{\perp} = 1$. \\
			Assume that $x_1 \neq 0,x_2 \neq 0 \in N(f)^{\perp}$. Then we have $f(x_1),f(x_2) \neq 0$. It exists $a \in \mathbb{C}$ such that
			\[
				f(x_1) + a f(x_2) = 0
			\]
			And also
			\[
				f(x_1+a x_2) = 0
			\]
			which gives 
			\[
				x_1+a x_2 \in N(f) \cap N(f)^{\perp} = \set{0}.
 			\] 
			Hence
			\[
				x_1 + a x_2 = 0
			\]
			Any two vectors are linearly dependent in $N(f)^{\perp}$ which gives \[
				\dim N(f)^{\perp} = 1
			\]
		\end{itemize}
		Take $y' \in N(f)^{\perp}$ with $\norm{y'} = 1$ and let \[
			x_f = \overline{f(y')}y'.
		\]
		We get
		\[
			\skal{x}{x_f} = \begin{cases}
				0, &\text{ if }x \in N(f)\\
				\skal{\lambda y'}{\overline{f(y')}y'}= f(y') \lambda \underset{=1}{\underbrace{\skal{y'}{y'}}}, &\text{ if }x = \lambda y'
			\end{cases}
		\]
		Furthermore
		\[
			\skal{x}{x_f} = \begin{cases}
				f(x), &\text{ if }x \in N(f)\\
				f(\lambda y') = f(x), &\text{ if }x = \lambda y'
			\end{cases}
		\]
		Since every element in $H$ is given by $x + \lambda y'$. For $x \in N(f)$ and $\lambda \in \mathbb{C}$. Using linearity we get \[
			f(x + \lambda y') = f(x) + f(\lambda y') = \skal{x}{x_f} + \skal{\lambda y'}{x_f} = \skal{x+ \lambda y'}{x_f}
		\]
		\item[uniqueness:] Assume there exists $x_1,x_2 \in H$ such that 
		\[
			f(x) = \skal{x}{x_1} = \skal{x}{x_2}, \qquad \forall\, x \in H
		\]
		We get
		\[
			\skal{x}{x_1-x_2} = 0, \qquad \forall\, x \in H.
		\]
		It holds in particular for $x = x_1 - x_2$ the following equality
		\[
			\skal{x_1-x_2}{x_1-x_2} = 0 \qquad \Rightarrow \qquad x_1-x_2 = 0.
		\]
		\item[norm equality] We must see that \[
			\norm{f} = \norm{x_f}_H
		\] From remark we have 
		\[
			f(x) = \skal{x}{x_f} \qquad \Rightarrow  \qquad \norm{f} \leq \norm{x_f}
		\]
		We have for $x_f \neq 0$:
		\[
			\norm{f} = \sup_{x \neq 0} \frac{\abs{f(x)}}{\norm{x}} \geq \frac{\abs{f(x_f)}}{\norm{x_f}} = \frac{\norm{x_f}^2}{\norm{x_f}} = \norm{x_f}
		\]
		This gives the desired result.
	\end{description}
\end{beweis}
\begin{beispiel}
	\[
		E = C_F = \set[(x_1,x_2,\dots)]{\text{only finite number of $x_i \neq 0$}} \subseteq l^2
	\]
	On $C_F$ consider $l^2$-inner-product
	\[
		\skal{x}{y} = \sum^{\infty}_{i=1} x_i \bar{y}_i \qquad \text{for }x,y \in C_F
	\]
	\begin{enumerate}
		\item $C_F$ is not a Hilbert space as it is not complete w.r.t
		\[
			\norm{x}_2 = \left( \sum^{\infty}_{i=1} \abs{x_i}^2 \right)^{\frac{1}{2}}
		\]
		Find a Cauchy sequence that is not convergent to an element in $C_F$. \\
		Find a proper closed subspace $M$ such that $M^{\perp}= \set{0}$ (This would mean in particular that $C_F \neq M + M^{\perp}$) \\
		Consider
		\[
			M = \set[(x_1,x_2,\dots) \in C_F]{\sum_{k=1}^{\infty} x_k \frac{1}{k}=0}
		\]
		\[
			x_f = (1, \frac{1}{2}, \frac{1}{3}, \dots) \in l^2
		\]
		\[
			M = \ker{f} \cap C_F
		\]
		where
		\begin{align*}
			f: l^2 &\to \mathbb{C} \\
			f(x) &= \skal{x}{x_f} = \sum_{k=1}^{\infty}x_k \frac{1}{k}
		\end{align*}
		\[
			M^{\perp} = \text{all elements in $C_F$ which are in $(\ker{f})^{\perp}$}
		\]
		From the proof of Riesz-Frechet theorem  we have $(\ker{f})^{\perp}$ is $1$-dimensional and \[
			x_f \in (\ker{f})^{\perp}
		\]
		Hence
		\[
			(\ker{f})^{\perp} = \set[\lambda x_f]{\lambda \in \mathbb{C}}
		\]
		We have
		\[
			\underset{= M ^{\perp}}{\underbrace{(\ker{f})^{\perp} \cap C_F}} = \set{0}.
		\]
		\item $(H,\skal{.}{.})$ Hilbert space and $\set{u_i} \subseteq H$ finite or infinite $i$. $\set{u_i}$ is an orthogonal system if
		\[
			\skal{u_i}{u_j}=0, \qquad \forall\, i \neq j.
		\]
		and an orthonormal system if
		\[
			\skal{u_i}{u_j} = \delta_{ij} = \begin{cases}
				0, &\text{ if }i \neq j\\
				1, &\text{ if }i = j
			\end{cases}
		\]
		\end{enumerate}
	\end{beispiel}
	\begin{proposition}
		Orthogonal system of non-zero vectors are linearly independent. (See linear algebra)
	\end{proposition}
	Having linearly independent family of vectors we can make it orthogonal with for example using Gram-Schmidt orthogonalization procedure.(See linear algebra for details). \\
	Recall that we can write a Fourier series of x with $\skal{x}{u_i}$ Fourier coefficients
		\[
			x \in H \qquad \Rightarrow \qquad x = \sum^{\infty}_{i=1} \skal{x}{u_i}u_i
		\]
		with $\set{u_i}$-ON-system. \\
		$C([- \pi, \pi])$ and $\set{u_k} = \set[\frac{1}{\sqrt{2 \pi}} e^{ikt}]{k \in \mathbb{Z}}$ equipped with the scalar product
		\[
			\skal{f}{g} = \int_{- \pi}^{\pi}f(t)\overline{g(t)} \,\mathrm{d}t
		\]
		It holds for the Fourier-series
		\[
			\skal{f}{u_k} = \hat f(k) = \frac{1}{\sqrt{2 \pi}}\int_{- \pi }^{\pi} f(t) e^{- ikt} \,\mathrm{d}t
		\]
		We want to see when
		\[
			\sum^{\infty}_{i=1} \skal{x}{u_i}u_i
		\]
		is convergent to $x$.
		\begin{definition}
			$\mathcal{A}_n$ ON-system is called an ON-basis for $H$ if its span is dense in $H$. We say that an ON-system is complete if every $x \in H$ is 
			\[
				\sum^{\infty}_{i=1} \skal{x}{u_i}u_i
			\]
		\end{definition}
	\begin{theorem}
		$(H, \skal{.}{.})$- Hilbert space, $\set{u_k}$ is ON-system in $H$. The following statements are equivalent.
		\begin{enumerate}[(1)]
			\item $\set{u_n}$ is a complete ON-system.
			\item $\set{u_n}$ is an ON-basis for $H$.
			\item (Parsevals's Identity) 
			\[
				\norm{x} = \left( \sum_{k=1}^{\infty} \abs{\skal{x}{u_k}}^2 \right)^{\frac{1}{2}}, \qquad \forall\, x \in H.
			\]
			\item $\skal{x}{y} = \sum_{k=1}^{\infty} \skal{x}{u_k} \overline{\skal{y}{u_k}}$ for all $x,y \in H$.
			\item $\skal{x}{u_k} = 0$ for all $k \in \mathbb{N}$ follows $x = 0$.
		\end{enumerate}
	\end{theorem}
\begin{beweis}
 	\begin{description}
 		\item[(1) $\Rightarrow $ (2):] We have 
		\[
			x = \sum^{\infty}_{i=1} \skal{x}{u_i}w_i
		\] 
		it means
		\[
			x = \lim_{n \to \infty} \sum^{n}_{i=1} \skal{x}{u_i}w_i \in \text{span }\set[u_i]{i \geq 1}
		\]
		This is implies that any $x \in H$ is in $\overline{\text{span } \set[u_i]{i \geq 1}}$, i.e. $\set{w_i}$ is ON-basis.
		\item[(2) $\Rightarrow$ (5):] Let $\set{u_i}$ be a ON-basis. Assume 
		\[
			\skal{x}{u_k} = 0, \qquad \forall\,  k \in \mathbb{N}
		\]
		Then
		\[
			\skal{x}{u} = 0, \qquad \forall\, u \in \text{span }\set[u_k]{k \geq 1}.
		\]
		By the property that strong convergence implies weak convergence we will have 
		\[
			\skal{x}{u}= 0, \qquad \forall\, u \in \text{span }\set[u_k]{k \geq 1} = H.
		\]
		In particular
		\[
			\skal{x}{u} = 0, \qquad \text{for }u =x
		\]
		which means
		\[
			\skal{x}{x} = 0 \qquad \Leftrightarrow \qquad  x=0.
		\]
		\item[(5) $\Rightarrow$ (1)] Recall Bessel's equality. For $\set{u_k}$- ON-system then 
		\[
			\norm{x- \sum^{k}_{i=1} \skal{x}{u_k}u_k}^2 = \norm{x}^2 - \sum^{k}_{i=1} \abs{\skal{x}{u_k}}^2
		\]
		Assume (5), i.e.
		\[
			\skal{x}{u_k} = 0, \qquad \forall\, k \qquad \Rightarrow \qquad x = 0
		\]
		We must see
		\[
			x = \sum_{k=1}^{n} \skal{x}{u_k}w_k \qquad \forall\, x \in H.
		\]
		From Bessel's equality we have
		\[
			\sum_{k=1}^{n} \abs{\skal{x}{w_k}} = \norm{x}^2 - \norm{x- \sum_{k=1}^{n} \skal{x}{u_k}w_k}^2 \leq \norm{x}^2, \qquad \forall\, k \in \mathbb{N}
		\]
		and hence $\sum_{k=1}^{n}\abs{\skal{x}{w_k}}^2$ is convergent. It implies that for $n>m$ we have
		\begin{align*}
			\norm{\sum_{k=1}^{n}\skal{x}{u_k}w_k - \sum_{k=1}^{n} \skal{x}{u_k}w_k}^2 
			&\stackrel{\hphantom{\text{pythagorian thm}}}{=} \norm{\sum_{k=m+1}^{n} \skal{x}{u_k}w_k}^2 \\
			&\stackrel{\text{pythagorian thm}}{=} \sum_{k=m+1}^{n} \abs{\skal{x}{u_k}}^2 \norm{w_k}^2 \\
			&\stackrel{\hphantom{\text{pythagorian thm}}}{\to } 0, \qquad n,m \to 0 \qquad (*)
		\end{align*}
		Note that if $\set{x_i}$ are paarwise orthogonal it holds
		\[
			\norm{\sum^{n}_{i=1}x_i}^2 = \sum^{n}_{i=1} \norm{x}^2.
		\]
		From $(*)$ we know that the partial sum
		\[
			S_n := \sum_{k=1}^{n}\skal{x}{u_k}w_k
		\]
		is a Cauchy sequence. As $H$ is a Hilbert space, $H$ is complete and hence $S_n$ has a limit in $H$. Write
		\[
			\sum_{i=1}^{\infty} \skal{x}{u_i}w_i 
		\]
		for the limit. We must see that the limit is $x$. Consider
		\[
			y := x - \sum_{i=1}^{\infty} \skal{x}{u_i}w_i
		\]
		Then 
		\[
			\skal{y}{u_i} = \skal{x}{w_i} - \skal{x}{w_i} = 0, \qquad \forall\, i
		\]
		By assumption (5) it follows 
		\[
			y = 0 \qquad \Leftrightarrow \qquad x = \sum^{\infty}_{i=1} \skal{x}{u_i}w_i
		\]
		\item[(1) $\Rightarrow$ (3):] From Bessel's equality we have again
		\[
			\norm{x- \sum^{n}_{i=1}\skal{x}{u_i}w_i}^2 = \norm{x}^2 - \sum^{n}_{i=1}\abs{\skal{x}{u_i}}^2
		\] 
		By assumption (1) the LHS tends to $0$ as $n \to \infty$. On the other hand the RHS goes to 
		\[
			\to \norm{x}^2 - \sum^{\infty}_{i=1} \abs{\skal{x}{u_i}}^2, \qquad n \to \infty.
		\]
		This gives 
		\[
			\norm{x}^2 - \sum_{i=1}^{\infty} \abs{\skal{x}{u_i}^2} = 0
		\]
		\item[(3) $\Rightarrow $ (5)] trivial.
		\item[(4) $\Rightarrow $ (5)] trivial (take $y=x$)
		\item[(1) $\Rightarrow $ (4)] We have
		\[
			x = \sum_{k=1}^{\infty} \skal{x}{u_k}u_k
		\]
		Then
		\[
			\skal{x}{y} = \sum_{k=1}^{\infty} \skal{x}{w_k}\skal{u_k}{y} = \sum_{k=1}^{\infty}\skal{x}{u_k}\overline{\skal{y}{u_k}}
		\]
 	\end{description}
\end{beweis}

\begin{beispiel}
	$L^2([- \pi, \pi])$ with
	\[
		\set[\frac{1}{\sqrt{2\pi}}e ^{int}]{n \in \mathbb{Z}}
	\]
	is an ON-system in $L^2([-\pi, \pi])$ where
	\[
		\skal{f}{g} = \int_{-\pi}^{\pi} f(t) \overline{g(t)} \,\mathrm{d}t
	\]
\end{beispiel}

\begin{satz}
	The system above is an ON-basis for $L^2([-\pi,\pi])$. In particular, for any $f \in L^2([-\pi,\pi])$
	\[
		f = \sum_{k \in \mathbb{Z}}^{} \hat f(k) e^{ikt}
	\]
	convergent in the $L^2$-norm.
	\[
		\norm{f}_{L^2} = \left( \int_{-\pi}^{\pi} \abs{f(t)}^2 \,\mathrm{d}t \right)^{\frac{1}{2}}
	\]
	which is equivalent to
	\[
		\norm{f - \sum_{k=-n}^{n} \hat f(k) e^{ikt}}^2_{L^2} \to 0
	\]
\end{satz}

\minisec{Sketch of the proof:}

\begin{enumerate}[(1)]
	\item Stein-Weierstraß-Theorem. $X$ compact set $C(X,\mathbb{C})$ continuous functions with complex values. Let $M \subseteq C(X,\mathbb{C})$ be a subspace that satisfies
	\begin{enumerate}[(a)]
		\item it seperates points of $X$, i.e. 
		\[
			\forall\, x_1,x_2 \in X, x_1 \neq x_2 \,\exists\, f \in M: \qquad f(x_1) \neq f(x_2)
		\]
		\item $M$ contains the constant function 1 ($f(x)= 1$ for all $x \in X$)
		\item It is closed under complex conjugation, i.e. 
		\[
			f \in M \qquad \Rightarrow \qquad \bar{f} \in M
		\]
		and closed under product, i.e.
		\[
			f_1,f_2 \in M \qquad \Rightarrow \qquad f_1 \cdot f_2 \in M
		\]
	\end{enumerate}
	Then $M$ is dense in $C(X,\mathbb{C})$ w.r.t. $\norm{.}_{\infty}$ (Continuous function by Polynomials) From this it follows
	\[
		M = \set{ \text{all complex polynomials}}
	\]
	are dense in $C([a,b])$.
	\item $C([a,b])$ is dense in $L^2([a,b])$ w.r.t. $\norm{.}_{L^2}$-norm. 
\end{enumerate}
We will use $(1)$ and $(2)$ to show that $\text{span }\set[\frac{1}{\sqrt{2 \pi}}e^{int}]{n \in \mathbb{Z}}$ is dense in $L^2([-\pi,\pi])$.
\begin{beweis}
	Let \[
		M := \text{span }\set[\frac{1}{\sqrt{2 \pi}}e^{int}]{n \in \mathbb{Z}} \subseteq \set[f \in C([-\pi, \pi])]{f( \pi) = f(-\pi)}
	\]
	$M$ seperates points, it contains the constant function $1$ and it is closed under complex conjugation. Furthermore $M$ is closed under taking products. With Stein-Weierstraß it follows that $M$ is dense in 
	\[
		\set[f \in C([-\pi,\pi])]{f(\pi)= f(-\pi)}.
	\]
	By (2) we have $C([-\pi,\pi])$ is dense in $L^2([-\pi,\pi])$ w.r.t. the $L^2$-norm. From this one can see that even $\set[f \in C([-\pi,\pi])]{f(\pi)= f(-\pi)}$ is dense in $L^2([-\pi,\pi])$:
	\[
		\forall\,  \varepsilon>0, \,\forall\, f \in L^2 \,\exists\,g \in C([-\pi,\pi]): \qquad \norm{f-g}_{L^2}^2 = \int_{-\pi}^{\pi}\abs{f(t)-g(t)}^2 \,\mathrm{d}t < \varepsilon
	\] 
	Define $g _{\varepsilon}$ such that it has a pike in $x = \pi - \varepsilon$ but it is continuous and is equal to $g$ for $x < \pi -\varepsilon$. Then
	\[
		g _{\varepsilon} \in C([-\pi,\pi]), \,g _{\varepsilon}(- \pi) = g _{\varepsilon} ( \pi).
	\]
	It holds
	\begin{align*}
		\norm{f - g _{\varepsilon}}_{L^2} &\leq \underset{< \sqrt{\varepsilon}}{\underbrace{\norm{f-g}_{L^2}}} + 
		\norm{g - g _{\varepsilon}}_{L^2} \\
		&\leq  \sqrt{\varepsilon} + \left( \int_{\pi - \varepsilon}^{\pi} \abs{g(t)-g _{\varepsilon}(t)} \,\mathrm{d}t  \right)^{\frac{1}{2}} \\
		&\leq \sqrt{\varepsilon} + \sqrt{\max_{x \in [-\pi - \varepsilon, \pi]} \abs{g- g _{\varepsilon}}\varepsilon} \\
		&= \sqrt{\varepsilon} + \sqrt{C} \sqrt{\varepsilon}
	\end{align*}
	We conclude: any $f = L^2-$limit $g_n$ with $g_n \in C([-\pi,\pi])$ and $g_n(- \pi ) = g_n( \pi)$. Each $g_n = \norm{.}_\infty$-norm limit of an element in 
	$ \text{span }\set[\frac{1}{\sqrt{2 \pi}}e^{int}]{n \in \mathbb{Z}}$ as
	\[
		\norm{g-f}_{L^2} \leq \norm{g-f}^{\frac{1}{2}}_{\infty} (2 \pi)^{\frac{1}{2}}
	\]
	Note that
	\[
		\left( \int_{- \pi}^{\pi} \abs{g(t)-f(t)}^2 \,\mathrm{d}t \right)^{\frac{1}{2}} \leq \max_{x \in [ - \pi, \pi]} \abs{g(t)- f(t)} \left( \int_{- \pi}^{\pi} \,\mathrm{d}t \right)^{\frac{1}{2}}
	\]
	We get that each $g_n$ can be approximated in the $L^2$-norm by elements in $\text{span }\set[\frac{1}{\sqrt{2 \pi}}e^{int}]{n \in \mathbb{Z}}$ hence
	\[
		\text{span }\set[\frac{1}{\sqrt{2 \pi}}e^{int}]{n \in \mathbb{Z}} \subseteq L^2([- \pi,\pi]).
	\]
\end{beweis}

\subsection{Linear operators on Hilbert spaces} 
\label{sub:linear_operators_on_hilbert_spaces}
Set $(H_1,\skal{.}{.}_1)$ and $(H_2, \skal{.}{.}_2)$ Hilbert spaces. A bounded linear mapping $A: H_1 \to H_2$ is called bounded linear operator. \\
Bounded means in our case
\[
	\norm{Ax}_2 \leq C \norm{x}_1 \qquad \forall\, x \in H \text{ and some constant }C
\] 
\begin{beispiel}
	Set $H_1 = H_2 = L^2([0,1])$ and $K: [0,1] \times [0,1] \to \mathbb{C}$. Assume that $K$ is continuous. Consider
	\[
		(Af)(x) = \int_{0}^{1}K(x,y)f(y) \,\mathrm{d}y
	\]
	$A$ is linear (trivial). Show that $A$ is bounded:
	\begin{align*}
		\norm{Af}_2 &\stackrel{\hphantom{\text{CS}}}{=} \int_{0}^{1} \abs{\int_{0}^{1}K(x,y)f(y) \,\mathrm{d}y}^2 \,\mathrm{d}x \\
		&\stackrel{\text{CS}}{\leq} \int_{0}^{1} \left( \int_{0}^{1}\abs{K(x,y)}^2 \,\mathrm{d}y  \cdot \int_{0}^{1}\abs{f(y)}^2 \,\mathrm{d}y \right)\,\mathrm{d}x \\
		&\stackrel{\hphantom{\text{CS}}}{=} \underset{< \infty}{\underbrace{\int_{0}^{1} \left( \int_{0}^{1} \abs{K(x,y)}^2 \,\mathrm{d}y \right) \,\mathrm{d}x}}
		\cdot \underset{= \norm{f}_2^2}{\underbrace{ \int_{0}^{1}\abs{f(y)}^2 \,\mathrm{d}y}}
	\end{align*}
	Hence
	\[
		\norm{A} \leq \left( \int_{0}^{1} \int_{0}^{1} \abs{K(x,y)}^2 \,\mathrm{d}x \,\mathrm{d}y \right)^{\frac{1}{2}}.
	\]
\end{beispiel}
	Products $(A \cdot B)$ of operators $H \to H$ with $A: H \to H$ and $B: H \to H$ are defined by
	\[
		(A \cdot B)(f) := A(Bf)
	\]
\begin{satz}
	If $A$ and $B$ are bounded, then $A \cdot B$ is also bounded and 
	\[
		\norm{AB} \leq \norm{A}\norm{B}.
	\]
	In particular: for all $n \in \mathbb{N}$ $A^n$ is bounded and 
	\[
		\norm{A^n} \leq \norm{A}^n
	\]
\end{satz}

%%% lecture 10
%%%lecture 10

\begin{beispiel}
	$E = L^2([0,1])$ and $f,g \in E$ with
	\[
		\skal{f}{g}_{L^2} = \int_{0}^{1} f(x) \overline{g(x)} \,\mathrm{d}x, \qquad \norm{f}_{L^2} = \left( \int_{0}^{1} \abs{f(x)}^2 \,\mathrm{d}x \right)^{\frac{1}{2}}.
	\]
	Set $h \in C([0,1] \times [0,1])$ and for $f \in L^2([0,1])$
	\[
		A(f)(x) = \int_{0}^{1} h(x,y)f(y) \,\mathrm{d}y, \qquad x \in [0,1].
	\]
	Then
	\[
		\norm{A} \leq \left( \int_{0}^{1} \left( \int_{0}^{1} \abs{h(x,y)}^2 \,\mathrm{d}y \right) \,\mathrm{d}x \right)^{\frac{1}{2}} < \infty.
	\]
\end{beispiel}
\begin{beispiel}
	Let $(E, \norm{.})$ be a normed space. Then there are no $A,B \in B(E,E)$ such that
	\[
		AB - BA = I
	\]
	where $I$ is the identity ($I(x)=x$ for $x \in E$).
	\begin{bemerkung}
		Consider $ f \in E = C^{\infty}([0,1])$ and 
		\[
			A = \diffd{}{x}, \qquad B=x.
		\]Then
		\[
			(AB - BA)(f)(x) = \diffd{}{x}(x(f(x))) - x \diffd{}{x}f(x) = f(x).
		\]
	\end{bemerkung}
	Argue by contradiction. \\
	Assume $A,B \in B(E,E)$ with $AB-BA = I$. \\
	Hint: Consider $A^nB-BA^n$ for $n = 1,2,\dots$. For $n =2$ we have
	\begin{align*}
		A^2B-BA^2 &= A^2B-ABA + ABA - BA^2 \\
		&= A(AB-BA) + (AB-BA)A \\
		&= 2A.
	\end{align*}
	For $n=3$ we have
	\begin{align*}
		A^3B-BA^3 &= A^3B-A^2BA + A^2BA - BA^3 \\
		&=A^2(AB-BA) + (A^2B-BA^2)A \\
		&= 3A^2.
	\end{align*}
	In general 
	\[
		A^nB-BA^n = nA^{n-1}, \qquad n=2,3,4,\dots \qquad (*)
	\]
	Check using an induction argument. We obtain
	\[
		n \norm{A^{n-1}} = \norm{A^nB-BA^n} \leq \norm{A^nB} + \norm{BA^n} \leq  2 \norm{A^{n-1}} \norm{A} \norm{B}
	\]
	Hence \[
		(2 \norm{A} \norm{B} -n) \norm{A^{n-1}} \geq 0, \qquad  \forall\,  n = 2,3,\dots.
	\]
	We conclude that $\norm{A^{n-1}}= 0$ for $n$ large enough. Clearly the same for $\norm{A^n}$. This yields $A^n=0$ for $n$ large enough. Repeated use of $(*)$ gives $A=0$. This contradicts $AB-BA= I$ so the implication in the example is proven.
\end{beispiel}

Recall a important theorem:

\begin{theorem}[Riesz representation theorem]
	$(E, \skal{.}{.})$ Hilbert space $f \in B(E,\mathbb{C})$. $f$ is bounded linear functional on $E$. This yields
	\[
		\exists\,! x_f \in E: \qquad f(x) = \skal{x}{x_f}, \qquad \forall\, x \in E.
	\]
	Also it holds
	\[
		\underset{\substack{\text{operator norm} \\ \text{of $f$}}}{\underbrace{\norm{f}}} = \underset{\substack{norm of} \\ \text{$x_f$ in $E$}}{\underbrace{\norm{x_f}}}.
	\]
\end{theorem}

\begin{definition}
	$\varphi: E \times E \to \mathbb{C}$ is called:
	\begin{itemize}
		\item Bilinear, if for scalars $\alpha$ and $\beta$ it holds
		\begin{align*}
			\varphi( \alpha x+ \beta y,z) &= \alpha \varphi(x,z)+ \beta \varphi(y,z) \qquad \forall\, x,y,z \in E \\
			\varphi(x,\alpha y + \beta z) &= \bar{\alpha} \varphi(x,z) + \overline{\beta} \varphi(y,z) \qquad \forall\, x,y,z \in E.
		\end{align*}
		\item Bounded, if there exists $M>0$ such that
		\[
			\abs{\varphi(x,y)} \leq M \norm{x}\norm{y}, \qquad \forall\, x,y \in E.
		\]
		\item Coercive, if there exists $K>0$ such that
		\[
			\varphi(x,x) \geq K \norm{x}^2, \qquad \forall\, x \in E.
		\]
	\end{itemize}
\end{definition}
Clearly $\skal{.}{.}$ in $E$ is a bilinear, bounded and coercive functional in $E$ (with $M=K=1$). \\
We will now introduce a Generalization of the Riesz representation theorem.

\begin{theorem}[Lax-Milgram]
	$(E, \skal{.}{.})$ Hilbert space. Let $\varphi: E \times E \to \mathbb{C}$ be a bilinear, bounded and coercive functional. $f: E \to \mathbb{C}$ bounded linear functional in $E$. Then there exists an unique $x_f \in E$ such that
	\[
		f(x) = \varphi(x,x_f), \qquad  \forall\, x \in E.
	\]
\end{theorem}

\begin{beweis}
	\begin{enumerate}[Step 1:]
		\item $\exists\,!$ $A \in B(E,E)$ with
		\[
			\varphi(x,y)= \skal{x}{A(y)}, \qquad \forall\, x,y \in E.
		\]
		\item $A$ is injective and surjective.
		\item Apply RRT with $ \tilde x_f = A^{-1}(x_f)$
		\begin{align*}
			f(x) &= \skal{x}{x_f} \\ &= \skal{x}{A(A^{-1}(x_f))} \\ &= \varphi(x, \tilde x_f), \qquad \forall\, x \in E.
		\end{align*}
	\end{enumerate}
	\begin{description}
		\item[Step 1:] Fix $y \in E$ and consider for $x \in E$
		\[
			x \stackrel{f_y}{\mapsto } \varphi(x,y) \in \mathbb{C}.
		\] 
		\textbf{Claim:} \text{    }$f_y: E \to \mathbb{C}$ is a bounded linear functional. \\
		For $x,y,z \in E$ and $\alpha,\beta$ scalars we have
		\begin{align*}
			f_y(\alpha x+ \beta z) &= \varphi(\alpha x + \beta z,y) \\
			&= \alpha \varphi(x,y) + \beta \varphi(z,y) \\
			&= \alpha f_y(x) + \beta f_y(z).
		\end{align*}
		Hence $f_y$ is linear. It is bounded because of
		\[
			\abs{f_y(x)} = \abs{\varphi(x,y)} \leq (M \norm{y})\norm{x}, \qquad \forall\, x \in E.
		\]
		So $f_y$ is bounded. \\
		RRT implies $f_y(x) = \skal{x}{A(y)}$ for all $x \in E$ for some $A(y) \in E$. \\ Now we have $A : E \to E$.
		\textbf{Claim:} \text{    }$A \in B(E,E)$. \\
		For $x,y,z \in E$ and scalars $\alpha, \beta$ we have
		\begin{align*}
			\skal{x}{A( \alpha y + \beta z)} &= \varphi(x, \alpha y + \beta z) \\
			&= \bar{\alpha} \varphi(x,y)+ \bar{\beta} \varphi(x,z) \\
			&= \bar{\alpha} \skal{x}{A(y)} + \bar{\beta} \skal{x}{A(z)} \\
			&= \skal{x}{\alpha A(y)}+ \skal{x}{\beta A(z)}. \\
		\end{align*}
		This is equivalent to
		\[
			\skal{x}{A(\alpha y + \beta z)- \alpha A(y) - \beta A(z)} = 0, \qquad  x \in E.
		\]
		This implies 
		\[
			\norm{A(\alpha y + \beta z) - \alpha A(y) - \beta A(z)} = 0.
		\]
		So \[
			A( \alpha y+ \beta z) = \alpha A(y) + \beta A(z) \qquad \forall\, y,z \in E \text{ and scalars }\beta,\alpha.
		\]
		Hence, $A$ is linear. We will now show that $A$ is bounded: \\
		We know because $\varphi$ is continuous that for all $x,y \in E$
		\[
			\abs{\skal{x}{A(y)}} = \abs{\varphi(x,y)} \leq M \norm{x} \norm{y}.
		\]
		Take $x = A(y)$ and get
		\[
			\norm{A(y)}^2 \leq  M \norm{A(y)} \norm{y} \qquad \forall\, y \in E
		\]
		which implies
		\[
			\norm{A(y)} \leq M \norm{y} \qquad \forall\, y \in E.
		\]
		Hence $\norm{A} \leq M < \infty$.
		\item[Step 2:] Note $\varphi(x,y) = \skal{x}{A(y)}$ for alle $x,y \in E$. \\
		\textbf{Claim:} \text{    }$A$ is injective, i.e.
		\[
			A(x_1) = A(x_2) \qquad \Rightarrow \qquad x_1 = x_2.
 		\]
		$\varphi$ is coercive so
		\[
			\norm{x}^2 \leq \frac{\varphi(x,x)}{K} = \frac{1}{K} \underset{>0}{\underbrace{\abs{\skal{x}{A(x)}}}} \leq \frac{1}{K} \norm{x} \norm{A(x)} \qquad \forall\,  x \in E.
		\]
		Hence \[
			\norm{x} \leq  \frac{1}{K} \norm{A(x)}, \qquad \forall\, x \in E.
		\]
		If $A(x_1) = A(x_2)$ we have $A(x_1-x_2) = 0 \in E$ then
		\[
			\norm{x_1 - x_2} \leq \frac{1}{K} \norm{A(x_1 -x_2)} = 0.
		\]
		We get $x_1 = x_2$. \\
		\textbf{Claim:} \text{    }$A$ is surjective, i.e. the image of $A$ is $E$: \[
			\mathcal{R}(A) = \set[A(x)]{x \in E} = E.
		\] 
		We first show that $\mathcal{R}(A)$ is a closed subspace of $E$. 
		\begin{itemize}
			\item $\mathcal{R}(A)$ is a subspace in $E$ since $A$ is linear.
			\item $\mathcal{R}(A)$ is closed since
			\[
				y_n \to y \qquad \text{in }(E, \norm{.}) \qquad \Rightarrow y \in \mathcal{R}(A).
			\]			
		\end{itemize}
		$\mathcal{R}(A)$ is linear. Take $y_1,y_2 \in \mathcal{R}(A)$ with preimages $x_1,x_2$ and yield
		\[
			\alpha_1 y_1 + \alpha_2 y_2 = \alpha_1 A(x_1) + \alpha_2 A(x_2) = A(\alpha_1 x_1 + \alpha_2 x_2).
		\]
		So \[
			\alpha_1 y_1 + \alpha_2 y_2 \in \mathcal{R}(A).
		\]
		Assume \[
			y_n \to y \qquad \text{in }(E, \norm{.}).
		\]
		For $n=1,2,\dots$ there are $x_1,x_2,\dots$ such that $y_n= A(x_n)$ for $n=1,2,\dots$. \\
		\textbf{Claim:} \text{    }$(x_n)_{n \in \mathbb{N}}$ is a Cauchy sequence in $E$ since
		\begin{align*}
			\norm{x_n-x_m} &\leq \frac{1}{K} \norm{A(x_n-x_m)} \\
			&= \frac{1}{K} \norm{A(x_n)- A(x_m)} \\
			&= \frac{1}{K} \norm{y_n -y_m} \to 0, \qquad n,m \to \infty
		\end{align*}
		since $(y_n)_{n \in \mathbb{N}}$ converges. \\
		Since $(E, \norm{.})$ is a Banach space $(x_n)_{n \in \mathbb{N}}$ converges in $(E, \norm{.})$. Call the limit $x \in E$. Hence
		\[
			A(x_n) \to y
		\]
		since $A$ is bounded, continuos and linear. So $y = A(x)$ and we get $y \in \mathcal{R}(A)$. \\
		Secondly $A$ is surjective, i.e. $\mathcal{R}(A)=E$. \\
		Assume that this is not true. The Orthogonal decomposition theorem gives
		\[
			E = \mathcal{R}(A) \oplus \mathcal{R}(A)^{\perp}.
		\]
		The first one is a closed subspace in $E$ and the second one is not empty by assumption. Fix $z \in \mathcal{R}(A)^{\perp} \setminus \set{0}$. Note 
		\[
			\varphi(x,y) = \skal{x}{A(y)} \qquad x,y \in E
		\]
		With $x = y = z$ we get
		\[
			\varphi(z,z) = \skal{z}{A(z)} = 0
		\]
		and 
		\[
			\varphi(z,z) \geq K \norm{z}^2 \geq 0 \qquad \Rightarrow  \,z=0.
		\]
		This is a contradiction. \\
		The Conclusion is 
		\[
			\mathcal{R}(A)^{\perp} = \set{0} \qquad \Rightarrow \qquad \mathcal{R}(A) = E.
		\]
		We have $\varphi(x,y) = \skal{x}{A(y)}$ for all $x,y \in E$ and $A \in B(E,E)$ surjective.
		\item[Step 3:] see above.
	\end{description}
\end{beweis}

\subsection{Adjoint operator} 
\label{sub:adjoint_operator}

$(E,\skal{.}{.})$ Hilbert space and $A \in B(E,E)$ with adjoint $A^{*}$, i.e.
\[
	\skal{A(x)}{y} = \skal{x}{A^*(y)} , \qquad \forall\, x,y \in E.
\]
Fix $y \in E$ and consider
\[
	x \stackrel{f_y}{\mapsto } \skal{A(x)}{y} \in \mathbb{C}. 
\]
\textbf{Claim:} \text{    }$f_y$ is a bounded linear functional on $E$
\begin{itemize}
	\item linear since $A$ is linear.
	\item bounded since $A$ is bounded with
	\[
		\abs{f_y(x)} \leq (\norm{A}\norm{y})\norm{x}, \qquad x \in E.
	\]
\end{itemize}
	RRT implies
	\[
		f_y(x) = \skal{x}{A^*(y)}, \qquad x \in E.
	\]
	We have $A^*: E \to E$ such that
	\[
		\skal{A(x)}{y} = \skal{x}{A^*(y)} , \qquad \forall\, x,y \in E.
	\]
\begin{proposition}
	$A \in B(E,E)$. Then $A^* \in B(E,E) $ and $\norm{A^*} = \norm{A}$.
\end{proposition}
\begin{beweis}
	$A^*$ linear:
	\[
		\skal{x}{A^* ( \alpha y + \beta z)} = \skal{x}{\alpha A^*(y) + \beta A^*(z)} \qquad \forall\, x,y \in E.
	\]
	$A^*$ bounded: \\ Take $x= A^*(y)$ and get
	\begin{align*}
		\norm{A^*(y)}^2 &= \abs{\skal{A(A^*(y))}{y}} \\
		&\leq \norm{A(A^*(y))} \norm{y} \\
		&\leq \norm{A} \norm{A^*(y)} \norm{y}, \qquad y \in E.
	\end{align*}
	We get
	\[
		\norm{A^*(y)} \leq \norm{A} \norm{y}, \qquad y \in E.
	\]
	Conclucion: $A^* \in B(E,E)$. We also get 
	\[
		\norm{A^*} \leq \norm{A}.
	\]
	But we also know that $A^{**} = A$ since
	\begin{align*}
		\skal{x}{A^{**}(y)} &= \skal{A^*(x)}{y} \\
		&= \overline{\skal{y}{A^*(x)}} \\
		&= \overline{\skal{A(y)}{x}} \\
		&= \skal{x}{A(y)}, \qquad x,y \in E.
	\end{align*}
	So \[
		\norm{A} = \norm{A^{**}} \leq \norm{A^*}
	\]
	which impllies
	\[
		\norm{A} = \norm{A^*}.
	\]
\end{beweis}
\begin{bemerkung}
	$A,B \in B(E,E)$ then
	\begin{align*}
		(A+B)^* &= A^* + B^* \\
		(AB)^* &= B^* A^* \\
		(\alpha A)^* &= \bar{\alpha}A^* \\
		A^{**} &= A \\
		I^* &= I.
	\end{align*}
\end{bemerkung}
\begin{beispiel}
	Continuity of the example above: For $f \in L^2([0,1])$ consider
	\[
		A(f)(x) = \int_{0}^{1}h(x,y)f(y) \,\mathrm{d}y, \qquad x \in [0,1].
	\]
	For $g \in L^2([0,1])$ it holds
	\begin{align*}
		\skal{A(f)}{g}_{L^2} &= \int_{0}^{1} A(f)(x)\overline{g(x)} \,\mathrm{d}x \\
		&= \int_{0}^{1}\int_{0}^{1} h(x,y)f(y) \,\mathrm{d}x \overline{g(x)} \,\mathrm{d}x \\
		&= \int_{0}^{1}f(y) \cdot \int_{0}^{1}h(x,y) \overline{g(x)} \,\mathrm{d}x \,\mathrm{d}y \\
		&= \int_{0}^{1} f(y) \cdot \overline{\int_{0}^{1} \overline{h(x,y)} g(x) \,\mathrm{d}x} \,\mathrm{d}y \\
		&= \skal{f}{A^*(g)}_{L^2}.
	\end{align*}
	This gives us 
	\[
		A^*(f)(x) = \int_{0}^{1} \overline{h(y,x)}f(y) \,\mathrm{d}y, \qquad x \in [0,1].
	\]
\end{beispiel}

\begin{beispiel}
	$A \in B(E,E)$. It follows
	\[
		\mathcal{R}(A)^{\perp} = N(A^*) = \set[x \in E]{A^*(x) = 0}
	\]
	since $x \in \mathcal{R}(A)^{\perp}$. It is equivalent that
	\[
		\skal{x}{A(y)} = 0, \qquad \forall\, y \in E
	\] 
	\[
		\Leftrightarrow \qquad \skal{A^*(x)}{y} = 0, \qquad \forall\, y \in E
	\]
	\[
		\Rightarrow \qquad A^*(x) = 0 \qquad \Leftrightarrow \qquad x \in N(A^*).
	\]
	We get
	\[
		N(A^*)^{\perp} = \overline{\mathcal{R}(A)}
	\]
	since
	\[
		N(A^*)^{\perp} = \left( R(A)^{\perp} \right)^{\perp} = \overline{\spn(\mathcal{R}(A))} = \overline{\mathcal{R}(A)}.
	\]
\end{beispiel}
\begin{bemerkung}
	$A \in B(E,E)$ is called self adjoint if $A^* = A$.
\end{bemerkung}
For $A \in B(E,E)$ we have
\[
	\norm{A} =\sup\limits_{\substack{\norm{x} = 1  \\ \norm{y}=1}} \abs{\skal{A(x)}{y}}
\]
since
\[
	\norm{\skal{A(x)}{y}} \leq \underset{\leq \norm{A}\norm{x}}{\underbrace{\norm{A(x)}}} \leq \norm{A}, \qquad \text{for }\norm{x}=\norm{y}=1.
\]
If $A(x) = 0$ for all $x \in E$ then $\norm{A}=0$ and also
\[
	\sup\limits_{\substack{\norm{x} = 1  \\ \norm{y}=1}} \abs{\skal{A(x)}{y}} = 0.
\]
For $x$ with $A(x) \neq 0$ then it is
\[
	A \left( \frac{1}{\norm{x}}x \right) \neq 0.
\]
For such an $x$ with $\norm{x}=1$ we have
\[
	\abs{\skal{A(x)}{\frac{1}{\norm{A(x)}}A(x)}} = \frac{1}{\norm{A(x)}} \norm{A(x)}^2 = \norm{A(x)}
\]
and
\[
	\norm{A} \leq \sup_{\norm{x}=1} \norm{A(x)} \leq \sup\limits_{\substack{\norm{x} = 1  \\ \norm{y}=1}} \abs{\skal{A(x)}{y}} \leq \norm{A}.
\]
\begin{proposition}
	Let $A \in B(E,E)$ be self-adjoint. Then
	\[
		\norm{A} = \sup_{\norm{x}=1} \abs{\skal{A(x)}{x}}.
	\]
\end{proposition}
\begin{beweis}
	Set 
	\[
		M = \sup_{\norm{x}=1} \abs{\skal{A(x)}{x}}.
	\]
	For $\norm{x}=1$ we have
	\[
		\abs{\skal{A(x)}{x}} \leq \norm{A(x)} \norm{x} \leq \norm{A}.
	\]
	Furthermore
	\[
		M \leq \norm{A}.
	\]
	It remains to prove: $\norm{A} \leq M$. \\
	For $x,z \in E$ consider:
	\begin{align*}
		\skal{A(x+z)}{x+z} - \skal{A(x-z)}{x-z} &= 2 \skal{A(x)}{z} + 2 \skal{A(z)}{x} \\
		&= 2 \left( \skal{A(x)}{z} + \skal{z}{A^*(x)} \right) \\
		&= 2 (\skal{A(x)}{z} + \skal{z}{A(x)}) \\
		&= 4 \re( \skal{A(x)}{z}).
	\end{align*}
	Assume now $A(x) \neq 0$ and set
	\[
		z = \frac{1}{\norm{A(x)}} A(x).
	\]
	Hence
	\[
		\norm{A(x)} = \frac{1}{4} \left( \skal{A(x+\frac{1}{\norm{A(x)}} A(x))}{x + \frac{1}{\norm{A(x)}} A(x)} 
		- \skal{A(x-\frac{1}{\norm{A(x)}} A(x))}{x- \frac{1}{\norm{A(x)}} A(x)} \right).
	\]
	Note \[
		\abs{\skal{A(y)}{y}} = \norm{y}^2 \abs{\skal{A(\frac{1}{\norm{y}}y)}{ \frac{1}{\norm{y}}y}} \leq M \norm{y}^2.
	\]
	We now obtain
	\begin{align*}
		\norm{A(x)} &\leq \frac{1}{4} \left( M \norm{x+\frac{1}{\norm{A(x)}} A(x)}^2 + M \norm{x - \frac{1}{\norm{A(x)}} A(x)}^2 \right) \\
		&= \frac{M}{4} 2 \left( \norm{x}^2 + \norm{\frac{1}{\norm{A(x)}} A(x)}^2 \right) \\
		&= \frac{M}{2} (\norm{x}^2 +1).
	\end{align*}
	So
	\[
		\norm{A} = \sup_{\norm{x}=1} \norm{A(x)} \leq M
	\]
	and this yields
	\[
		\norm{A}= M.
	\]
\end{beweis}

%%% lecture 11
%%% lecture 11

\begin{definition}[compact]
	If $A: E \to E$ is linear, then we say that $A$ is \underline{compact} if for all bounded sequences $(x_n)_{n=1}^{\infty}$ in $E$, $ (A(x_n))_{n=1}^{\infty}$ has a bounded subsequence in $E$
\end{definition}

\begin{lemma}
	$A$ is compact and linear \qquad  $\Rightarrow $ \qquad  $A$ is bounded.
\end{lemma}
\begin{beweis}
	If $A$ is not bounded then there exists a sequence $(y_n)_{n=1}^{\infty}$ in $E$ such that
	\[
		\norm{A(y_n)} \geq n \norm{y_n}, \qquad \text{for }n=1,2,\dots
	\]
	Set $x_n = \frac{y_n}{\norm{y_n}}$ for $n = 1,2,\dots$. Here $\norm{x_n}=1$ for all $n \in \mathbb{N}$ and 
	\[
		\norm{A(x_n)} = \norm{A \left( \frac{1}{\norm{y_n}}y_n \right)} = \frac{1}{\norm{y_n}} \norm{A(y_n)} >n, \qquad \forall\, n \in \mathbb{N}.
	\]
	$(A(x_n))_{n=1}^{\infty}$ has no converging subsequence since $\norm{A(x_n)} \to \infty$ for $n \to \infty$.
\end{beweis}
\begin{bemerkung}
	\begin{itemize}
		\item $A \in B(E,E)$ and $F \subset E$ where $F$ is bounded. Then
		\[
			A(F) = \set[A(x)]{x \in F} 
		\]
		is bounded.
		\item $A \in B(E,E)$ compact and $F \subset E$, $F$ bounded. Then $\overline{A(F)}$ is compact.
	\end{itemize}
\end{bemerkung}

\begin{lemma}
	$A,B$ compact linear operators $E \to E$ and $\alpha$ and $\beta$ scalars. Then $\alpha A+ \beta A$ is compact.
\end{lemma}
\begin{beweis}
	Fix an arbitrary bounded sequence $(x_n)_{n=1}^{\infty}$ in $E$. Since $A$ is compact there exists a converging subsequence $(A(x_{n_k}))_{k=1}^{\infty}$ of $(A(x_n))_{n=1}^{\infty}$. \\ Clearly $(\alpha A(x_n))_{n=1}^{\infty}$ converges in $E$. \\ Since $B$ is compact there exists a converging subsequence $(B(x_{n_k}))_{k=1}^{\infty}$ of $(B(x_n))_{n=1}^{\infty}$.  \\ Clearly $(\beta B(x_{n_k}))_{k=1}^{\infty}$ converges in $E$. Hence
	\[
		\left( \alpha A(x _{n_k}) + \beta B(x _{n_k}) \right)_{k=1}^{\infty} = \left( (\alpha A + \beta B)(x _{n_k}) \right)_{k=1}^{\infty}
	\]
	converges in $E$.
\end{beweis} 
Set \[
	K(E,E) := \text{set of all compact linear mappings $E \to E$}
\]
We have $K(E,E)$ is a subspace in $(B(E,E), \norm{.}_{E \to E})$.

\begin{proposition}
	$K(E,E)$ is a \underline{closed} subspace in $(B(E,E), \norm{.}_{E \to E})$.
\end{proposition}
Before the proof we note:
\begin{enumerate}
	\item Assume $(E,\skal{.}{.})$ to be a Hilbert space and $A \in B(E,E)$
	\begin{align*}
		x_n \to x \,\text{ in }E \qquad &\Rightarrow \qquad A(x_n) \to A(x)\, \text{ in }E \\
		x_n \rightharpoonup x \,\text{ in }E \qquad &\Rightarrow \qquad A(x_n) \rightharpoonup A(x)\, \text{ in }E
	\end{align*}
	since for $y \in E$ we have
	\[
		\skal{A(x_n)}{y} = \skal{x_n}{A^*(y)} \stackrel{n \to \infty}{\to } \skal{x}{A^*(y)} = \skal{A(x)}{y}
	\]
	\item $A \in K(E,E)$ and $x_n \rightharpoonup x$ in $E$ 
	\[
		\Rightarrow A(x_n) \to A(x) \qquad \text{in }E.
	\]
	\item $A \in B(E,E)$ finite-rank operator, i.e.
	\[
		\dim \mathcal{R}(A) < \infty \qquad \Rightarrow \qquad A \in K(E,E)
	\]
	since: Let $e_1,e_2, \dots, e_N$ be an ON-basis for $\mathcal{R}(A)$ with $N = \dim( \mathcal{R}(A))$. We have
	\[
		A(x) = \skal{A(x)}{e_1}e_1 + \dots + \skal{A(x)}{e_N}e_N
	\]
	Fix an arbitrary bounded sequence $(x_n)_{n=1}^{\infty}$ in $E$. $A$ is bounded which implies that $(A(x_n))_{n=1}^{\infty}$ is a bounded sequence. Furthermore
	\[
		(\skal{A(x_n)}{e_1})_{n=1}^{\infty}
	\]
	is a bounded sequence in $\mathbb{C}$. Bolzano Weierstrass theorem implies that $(\skal{A(x_n)}{e_1})_{n=1}^{\infty}$ has a converging subsequence 
	$(\skal{A(x _{n_k})}{e_1})_{k=1}^{\infty}$. Clearly $(\skal{A(x _{n_k})}{e_1} e_1)_{k=1}^{\infty}$ converges in $E$. \\
	Hence 
	\[
		A(x) = \skal{A(x)}{e_1}e_1 + \dots + \skal{A(x)}{e_N}e_N
	\]
	is a compact mapping since $K(E,E)$ is a subspace of $B(E,E)$.
\end{enumerate}

\begin{beweis}
	Assume $(A_n)_{n=1}^{\infty} \subseteq K(E,E)$ such that $A_n \to A$ in $(B(E,E), \norm{.}_{E \to E})$. \\
	 We have to show: $A \in K(E,E)$ \\
	 Fix an arbitrary bounded sequence $(x_n)_{n=1}^{\infty}$ in $E$. We want to show that $(A(x_n))_{n=1}^{\infty}$ has a converging subsequence in $E$. \\
	 Set \[
	 	M = \sup_{n} \norm{x_n} < \infty
	 \]
	 \begin{align*}
	 	A_1 \in K(E,E) \qquad &\Rightarrow \qquad (A_1(x_n))_{n=1}^{\infty} \text{ has a converging subsequence }(A_1(x _{n_k}))_{k=1}^{\infty} \\
		A_2 \in K(E,E) \qquad &\Rightarrow \qquad (A_2(x_n))_{n=1}^{\infty} \text{ has a converging subsequence }(A_2(x _{n_k}))_{k=1}^{\infty}
	 \end{align*}
	 proceed inductively:
	 \[
	 	A_k \in K(E,E) \qquad \Rightarrow \qquad (A_k(x_n))_{n=1}^{\infty} \text{ has a converging subsequence }(A_k(x _{n_l}))_{l=1}^{\infty}
	 \] 
	 Also: $(A_l(x_{n,k})_{n=1}^{\infty}$ converges in $E$ for $l=1,2,\dots,k$. \\ Here $(A_k(y_n))_{n=1}^{\infty}$ converges for $k=1,2,\dots$. \\
	 So since $(E,\norm{.})$ is a Banach space it is enough to show that $(A(y_n))_{n=1}^{\infty}$ is a Cauchy sequence in $(E,\norm{.})$. \\
	 Fix an arbitrary $\varepsilon >0$. We have
	 \[
	 	\norm{A(y_n)-A(y_m)} \leq \underset{\leq \norm{A-A_k}_{E \to E} \norm{y_n}}{\underbrace{\norm{A(y_n)-A_k(y_n)}}} + \norm{A_k(y_n)- A_k(y_m)} + \norm{A_k(y_m)-A(y_m)}
	 \]
	 Fix $k$ large enough such that
	 \[
	 	\norm{A_k-A} < \frac{\varepsilon}{3M}
	 \]
	 Then 
	 \[
	 	\norm{A(y_n)-A(y_m)} < \frac{2}{3} \varepsilon+ \norm{A_k(y_n)-A_k(y_m)}
	 \]
	 $(A_k(y_n))_{n=1}^{\infty}$ converges in $E$. This implies the existence of $N$ such that
	 \[
	 	\forall\, n,m \geq N: \qquad \norm{A_k(y_n)-A_k(y_n)}< \varepsilon
	 \]
	 \[
	 	\Rightarrow \norm{A(y_n)-A(y_m)} < \varepsilon, \qquad \forall\, n,m \geq N
	 \]
	 and thus $(A(y_n))_{n=1}^{\infty}$ is a Cauchy sequence.
\end{beweis}

\begin{proposition}
	Let $(E, \skal{.}{.})$ be a seperable Hilbert space and $A \in K(E,E)$. then there exist finite-ranked operators $A_n \in K(E,E)$ such that
	\[
		\norm{A-A_n}_{E \to E} \to 0, \qquad n \to \infty.
	\]
\end{proposition}
\begin{beweis}
	Let $(x_n)_{n=1}^{\infty}$ be an ON-basis for $E$. For 
	\[
		x = \sum_{k=1}^{\infty} \skal{x}{x_k}x_k, \qquad x \in E.
	\]
	Set 
	\[
		A_n(x) = A \left( \sum_{k=1}^{n}\skal{x}{x_k}x_k \right) = \sum_{k=1}^{n} \skal{x}{x_k}A(x_k), \qquad x \in E, \qquad n=1,2,\dots
	\]
	Here $\dim( \mathcal{R}(A_n)) \leq n$ for $n =1,2,\dots$. \\
	So $A_n$ is a finite ranked operator in $E$ for $n=1,2,\dots$. \\
	Fix $x \in E$ with $\norm{x}=1$ and consider:
	\[
		\norm{(A-A_n)(x)}^2 = \norm{A(\sum_{k=n+1}^{\infty} \skal{x}{x_k}x_k)}^2 \leq \sup\limits_{\substack{\norm{y}=1, \\ y \in \set{x_1,\dots,x_n}^{\perp}}}
		\norm{A(y)}^2
	\]
	and thus
	\[
		\norm{A-A_n}_{E \to E}^2 \leq \sup\limits_{\substack{\norm{y}=1, \\ y \in \set{x_1,\dots,x_n}^{\perp}}} \norm{A(y)}^2
	\]
	Set 
	\[
		u_n := \sup\limits_{\substack{\norm{y}=1, \\ y \in \set{x_1,\dots,x_n}^{\perp}}} \norm{A(y)}^2 < \infty, \qquad n =1,2,\dots
	\]
	Here $a_n \geq a_{n+1} \geq 0$ for $n=1,2,\dots$. \\
	Clearly $(a_n)_{n=1}^{\infty}$ converges in $\mathbb{R}$. Set $a = \lim_{n \to \infty}a_n$. It remains to prove $a =0$.
	Assume $a>0$. Then there exists $(y_n)_{n=1}^{\infty}$ in $E$ such that
	\begin{enumerate}
		\item $\norm{y_n}=1$
		\item $y \in \set{x_1,\dots,x_n}^{\perp}$
		\item $\norm{A(y_n)}^2 \geq \frac{1}{2} a$
	\end{enumerate}
	\textbf{Claim:} \text{    } $y_n \rightharpoonup 0$ in $(E,\skal{.}{.})$ since: \\
	fix an arbitrary $x \in E$ and
	\begin{align*}
		\abs{\skal{y_n}{x}} &= \abs{\skal{y_n}{\sum_{k=1}^{\infty} \skal{x}{x_k}x_k}} \\
		&= \abs{\skal{y_n}{\sum_{k=n+1}^{\infty}\skal{x}{x_k}x_k}} \\
		&\leq \norm{y_n} \cdot \norm{\sum_{k=n+1}^{\infty}\skal{x}{x_k}x_k} \\
		&= \sqrt{\sum_{k=n+1}^{\infty}\abs{\skal{x}{x_k}}^2} \to 0, \qquad n \to \infty
	\end{align*}
	(Note that $\sum_{k=1}^{\infty}\abs{\skal{x}{x_k}}^2 = \norm{x}^2 < \infty$) \\
	We have $y_n \rightharpoonup 0$ in $(E, \skal{.}{.})$ and
	\[
		A \in B(E,E) \qquad \Rightarrow \qquad A(y_n) \to A(0)= 0
	\]
	Contradiction to (3) above which gives us $a=0$.
\end{beweis}

\begin{proposition}
	$(E,\skal{.}{.})$ Hilbert space and $A \in K(E,E)$. Then \[
		x_n \rightharpoonup x \text{ in } (E, \skal{.}{.}) \qquad \Rightarrow \qquad A(x_n) \to A(x) \text{ in }(E,\skal{.}{.})
	\]
\end{proposition}
\begin{beweis}
	$x_n \rightharpoonup x$ in $(E,\skal{.}{.})$ implies that $\sup_n \norm{x_n}< \infty$ (according to important theorem). Since $A \in K(E,E)$, we know that $(A(x_n))_{n=1}^{\infty}$ has a converging subsequence $(A(x _{n_k})_{k=1}^{\infty}$ since $(x_n)_{n=1}^{\infty}$ is bounded. \\
	Say $A(x _{n_k}) \to y$ in $E$. $A \in K(E,E) \subset B(E,E)$ and $x_n \rightharpoonup x$ in $(E, \skal{.}{.})$. \\
	This implies
	\[
		A(x_n) \rightharpoonup A(x) \qquad \text{in }(E,\skal{.}{.})
	\]
	We get that $y = A(x)$. We have $A(x _{n_k}) \to A(x)$ in $E$. \\
	Assume that $A(x_n) \not A(x)$ in $E$. \\
	Then there exists an $\varepsilon >0$ and a subsequence $(A(\tilde x_n))_{n=1}^{\infty}$ of $(A(x_n))_{n=1}^{\infty}$ such that
	\[
		\norm{A( \tilde x_n) - A(x)} \geq \varepsilon, \qquad \forall\,  n
	\]
	But $\tilde x_n \rightharpoonup x$ in $(E,\skal{.}{.})$ and to be compact implies that $(A(\tilde x_n))_{n=1}^{\infty}$ has a converging subsequence $(A( \tilde x _{n_k})_{k=1}^{\infty}$ that converges to $A(x)$ (same argument as before)
	Conclusion: $A(x_n) \to A(x)$ in $(E, \skal{.}{.})$.
\end{beweis}

\begin{proposition}
	$A \in K(E,E)$ and $(E, \skal{.}{.})$ Hilbert space \qquad $\Rightarrow $ $A^* \in K(E,E)$
\end{proposition}
\begin{beweis}
	Fix any bounded sequence $(x_n)_{n=1}^{\infty}$ in $E$. 
	\begin{align*}
		\norm{A^*(x_n)-A^*(x_m)} &= \skal{A^*(x_n)-A^*(x_m)}{A^*(x_n)-A^*(x_m)} \\
		&= \skal{x_n-x_m}{A(A^*(x_n))-A(A^*(x_m))} 
	\end{align*}
	then use $A \in K(E,E)$.
\end{beweis}

\begin{proposition}
	$A \in K(E,E)$, $B \in B(E,E)$ \qquad $\Rightarrow $ $AB,BA \in K(E,E)$.
\end{proposition}

\begin{beispiel}
	We already know this example: $k \in C([0,1] \times [0,1])$ with
	\[
		A(f)(x) = \int_{0}^{1} k(x,y)f(y) \,\mathrm{d}y, \qquad x \in [0,1], \qquad f \in L^2([0,1])
	\]
	We know that $A \in B(L^2([0,1]), L^2([0,1]))$ 
	\[
		\norm{A}_{L^2 \to L^2} \leq \norm{k}_{L^2([0,1] \times [0,1])}
	\]
	\textbf{Claim:} \text{    }$A \in K(L^2([0,1]),L^2([0,1]))$. \\
	Approximate $A$ by finite-ranked operators. \\
	Note: set $A = A_k$ and $B = A_{k_n}$ where $k_n$ is a nice function on $[0,1] \times [0,1]$ and
	\[
		A-B = A_k - A_{k_n} = A_{k-k_n}
	\]
	So \[
		\norm{A-B}_{L^2 \to L^2} \leq \norm{k - k_n}
	\]
	Set \begin{align*}
		I_f = [x_j - \frac{1}{N}, x_j], \qquad j = 1, \dots, N, \qquad x_j = \frac{j}{N} \\
		\tilde I_l =[y_l - \frac{1}{N}, y_j], \qquad l = 1, \dots, N, \qquad y_l = \frac{l}{N} 
	\end{align*}
	Set \[
		k_n(x,y) = \sum_{j=1}^{N} \sum_{l=1}^{N} k(x_j,y_l) \chi_{I_j}(x) \chi_{\tilde I_l}(y)
	\]
	where \[
		\chi_{I_j}(x) = \begin{cases}
			1, &\text{ if }x \in I_j\\
			0, &\text{elsewhere}
		\end{cases}
	\]
\end{beispiel}
Since $k \in C([0,1] \times [0,1])$ and $[0,1] \times [0,1]$ compact in $\mathbb{R}^2$ then $k$ is uniformly continous on $[0,1] \times [0,1]$. We fix $\varepsilon >0$. \\
\textbf{Claim:} \text{    }It exists an $N$ such that
\[
	\sup_{\substack{(x,y) \in  \\ [0,1] \times [0,1]}} \abs{k(x,y)- k_n(x,y)} < \infty
\]
\[
	A_{k_N}(f)(x) = \int_{0}^{1} k_N(x,y) f(y) \,\mathrm{d}y = \sum_{j=1}^{N} \underset{\text{scalar}}{\underbrace{\sum_{l=1}^{N} k(x_i,y_l) \int_{0}^{1} \chi_{\tilde I_l}(y) f(y) \,\mathrm{d}y \chi_{I_j}(x)}}
\]
\[
	\dim(\mathcal{R}(A_{k_N})) = N < \infty.
\]
Hence $A_{k_N} \in K(L^2([0,1]), L^2([0,1]))$ for all $N$. \\
Moreover 
\[
	\norm{A-A_{k_N}}_{L^2 \to L^2} \leq \norm{k - k_N}_{L^2([0,1] \times [0,1])} < \varepsilon
\] 
for $N$ large enough. $K(E,E)$ is a closed set in $(B(E,E), \norm{.}_{L^2 \to L^2})$ so $A \in K(L^2,L^2)$

\begin{beispiel}
	$(E,\skal{.}{.})$ Hilbert space, $(x_{n})_{n=1}^{\infty}$ ON-basis and $(\lambda_n)_{n=1}^{\infty}$ sequence of scalars. Set
\[
	T(x) = \sum_{n=1}^{\infty}\lambda_n \skal{x}{x_n} x_n, \qquad x \in E
\]	
\textbf{Claim:} \text{    }     
\begin{enumerate}[1)]
	\item $T \in B(E,E) \qquad \Leftrightarrow \qquad (\lambda_n)^{\infty}_{n=1} \text{ is a bounded sequence in $\mathbb{C}$}$.
	\item$T \in K(E,E) \qquad \Leftrightarrow \qquad \lambda_n \to 0 \text{ for }n \to \infty$.
\end{enumerate}
Note $x \in E$ and the Parseval's formula
\[
	\norm{x}^2 = \sum_{n=1}^{\infty} \abs{\skal{x}{x_n}}^2
\]
For $T(x) \in E$ we have
\[
	\norm{T(x)}^2 = \sum_{n=1}^{\infty} \abs{\lambda_n}^2 \abs{\skal{x}{x_n}}^2
\]
If $(\lambda_n)_{n=1}^{\infty}$ bounded sequence in $\mathbb{C}$. Then $\sup \abs{\lambda_n} \equiv M < \infty$ and
\[
	\norm{T(x)}^2 \leq \sum_{n=1}^{\infty} M^2 \abs{\skal{x}{x_n}}^2 = M^2 \norm{x}^2
\]
If $(\lambda_n)_{n=1}^{\infty}$ is not bounded then there exists a sequence $(\lambda_{n_k})_{k=1}^{\infty}$ such that $\abs{\lambda_{n_k}} \to \infty$ as $k \to \infty$. But
\[
	\norm{T(x_{n_k})} = \abs{\lambda_{n_k}}\norm{x_{n_k}} = \abs{\lambda_{n_k}} \to \infty, \qquad k \to \infty
\]
\[
	\sup\limits_{\norm{x}=1} \norm{T(x)} = \infty
\]
So 1) is done. For 2) we assume $\lambda_n \to 0$ for $n \to \infty$. Set 
\[
	T_k(x)= \sum_{n=1}^{k}\lambda_n \skal{x}{x_n}x_n, \qquad x \in E
\]
$T_k$ is a finite rank operator for $k=1,2,\dots$. SO $T_k \in K(E,E)$ for all $k$.
\begin{align*}
	\norm{T-T_k}_{E \to E} &= \sup_{\norm{x}=1} \norm{(T-T_k)(x)} \\ &= \sup_{\norm{x}=1} \norm{\sum_{k=n+1}^{\infty} \lambda_n \skal{x}{x_n}x_n} \\
	& \leq \sup\limits_{n= k+1,k+2,\dots}\abs{\lambda_n} \to 0, \qquad k \to \infty
\end{align*}
Assume $\lambda_n \not \to 0$ for $n \to \infty$. Then there exists $\varepsilon >0$ and a sequence $(\lambda_{n_k})_{k=1}^{\infty}$ such that 
\[
	\abs{\lambda_{n_k}} \geq \varepsilon
\]
Note
\[
	T(x_{n_k})= \lambda_{n_k}x_{n_k}, \qquad k =1,2,\dots
\]
\[
	\norm{T(x_{n_k})} = \abs{\lambda_{n_k}} \norm{x_{n_k}} = \abs{\lambda_{n_k}} \geq \varepsilon, \qquad k = 1,2,\dots
\]
$x_{n_k} \stackrel{\text{w}}{\to }0$ in $(E,\skal{.}{.})$ since for $y \in E$
\[
	\skal{x_{n_k}}{y} = \skal{x_{n_k}}{\sum_{n=1}^{\infty}\skal{y}{x_n}x_n} = \overline{\skal{y}{x_{n_k}}} \to 0
\]
since
\[
	\sum_{n=1}^{\infty}\abs{\skal{y}{x_n}}^2 = \norm{y}^2 < \infty
\]
If $T \in K(E,E)$ then $T(x_{n_k}) \to T(0)=0$ but \[
	\norm{T(x_{n_k})} \geq \varepsilon, \qquad \text{for all }k
\]
Hence
\[
	T \not \in K(E,E)
\]
\end{beispiel}
\begin{beispiel}
	$(E,\skal{.}{.})$ Hilbert space, $A \in K(E,E)$ and $I(x)=x$ for all $x \in E$. It follows
	\[
		\Rightarrow  \qquad R(I-A) \text{ closed in $E$}
	\]
\begin{bemerkung}
	\begin{align*}
		R(I-A)^{\perp} &= \mathcal{N}((I-A)^{*}) = \mathcal{N}(I-A^{*}) \\
		\overline{R(I-A)} &= R(I-A)^{\perp \perp} = \mathcal{N}(I-A^{*})^{\perp}
	\end{align*}
	If $A \in K(E,E)$ then
	\[
		\overline{R(I-A)} = R(I-A).
	\]
	Solve \[
		x = A(x) + y \qquad \Leftrightarrow \qquad (I-A)(x) = y
	\]
	Compare 'Fredholm alternative'.
\end{bemerkung}
\end{beispiel}
\begin{beweis}
	Take a sequence $(y_n)_{n \in \mathbb{N}} \subseteq R(I-A)$ such that $y_n \to y$ in $(E,\norm{.})$. \\
	To show: $y \in R(I-A)$, i.e. $y = (I-A)(x)$ for some $x \in E$ and $y_n = (I-A)(x_n)$ for some $x_n \in E$.
	\[
		x_n \in E = \mathcal{N}(I-A) + \mathcal{N}(I-A)^{\perp}
	\]
	such that
	\[
		x_n = \tilde x_n + \hat x_n 
	\]
	with
	\[
		\norm{x_n}^2 = \norm{\tilde x_n}^2 + \norm{\hat x_n}^2
	\]
	Step 1: Show $(\hat x_n)_{n=1}^{\infty}$ bounded in $E$. \\
	Step 2: $y_n = (I-A)( \hat x_n) = \hat x_n - A( \hat x_n)$. \\
\end{beweis}

%%% lecture 12
%%%lecture 12

\minisec{recall:}

$(E,\skal{.}{.})$ Hilbert space and $(x_n)_{n=1}^{\infty}$ ON-basis and $(\lambda_n)_{n=1}^{\infty}$ sequence of complex numbers.
Set 
\[
	A(x) = \sum_{n=1}^{\infty}\lambda_n \skal{x}{x_n}x_n.
\]
We have: 
\begin{itemize}
	\item $A: E \to E$ if $(\lambda_n)_{n=1}^{\infty} \in l^{\infty}$ \\
	if $(\lambda_n)_{n=1}^{\infty}$ is not bounded, there exists a subsequence $(\lambda_{n_k})_{k \in \mathbb{N}}$ such that
	\[
		\abs{\lambda_{n_k}} \geq k, \qquad k=1,2,\dots.
	\] 
	Set 
	\[
		x = \sum_{k=1}^{\infty}\frac{1}{k}x_{n_k}.
	\]
	Clearly $x \in E$ since $\left( \frac{1}{k} \right)_{k=1}^{\infty} \in l^{\infty}$. But
	\[
		T(x) = \sum_{k=1}^{\infty} \lambda_{n_k} \frac{1}{k} x_{n_k} \not \in E
	\]
	since $\left( \lambda_{n_k} \cdot \frac{1}{k} \right)_{k=1}^{\infty} \not \in l^2$. \\
	Note
	\[
		A \in B(E,E) \qquad \Leftrightarrow \qquad (\lambda_n)_{n=1}^{\infty} \in l^{\infty}
	\]
	and $\norm{A} = \sup_{n} \abs{\lambda_n}$.
	\item $A \in K(E,E)$ iff $\lambda_n \to 0$ for $n \to \infty$.
	\item $A$ is self adjoint iff $\lambda_n \in \mathbb{R}$ for all $n \in \mathbb{N}$.
\end{itemize}
\minisec{Basis facts:}
Set $A \in B(E,E)$ where $(E,\skal{.}{.})$ is a Hilbert space. Then:
\begin{itemize}
	\item If $A$ is self-adjoint we have
	\[
		\norm{A} = \sup_{\norm{x}=1} \abs{\skal{A(x)}{x}}.
	\]
	\item If $A$ is self-adjoint it follows
	\[
		\skal{A(x)}{x} \in \mathbb{R}, \qquad \forall\, x \in E
	\]
	since
	\begin{align*}
		\skal{A(x)}{x} = \skal{x}{A^*(x)} \stackrel{\text{self-adjoint}}{=} \skal{x}{A(x)} = \overline{\skal{A(x)}{x}}.
	\end{align*} 
	\item $K(E,E)$ (Set of all compact linear operators) closed subspace in $(B(E,E),\norm{.}_{E \to E})$.
	\item $A \in K(E,E)$ and $x_n \rightharpoonup x$ in $E$. Then
	\[
		A(x_n) \to A(x), \qquad \text{in }E.
	\]
	\item $A \in K(E,E)$ and $B \in B(E,E)$. Then
	\begin{itemize}
		\item $AB, BA \in K(E,E)$,
		\item $A^* \in K(E,E)$,
		\item $\mathcal{R}(B)^{\perp} = \mathcal{N}(B^*)$ \\
		$\overline{\mathcal{R}(B)} = \mathcal{N}(B^*)^{\perp}$,
		\item $\mathcal{R}(I-A)$ is a closed subspace in $E$.
	\end{itemize}
	\item $E = \mathcal{R}(I-A) \oplus \mathcal{R}(I-A)^{\perp} = \mathcal{R}(I-A) \oplus \mathcal{N}(I-A^*)$.
	\item For any $A \in K(E,E)$
	\[
		\dim(\mathcal{N}(I-A)) = \dim(\set[x \in E]{x-A(x)=0}) < \infty
	\]
	since: if $\dim(\mathcal{N}(I-A))= \infty$ then there exists an ON- sequence $(x_n)_{n=1}^{\infty}$ in $\mathcal{N}(I-A)$. Then
	\[
		x_n \rightharpoonup E, \qquad \text{since } \skal{x_n}{y} \to 0, n \to \infty
	\]
	since for $y \in \overline{\spn\set[x_n]{n=1,2,\dots}}$ then
	\[
		\norm{y}^2 = \sum_{n=1}^{\infty} \abs{\skal{x_n}{y}}^2 < \infty.
	\]
	$A \in K(E,E)$ implies that $A(x_n) \to A(0)=0$ in $E$. But
	\[
		x_n = A(x_n) \to 0 \qquad \text{in }E, \qquad \norm{x_n}=1 \text{ for all }n
	\]
	This is a contradiction. \\
	Conclusion: $\dim(I-A) < \infty$.
\end{itemize}
From above we have for $A \in K(E,E)$ \[
	E = \mathcal{R}(I-A) \oplus \mathcal{N}(I-A^*).
\]
Consider the equation
\[
	x = A(x)+ y \qquad \qquad (1).
\]
(1) has a solution provided by $y \in \mathcal{R}(I-A)$. That is the case if $y \perp z$ for all $z \in \mathcal{N}(I-A^*)$. \\
Since $\dim(\mathcal{N}(I-A^*)) < \infty$, this is just finitely many conditions.


\begin{theorem}[Fredholm alternativ]
$A \in K(E,E)$ where $E$ is a Hilbert space. then exactly one of the statements below holds:
\begin{enumerate}
	\item $x = A(x) + y$ is solvable for every $y \in E$.
	\item $x = A(x)$ has a non trivial solution $x \in E$, i.e. $x \neq 0$.
\end{enumerate}
(No assumption on $A$ being self-adjoint.)
\end{theorem}
\begin{bemerkung}
	The statement in Fredholm Alternativ also holds if $(E,\norm{.})$ is a Banach space.
\end{bemerkung}
\begin{beweis}
	\begin{description}
		\item[(1) $\Rightarrow$ $ \lnot$ (2):] We want to show that there are no non-trivial solutions for $x = A(x)$. \\
		Assume that there exists a non-trivial solution $x_1 \in E$ to $x = A(x)$, i.e.
		\[
			(I-A)(x_1)=0, \qquad \text{with }x_1 \neq 0.
		\]
		If $(1)$ holds true there exists a $x_2 \in E$ such that
		\[
			(I-A)(x_2) = x_1 \neq 0.
		\]
		But
		\[
			(I-A)(x_1) = (I-A)^2(x_2) = 0.
		\]
		With $(1)$ there exists $x_3 \in E$ such that 
		\[
			(I-A)(x_3) = x_2
		\]
		which implies
		\[
			(I-A)^2(x_3) = (I-A)(x_2) = x_1 \neq 0.
		\]
		But once again
		\[
			(I-A)^3(x_3)=0.
		\]
		Proceed inductively gives us a sequence $(x_k)_{k=1}^{\infty}$ such that
		\[
			(I-A)^k(x_k) =0, \qquad \text{but }(I-A)^{k-1}(x_k) \neq 0.
		\]
		We obtain
		\[
			\mathcal{N}(I-A) \subsetneq \mathcal{N}((I-A)^2) \subsetneq \mathcal{N}((I-A)^3) \subsetneq \dots.
		\]
		This is a sequence of proper closed subspaces.  \\
		Apply now Riesz-Lemma: \\
		There exists a sequence $(y_k)_{k=1}^{\infty}$ with $\norm{y_k} = 1$ and $\norm{y_k - x} \geq \frac{1}{2}$ for all $x \in \mathcal{N}((I-A)^{k-1})$ and 
		$y_k \in \mathcal{N}((I-A)^k)$. \\
		\textbf{Claim:} \text{    }$\norm{A(y_n)- A(y_m)} \geq \frac{1}{2}$ for all $n >m$. \\
		\begin{align*}
			\norm{A(y_m)- A(y_n)} &= \norm{\underset{\in \mathcal{N}((I-A)^{n-1})}{\underbrace{(I-A)(y_n)}} - y_n + \underset{\in \mathcal{N}((I-A)^{n-1})}{\underbrace{A(y_m)}}} \\
			&=  \norm{y_n - \underset{\in \mathcal{N}((I-A)^{n-1})}{\underbrace{((I-A)(y_n) + A(y_m))}}} \geq \frac{1}{2}.
		\end{align*}
		So $(A(y_n))_{n=1}^{\infty}$ can not converge in $E$. But $A$ is compact and $\norm{y_n}=1$ for all $n$. This is a contradiction. \\
		Conclusion: There is no non-trivial solution of $A(x)=x$.
		\item[$\lnot$ (2) $\Rightarrow$ (1)] Assume that $x = A(x)$ has no non-trivial solution $x \in E$. We want to show that $(1)$ holds.
		\[
			E = \mathcal{R}(I-A^*) \oplus \mathcal{N}(I-A), \qquad \text{ with }\mathcal{N}(I-A) = \set{0}.
		\]
		Hence 
		\[
			x = A^*(x)+y
		\] is solvable for every $y \in E$. From the first part of the proof it follows that 
		\[
			\mathcal{N}(I-A^*) = \set{0}.
		\]
		But then
		\[
			E = \mathcal{R}(I-A) \oplus \mathcal{N}(I-A^*) = \mathcal{R}(I-A).
		\]
		Conclusion: $x = A(x) + y$ is solvable for all $y \in E$.
	\end{description}
\end{beweis}

\begin{beispiel}
	$L^2([0,1])$, $k \in C([0,1] \times [0,1])$ and
	\[
		A(f)(x) = \int_{0}^{1}k(x,y)f(y) \,\mathrm{d}y , \qquad x \in [0,1].
	\]
	Then
	\begin{itemize}
		\item $A \in B(L^2,L^2)$ with $\norm{A}_{L^2 \to L^2} \leq \norm{k}_{L^2([0,1] \times [0,1])}$,
		\item $A$ self-adjoint if $k(x,y) = \overline{k(y,x)}$ for all $x,y \in [0,1]$,
		\item $A \in K(E,E)$ (by approximation by finite rank operators). 
	\end{itemize}
\end{beispiel}

\begin{theorem}[Hilbert-Schmidt-Theorem]
	$(E, \skal{.}{.})$ Hilbert spaces and $A \in K(E,E)$ self adjoint. Then there exists a sequence of non-zero eigenvalues of $A$ denoted
	$(\lambda_n)_{n=1}^{N}$ for $N$ finite or infinite, corresponding to eigenvectors $(u_n)_{n=1}^{N}$. Respectively where $(u_n)_{n=1}^{N}$ is an ON-sequence, and
	\[
		\abs{\lambda_1} \geq \abs{\lambda_2} \geq \dots
	\]
	with 
	\[
		\lim_{n \to \infty} \lambda_n = 0, \qquad \text{if } N= \infty
	\]
	such that for $x \in E$
	\[
		x = \sum_{n=1}^{N}\skal{x}{u_n}u_n +v, \qquad v \in \mathcal{N}(A).
	\]
	Moreover
	\[
		A(x) = \sum_{n=1}^{N} \lambda_n \skal{x}{u_n}u_n.
	\]
\end{theorem}
\begin{bemerkung}
	With notation from the theorem above we have
	\begin{enumerate}
		\item \[
		A^k(x) = \sum_{n=1}^{N} \lambda_n^k \skal{x}{u_n}u_n, \qquad k=1,2,\dots.
	\]
	\item If $A$ is injective, i.e. $\mathcal{N}(A) = \set{0}$ then the Eigenvectors $(u_n)_{n=1}^{N}$ form an ON-basis for $E$.
	\end{enumerate}
\end{bemerkung}
\begin{definition*}[Eigenvalues and Eigenvectors for $A \in B(E,E)$]
	$\lambda \in \mathbb{C}$ is called an eigenvalue of $A$ if there exists an $0 \neq x \in E$ such that
	\[
		A(x) = \lambda x.
	\]
\end{definition*}
\begin{bemerkung}[properties for Eigenvalues and Eigenvectors]
	\begin{enumerate}
		\item $\abs{\lambda} \leq \norm{A}$ since
		\[
			\abs{\lambda} \norm{x} =\norm{\lambda x} = \norm{A(x)} \leq \norm{A} \cdot \norm{x}.
		\]
		\item $A$ self-adjoint and $\lambda$ eigenvalue. Then
		\[
			\Rightarrow \lambda \in \mathbb{R}
		\]
		since
		\begin{align*}
			\lambda \skal{x}{x} &= \skal{\lambda x}{x} \\ &= \skal{A(x)}{x} \\ &= \skal{x}{A^*(x)} \\ &= \skal{x}{A(x)} \\ &= \skal{x}{\lambda x} \\ & = \bar{\lambda} \skal{x}{x}.
		\end{align*}
		So \[
			\lambda = \bar{\lambda}, \qquad \Rightarrow \lambda \in \mathbb{R}.
		\]
		\item $A$ self-adjoint, $A(x) = \lambda x$ and $A(y) = \mu y$, where $x,y \neq 0$ and $\lambda \neq  \mu $.
		\[
			\Rightarrow \,x \perp y
		\]
		since
		\[
			\lambda \skal{x}{y} = \dots = \bar{\mu} \skal{x}{y}.
		\]
		So
		\[
			\underset{\neq 0}{\underbrace{( \lambda - \mu)}}\skal{x}{y} = 0.
		\]
		\item $A \in K(E,E)$ and $\lambda \neq 0$ eigenvalue of $A$. Then
		\[
			\dim E_\lambda = \dim \set[x \in E]{A(x)= \lambda x} < \infty. 
		\]
	\end{enumerate}
\end{bemerkung}
\begin{proposition}
	$(E, \skal{.}{.})$ Hilbert space and $A \in K(E,E)$ self-adjoint. Then
	\[
		\Rightarrow \,\norm{A} \qquad \text{or} \qquad  - \norm{A}
	\]
	is an eigenvalue of $A$.
\end{proposition}
\begin{beweis}
	$A = 0$ then the statement is trivial. \\
	Assume $A \neq 0$. \\
	$A$ self-adjoint implies that 
	\[
		\norm{A} = \sup_{\norm{x}=1} \abs{\skal{A(x)}{x}}.
	\]
	Also self-adjoint implies that for all $x \in E$ we have
	\[
		\skal{A(x)}{x} \in \mathbb{R}.
	\]
	Hence there exists a sequence $(x_n)_{n=1}^{\infty}$ in $E$ with $\norm{x_n}=1$ for all $n$ such that
	\[
		\skal{A(x_n)}{x_n} \to \lambda, \qquad n \to \infty.
	\]
	where $\lambda \in \mathbb{R}$ and $\abs{\lambda} = \norm{A}$. \\
	\textbf{Claim:} \text{    }$A(x_n) - \lambda x_n \to 0$ in $E$. \\
	\begin{align*}
		\norm{A(x_n)- \lambda x_n}^2 &= \skal{A(x_n)-\lambda x_n}{A(x_n)- \lambda x_n} \\
		&= \underset{\substack{= \norm{A(x_n)}^2 \\ \leq \norm{A}^2 \norm{x_n}^2 \\ = \norm{A}^2}}{\underbrace{\skal{A(x_n)}{A(x_n)}}}- 
		\overset{\to \abs{\lambda}^2 = \norm{A}^2}{\overbrace{\bar{\lambda} 
		\underset{ \to  \lambda}{\underbrace{\skal{A(x_n)}{x_n}}}}}- \overset{\to \lambda^2 = \norm{A}^2}{\overbrace{\lambda \underset{\to \lambda}{\underbrace{\skal{x_n}{A(x_n)}}}}} + \underset{= \norm{A}^2}{\underbrace{\abs{\lambda}^2}} \underset{=1}{\underbrace{\skal{x_n}{x_n}}} \\
		&\to 0, \qquad n \to \infty.
	\end{align*}
	$A \in K(E,E)$ and $\norm{x_n}=1$ for all $n$ we get that 
	\[
		(A(x_n))_{n=1}^{\infty} 
	\]
	has a converging subsequence $(A(x_{n_k})_{k=1}^{\infty}$ in $E$. \\
	Call the limit element $y \in E$ so
	\[
		A(x _{n_k}) \to y \qquad \text{ in }E. 
	\]
	\[
		\begin{cases}
			A(x_n) - \lambda x_n &\to 0\\
			A(x _{n_k}) &\to y
		\end{cases} \qquad \text{in }E
	\]
	implies
	\[
		x _{n_k} \to  \frac{1}{\lambda}y \qquad \text{ in }E
	\]
	(note $\abs{\lambda}>0$ since $A \neq 0$). \\
	Set $x = \frac{1}{\lambda}y$. So $x _{n_k} \to x$ in $E$. Consider
	\begin{align*}
		\norm{A(x)- \lambda x} &\leq  \norm{A(x)-A(x _{n_k})} + \norm{A(x _{n_k})- y} \to 0, \qquad k \to \infty
	\end{align*}
	Conclusion: \[
		A(x) = \lambda x.
	\]
	where $\norm{x}=1$ since $1 = \norm{x _{n_k}} \to  \norm{x}$ as $k \to \infty$.
\end{beweis}
We are now going to prove the Hilbert-Schmidt theorem:

\begin{beweis}
	If $A=0$ the theorem is trivial. \\ Assume $A \neq 0$. \\
	By the proposition above there exists an eigenvalue $\lambda_1$ of $A$ with $\abs{\lambda_1}= \norm{A}$ and an eigenvector $u_1$ with 
	$\norm{u_1}=1$ corresponding to the eigenvalue $\lambda_1$. \\
	Set $Q_1 = \set{u_1}^{\perp}$. $Q_1$ is a closed subspace of $E$ and hence $Q_1$ is a Hilbert space. \\
	For $x \in Q_1$ we have $A(x) \in Q_1$ since for $x \in Q_1$ we have
	\begin{align*}
		\skal{A(x)}{u_1} &= \skal{x}{A^*(u_1)}  \\ &= \skal{x}{A(u_1)} \\&= \skal{x}{\underset{\in  \mathbb{R}}{\underbrace{\lambda_1}} u_1} \\
		&= \lambda_1 \skal{x}{u_1} = 0.
	\end{align*}
	Now
	\[
		A  \big|_{Q_1}^{} : Q_1 \to Q_1
	\]
	is compact and also self-adjoint. By proposition above there exists an eigenvalue $\lambda_2$ of $A  \big|_{Q_1}^{}$ and a corresponding eigenvector $u_2$ with $\norm{u_2}=1$ where
	\[
		\abs{\lambda_2} = \norm{A  \big|_{Q_1}^{}} \leq \norm{A} = \abs{ \lambda_1}.
	\]
	Here $A(u_2) = \lambda_2 u_2$ so $\lambda_2$ is an eigenvalue of $A$. Set $Q_2 = \set{u_1,u_2}^{\perp}$. $Q_2$ is a closed subspace of $E$ and we have
	\[
		x \in Q_2 \qquad \Rightarrow \qquad A(x) \in Q_2 
	\]
	since $x \in Q_2$ we have
	\begin{align*}
		\skal{A(x)}{u_1} &= \skal{x}{A(u_1)} = \skal{x}{\lambda_1 u_1} = 0 \\
		\skal{A(x)}{u_2} &= \skal{x}{A(u_2)} = \skal{x}{\lambda_2 u_2} = 0.
	\end{align*}
	Proceed inductively.
	\begin{description}
		\item[Case 1:] For a positive integer $k$ we have 
		\[
			\abs{\lambda_1} \geq  \abs{\lambda_2} \geq  \dots \geq  \abs{\lambda_k}>0
		\]
		with corresponding eigenvectors $u_1,u_2, \dots,u_k$ but $A  \big|_{Q_k}^{}$ with $Q_k = \set{u_1,u_2,\dots,u_k}^{\perp}$, then is the zero-mapping $Q_k \to Q_k$. This corresponds to $N=k$ and 
		\[
			x = \sum_{n=1}^{k} \skal{x}{u_n}u_n + v, \qquad \text{where }v \in \mathcal{N}(A).
		\]
		\item[Case 2:] The process never terminates. We get 
		\[
			\abs{\lambda_1} \geq \abs{\lambda_2} \geq \dots \geq \abs{\lambda_n} \geq \dots
		\]
		with corresponding eigenvectors $u_1,u_2, \dots,u_n, \dots$. \\
		We have $(u_n)_{n=1}^{\infty}$ ON-sequence in $E$ corresponding to the non-zero eigenvalue $(\lambda_n)_{n=1}^{\infty}$. $A \in K(E,E)$ and $u_n \to 0$ in $E$ since 
		$(u_n)_{n=1}^{\infty}$ is ON-sequence. \\
		Then this implies $A(u_n) \to 0$ in $E$. So \[
			\abs{\lambda_n} = \norm{\lambda_n u_n} = \norm{A(u_n)} \to 0, \qquad n \to \infty.
		\]
		Hence \[
			\lim_{n \to \infty} \lambda_n = 0.
		\]
		Set 
		\[
			S := \overline{\spn\set{u_1, \dots,u_n, \dots}} = \set[\sum_{k=1}^{\infty}a_n u_n]{ (a_n)_{n=1}^{\infty} \in l^\infty}.
		\]
		S is a closed subspace of $E$. \\ We have $E = S \oplus S^{\perp}$ where $S^{\perp} \subseteq Q_k = \set{u_1,\dots,u_k}^{\perp}$ for all $k \in \mathbb{N}$.
		For $x \in E$ we have
		\[
			\underset{\in S}{\underbrace{\sum_{k=1}^{\infty} \skal{x}{u_k}u_k}} + \underset{\in S^{\perp}}{\underbrace{v}}
		\]
		since $(\skal{x}{u_k})_{k=1}^{\infty} \in l^\infty$ by Bessel's inequality. 
		To show: $A(v)=0$. Clearly, $v \in Q_k$ for all $k$. If $v = 0$ there is nothing to prove. For $v \neq 0$ set $w= \frac{1}{\norm{v}}v$ and get
		\begin{align*}
			\abs{\skal{A(v)}{v}} &= \norm{v}^2 \abs{ \skal{A(w)}{w}} \\
			&\leq \norm{v}^2 \underset{= \norm{A  \big|_{Q_k}^{}} = \abs{\lambda_{k+1}} \to 0}{\underbrace{\sup_{ \substack{\norm{z}=1 \\ z \in Q_k}} \abs{\skal{A(z)}{z}}}}
		\end{align*}
		\textbf{Claim:} \text{    }$A  \big|_{S^{\perp}}^{} = 0$ and hence $v \in S^{\perp}$ implies $A(v) = 0$.
	\end{description}
\end{beweis}

\begin{theorem}[Spectral mapping theorem]
	$(E,\skal{.}{.})$ seperable Hilbert space and $\infty$-dimensional $A \in K(E,E)$ self-adjoint. Then there exists a ON-basis of eigenvectors $(\tilde u_n)_{n=1}^{\infty}$ corresponding to the eigenvalues $(\tilde \lambda_n)_{n=1}^{\infty}$ if $A$ where $\lim_{n \to \infty} \tilde \lambda_n = 0$.
\end{theorem}
\begin{beweis}[consequence of HS-theorem]
	We have by HS-theorem an ON-sequence $(u_n)_{n=1}^{\infty}$ of eigenvectors corresponding to the non-zero eigenvalues $(\lambda_n)_{n=1}^{N}$. \\
	Set
	\[
		S = \overline{\spn\set{u_1,\dots,u_n, \dots}}.
	\]
	$E$ is seperable implies $E$ has an ON-basis $(v_n)_{n=1}^{\infty}$. By Gram-Schmidt Orthoganlization procedure we can obtain an ON-basis $(w_n)_{n=1}^{M}$ for $S^{\perp}$. Have $M$ finite or infinite. \begin{align*}
		S &:\,u_1,u_2, \dots \qquad \text{ON-basis finite or infinite} \\
		S^{\perp} &:\,w_1,w_2, \dots \qquad \text{ON-basis finite or infinite}
	\end{align*}
	Consider the ON-sequence $u_1,w_1,u_2,w_2, \dots = \tilde u_1, \tilde u_2, \dots$. This gives an ON-basis for $E$ consisting of eigenvectors to $A$. Also
	\[
		\lim_{n \to \infty}\tilde \lambda_n = 0.
	\]
\end{beweis}

\begin{itemize}
	\item $(E,\skal{.}{.})$ complex Hilbert space.
	\item $A \in \mathcal{B(E,E)}$. 
	\item Consider the equation
\[
	x = A(x)+ y, \qquad y \in E.
\]
\[
	(I-A)(x)= y.
\]
\item Consider this problem for $\lambda \in \mathbb{C}$. \\
\item Set \[
	\rho(A) := \set[\lambda \in \mathbb{C}]{(A- \lambda I)^{-1} \in \mathcal{B}(E,E)}
\]
\item $\rho(A)$ is called the resolvent set for $A$.
\item Set \[
	\sigma(A) = \mathbb{C} \setminus \rho(A).
\]
\item $\sigma(A)$ is called the spectrum of $A$.
\item Clearly, a necessary condition for $(A-\lambda I)^{-1} \in \mathcal{B}(E,E)$ is that 
\[
	A - \lambda I:E \to E
\]
is a bijection.
\item Linearity for $(A-\lambda I)^{-1}$ follows from the linearity of $A-\lambda I$.

\end{itemize}
\begin{theorem}[Banachs's inverse mapping theorem]
	$(E,\norm{.})$ Banach space, $A \in \mathcal{B}(E,E)$. $A-\lambda I: E \to E$ bijection. Then
	\[
		\Rightarrow (A- \lambda I)^{-1} \in \mathcal{B}(E,E)
	\]
\end{theorem}
\begin{beweis}
	based on the Open mapping theorem. Proof is omitted. Assume $\lambda \in \sigma(A)$. Then $A-\lambda I:E \to E$ is not a bijection.

\begin{itemize}
	\item If $A-\lambda I: E \to E$ is not injective then there exists $0 \neq x \in  E $ such that
	\[
		(A- \lambda I)(x) = 0,
	\]
	i.e. $\lambda$ is an eigenvalue of $A$.
	Set	
	\[
		\sigma_p(A) = \set[\lambda \in \mathbb{C}]{\lambda \text{eigenvalue of $A$}}.
	\]
	\item If $A-\lambda I$ is injective, densely defined but not bounded then $\lambda \in \sigma(A)$. The set of such $\lambda$'s is called the continuous spectrum of $A$, denoted $sigma_c(A)$
	\item If $A-\lambda I$ is not surjective then the set of such $\lambda$'s is called the residual spectrum, denoted $\sigma_r(A)$.
\end{itemize}
\end{beweis}
 \begin{lemma*}
 	$(E,\norm{.})$ Banach space, $A \in \mathcal{B}(E,E)$ with $\norm{A} < 1$. Then
	\[
		(I-A)^{-1} \in \mathcal{B}(E,E)
	\]
	and
	\[
		(I-A)^{-1} = I + \sum_{n=1}^{\infty} A^n.
	\]
	This series is called a Neumannseries. 
 \end{lemma*}
\begin{beweis}
	Observe
	\[
		\norm{A^n} = \norm{A \cdot A \cdots A} \leq \norm{A}^n, \qquad n = 1,2,\dots
	\]
	and
	\[
		\sum_{n=1}^{\infty}\norm{A^n}< \infty.
	\]
	Since $E$ is a Banach space we have
	\[
		\sum_{n=1}^{\infty}A^n 
	\]
	converges in $\mathcal{B(E,E)}$. Since $E$ Banach space implies $\mathcal{B}(E,E)$ is a Banach space. \\
	Note
	\[
		(I-A)\left( I+ \sum_{n=1}^{N} A^n \right) = I-A^{N+1} \to I, \qquad \text{in }\mathcal{B}(E,E).
	\]
	\[
		\left(I + \sum_{n=1}^{N}A^n \right)(I-A) = I-A^{N+1} \to I, \qquad \text{in }\mathcal{B}(E,E).
	\]
	We get
	\[
		\left( I + \sum_{n=1}^{\infty} \right)(I-A) = I = (I-A)(I + \sum_{n=1}^{\infty}A^n).
	\]
	We have $(I-A)^{-1} $ exists and is equal to $I+ \sum_{n=1}^{A^n}$.
\end{beweis}
 \begin{lemma*}
 	$(E,\norm{.})$ Banach space and $A \in \mathcal{B}(E,E)$. Then
	\begin{enumerate}
		\item $\sigma(A) \neq \emptyset$.
		\item $\sigma(A)$ closed set in $\mathbb{C}$.
		\item $\sigma(A) \subseteq \overline{B(0,\norm{A})}$
	\end{enumerate}
 \end{lemma*}
\begin{beweis}
	\begin{enumerate}
		\item omitted.
		\item Enough to prove that $\rho(A)$ is an open set in $\mathbb{C}$. \\
		Fix $\lambda_0 \in \rho(A)$. So $(A- \lambda_0 I)^{-1} \in \mathcal{B}(E,E)$. \\ Note:
		\begin{align*}
			A-\lambda I &= A - \lambda_0 I - (\lambda - \lambda_0)I \\
			&= \underset{\substack{\text{invertible} \\ \text{ since }\lambda_0 \in \rho(A)}}{\underbrace{(A-\lambda_0 I)}} \underset{\substack{ \text{ invertible if} \\ \norm{(\lambda-\lambda_0)(A- \lambda_0 I)^{-1}}<1 \\ \text{by previous lemma, i.e.} \\ \abs{\lambda - \lambda_0} < \frac{1}{\norm{(A- \lambda_0 I)^{-1}}}}}{\underbrace{\left( I - (\lambda - \lambda_0)(A- \lambda_0 I)^{-1} \right)}}.
		\end{align*}
		Clearly, $A-\lambda I$ is invertible if
		\[
			\abs{\lambda - \lambda_0} < \frac{1}{\norm{(A- \lambda_0 I)^{-1}}}.
		\]
		\item It is enough to show that $\lambda \in \rho(A)$ if \[
			\abs{\lambda} > \norm{A}.
		\]
		Note 
		\[
			A- \lambda I = - \lambda (I - \frac{1}{\lambda}A).
		\]
		Here 
		\[
			\norm{- \frac{1}{\lambda}A} = \frac{1}{\abs{\lambda}} \norm{A} < 1.
		\]
		$I - \frac{1}{\lambda}A$ is invertible by previous lemma. So $\rho(A)$.
	\end{enumerate}
\end{beweis}
Now assume $(E,\skal{.}{.})$ is a complex Hilbert space with infinite dimension. $A \in \mathcal{K}(E,E)$ (We don't assume $A$ is self-adjoint). Then
\begin{enumerate}
	\item $\lambda \in \sigma(A) \setminus \set{0}$ $\qquad $ $\Rightarrow$ is an eigenvalue of $A$. 
	\item $\lambda \in \sigma(A) \setminus \set{0}$ $\qquad $ $\Rightarrow $ $\dim \set[x \in E]{A(x) = \lambda x} < \infty$.
	\item $O$ is the only cluster point for $\sigma(A)$ 
	\item $0 \in \sigma(A)$ since if $0 \not \in \sigma(A)$ then $A^{-1} \in \mathcal{B}(E,E)$ and \[
		\underset{\in \mathcal{K}(E,E)}{\underbrace{\underset{\in \mathcal{K}(E,E)}{\underbrace{A}} \underset{\in \mathcal{B}(E,E)}{\underbrace{A^{-1}}}}} = I.
	\]
	But $I \not \in \mathcal{K}(E,E)$ since $E$ $\infty$-dimensional. Just take an ON-sequence $(x_n)_{n=1}^{\infty}$ in $E$. Then \[
		x_n \rightharpoonup 0, \qquad \text{ in }E
	\]
	but $\norm{x_n}=1$ for all $n$ and if $I \in \mathcal{K}(E,E)$ then 
	\[
		x_n = I(x_n) \to I(0) = 0, \qquad  \text{ in }E
	\]
	which implies that $\norm{x_n}\to 0$ for $n \to \infty$. Moreover (by Hilbert-Schmidt theorem) $(E, \skal{.}{.})$ complex Hilbert space, seperable and $\infty$-dim. $A \in \mathcal{K}(E,E)$ and self-adjoint it follows
	\[
		\Rightarrow \qquad (u_n)_{n=1}^{\infty} \text{ ON-basis for $E$ where}
	\]
	\[
		A(u_n) = \lambda_n u_n, \qquad n=1,2,\dots.
	\]
	($\lambda_n$ eigenvalue of $A$ with normalised eigenvector $u_n$) with
	\[
		\lim_{n \to \infty}\lambda_n = 0.
	\]
	For $x \in E$
	\[
		x = \sum_{n=1}^{\infty} \skal{x}{u_n}u_n
	\]
	and
	\[
		A(x) = \sum_{n=1}^{\infty}\lambda \skal{x}{\lambda_n}u_n
	\]
\end{enumerate}
\minisec{Fredholm Alternativ:}
$E,A$ as above. Then
\begin{enumerate}
	\item $x = A(x) + y$ is seperable for all $y \in E$. \\
	iff
	\item $x = A(x)$ has no non-trivial solution $x \in E$.
\end{enumerate}
Exactly one of the statements hold:
\begin{enumerate}
	\item (1) from above
	\item (2) has no non-trivial solution $x \in E$.
\end{enumerate}
In general (1) is seperable for $y \in E$ iff 
\[
	y \in \set[x \in E]{A(x)=x}^{\perp}.
\]
If so: If $x$ is a solution to (1) then also $x + \tilde x$ is a solution to (1) where
\[
	\tilde x \in \set[x \in E]{A(x)=x}
\]
\begin{beweis}
	Look at (1). Let $(u_n)_{n=1}^{\infty}$ be the ON-basis from the previous theorem.
	\[
		x = \sum_{n=1}^{\infty}\skal{x}{u_n}u_n, \qquad y = \sum_{n=1}^{\infty}\skal{y}{u_n}u_n.
	\]
	\[
		A(x) = \sum_{n=1}^{\infty} \lambda_n \skal{x}{u_n}u_n.
	\]
	(1) taked the form
	\[
		\sum_{n=1}^{\infty} \left( \skal{x}{u_n} - \lambda_n \skal{x}{u_n} - \skal{y}{u_n} \right) u_n = 0.
	\]
	This implies 
	\[
		(I- \lambda)\skal{x}{u_n}- \skal{y}{u_n} = 0, \qquad n=1,2,\dots.
	\]
	If $\lambda_n \neq 1$ then
	\[
		\skal{x}{u_n} = \frac{\skal{y}{u_n}}{1 - \lambda_n}.
	\]
	If $\lambda_n = 1$ then $-y$ must be orthogonal to every $u_n$ corresponding by the eigenvalue $1$.
	\[
		\sum_{n=1}^{\infty} \frac{\skal{y}{u_n}}{1- \lambda_n}u_n \in E
	\]
	since
	\[
		(\frac{\skal{y}{u_n}}{1- \lambda_n})_{n=1}^{\infty} \in l^2
	\]
	since
	\[
		\sup_{\substack{n \\ \lambda_n \neq 1}} \abs{\frac{1}{1- \lambda_n}} < \infty
	\]
	since
	\[
		\lim_{n \to \infty}\lambda_n = 0
	\]
	and
	\[
		(\skal{y}{u_n})_{n=1}^{\infty} \in l^2.
	\]
\end{beweis}

\section{Boundary Value Problems for ODE's} 
\label{sec:boundary_value_problems_for_ode_s}
Consider
\[
	(*)\qquad \begin{cases}
		Lu &=f \in C([0,1]) \\
		R_ju &= 0 \qquad j=1,2,\dots,n 
	\end{cases}
\]
(homogenuous boundary conditions),
where
\[
	Lu := u^{(n)} + C_{n-1}(x)u^{(n-1)} + \dots + c_1(x)u' + c_0(x)u, \qquad u \in C^n([0,1])
\]
with
\[
	c_0(x),c_1(x), \dots, c_{n-1}(x) \in C([0,1])-
\]
\[
	R_j = \sum_{k=0}^{n-1} \left( \alpha_{jk} u^{(k)}(0) + \beta_{jk}u^{(k)}(1) \right), \qquad j=1,2,\dots,n
\]
with
\[
	\alpha_{jk}, \beta_{jk} \in \mathbb{C}, \qquad j=1,\dots,n, \qquad k=0,\dots,n-1
\]
Reformulate (*).
\[
	u(x) = \int_{0}^{1} \underset{\substack{\text{Green's function} \\ \text{for $L$ and $R_j$} \\ j=1,\dots,n}}{\underbrace{g(x,y)}}f(y) \,\mathrm{d}y \qquad \in C^n([0,1])
\]
and satisfies the boundary conditions $R_j = 0$ for $j=1,2,\dots,n$. \\
Consider the problem
\[
(**) \qquad 	\begin{cases}
		Lu &= f(x,u), \qquad x \in [0,1) \\
		R_ju &= 0, \qquad j=1,2,\dots,n.
		
	\end{cases}
\]
The reformulation above gives
\[
	u(x) = \int_{0}^{1}g(x,y)f(y,u(y)) \,\mathrm{d}y, \qquad x \in [0,1].
\]
To find a solution set
\[
	T(u)(x) = \int_{0}^{1}g(x,y)f(y,u(y)) \,\mathrm{d}y, \qquad x \in [0,1].
\]
\[
	T: C([0,1]) \to C([0,1])
\]
A fixed point to $T$ gives a solution to $(**)$. Note that if $u \in C([0,1])$ then
\[
	T(u) \in C^n([0,1]) 
\]
and satisfies $R_j=0$ for $j=1,2,\dots$. \\
Given $L$ and $R_j$ for $j=1,2,\dots,n$ find the corresponding Green's function.

\begin{beispiel}
	\[
		\begin{cases}
			Lu &= u''-u, \qquad \text{ on }[0,1]\\
			R_1u &=u(0)=0 \\
			R_2u &=u(1)=0
		\end{cases}
	\]
\end{beispiel}
	\begin{theorem}
		$Lu=f \in C([0,1])$, where
		\[
			Lu: = u^{(n)}+ c_{n-1}(x)u^{(n-1)} + \dots + c_1(x)u' + c_0(x)u
		\]
		and $\xi = (\xi_1, \dots, \xi_n) \in \mathbb{C}^n$. Then for $x_0 \in [0,1]$
		\[
			\Rightarrow \qquad \exists\,!\,\,u \in C^n([0,1]) \text{ with } Lu=f.
		\]
		and 
		\[
			(u,u',\dots,u^{(n-1)}) \big|_{x_0}^{} = \xi.
		\]
	\end{theorem}
	\begin{beweis}
		Reformulate the problem as a system of first order differential equations.
	\[
		\begin{cases}
			Lu &=f \\
			(u,u',\dots,u^{(n-1)})  \big|_{x_0}^{} &= \xi
		\end{cases}
	\]
	corresponds to
	\[
		\begin{cases}
			\tilde u' = \tilde f \\
			\tilde u(x_0) = \xi
		\end{cases}
	\]
	and is equivalent to
	\[
		\tilde u(x) = \xi + \int_{x_0}^{x} \tilde f(s) \,\mathrm{d}s.
	\]
	$\tilde f$ contains $\tilde u$ implicitly. The statement of the proof follows from an application of Banach's fixed point theorem. (See course homepage and proof of picard's existence theorem.)
	\end{beweis}
Set 
\[
 	\mathcal{N}(L) = \set[u \in C^n([0,1])]{Lu=0}
\]
\textbf{Claim:} \text{    }$\dim \mathcal{N}(L)=n$ \\
Set \[
	C_R^n([0,1]) = \set[u \in C^n([0,1])]{R_ju = 0, j = 1,2,\dots,n}
\]
and $L_0 = L  \big|_{C^n_R([0,1])}^{}$. Let $u_1,\dots,u_m \in \mathcal{N}(L)$

\begin{theorem}
	The following statements are equivalent. Let $u_1,\dots,u_n \in \mathcal{N}(L)$
	\begin{enumerate}
		\item $W(x) \neq 0$ for all $x \in [0,1]$. 
		\item $W(x)\neq  0$ for some $x \in [0,1]$.
		\item $u_1,u_2, \dots,u_n$ is a basis for $\mathcal{N}(L)$.
	\end{enumerate}
	where 
	\[
		w(x) = \det \left( \begin{pmatrix}
			u_1(x) & \dots & u_n(x) \\
			u'_1(x) & \dots & u'_n(x) \\
			\vdots & & \vdots \\
			u_1^{(n-1)} & \dots & u_n^{(n-1)(x)}
		\end{pmatrix} \right), \qquad x \in [0,1].
	\]
\end{theorem}
\begin{theorem}
	With the notation from above the following statements are equivalent.
	\begin{enumerate}
		\item $L_0 : C^n_R( [0,1]) \to C([0,1])$ is a bijection.
		\item $\det(R_ju_k)_{1 \leq j,k \leq n} \neq 0$.
	\end{enumerate}
\end{theorem}

\begin{beispiel}[continue]
	From the example above we get
	\[
		u_1(x) = e^x, \qquad u_2(x) = e^{-x}.
	\]
	\[
		u(x) = A e^x + B e^{-x}
	\]
	and
	\begin{align*}
		R_1u_1 &= u_1(0)=e^0 =1 \\
		R_1u_2 &= u_2(0)=e^0 =1 \\
		R_2u_1 &= u_1(1)=e \\
		R_2u_2 &= u_2(1)=\frac{1}{e}
	\end{align*}
	and 
	\[
		\det(R_ju_k) = \det( \begin{pmatrix}
			1 & 1 \\ e & \frac{1}{e} 
		\end{pmatrix}) = \frac{1}{e} - e \neq 0.
	\]
\end{beispiel}

\begin{theorem}
	Assume $u_1, \dots,u_n$ basis for $\mathcal{N}(L)$ and $\det(R_ju_k) \neq 0$. Set $G=L_0^{-1}$. \[
		\Rightarrow \qquad \exists\,!\, \text{continuous }g \in C([0,1] \times [0,1])
	\]
	such that
	\[
		G(f) = \int_{0}^{1}g(x,y)f(y) \,\mathrm{d}y
	\] is a solution of
	\[
		\begin{cases}
			Lu &= f \\
			R_ju &=0, \qquad j=1,\dots,n
		\end{cases}.
	\]
	Here 
	\[
		g(x,y)= \underset{\equiv e(x,y)}{\underbrace{\left( \sum_{k=1}^{n}a_k(y)u_k(x) \right)}} \theta(x-y)+ \sum_{k=1}^{n}b_k(y)u_k(x).
	\]
	where
	\begin{align*}
		e^{(k)}_x (y,y) &= 0, \qquad k = 0,1,\dots,n-2 \\
		e^{(n-1)}_x(y,y) &=1
	\end{align*}
	Note
	\[
		Lu = 1 u^{(n)} + c_{n-1}u^{(n-1)} + \dots + c_0 u.
	\]
	and 
	\begin{align*}
		R_j(g(.,y)) = 0, \qquad 0 < y < 1, \qquad j=1,2,\dots,n
	\end{align*}
\end{theorem}
Note 
\begin{align*}
	\int_{0}^{1}g(x,y)f(y) \,\mathrm{d}y &= \int_{0}^{1}e(x,y) \theta(x-y)f(y) \,\mathrm{d}y + \int_{0}^{1} \sum_{k=1}^{n} b_k(y)u_k(x)f(y) \,\mathrm{d}y \\
	&= \underset{L[...]=f}{\underbrace{\underset{=L [...] = f}{\underbrace{\int_{0}^{x}\sum_{k=1}^{\infty}a_k(y)u_k(x) f(y) \,\mathrm{d}y}} + \underset{L[...]= 0}{\underbrace{\sum_{k=1}^{N} \int_{0}^{1}b_k(y)f(y) \,\mathrm{d}y u_k(x)}}}}
\end{align*}
Calculate $g(x,y)$ for $n=2$: \\
Set \begin{align*}
	e(x,y) = a_1(y)u_1(x)+ a_2(y)u_2(x)
\end{align*}
\[
	\begin{cases}
		e(y,y) &= a_1(y)e^y + a_2(y)e^{-y} = 0 \\
		e'_x(y,y) &= a_1(y)e^y - a_2(y) e^{-y}=1 \\
	\end{cases}
\]
So we get
\begin{align*}
	a_1(y) &= \frac{1}{2} e^{-y} \\
	a_2(y) &= - \frac{1}{2} e^{-y}
\end{align*}
and
\begin{align*} 
	e(x,y) &= \frac{1}{2} e^{-y} e^x - \frac{1}{2} e^{y}e^{-x}  \\
	&= \frac{1}{2} (e^{x-y} - e^{y-x}), \qquad (x,y) \in [0,1] \times [0,1]
\end{align*}
Set 
\[
	g(x,y) = e(x,y)\theta(x-y) + b_1(y)u_1(x)+ b_2(y)u_2(x)
\]
For $0<y<1$
\begin{align*}
	R_1g(.,y)=0, &\text{ i.e. }g(0,y)=0, \qquad \text{for }y \in (0,1), \\
	 &\text{ i.e. }b_1(y)u_1(0)+ b_2u_2(0) = 0 \text{ for }y \in (0,1), \\
	 &\text{ So $b_1(y) + b_2(y) = 0$}.
\end{align*}
\begin{align*}
	R_2g(.,y)=0, &\text{ i.e. }g(1,y)=0, \qquad \text{for }y \in (0,1), \\
	 &\text{ i.e. }e(1,y) + b_1(y)u_1(1)+ b_2(y)u_2(1) = 0 \text{ for }y \in (0,1), \\
	 &\text{ So }\frac{1}{2}\left( e^{1-y}- e^{y-1} \right) + b_1(y)e + b_2(y)e^{-1}=0 \text{ for }y \in (0,1).
\end{align*}
So we have in total
\[
	\begin{cases}
		b_1(y) + b_2(y) &= 0 \\
		\frac{1}{2}\left( e^{1-y}- e^{y-1} \right) + b_1(y)e + b_2(y)e^{-1} &=0 
	\end{cases}.
\]
We obtain
\[
	\begin{cases}
		b_1(y) &= -b_2(y) \\
		b_2(y)\left(  e^{-1} - e \right) &= \frac{1}{2}\left( e^{y-1}- e^{1-y} \right)
	\end{cases}.
\]
So
\[
	b_2(y) = \frac{\frac{1}{2}(e^{y-1}-e^{1-y})}{\left( e^{-1}-e \right)} = \frac{1}{2} \frac{e^{1-y}-e^{y}}{e^2-1}
\]
and
\[
	b_1(y) = \frac{1}{2} \frac{e^y-e^{2-y}}{e^2 -1}.
\]
We obtain
\[
	g(x,y) = \frac{1}{2}(e^{x-y}- e^{y-x})\theta(x-t) + \frac{1}{2} \frac{e^{x+y-e^{x+2-y}}}{e^2-1} + \frac{1}{2} \frac{e^{2-y-x}-e^{y-x}}{e^2-1}.
\]

Question: $g(x,y)=g(y,x)$ for all $x,y \in [0,1]$? \\
In general, we say that $L_0 = L  \big|_{C^n_R([0,1])}^{}$ is \underline{symmetrie} if 
\[
	\skal{L_0(u)}{v}_{L^2} = \skal{u}{L_0(v)}_{L^2}, \qquad \forall\, u,v \in C_R^n([0,1])
\]

\begin{beispiel}[continue]
	As above we have
	\[
		L(u) = u'' - u
	\] with boundary conditions
	\[
		u(0)= u(1)=0
	\]
	Set $u,v \in C_R^2([0,1])$
	\begin{align*}
		\skal{L_0(u)}{v}_{L^2} &= \int_{0}^{1} L_0(u)\bar{v} \,\mathrm{d}x  \\ 
		&= \int_{0}^{1}u'' \bar{v}- u \bar{v} \,\mathrm{d}x \\  
		&= - \int_{0}^{1} u' \bar{v} + u \bar{v} \,\mathrm{d}x 
		+ \underset{=u'(1)\underset{=0}{\underbrace{\bar{v}(1)}}-u'(0) \underset{=0}{\underbrace{\bar{v}(0)}}}{\underbrace{u'\bar{v}  \Big|_{0}^{1}}} \\
		&= - \int_{0}^{1}\left( u' \bar{v}' + u \bar{v} \right) \,\mathrm{d}x \\
		&= \int_{0}^{1} u( \bar{v}'' - \bar{v}) \,\mathrm{d}x \\
		&= \int_{0}^{1} u \overline{L_0v} \,\mathrm{d}x \\
		&= \skal{u}{L_0v}_{L^2}
	\end{align*}
\end{beispiel}
\cleardoubleoddemptypage
\pagenumbering{Alph}
\setcounter{page}{1}

\end{document}