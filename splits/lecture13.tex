\begin{itemize}
	\item $(E,\skal{.}{.})$ complex Hilbert space.
	\item $A \in \mathcal{B(E,E)}$. 
	\item Consider the equation
\[
	x = A(x)+ y, \qquad y \in E.
\]
\[
	(I-A)(x)= y.
\]
\item Consider this problem for $\lambda \in \mathbb{C}$. \\
\item Set \[
	\rho(A) := \set[\lambda \in \mathbb{C}]{(A- \lambda I)^{-1} \in \mathcal{B}(E,E)}
\]
\item $\rho(A)$ is called the resolvent set for $A$.
\item Set \[
	\sigma(A) = \mathbb{C} \setminus \rho(A).
\]
\item $\sigma(A)$ is called the spectrum of $A$.
\item Clearly, a necessary condition for $(A-\lambda I)^{-1} \in \mathcal{B}(E,E)$ is that 
\[
	A - \lambda I:E \to E
\]
is a bijection.
\item Linearity for $(A-\lambda I)^{-1}$ follows from the linearity of $A-\lambda I$.

\end{itemize}
\begin{theorem}[Banachs's inverse mapping theorem]
	$(E,\norm{.})$ Banach space, $A \in \mathcal{B}(E,E)$. \\$A-\lambda I: E \to E$ bijection. Then
	\[
		\Rightarrow (A- \lambda I)^{-1} \in \mathcal{B}(E,E)
	\]
\end{theorem}
\begin{beweis}
	based on the Open mapping theorem. Proof is omitted. Assume $\lambda \in \sigma(A)$. Then $A-\lambda I:E \to E$ is not a bijection.

\begin{itemize}
	\item If $A-\lambda I: E \to E$ is not injective then there exists $0 \neq x \in  E $ such that
	\[
		(A- \lambda I)(x) = 0,
	\]
	i.e. $\lambda$ is an eigenvalue of $A$.
	Set	
	\[
		\sigma_p(A) = \set[\lambda \in \mathbb{C}]{\lambda \text{eigenvalue of $A$}}.
	\]
	\item If $A-\lambda I$ is injective, densely defined but not bounded then $\lambda \in \sigma(A)$. The set of such $\lambda$'s is called the continuous spectrum of $A$, denoted $\sigma_c(A)$
	\item If $A-\lambda I$ is not surjective then the set of such $\lambda$'s is called the residual spectrum, denoted $\sigma _r(A)$.
\end{itemize}
\end{beweis}
 \begin{lemma*}
 	$(E,\norm{.})$ Banach space, $A \in \mathcal{B}(E,E)$ with $\norm{A} < 1$. Then
	\[
		(I-A)^{-1} \in \mathcal{B}(E,E)
	\]
	and
	\[
		(I-A)^{-1} = I + \sum_{n=1}^{\infty} A^n.
	\]
	This series is called a Neumannseries. 
 \end{lemma*}
\begin{beweis}
	Observe
	\[
		\norm{A^n} = \norm{A \cdot A \cdots A} \leq \norm{A}^n, \qquad n = 1,2,\dots
	\]
	and
	\[
		\sum_{n=1}^{\infty}\norm{A^n}< \infty.
	\]
	Since $E$ is a Banach space we have
	\[
		\sum_{n=1}^{\infty}A^n 
	\]
	converges in $\mathcal{B(E,E)}$. Since $E$ Banach space implies $\mathcal{B}(E,E)$ is a Banach space. \\
	Note
	\[
		(I-A)\left( I+ \sum_{n=1}^{N} A^n \right) = I-A^{N+1} \to I, \qquad \text{in }\mathcal{B}(E,E).
	\]
	\[
		\left(I + \sum_{n=1}^{N}A^n \right)(I-A) = I-A^{N+1} \to I, \qquad \text{in }\mathcal{B}(E,E).
	\]
	We get
	\[
		\left( I + \sum_{n=1}^{\infty} \right)(I-A) = I = (I-A)(I + \sum_{n=1}^{\infty}A^n).
	\]
	We have $(I-A)^{-1} $ exists and is equal to $I+ \sum_{n=1}^{A^n}$.
\end{beweis}
 \begin{lemma*}
 	$(E,\norm{.})$ Banach space and $A \in \mathcal{B}(E,E)$. Then
	\begin{enumerate}
		\item $\sigma(A) \neq \emptyset$.
		\item $\sigma(A)$ closed set in $\mathbb{C}$.
		\item $\sigma(A) \subseteq \overline{B(0,\norm{A})}$
	\end{enumerate}
 \end{lemma*}
\begin{beweis}
	\begin{enumerate}
		\item omitted.
		\item Enough to prove that $\rho(A)$ is an open set in $\mathbb{C}$. \\
		Fix $\lambda_0 \in \rho(A)$. So $(A- \lambda_0 I)^{-1} \in \mathcal{B}(E,E)$. \\ Note:
		\begin{align*}
			A-\lambda I &= A - \lambda_0 I - (\lambda - \lambda_0)I \\
			&= \underset{\substack{\text{invertible} \\ \text{ since }\lambda_0 \in \rho(A)}}{\underbrace{(A-\lambda_0 I)}} \underset{\substack{ \text{ invertible if} \\ \norm{(\lambda-\lambda_0)(A- \lambda_0 I)^{-1}}<1 \\ \text{by previous lemma, i.e.} \\ \abs{\lambda - \lambda_0} < \frac{1}{\norm{(A- \lambda_0 I)^{-1}}}}}{\underbrace{\left( I - (\lambda - \lambda_0)(A- \lambda_0 I)^{-1} \right)}}.
		\end{align*}
		Clearly, $A-\lambda I$ is invertible if
		\[
			\abs{\lambda - \lambda_0} < \frac{1}{\norm{(A- \lambda_0 I)^{-1}}}.
		\]
		\item It is enough to show that $\lambda \in \rho(A)$ if \[
			\abs{\lambda} > \norm{A}.
		\]
		Note 
		\[
			A- \lambda I = - \lambda (I - \frac{1}{\lambda}A).
		\]
		Here 
		\[
			\norm{- \frac{1}{\lambda}A} = \frac{1}{\abs{\lambda}} \norm{A} < 1.
		\]
		$I - \frac{1}{\lambda}A$ is invertible by previous lemma. So $\rho(A)$.
	\end{enumerate}
\end{beweis}
Now assume $(E,\skal{.}{.})$ is a complex Hilbert space with infinite dimension. $A \in \mathcal{K}(E,E)$ (We don't assume $A$ is self-adjoint). Then
\begin{enumerate}
	\item $\lambda \in \sigma(A) \setminus \set{0}$ $\qquad $ $\Rightarrow$ is an eigenvalue of $A$. 
	\item $\lambda \in \sigma(A) \setminus \set{0}$ $\qquad $ $\Rightarrow $ $\dim \set[x \in E]{A(x) = \lambda x} < \infty$.
	\item $O$ is the only cluster point for $\sigma(A)$ 
	\item $0 \in \sigma(A)$ since if $0 \not \in \sigma(A)$ then $A^{-1} \in \mathcal{B}(E,E)$ and \[
		\underset{\in \mathcal{K}(E,E)}{\underbrace{\underset{\in \mathcal{K}(E,E)}{\underbrace{A}} \underset{\in \mathcal{B}(E,E)}{\underbrace{A^{-1}}}}} = I.
	\]
	But $I \not \in \mathcal{K}(E,E)$ since $E$ $\infty$-dimensional. Just take an ON-sequence $(x_n)_{n=1}^{\infty}$ in $E$. Then \[
		x_n \rightharpoonup 0, \qquad \text{ in }E
	\]
	but $\norm{x_n}=1$ for all $n$ and if $I \in \mathcal{K}(E,E)$ then 
	\[
		x_n = I(x_n) \to I(0) = 0, \qquad  \text{ in }E
	\]
	which implies that $\norm{x_n}\to 0$ for $n \to \infty$. Moreover (by Hilbert-Schmidt theorem) $(E, \skal{.}{.})$ complex Hilbert space, seperable and $\infty$-dim. $A \in \mathcal{K}(E,E)$ and self-adjoint it follows
	\[
		\Rightarrow \qquad (u_n)_{n=1}^{\infty} \text{ ON-basis for $E$ where}
	\]
	\[
		A(u_n) = \lambda_n u_n, \qquad n=1,2,\dots.
	\]
	($\lambda_n$ eigenvalue of $A$ with normalised eigenvector $u_n$) with
	\[
		\lim_{n \to \infty}\lambda_n = 0.
	\]
	For $x \in E$
	\[
		x = \sum_{n=1}^{\infty} \skal{x}{u_n}u_n
	\]
	and
	\[
		A(x) = \sum_{n=1}^{\infty}\lambda \skal{x}{\lambda_n}u_n
	\]
\end{enumerate}
\minisec{Fredholm Alternativ:}
$E,A$ as above. Then
\begin{enumerate}
	\item $x = A(x) + y$ is seperable for all $y \in E$. \\
	iff
	\item $x = A(x)$ has no non-trivial solution $x \in E$.
\end{enumerate}
Exactly one of the statements hold:
\begin{enumerate}
	\item (1) from above
	\item (2) has no non-trivial solution $x \in E$.
\end{enumerate}
In general (1) is seperable for $y \in E$ iff 
\[
	y \in \set[x \in E]{A(x)=x}^{\perp}.
\]
If so: If $x$ is a solution to (1) then also $x + \tilde x$ is a solution to (1) where
\[
	\tilde x \in \set[x \in E]{A(x)=x}
\]
\begin{beweis}
	Look at (1). Let $(u_n)_{n=1}^{\infty}$ be the ON-basis from the previous theorem.
	\[
		x = \sum_{n=1}^{\infty}\skal{x}{u_n}u_n, \qquad y = \sum_{n=1}^{\infty}\skal{y}{u_n}u_n.
	\]
	\[
		A(x) = \sum_{n=1}^{\infty} \lambda_n \skal{x}{u_n}u_n.
	\]
	(1) taked the form
	\[
		\sum_{n=1}^{\infty} \left( \skal{x}{u_n} - \lambda_n \skal{x}{u_n} - \skal{y}{u_n} \right) u_n = 0.
	\]
	This implies 
	\[
		(I- \lambda)\skal{x}{u_n}- \skal{y}{u_n} = 0, \qquad n=1,2,\dots.
	\]
	If $\lambda_n \neq 1$ then
	\[
		\skal{x}{u_n} = \frac{\skal{y}{u_n}}{1 - \lambda_n}.
	\]
	If $\lambda_n = 1$ then $-y$ must be orthogonal to every $u_n$ corresponding by the eigenvalue $1$.
	\[
		\sum_{n=1}^{\infty} \frac{\skal{y}{u_n}}{1- \lambda_n}u_n \in E
	\]
	since
	\[
		(\frac{\skal{y}{u_n}}{1- \lambda_n})_{n=1}^{\infty} \in l^2
	\]
	since
	\[
		\sup_{\substack{n \\ \lambda_n \neq 1}} \abs{\frac{1}{1- \lambda_n}} < \infty
	\]
	since
	\[
		\lim_{n \to \infty}\lambda_n = 0
	\]
	and
	\[
		(\skal{y}{u_n})_{n=1}^{\infty} \in l^2.
	\]
\end{beweis}

\section{Boundary Value Problems for ODE's} 
\label{sec:boundary_value_problems_for_ode_s}
Consider
\[
	(*)\qquad \begin{cases}
		Lu &=f \in C([0,1]) \\
		R_ju &= 0 \qquad j=1,2,\dots,n 
	\end{cases}
\]
(homogenuous boundary conditions),
where
\[
	Lu := u^{(n)} + C_{n-1}(x)u^{(n-1)} + \dots + c_1(x)u' + c_0(x)u, \qquad u \in C^n([0,1])
\]
with
\[
	c_0(x),c_1(x), \dots, c_{n-1}(x) \in C([0,1])-
\]
\[
	R_j = \sum_{k=0}^{n-1} \left( \alpha_{jk} u^{(k)}(0) + \beta_{jk}u^{(k)}(1) \right), \qquad j=1,2,\dots,n
\]
with
\[
	\alpha_{jk}, \beta_{jk} \in \mathbb{C}, \qquad j=1,\dots,n, \qquad k=0,\dots,n-1
\]
Reformulate (*).
\[
	u(x) = \int_{0}^{1} \underset{\substack{\text{Green's function} \\ \text{for $L$ and $R_j$} \\ j=1,\dots,n}}{\underbrace{g(x,y)}}f(y) \,\mathrm{d}y \qquad \in C^n([0,1])
\]
and satisfies the boundary conditions $R_j = 0$ for $j=1,2,\dots,n$. \\
Consider the problem
\[
(**) \qquad 	\begin{cases}
		Lu &= f(x,u), \qquad x \in [0,1) \\
		R_ju &= 0, \qquad j=1,2,\dots,n.
		
	\end{cases}
\]
The reformulation above gives
\[
	u(x) = \int_{0}^{1}g(x,y)f(y,u(y)) \,\mathrm{d}y, \qquad x \in [0,1].
\]
To find a solution set
\[
	T(u)(x) = \int_{0}^{1}g(x,y)f(y,u(y)) \,\mathrm{d}y, \qquad x \in [0,1].
\]
\[
	T: C([0,1]) \to C([0,1])
\]
A fixed point to $T$ gives a solution to $(**)$. Note that if $u \in C([0,1])$ then
\[
	T(u) \in C^n([0,1]) 
\]
and satisfies $R_j=0$ for $j=1,2,\dots$. \\
Given $L$ and $R_j$ for $j=1,2,\dots,n$ find the corresponding Green's function.

\begin{beispiel}
	\[
		\begin{cases}
			Lu &= u''-u, \qquad \text{ on }[0,1]\\
			R_1u &=u(0)=0 \\
			R_2u &=u(1)=0
		\end{cases}
	\]
\end{beispiel}
	\begin{theorem}
		$Lu=f \in C([0,1])$, where
		\[
			Lu: = u^{(n)}+ c_{n-1}(x)u^{(n-1)} + \dots + c_1(x)u' + c_0(x)u
		\]
		and $\xi = (\xi_1, \dots, \xi_n) \in \mathbb{C}^n$. Then for $x_0 \in [0,1]$
		\[
			\Rightarrow \qquad \exists\,!\,\,u \in C^n([0,1]) \text{ with } Lu=f.
		\]
		and 
		\[
			(u,u',\dots,u^{(n-1)}) \big|_{x_0}^{} = \xi.
		\]
	\end{theorem}
	\begin{beweis}
		Reformulate the problem as a system of first order differential equations.
	\[
		\begin{cases}
			Lu &=f \\
			(u,u',\dots,u^{(n-1)})  \big|_{x_0}^{} &= \xi
		\end{cases}
	\]
	corresponds to
	\[
		\begin{cases}
			\tilde u' = \tilde f \\
			\tilde u(x_0) = \xi
		\end{cases}
	\]
	and is equivalent to
	\[
		\tilde u(x) = \xi + \int_{x_0}^{x} \tilde f(s) \,\mathrm{d}s.
	\]
	$\tilde f$ contains $\tilde u$ implicitly. The statement of the proof follows from an application of Banach's fixed point theorem. (See course homepage and proof of picard's existence theorem.)
	\end{beweis}
Set 
\[
 	\mathcal{N}(L) = \set[u \in C^n((0,1))]{Lu=0}
\]
\textbf{Claim:} \text{    }$\dim \mathcal{N}(L)=n$ \\
Set \[
	C_R^n([0,1]) = \set[u \in C^n((0,1))]{R_ju = 0, j = 1,2,\dots,n}
\]
and $L_0 = L  \big|_{C^n_R([0,1])}^{}$. Let $u_1,\dots,u_m \in \mathcal{N}(L)$

\begin{theorem}
	The following statements are equivalent. Let $u_1,\dots,u_n \in \mathcal{N}(L)$
	\begin{enumerate}
		\item $W(x) \neq 0$ for all $x \in [0,1]$. 
		\item $W(x)\neq  0$ for some $x \in [0,1]$.
		\item $u_1,u_2, \dots,u_n$ is a basis for $\mathcal{N}(L)$.
	\end{enumerate}
	where 
	\[
		W(x) = \det \left( \begin{pmatrix}
			u_1(x) & \dots & u_n(x) \\
			u'_1(x) & \dots & u'_n(x) \\
			\vdots & & \vdots \\
			u_1^{(n-1)} & \dots & u_n^{(n-1)(x)}
		\end{pmatrix} \right), \qquad x \in [0,1].
	\]
\end{theorem}
\begin{theorem}
	With the notation from above the following statements are equivalent.
	\begin{enumerate}
		\item $L_0 : C^n_R( [0,1]) \to C([0,1])$ is a bijection.
		\item $\det(R_ju_k)_{1 \leq j,k \leq n} \neq 0$.
	\end{enumerate}
\end{theorem}

\begin{beispiel}[continue]
	From the example above we get
	\[
		u_1(x) = e^x, \qquad u_2(x) = e^{-x}.
	\]
	\[
		u(x) = A e^x + B e^{-x}
	\]
	and
	\begin{align*}
		R_1u_1 &= u_1(0)=e^0 =1 \\
		R_1u_2 &= u_2(0)=e^0 =1 \\
		R_2u_1 &= u_1(1)=e \\
		R_2u_2 &= u_2(1)=\frac{1}{e}
	\end{align*}
	and 
	\[
		\det(R_ju_k) = \det( \begin{pmatrix}
			1 & 1 \\ e & \frac{1}{e} 
		\end{pmatrix}) = \frac{1}{e} - e \neq 0.
	\]
\end{beispiel}

\begin{theorem}
	Assume $u_1, \dots,u_n$ basis for $\mathcal{N}(L)$ and $\det(R_ju_k) \neq 0$. Set $G=L_0^{-1}$. \[
		\Rightarrow \qquad \exists\,!\, \text{continuous }g \in C([0,1] \times [0,1])
	\]
	such that
	\[
		G(f) = \int_{0}^{1}g(x,y)f(y) \,\mathrm{d}y
	\] is a solution of
	\[
		\begin{cases}
			Lu &= f \\
			R_ju &=0, \qquad j=1,\dots,n
		\end{cases}.
	\]
	Here 
	\[
		g(x,y)= \underset{\equiv e(x,y)}{\underbrace{\left( \sum_{k=1}^{n}a_k(y)u_k(x) \right)}} \theta(x-y)+ \sum_{k=1}^{n}b_k(y)u_k(x).
	\]
	where
	\begin{align*}
		e^{(k)}_x (y,y) &= 0, \qquad k = 0,1,\dots,n-2 \\
		e^{(n-1)}_x(y,y) &=1
	\end{align*}
	Note
	\[
		Lu = 1 u^{(n)} + c_{n-1}u^{(n-1)} + \dots + c_0 u.
	\]
	and 
	\begin{align*}
		R_j(g(.,y)) = 0, \qquad 0 < y < 1, \qquad j=1,2,\dots,n
	\end{align*}
\end{theorem}
Note 
\begin{align*}
	\int_{0}^{1}g(x,y)f(y) \,\mathrm{d}y &= \int_{0}^{1}e(x,y) \theta(x-y)f(y) \,\mathrm{d}y + \int_{0}^{1} \sum_{k=1}^{n} b_k(y)u_k(x)f(y) \,\mathrm{d}y \\
	&= \underset{L[...]=f}{\underbrace{\underset{=L [...] = f}{\underbrace{\int_{0}^{x}\sum_{k=1}^{\infty}a_k(y)u_k(x) f(y) \,\mathrm{d}y}} + \underset{L[...]= 0}{\underbrace{\sum_{k=1}^{N} \int_{0}^{1}b_k(y)f(y) \,\mathrm{d}y u_k(x)}}}}
\end{align*}
Calculate $g(x,y)$ for $n=2$: \\
Set \begin{align*}
	e(x,y) = a_1(y)u_1(x)+ a_2(y)u_2(x)
\end{align*}
\[
	\begin{cases}
		e(y,y) &= a_1(y)e^y + a_2(y)e^{-y} = 0 \\
		e'_x(y,y) &= a_1(y)e^y - a_2(y) e^{-y}=1 \\
	\end{cases}
\]
So we get
\begin{align*}
	a_1(y) &= \frac{1}{2} e^{-y} \\
	a_2(y) &= - \frac{1}{2} e^{-y}
\end{align*}
and
\begin{align*} 
	e(x,y) &= \frac{1}{2} e^{-y} e^x - \frac{1}{2} e^{y}e^{-x}  \\
	&= \frac{1}{2} (e^{x-y} - e^{y-x}), \qquad (x,y) \in [0,1] \times [0,1]
\end{align*}
Set 
\[
	g(x,y) = e(x,y)\theta(x-y) + b_1(y)u_1(x)+ b_2(y)u_2(x)
\]
For $0<y<1$
\begin{align*}
	R_1g(.,y)=0, &\text{ i.e. }g(0,y)=0, \qquad \text{for }y \in (0,1), \\
	 &\text{ i.e. }b_1(y)u_1(0)+ b_2u_2(0) = 0 \text{ for }y \in (0,1), \\
	 &\text{ So $b_1(y) + b_2(y) = 0$}.
\end{align*}
\begin{align*}
	R_2g(.,y)=0, &\text{ i.e. }g(1,y)=0, \qquad \text{for }y \in (0,1), \\
	 &\text{ i.e. }e(1,y) + b_1(y)u_1(1)+ b_2(y)u_2(1) = 0 \text{ for }y \in (0,1), \\
	 &\text{ So }\frac{1}{2}\left( e^{1-y}- e^{y-1} \right) + b_1(y)e + b_2(y)e^{-1}=0 \text{ for }y \in (0,1).
\end{align*}
So we have in total
\[
	\begin{cases}
		b_1(y) + b_2(y) &= 0 \\
		\frac{1}{2}\left( e^{1-y}- e^{y-1} \right) + b_1(y)e + b_2(y)e^{-1} &=0 
	\end{cases}.
\]
We obtain
\[
	\begin{cases}
		b_1(y) &= -b_2(y) \\
		b_2(y)\left(  e^{-1} - e \right) &= \frac{1}{2}\left( e^{y-1}- e^{1-y} \right)
	\end{cases}.
\]
So
\[
	b_2(y) = \frac{\frac{1}{2}(e^{y-1}-e^{1-y})}{\left( e^{-1}-e \right)} = \frac{1}{2} \frac{e^{1-y}-e^{y}}{e^2-1}
\]
and
\[
	b_1(y) = \frac{1}{2} \frac{e^y-e^{2-y}}{e^2 -1}.
\]
We obtain
\[
	g(x,y) = \frac{1}{2}(e^{x-y}- e^{y-x})\theta(x-t) + \frac{1}{2} \frac{e^{x+y-e^{x+2-y}}}{e^2-1} + \frac{1}{2} \frac{e^{2-y-x}-e^{y-x}}{e^2-1}.
\]

Question: $g(x,y)=g(y,x)$ for all $x,y \in [0,1]$? \\
In general, we say that $L_0 = L  \big|_{C^n_R([0,1])}^{}$ is \underline{symmetrie} if 
\[
	\skal{L_0(u)}{v}_{L^2} = \skal{u}{L_0(v)}_{L^2}, \qquad \forall\, u,v \in C_R^n([0,1])
\]

\begin{beispiel}[continue]
	As above we have
	\[
		L(u) = u'' - u
	\] with boundary conditions
	\[
		u(0)= u(1)=0
	\]
	Set $u,v \in C_R^2([0,1])$
	\begin{align*}
		\skal{L_0(u)}{v}_{L^2} &= \int_{0}^{1} L_0(u)\bar{v} \,\mathrm{d}x  \\ 
		&= \int_{0}^{1}u'' \bar{v}- u \bar{v} \,\mathrm{d}x \\  
		&= - \int_{0}^{1} u' \bar{v} + u \bar{v} \,\mathrm{d}x 
		+ \underset{=u'(1)\underset{=0}{\underbrace{\bar{v}(1)}}-u'(0) \underset{=0}{\underbrace{\bar{v}(0)}}}{\underbrace{u'\bar{v}  \Big|_{0}^{1}}} \\
		&= - \int_{0}^{1}\left( u' \bar{v}' + u \bar{v} \right) \,\mathrm{d}x \\
		&= \int_{0}^{1} u( \bar{v}'' - \bar{v}) \,\mathrm{d}x \\
		&= \int_{0}^{1} u \overline{L_0v} \,\mathrm{d}x \\
		&= \skal{u}{L_0v}_{L^2}
	\end{align*}
\end{beispiel}
