%%% lecture 8b

\begin{theorem}[Orthogonal Decompostion Theorem]
	Let $(E, \skal{.}{.})$ be a Hilbert space and $S$ be a closed subspace of $E$. 
	\[
		\Rightarrow \qquad E = S \oplus S^{\perp}
	\]
	which means that for every $x \in E$ there exists an unique decomposition
	\[
		x = y + z
	\]
	with $y \in S$ and $z \in S^{\perp}$.
\end{theorem}
\begin{beispiel}
	Let $A \subseteq E$ where $E$ is a Hilbert space. It follows
	\[
		\overline{\spn  A} = \left( A^{\perp} \right)^{\perp}.
	\]
	Note 
	\[
		A \subseteq \underset{\text{subspace of $E$}}{\underbrace{ \left( A^{\perp} \right)^{\perp}}} \qquad \Rightarrow \qquad \spn  A \subseteq 
		\underset{\text{closed}}{\underbrace{ \left( A^{\perp} \right)^{\perp}}} \qquad \Rightarrow \qquad \overline{\spn  A} \subseteq 
		\left( A^{\perp} \right)^{\perp}
	\]
	\[
		A \subseteq \overline{\spn  A} \qquad \Rightarrow \qquad \overline{\spn  A}^{\perp} \subseteq A^{\perp} \qquad \Rightarrow \qquad 
		\left( A^{\perp} \right)^{\perp} \subseteq \left( \overline{\spn  A}^{\perp} \right)^{\perp}.
	\]
	Hence
	\[
		\overline{\spn  A} \subseteq \left( A^{\perp} \right)^{\perp} \subseteq \left( \overline{\spn  A}^{\perp} \right)^{\perp}.
	\]
	By the Orthogonal Decomposition Theorem we get
	\[
		E = \overline{\spn  A} \oplus \overline{\spn  A}^{\perp} = \overline{\spn  A}^{\perp} \oplus \left( \overline{\spn  A}^{\perp} \right)^{\perp}, 
	\]
	which implies
	\[
		\overline{\spn  A} = \left( \overline{\spn  A}^{\perp} \right)^{\perp},
	\]
	\[
		\Rightarrow \qquad \left( A^{\perp} \right)^{\perp} = \overline{\spn  A}.
	\]
\end{beispiel}

Now we are going to prove the Orthogonal Decomposition Theorem.

\begin{beweis}
	\begin{description}
		\item[Step 1:] $S$ is a closed convex set in a Hilbert space $E$. This implies that 
		\[
			\forall\,  x \in E \, \exists\,! \,y \in S: \qquad \norm{x-y} \leq \norm{x- \tilde y} \qquad \forall\, \tilde y \in S.
		\] 
		which means
		\[
			\norm{x-y} = \inf_{\tilde y \in S}\norm{x-\tilde y}.
		\]
		Fix $x \not \in S$ with
		\[
			\inf_{ \tilde y \in S} \norm{x- \tilde y} = d > 0.
		\]
		Take a sequence $(y_n)_{n=1}^{\infty}$ in $S$ such that 
		\[
			\norm{x-y_n} \to d, \qquad n \to \infty.
		\]
		\textbf{Claim:} \text{    } This is a Cauchy sequence. \\
		(use Parallelogram-law for $\norm{.}$)
		\item[Step 2:] $S$ as in ODT. \\
		Note: $S$ must be convex. \\
		Fix $x \in E$, choose $y \in S$ with
		\[
			\norm{x-y} \leq  \norm{x- \tilde y}, \qquad \forall\,  \tilde y \in S.
		\]
		Set
		\[
			\underset{\in E}{\underbrace{x}} = \underset{\in S}{\underbrace{y}} + (x-y).
		\]
		To show: $x-y \in S^{\perp}$. A variational argument of this is 
		\[
			\skal{x-y}{v}= 0, \qquad \forall\, v \in S.
		\]
		We know
		\begin{align*}
			\norm{x-y}^2 &\leq \norm{x-y + \alpha v}^2 \qquad \forall\, \text{scalars }\alpha \\
			\norm{x-y}^2 &\leq \skal{x-y+\alpha v}{x-y+ \alpha v} \\
			&= \norm{x-y}^2+ \alpha \skal{v}{x-y} + \bar{\alpha} \skal{x-y}{v} + \abs{\alpha}^2 \norm{v}^2
		\end{align*}
		and
		\[
			0 \leq 2 \re( \alpha \skal{x-y}{v}) + \abs{\alpha}^2 \norm{v}^2.
		\]
		Set 
		\[
			\alpha = t \overline{\skal{x-y}{v}}, \qquad t \in \mathbb{R},
		\]
		\[
			\Rightarrow \qquad 0 \leq  2 t \abs{\skal{x-y}{v}}^2 + t^2 \abs{\skal{x-y}{v}}^2 \norm{v}^2.
		\]
		Assume $\skal{x-y}{v} \neq 0$: \\
		We have 
		\begin{align*}
			0 &\leq 2t + t^2 \norm{v}^2 \qquad \forall\, t \in \mathbb{R} \\
			\Rightarrow \qquad -2t &\leq  t^2 \norm{v}^2, \qquad \text{Let }t <0 \\
			\Leftrightarrow \qquad 2 &\leq -t \norm{v}^2, \qquad t<0.
		\end{align*}
		Let $t \to 0$, then
		\[
			2 \leq 0
		\]
		which is a contradiction.
	\end{description}
\end{beweis}