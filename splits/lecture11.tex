%%% lecture 11

\begin{definition}[compact]
	If $A: E \to E$ is linear, then we say that $A$ is \underline{compact} if for all bounded sequences $(x_n)_{n=1}^{\infty}$ in $E$, $ (A(x_n))_{n=1}^{\infty}$ has a bounded subsequence in $E$.
\end{definition}

\begin{lemma}
	$A$ is compact and linear \qquad  $\Rightarrow $ \qquad  $A$ is bounded.
\end{lemma}
\begin{beweis}
	If $A$ is not bounded then there exists a sequence $(y_n)_{n=1}^{\infty}$ in $E$ such that
	\[
		\norm{A(y_n)} \geq n \norm{y_n}, \qquad \text{for }n=1,2,\dots.
	\]
	Set $x_n = \frac{y_n}{\norm{y_n}}$ for $n = 1,2,\dots$. Here $\norm{x_n}=1$ for all $n \in \mathbb{N}$ and 
	\[
		\norm{A(x_n)} = \norm{A \left( \frac{1}{\norm{y_n}}y_n \right)} = \frac{1}{\norm{y_n}} \norm{A(y_n)} >n, \qquad \forall\, n \in \mathbb{N}.
	\]
	$(A(x_n))_{n=1}^{\infty}$ has no converging subsequence since $\norm{A(x_n)} \to \infty$ for $n \to \infty$.
\end{beweis}
\begin{bemerkung}
	\begin{itemize}
		\item $A \in B(E,E)$ and $F \subset E$ where $F$ is bounded. Then
		\[
			A(F) = \set[A(x)]{x \in F} 
		\]
		is bounded.
		\item $A \in B(E,E)$ compact and $F \subset E$, $F$ bounded. Then $\overline{A(F)}$ is compact.
	\end{itemize}
\end{bemerkung}

\begin{lemma}
	$A,B$ compact linear operators $E \to E$ and $\alpha$ and $\beta$ scalars. Then $\alpha A+ \beta A$ is compact.
\end{lemma}
\begin{beweis}
	Fix an arbitrary bounded sequence $(x_n)_{n=1}^{\infty}$ in $E$. Since $A$ is compact there exists a converging subsequence $(A(x_{n_k}))_{k=1}^{\infty}$ of $(A(x_n))_{n=1}^{\infty}$. \\ Clearly $(\alpha A(x_n))_{n=1}^{\infty}$ converges in $E$. \\ Since $B$ is compact there exists a converging subsequence $(B(x_{n_k}))_{k=1}^{\infty}$ of $(B(x_n))_{n=1}^{\infty}$.  \\ Clearly $(\beta B(x_{n_k}))_{k=1}^{\infty}$ converges in $E$. Hence
	\[
		\left( \alpha A(x _{n_k}) + \beta B(x _{n_k}) \right)_{k=1}^{\infty} = \left( (\alpha A + \beta B)(x _{n_k}) \right)_{k=1}^{\infty}
	\]
	converges in $E$.
\end{beweis} 
Set \[
	K(E,E) := \text{set of all compact linear mappings $E \to E$}.
\]
We have $K(E,E)$ is a subspace in $(B(E,E), \norm{.}_{E \to E})$.

\begin{proposition}
	$K(E,E)$ is a \underline{closed} subspace in $(B(E,E), \norm{.}_{E \to E})$.
\end{proposition}
Before the proof we note:
\begin{enumerate}
	\item Assume $(E,\skal{.}{.})$ to be a Hilbert space and $A \in B(E,E)$.
	\begin{align*}
		x_n \to x \,\text{ in }E \qquad &\Rightarrow \qquad A(x_n) \to A(x)\, \text{ in }E \\
		x_n \rightharpoonup x \,\text{ in }E \qquad &\Rightarrow \qquad A(x_n) \rightharpoonup A(x)\, \text{ in }E
	\end{align*}
	since for $y \in E$ we have
	\[
		\skal{A(x_n)}{y} = \skal{x_n}{A^*(y)} \stackrel{n \to \infty}{\to } \skal{x}{A^*(y)} = \skal{A(x)}{y}.
	\]
	\item $A \in K(E,E)$ and $x_n \rightharpoonup x$ in $E$ 
	\[
		\Rightarrow A(x_n) \to A(x) \qquad \text{in }E.
	\]
	\item $A \in B(E,E)$ finite-rank operator, i.e.
	\[
		\dim \mathcal{R}(A) < \infty \qquad \Rightarrow \qquad A \in K(E,E)
	\]
	since: Let $e_1,e_2, \dots, e_N$ be an ON-basis for $\mathcal{R}(A)$ with $N = \dim( \mathcal{R}(A))$. We have
	\[
		A(x) = \skal{A(x)}{e_1}e_1 + \dots + \skal{A(x)}{e_N}e_N.
	\]
	Fix an arbitrary bounded sequence $(x_n)_{n=1}^{\infty}$ in $E$. $A$ is bounded which implies that $(A(x_n))_{n=1}^{\infty}$ is a bounded sequence. Furthermore
	\[
		(\skal{A(x_n)}{e_1})_{n=1}^{\infty}
	\]
	is a bounded sequence in $\mathbb{C}$. Bolzano Weierstrass theorem implies that $(\skal{A(x_n)}{e_1})_{n=1}^{\infty}$ has a converging subsequence 
	$(\skal{A(x _{n_k})}{e_1})_{k=1}^{\infty}$. Clearly $(\skal{A(x _{n_k})}{e_1} e_1)_{k=1}^{\infty}$ converges in $E$. \\
	Hence 
	\[
		A(x) = \skal{A(x)}{e_1}e_1 + \dots + \skal{A(x)}{e_N}e_N
	\]
	is a compact mapping since $K(E,E)$ is a subspace of $B(E,E)$.
\end{enumerate}

\begin{beweis}
	Assume $(A_n)_{n=1}^{\infty} \subseteq K(E,E)$ such that $A_n \to A$ in $(B(E,E), \norm{.}_{E \to E})$. \\
	 We have to show: $A \in K(E,E)$ \\
	 Fix an arbitrary bounded sequence $(x_n)_{n=1}^{\infty}$ in $E$. We want to show that $(A(x_n))_{n=1}^{\infty}$ has a converging subsequence in $E$. \\
	 Set \[
	 	M = \sup_{n} \norm{x_n} < \infty.
	 \]
	 \begin{align*}
	 	A_1 \in K(E,E) \qquad &\Rightarrow \qquad (A_1(x_n))_{n=1}^{\infty} \text{ has a converging subsequence }(A_1(x _{n_k}))_{k=1}^{\infty} \\
		A_2 \in K(E,E) \qquad &\Rightarrow \qquad (A_2(x_n))_{n=1}^{\infty} \text{ has a converging subsequence }(A_2(x _{n_k}))_{k=1}^{\infty}
	 \end{align*}
	 proceed inductively:
	 \[
	 	A_k \in K(E,E) \qquad \Rightarrow \qquad (A_k(x_n))_{n=1}^{\infty} \text{ has a converging subsequence }(A_k(x _{n_l}))_{l=1}^{\infty}
	 \] 
	 Also: $(A_l(x_{n,k})_{n=1}^{\infty}$ converges in $E$ for $l=1,2,\dots,k$. \\ Here $(A_k(y_n))_{n=1}^{\infty}$ converges for $k=1,2,\dots$. \\
	 So since $(E,\norm{.})$ is a Banach space it is enough to show that $(A(y_n))_{n=1}^{\infty}$ is a Cauchy sequence in $(E,\norm{.})$. \\
	 Fix an arbitrary $\varepsilon >0$. We have
	 \[
	 	\norm{A(y_n)-A(y_m)} \leq \underset{\leq \norm{A-A_k}_{E \to E} \norm{y_n}}{\underbrace{\norm{A(y_n)-A_k(y_n)}}} + \norm{A_k(y_n)- A_k(y_m)} + \norm{A_k(y_m)-A(y_m)}.
	 \]
	 Fix $k$ large enough such that
	 \[
	 	\norm{A_k-A} < \frac{\varepsilon}{3M}.
	 \]
	 Then 
	 \[
	 	\norm{A(y_n)-A(y_m)} < \frac{2}{3} \varepsilon+ \norm{A_k(y_n)-A_k(y_m)}
	 \]
	 $(A_k(y_n))_{n=1}^{\infty}$ converges in $E$. This implies the existence of $N$ such that
	 \[
	 	\forall\, n,m \geq N: \qquad \norm{A_k(y_n)-A_k(y_n)}< \varepsilon
	 \]
	 \[
	 	\Rightarrow \norm{A(y_n)-A(y_m)} < \varepsilon, \qquad \forall\, n,m \geq N
	 \]
	 and thus $(A(y_n))_{n=1}^{\infty}$ is a Cauchy sequence.
\end{beweis}

\begin{proposition}
	Let $(E, \skal{.}{.})$ be a seperable Hilbert space and $A \in K(E,E)$. then there exist finite-ranked operators $A_n \in K(E,E)$ such that
	\[
		\norm{A-A_n}_{E \to E} \to 0, \qquad n \to \infty.
	\]
\end{proposition}
\begin{beweis}
	Let $(x_n)_{n=1}^{\infty}$ be an ON-basis for $E$. For 
	\[
		x = \sum_{k=1}^{\infty} \skal{x}{x_k}x_k, \qquad x \in E.
	\]
	Set 
	\[
		A_n(x) = A \left( \sum_{k=1}^{n}\skal{x}{x_k}x_k \right) = \sum_{k=1}^{n} \skal{x}{x_k}A(x_k), \qquad x \in E, \qquad n=1,2,\dots.
	\]
	Here $\dim( \mathcal{R}(A_n)) \leq n$ for $n =1,2,\dots$. \\
	So $A_n$ is a finite ranked operator in $E$ for $n=1,2,\dots$. \\
	Fix $x \in E$ with $\norm{x}=1$ and consider:
	\[
		\norm{(A-A_n)(x)}^2 = \norm{A(\sum_{k=n+1}^{\infty} \skal{x}{x_k}x_k)}^2 \leq \sup\limits_{\substack{\norm{y}=1, \\ y \in \set{x_1,\dots,x_n}^{\perp}}}
		\norm{A(y)}^2
	\]
	and thus
	\[
		\norm{A-A_n}_{E \to E}^2 \leq \sup\limits_{\substack{\norm{y}=1, \\ y \in \set{x_1,\dots,x_n}^{\perp}}} \norm{A(y)}^2.
	\]
	Set 
	\[
		u_n := \sup\limits_{\substack{\norm{y}=1, \\ y \in \set{x_1,\dots,x_n}^{\perp}}} \norm{A(y)}^2 < \infty, \qquad n =1,2,\dots.
	\]
	Here $a_n \geq a_{n+1} \geq 0$ for $n=1,2,\dots$. \\
	Clearly $(a_n)_{n=1}^{\infty}$ converges in $\mathbb{R}$. Set $a = \lim_{n \to \infty}a_n$. It remains to prove $a =0$.
	Assume $a>0$. Then there exists $(y_n)_{n=1}^{\infty}$ in $E$ such that
	\begin{enumerate}
		\item $\norm{y_n}=1$,
		\item $y \in \set{x_1,\dots,x_n}^{\perp}$,
		\item $\norm{A(y_n)}^2 \geq \frac{1}{2} a$.
	\end{enumerate}
	\textbf{Claim:} \text{    } $y_n \rightharpoonup 0$ in $(E,\skal{.}{.})$ since: \\
	Fix an arbitrary $x \in E$ and
	\begin{align*}
		\abs{\skal{y_n}{x}} &= \abs{\skal{y_n}{\sum_{k=1}^{\infty} \skal{x}{x_k}x_k}} \\
		&= \abs{\skal{y_n}{\sum_{k=n+1}^{\infty}\skal{x}{x_k}x_k}} \\
		&\leq \norm{y_n} \cdot \norm{\sum_{k=n+1}^{\infty}\skal{x}{x_k}x_k} \\
		&= \sqrt{\sum_{k=n+1}^{\infty}\abs{\skal{x}{x_k}}^2} \to 0, \qquad n \to \infty.
	\end{align*}
	(Note that $\sum_{k=1}^{\infty}\abs{\skal{x}{x_k}}^2 = \norm{x}^2 < \infty$) \\
	We have $y_n \rightharpoonup 0$ in $(E, \skal{.}{.})$ and
	\[
		A \in B(E,E) \qquad \Rightarrow \qquad A(y_n) \to A(0)= 0.
	\]
	Contradiction to (3) above which gives us $a=0$.
\end{beweis}

\begin{proposition}
	$(E,\skal{.}{.})$ Hilbert space and $A \in K(E,E)$. Then \[
		x_n \rightharpoonup x \text{ in } (E, \skal{.}{.}) \qquad \Rightarrow \qquad A(x_n) \to A(x) \text{ in }(E,\skal{.}{.}).
	\]
\end{proposition}
\begin{beweis}
	$x_n \rightharpoonup x$ in $(E,\skal{.}{.})$ implies that $\sup_n \norm{x_n}< \infty$ (according to important theorem). Since $A \in K(E,E)$, we know that $(A(x_n))_{n=1}^{\infty}$ has a converging subsequence $(A(x _{n_k})_{k=1}^{\infty}$ since $(x_n)_{n=1}^{\infty}$ is bounded. \\
	Say $A(x _{n_k}) \to y$ in $E$. $A \in K(E,E) \subset B(E,E)$ and $x_n \rightharpoonup x$ in $(E, \skal{.}{.})$. \\
	This implies
	\[
		A(x_n) \rightharpoonup A(x) \qquad \text{in }(E,\skal{.}{.}).
	\]
	We get that $y = A(x)$. We have $A(x _{n_k}) \to A(x)$ in $E$. \\
	Assume that $A(x_n) \not A(x)$ in $E$. \\
	Then there exists an $\varepsilon >0$ and a subsequence $(A(\tilde x_n))_{n=1}^{\infty}$ of $(A(x_n))_{n=1}^{\infty}$ such that
	\[
		\norm{A( \tilde x_n) - A(x)} \geq \varepsilon, \qquad \forall\,  n.
	\]
	But $\tilde x_n \rightharpoonup x$ in $(E,\skal{.}{.})$ and to be compact implies that $(A(\tilde x_n))_{n=1}^{\infty}$ has a converging subsequence $(A( \tilde x _{n_k})_{k=1}^{\infty}$ that converges to $A(x)$ (same argument as before)
	Conclusion: $A(x_n) \to A(x)$ in $(E, \skal{.}{.})$.
\end{beweis}

\begin{proposition}
	$A \in K(E,E)$ and $(E, \skal{.}{.})$ Hilbert space \qquad $\Rightarrow $ $A^* \in K(E,E)$.
\end{proposition}
\begin{beweis}
	Fix any bounded sequence $(x_n)_{n=1}^{\infty}$ in $E$. 
	\begin{align*}
		\norm{A^*(x_n)-A^*(x_m)} &= \skal{A^*(x_n)-A^*(x_m)}{A^*(x_n)-A^*(x_m)} \\
		&= \skal{x_n-x_m}{A(A^*(x_n))-A(A^*(x_m))} 
	\end{align*}
	then use $A \in K(E,E)$.
\end{beweis}

\begin{proposition}
	$A \in K(E,E)$, $B \in B(E,E)$ \qquad $\Rightarrow $ $AB,BA \in K(E,E)$.
\end{proposition}

\begin{beispiel}
	We already know this example: $k \in C([0,1] \times [0,1])$ with
	\[
		A(f)(x) = \int_{0}^{1} k(x,y)f(y) \,\mathrm{d}y, \qquad x \in [0,1], \qquad f \in L^2([0,1]).
	\]
	We know that $A \in B(L^2([0,1]), L^2([0,1]))$ 
	\[
		\norm{A}_{L^2 \to L^2} \leq \norm{k}_{L^2([0,1] \times [0,1])}.
	\]
	\textbf{Claim:} \text{    }$A \in K(L^2([0,1]),L^2([0,1]))$. \\
	Approximate $A$ by finite-ranked operators. \\
	Note: set $A = A_k$ and $B = A_{k_n}$ where $k_n$ is a nice function on $[0,1] \times [0,1]$ and
	\[
		A-B = A_k - A_{k_n} = A_{k-k_n}.
	\]
	So \[
		\norm{A-B}_{L^2 \to L^2} \leq \norm{k - k_n}.
	\]
	Set \begin{align*}
		I_f = [x_j - \frac{1}{N}, x_j], \qquad j = 1, \dots, N, \qquad x_j = \frac{j}{N} \\
		\tilde I_l =[y_l - \frac{1}{N}, y_j], \qquad l = 1, \dots, N, \qquad y_l = \frac{l}{N}. 
	\end{align*}
	Set \[
		k_n(x,y) = \sum_{j=1}^{N} \sum_{l=1}^{N} k(x_j,y_l) \chi_{I_j}(x) \chi_{\tilde I_l}(y)
	\]
	where \[
		\chi_{I_j}(x) = \begin{cases}
			1, &\text{ if }x \in I_j\\
			0, &\text{elsewhere}.
		\end{cases}
	\]
\end{beispiel}
Since $k \in C([0,1] \times [0,1])$ and $[0,1] \times [0,1]$ compact in $\mathbb{R}^2$ then $k$ is uniformly continous on $[0,1] \times [0,1]$. We fix $\varepsilon >0$. \\
\textbf{Claim:} \text{    }It exists an $N$ such that
\[
	\sup_{\substack{(x,y) \in  \\ [0,1] \times [0,1]}} \abs{k(x,y)- k_n(x,y)} < \infty,
\]
\[
	A_{k_N}(f)(x) = \int_{0}^{1} k_N(x,y) f(y) \,\mathrm{d}y = \sum_{j=1}^{N} \underset{\text{scalar}}{\underbrace{\sum_{l=1}^{N} k(x_i,y_l) \int_{0}^{1} \chi_{\tilde I_l}(y) f(y) \,\mathrm{d}y \chi_{I_j}(x)}}.
\]
\[
	\dim(\mathcal{R}(A_{k_N})) = N < \infty.
\]
Hence $A_{k_N} \in K(L^2([0,1]), L^2([0,1]))$ for all $N$. \\
Moreover 
\[
	\norm{A-A_{k_N}}_{L^2 \to L^2} \leq \norm{k - k_N}_{L^2([0,1] \times [0,1])} < \varepsilon
\] 
for $N$ large enough. $K(E,E)$ is a closed set in $(B(E,E), \norm{.}_{L^2 \to L^2})$ so $A \in K(L^2,L^2)$.

\begin{beispiel}
	$(E,\skal{.}{.})$ Hilbert space, $(x_{n})_{n=1}^{\infty}$ ON-basis and $(\lambda_n)_{n=1}^{\infty}$ sequence of scalars. Set
\[
	T(x) = \sum_{n=1}^{\infty}\lambda_n \skal{x}{x_n} x_n, \qquad x \in E.
\]	
\textbf{Claim:} \text{    }     
\begin{enumerate}[1)]
	\item $T \in B(E,E) \qquad \Leftrightarrow \qquad (\lambda_n)^{\infty}_{n=1} \text{ is a bounded sequence in $\mathbb{C}$}$.
	\item$T \in K(E,E) \qquad \Leftrightarrow \qquad \lambda_n \to 0 \text{ for }n \to \infty$.
\end{enumerate}
Note $x \in E$ and the Parseval's formula
\[
	\norm{x}^2 = \sum_{n=1}^{\infty} \abs{\skal{x}{x_n}}^2.
\]
For $T(x) \in E$ we have
\[
	\norm{T(x)}^2 = \sum_{n=1}^{\infty} \abs{\lambda_n}^2 \abs{\skal{x}{x_n}}^2.
\]
If $(\lambda_n)_{n=1}^{\infty}$ bounded sequence in $\mathbb{C}$. Then $\sup \abs{\lambda_n} \equiv M < \infty$ and
\[
	\norm{T(x)}^2 \leq \sum_{n=1}^{\infty} M^2 \abs{\skal{x}{x_n}}^2 = M^2 \norm{x}^2.
\]
If $(\lambda_n)_{n=1}^{\infty}$ is not bounded then there exists a sequence $(\lambda_{n_k})_{k=1}^{\infty}$ such that $\abs{\lambda_{n_k}} \to \infty$ as $k \to \infty$. But
\[
	\norm{T(x_{n_k})} = \abs{\lambda_{n_k}}\norm{x_{n_k}} = \abs{\lambda_{n_k}} \to \infty, \qquad k \to \infty
\]
\[
	\sup\limits_{\norm{x}=1} \norm{T(x)} = \infty.
\]
So 1) is done. For 2) we assume $\lambda_n \to 0$ for $n \to \infty$. Set 
\[
	T_k(x)= \sum_{n=1}^{k}\lambda_n \skal{x}{x_n}x_n, \qquad x \in E
\]
$T_k$ is a finite rank operator for $k=1,2,\dots$. SO $T_k \in K(E,E)$ for all $k$.
\begin{align*}
	\norm{T-T_k}_{E \to E} &= \sup_{\norm{x}=1} \norm{(T-T_k)(x)} \\ &= \sup_{\norm{x}=1} \norm{\sum_{k=n+1}^{\infty} \lambda_n \skal{x}{x_n}x_n} \\
	& \leq \sup\limits_{n= k+1,k+2,\dots}\abs{\lambda_n} \to 0, \qquad k \to \infty
\end{align*}
Assume $\lambda_n \not \to 0$ for $n \to \infty$. Then there exists $\varepsilon >0$ and a sequence $(\lambda_{n_k})_{k=1}^{\infty}$ such that 
\[
	\abs{\lambda_{n_k}} \geq \varepsilon.
\]
Note
\[
	T(x_{n_k})= \lambda_{n_k}x_{n_k}, \qquad k =1,2,\dots
\]
\[
	\norm{T(x_{n_k})} = \abs{\lambda_{n_k}} \norm{x_{n_k}} = \abs{\lambda_{n_k}} \geq \varepsilon, \qquad k = 1,2,\dots
\]
$x_{n_k} \stackrel{\text{w}}{\to }0$ in $(E,\skal{.}{.})$ since for $y \in E$
\[
	\skal{x_{n_k}}{y} = \skal{x_{n_k}}{\sum_{n=1}^{\infty}\skal{y}{x_n}x_n} = \overline{\skal{y}{x_{n_k}}} \to 0
\]
since
\[
	\sum_{n=1}^{\infty}\abs{\skal{y}{x_n}}^2 = \norm{y}^2 < \infty.
\]
If $T \in K(E,E)$ then $T(x_{n_k}) \to T(0)=0$ but \[
	\norm{T(x_{n_k})} \geq \varepsilon, \qquad \text{for all }k.
\]
Hence
\[
	T \not \in K(E,E).
\]
\end{beispiel}
\begin{beispiel}
	$(E,\skal{.}{.})$ Hilbert space, $A \in K(E,E)$ and $I(x)=x$ for all $x \in E$. It follows
	\[
		\Rightarrow  \qquad \mathcal{R}(I-A) \text{ closed in $E$}.
	\]
\begin{bemerkung}
	\begin{align*}
		\mathcal{R}(I-A)^{\perp} &= \mathcal{N}((I-A)^{*}) = \mathcal{N}(I-A^{*}) \\
		\overline{\mathcal{R}(I-A)} &= \left( \mathcal{R}(I-A)^{\perp} \right)^{\perp} = \mathcal{N}(I-A^{*})^{\perp}.
	\end{align*}
	If $A \in K(E,E)$ then
	\[
		\overline{\mathcal{R}(I-A)} = \mathcal{R}(I-A).
	\]
	Solve \[
		x = A(x) + y \qquad \Leftrightarrow \qquad (I-A)(x) = y
	\]
	Compare 'Fredholm alternative'.
\end{bemerkung}
\end{beispiel}
\begin{beweis}
	Take a sequence $(y_n)_{n \in \mathbb{N}} \subseteq R(I-A)$ such that $y_n \to y$ in $(E,\norm{.})$. \\
	To show: $y \in \mathcal{R}(I-A)$, i.e. $y = (I-A)(x)$ for some $x \in E$ and $y_n = (I-A)(x_n)$ for some $x_n \in E$.
	\[
		x_n \in E = \mathcal{N}(I-A) + \mathcal{N}(I-A)^{\perp}
	\]
	such that
	\[
		x_n = \tilde x_n + \hat x_n 
	\]
	with
	\[
		\norm{x_n}^2 = \norm{\tilde x_n}^2 + \norm{\hat x_n}^2.
	\]
	Step 1: Show $(\hat x_n)_{n=1}^{\infty}$ bounded in $E$. \\
	Step 2: $y_n = (I-A)( \hat x_n) = \hat x_n - A( \hat x_n)$. \\
\end{beweis}