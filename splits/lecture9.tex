%% lecture 9


\subsection{Bounded linear functionals on Hilbert spaces} 
\label{sub:bounded_linear_functionals_on_hilbert_spaces}

Consider $(H, \skal{.}{.})$- Hilbert space (inner product space which is complete w.r.t. to a norm $\norm{x}= \sqrt{\skal{x}{x}}$). \\
Let $M$ be a closed subspace of $H$. 
\[
	\mathcal{M}^{\perp} = \set[y \in H]{\skal{x}{y}= 0, \, \forall\,  x \in M}.
\]
Then we know $H = M + M^{\perp}$, i.e. for any $x \in H$ there exists a unique $y \in M$ and $z \in M^{\perp}$ such that
\[
	x = y + z.
\]
\begin{theorem}[Riesz-Frechét represantation theorem]
	Let $(H, \skal{.}{.})$ be a Hilbertspace. Let $f$ be a bounded linear functional on $H$. Then there exists a unique $x_f \in H$ such that
	\[
		f(x) = \skal{x}{x_f}, \qquad \forall\,  x \in H.
	\]
	Moreover \[
		\norm{f} = \norm{x_f}_H.
	\]
\end{theorem}
\begin{bemerkung}
	If $f: H \to \mathbb{C}$ is of the form
	\[
		f(x) = \skal{x}{y}, \qquad \text{for all } x \in H \text{ and some }y \in H.
	\]
	Then $f$ is bounded and linear (easy with Cauchy-Schwarz and properties of the scalarproduct).
\end{bemerkung}
\begin{beweis}
	\begin{description}
		\item[Existence of $x_f$:] If $f$ is a zero linear functional, i.e. $f(x)= 0$ for all $x \in H$ take $x_f = 0$. Assume now that $f$ is not the zero functional. Consider \[
			N(f) := \ker{f} = \set[x \in H]{f(x)= 0}.
		\] 
		Then $N(f)$ is a closed subspace of $H$: \\
		For $x_1,x_2 \in N(f), \,\alpha, \beta \in \mathbb{C}$ it holds
		\[
			f( \alpha x_1, \beta x_2) \stackrel{\text{lin}}{=} \alpha f(x_1) + \beta f(x_2).
		\]
		Hence $\alpha x_1 + \beta x_2 \in N(f)$ and $N(f)$ is a subspace. $N(f)$ is closed since if $x_n \in N(f)$ with $x_n \to x$ strongly. Then 
		\[
			f(x_n) \to f(x)
		\] 
		because of bounded and hence continuous. But we know that $f(x_n) = 0$ so the limit has to be $f(x)=0$, i.e $x \in N(f)$. $N(f)$ is a proper closed subspace. ($N(f) \neq H$). Consider now $N(f)^{\perp}$ which is non-zero. \begin{itemize}
			\item $\dim N(f)^{\perp} = 1$. \\
			Assume that $x_1 \neq 0,x_2 \neq 0 \in N(f)^{\perp}$. Then we have $f(x_1),f(x_2) \neq 0$. It exists $a \in \mathbb{C}$ such that
			\[
				f(x_1) + a f(x_2) = 0.
			\]
			And also
			\[
				f(x_1+a x_2) = 0
			\]
			which gives 
			\[
				x_1+a x_2 \in N(f) \cap N(f)^{\perp} = \set{0}.
 			\] 
			Hence
			\[
				x_1 + a x_2 = 0.
			\]
			Any two vectors are linearly dependent in $N(f)^{\perp}$ which gives \[
				\dim N(f)^{\perp} = 1.
			\]
		\end{itemize}
		Take $y' \in N(f)^{\perp}$ with $\norm{y'} = 1$ and let \[
			x_f = \overline{f(y')}y'.
		\]
		We get
		\[
			\skal{x}{x_f} = \begin{cases}
				0, &\text{ if }x \in N(f)\\
				\skal{\lambda y'}{\overline{f(y')}y'}= f(y') \lambda \underset{=1}{\underbrace{\skal{y'}{y'}}}, &\text{ if }x = \lambda y'
			\end{cases}.
		\]
		Furthermore
		\[
			\skal{x}{x_f} = \begin{cases}
				f(x), &\text{ if }x \in N(f)\\
				f(\lambda y') = f(x), &\text{ if }x = \lambda y'
			\end{cases}.
		\]
		Since every element in $H$ is given by $x + \lambda y'$. For $x \in N(f)$ and $\lambda \in \mathbb{C}$. Using linearity we get \[
			f(x + \lambda y') = f(x) + f(\lambda y') = \skal{x}{x_f} + \skal{\lambda y'}{x_f} = \skal{x+ \lambda y'}{x_f}
		\]
		\item[uniqueness:] Assume there exists $x_1,x_2 \in H$ such that 
		\[
			f(x) = \skal{x}{x_1} = \skal{x}{x_2}, \qquad \forall\, x \in H.
		\]
		We get
		\[
			\skal{x}{x_1-x_2} = 0, \qquad \forall\, x \in H.
		\]
		It holds in particular for $x = x_1 - x_2$ the following equality
		\[
			\skal{x_1-x_2}{x_1-x_2} = 0 \qquad \Rightarrow \qquad x_1-x_2 = 0.
		\]
		\item[norm equality] We must see that \[
			\norm{f} = \norm{x_f}_H.
		\] From remark we have 
		\[
			f(x) = \skal{x}{x_f} \qquad \Rightarrow  \qquad \norm{f} \leq \norm{x_f}.
		\]
		We have for $x_f \neq 0$:
		\[
			\norm{f} = \sup_{x \neq 0} \frac{\abs{f(x)}}{\norm{x}} \geq \frac{\abs{f(x_f)}}{\norm{x_f}} = \frac{\norm{x_f}^2}{\norm{x_f}} = \norm{x_f}.
		\]
		This gives the desired result.
	\end{description}
\end{beweis}
\begin{beispiel}
	\[
		E = C_F = \set[(x_1,x_2,\dots)]{\text{only finite number of $x_i \neq 0$}} \subseteq l^2.
	\]
	On $C_F$ consider $l^2$-inner-product
	\[
		\skal{x}{y} = \sum^{\infty}_{i=1} x_i \bar{y}_i \qquad \text{for }x,y \in C_F.
	\]
	\begin{enumerate}
		\item $C_F$ is not a Hilbert space as it is not complete w.r.t
		\[
			\norm{x}_2 = \left( \sum^{\infty}_{i=1} \abs{x_i}^2 \right)^{\frac{1}{2}}.
		\]
		Find a Cauchy sequence that is not convergent to an element in $C_F$. \\
		Find a proper closed subspace $M$ such that $M^{\perp}= \set{0}$ (This would mean in particular that $C_F \neq M + M^{\perp}$) \\
		Consider
		\[
			M = \set[(x_1,x_2,\dots) \in C_F]{\sum_{k=1}^{\infty} x_k \frac{1}{k}=0},
		\]
		\[
			x_f = (1, \frac{1}{2}, \frac{1}{3}, \dots) \in l^2,
		\]
		\[
			M = \ker{f} \cap C_F
		\]
		where
		\begin{align*}
			f: l^2 &\to \mathbb{C} \\
			f(x) &= \skal{x}{x_f} = \sum_{k=1}^{\infty}x_k \frac{1}{k},
		\end{align*}
		\[
			M^{\perp} = \text{all elements in $C_F$ which are in $(\ker{f})^{\perp}$}.
		\]
		From the proof of Riesz-Frechet theorem  we have $(\ker{f})^{\perp}$ is $1$-dimensional and \[
			x_f \in (\ker{f})^{\perp}.
		\]
		Hence
		\[
			(\ker{f})^{\perp} = \set[\lambda x_f]{\lambda \in \mathbb{C}}.
		\]
		We have
		\[
			\underset{= M ^{\perp}}{\underbrace{(\ker{f})^{\perp} \cap C_F}} = \set{0}.
		\]
		\item $(H,\skal{.}{.})$ Hilbert space and $\set{u_i} \subseteq H$ finite or infinite $i$. $\set{u_i}$ is an orthogonal system if
		\[
			\skal{u_i}{u_j}=0, \qquad \forall\, i \neq j
		\]
		and an orthonormal system if
		\[
			\skal{u_i}{u_j} = \delta_{ij} = \begin{cases}
				0, &\text{ if }i \neq j\\
				1, &\text{ if }i = j
			\end{cases}.
		\]
		\end{enumerate}
	\end{beispiel}
	\begin{proposition}
		Orthogonal system of non-zero vectors are linearly independent. (See linear algebra)
	\end{proposition}
	Having linearly independent family of vectors we can make it orthogonal with for example using Gram-Schmidt orthogonalization procedure.(See linear algebra for details). \\
	Recall that we can write a Fourier series of x with $\skal{x}{u_i}$ Fourier coefficients
		\[
			x \in H \qquad \Rightarrow \qquad x = \sum^{\infty}_{i=1} \skal{x}{u_i}u_i
		\]
		with $\set{u_i}$-ON-system. \\
		$C([- \pi, \pi])$ and $\set{u_k} = \set[\frac{1}{\sqrt{2 \pi}} e^{ikt}]{k \in \mathbb{Z}}$ equipped with the scalar product
		\[
			\skal{f}{g} = \int_{- \pi}^{\pi}f(t)\overline{g(t)} \,\mathrm{d}t.
		\]
		It holds for the Fourier-series
		\[
			\skal{f}{u_k} = \hat f(k) = \frac{1}{\sqrt{2 \pi}}\int_{- \pi }^{\pi} f(t) e^{- ikt} \,\mathrm{d}t.
		\]
		We want to see when
		\[
			\sum^{\infty}_{i=1} \skal{x}{u_i}u_i
		\]
		is convergent to $x$.
		\begin{definition}
			$\mathcal{A}_n$ ON-system is called an ON-basis for $H$ if its span is dense in $H$. We say that an ON-system is complete if every $x \in H$ is 
			\[
				\sum^{\infty}_{i=1} \skal{x}{u_i}u_i.
			\]
		\end{definition}
	\begin{theorem}
		$(H, \skal{.}{.})$- Hilbert space, $\set{u_k}$ is ON-system in $H$. The following statements are equivalent.
		\begin{enumerate}[(1)]
			\item $\set{u_n}$ is a complete ON-system.
			\item $\set{u_n}$ is an ON-basis for $H$.
			\item (Parsevals's Identity) 
			\[
				\norm{x} = \left( \sum_{k=1}^{\infty} \abs{\skal{x}{u_k}}^2 \right)^{\frac{1}{2}}, \qquad \forall\, x \in H.
			\]
			\item $\skal{x}{y} = \sum_{k=1}^{\infty} \skal{x}{u_k} \overline{\skal{y}{u_k}}$ for all $x,y \in H$.
			\item $\skal{x}{u_k} = 0$ for all $k \in \mathbb{N}$ follows $x = 0$.
		\end{enumerate}
	\end{theorem}
\begin{beweis}
 	\begin{description}
 		\item[(1) $\Rightarrow $ (2):] We have 
		\[
			x = \sum^{\infty}_{i=1} \skal{x}{u_i}u_i
		\] 
		it means
		\[
			x = \lim_{n \to \infty} \sum^{n}_{i=1} \skal{x}{u_i}u_i \in \spn\set[u_i]{i \geq 1}.
		\]
		This is implies that any $x \in H$ is in $\overline{\spn \set[u_i]{i \geq 1}}$, i.e. $\set{u_i}$ is ON-basis.
		\item[(2) $\Rightarrow$ (5):] Let $\set{u_i}$ be a ON-basis. Assume 
		\[
			\skal{x}{u_k} = 0, \qquad \forall\,  k \in \mathbb{N}.
		\]
		Then
		\[
			\skal{x}{u} = 0, \qquad \forall\, u \in \spn\set[u_k]{k \geq 1}.
		\]
		By the property that strong convergence implies weak convergence we will have 
		\[
			\skal{x}{u}= 0, \qquad \forall\, u \in \spn\set[u_k]{k \geq 1} = H.
		\]
		In particular
		\[
			\skal{x}{u} = 0, \qquad \text{for }u =x
		\]
		which means
		\[
			\skal{x}{x} = 0 \qquad \Leftrightarrow \qquad  x=0.
		\]
		\item[(5) $\Rightarrow$ (1)] Recall Bessel's equality. For $\set{u_k}$- ON-system then 
		\[
			\norm{x- \sum^{k}_{i=1} \skal{x}{u_k}u_k}^2 = \norm{x}^2 - \sum^{k}_{i=1} \abs{\skal{x}{u_k}}^2
		\]
		Assume (5), i.e.
		\[
			\skal{x}{u_k} = 0, \qquad \forall\, k \qquad \Rightarrow \qquad x = 0
		\]
		We must see
		\[
			x = \sum_{k=1}^{n} \skal{x}{u_k}u_k \qquad \forall\, x \in H.
		\]
		From Bessel's equality we have
		\[
			\sum_{k=1}^{n} \abs{\skal{x}{u_k}}^2 = \norm{x}^2 - \norm{x- \sum_{k=1}^{n} \skal{x}{u_k}u_k}^2 \leq \norm{x}^2, \qquad \forall\, k \in \mathbb{N}
		\]
		and hence $\sum_{k=1}^{n}\abs{\skal{x}{u_k}}^2$ is convergent. It implies that for $n>m$ we have
		\begin{align*}
			\norm{\sum_{k=1}^{n}\skal{x}{u_k}u_k - \sum_{k=1}^{m} \skal{x}{u_k}u_k}^2 
			&\stackrel{\hphantom{\text{pythagorian thm}}}{=} \norm{\sum_{k=m+1}^{n} \skal{x}{u_k}u_k}^2 \\
			&\stackrel{\text{pythagorian thm}}{=} \sum_{k=m+1}^{n} \abs{\skal{x}{u_k}}^2 \norm{u_k}^2 \\
			&\stackrel{\hphantom{\text{pythagorian thm}}}{\to } 0, \qquad n,m \to 0 \qquad (*).
		\end{align*}
		Note that if $\set{x_i}$ are paarwise orthogonal it holds
		\[
			\norm{\sum^{n}_{i=1}x_i}^2 = \sum^{n}_{i=1} \norm{x}^2.
		\]
		From $(*)$ we know that the partial sum
		\[
			S_n := \sum_{k=1}^{n}\skal{x}{u_k}u_k
		\]
		is a Cauchy sequence. As $H$ is a Hilbert space, $H$ is complete and hence $S_n$ has a limit in $H$. Write
		\[
			\sum_{i=1}^{\infty} \skal{x}{u_i}u_i 
		\]
		for the limit. We must see that the limit is $x$. Consider
		\[
			y := x - \sum_{i=1}^{\infty} \skal{x}{u_i}u_i.
		\]
		Then 
		\[
			\skal{y}{u_i} = \skal{x}{u_i} - \skal{x}{u_i} = 0, \qquad \forall\, i.
		\]
		By assumption (5) it follows 
		\[
			y = 0 \qquad \Leftrightarrow \qquad x = \sum^{\infty}_{i=1} \skal{x}{u_i}u_i.
		\]
		\item[(1) $\Rightarrow$ (3):] From Bessel's equality we have again
		\[
			\norm{x- \sum^{n}_{i=1}\skal{x}{u_i}u_i}^2 = \norm{x}^2 - \sum^{n}_{i=1}\abs{\skal{x}{u_i}}^2.
		\] 
		By assumption (1) the LHS tends to $0$ as $n \to \infty$. On the other hand the RHS goes to 
		\[
			\to \norm{x}^2 - \sum^{\infty}_{i=1} \abs{\skal{x}{u_i}}^2, \qquad n \to \infty.
		\]
		This gives 
		\[
			\norm{x}^2 - \sum_{i=1}^{\infty} \abs{\skal{x}{u_i}}^2 = 0.
		\]
		\item[(3) $\Rightarrow $ (5)] trivial.
		\item[(4) $\Rightarrow $ (5)] trivial (take $y=x$).
		\item[(1) $\Rightarrow $ (4)] We have
		\[
			x = \sum_{k=1}^{\infty} \skal{x}{u_k}u_k.
		\]
		Then
		\[
			\skal{x}{y} = \sum_{k=1}^{\infty} \skal{x}{u_k}\skal{u_k}{y} = \sum_{k=1}^{\infty}\skal{x}{u_k}\overline{\skal{y}{u_k}}.
		\]
 	\end{description}
\end{beweis}

\begin{beispiel}
	$L^2([- \pi, \pi])$ with
	\[
		\set[\frac{1}{\sqrt{2\pi}}e ^{int}]{n \in \mathbb{Z}}
	\]
	is an ON-system in $L^2([-\pi, \pi])$ where
	\[
		\skal{f}{g} = \int_{-\pi}^{\pi} f(t) \overline{g(t)} \,\mathrm{d}t.
	\]
\end{beispiel}

\begin{satz}
	The system above is an ON-basis for $L^2([-\pi,\pi])$. In particular, for any $f \in L^2([-\pi,\pi])$
	\[
		f = \sum_{k \in \mathbb{Z}}^{} \hat f(k) e^{ikt}
	\]
	convergent in the $L^2$-norm.
	\[
		\norm{f}_{L^2} = \left( \int_{-\pi}^{\pi} \abs{f(t)}^2 \,\mathrm{d}t \right)^{\frac{1}{2}}
	\]
	which is equivalent to
	\[
		\norm{f - \sum_{k=-n}^{n} \hat f(k) e^{ikt}}^2_{L^2} \to 0.
	\]
\end{satz}

\minisec{Sketch of the proof:}

\begin{enumerate}[(1)]
	\item Stein-Weierstraß-Theorem. $X$ compact set $C(X,\mathbb{C})$ continuous functions with complex values. Let $M \subseteq C(X,\mathbb{C})$ be a subspace that satisfies:
	\begin{enumerate}[(a)]
		\item it seperates points of $X$, i.e. 
		\[
			\forall\, x_1,x_2 \in X, x_1 \neq x_2 \,\exists\, f \in M: \qquad f(x_1) \neq f(x_2).
		\]
		\item $M$ contains the constant function 1 ($f(x)= 1$ for all $x \in X$).
		\item It is closed under complex conjugation, i.e. 
		\[
			f \in M \qquad \Rightarrow \qquad \bar{f} \in M
		\]
		and closed under product, i.e.
		\[
			f_1,f_2 \in M \qquad \Rightarrow \qquad f_1 \cdot f_2 \in M.
		\]
	\end{enumerate}
	Then $M$ is dense in $C(X,\mathbb{C})$ w.r.t. $\norm{.}_{\infty}$ (Continuous function by Polynomials) From this it follows
	\[
		M = \set{ \text{all complex polynomials}}
	\]
	are dense in $C([a,b])$.
	\item $C([a,b])$ is dense in $L^2([a,b])$ w.r.t. $\norm{.}_{L^2}$-norm. 
\end{enumerate}
We will use $(1)$ and $(2)$ to show that $\spn\set[\frac{1}{\sqrt{2 \pi}}e^{int}]{n \in \mathbb{Z}}$ is dense in $L^2([-\pi,\pi])$.
\begin{beweis}
	Let \[
		M := \spn\set[\frac{1}{\sqrt{2 \pi}}e^{int}]{n \in \mathbb{Z}} \subseteq \set[f \in C([-\pi, \pi])]{f( \pi) = f(-\pi)}.
	\]
	$M$ seperates points, it contains the constant function $1$ and it is closed under complex conjugation. Furthermore $M$ is closed under taking products. With Stein-Weierstraß it follows that $M$ is dense in 
	\[
		\set[f \in C([-\pi,\pi])]{f(\pi)= f(-\pi)}.
	\]
	By (2) we have $C([-\pi,\pi])$ is dense in $L^2([-\pi,\pi])$ w.r.t. the $L^2$-norm. From this one can see that even $\set[f \in C([-\pi,\pi])]{f(\pi)= f(-\pi)}$ is dense in $L^2([-\pi,\pi])$:
	\[
		\forall\,  \varepsilon>0, \,\forall\, f \in L^2 \,\exists\,g \in C([-\pi,\pi]): \qquad \norm{f-g}_{L^2}^2 = \int_{-\pi}^{\pi}\abs{f(t)-g(t)}^2 \,\mathrm{d}t < \varepsilon.
	\] 
	Define $g _{\varepsilon}$ such that it has a pike in $x = \pi - \varepsilon$ but it is continuous and is equal to $g$ for $x < \pi -\varepsilon$. Then
	\[
		g _{\varepsilon} \in C([-\pi,\pi]), \,g _{\varepsilon}(- \pi) = g _{\varepsilon} ( \pi).
	\]
	It holds
	\begin{align*}
		\norm{f - g _{\varepsilon}}_{L^2} &\leq \underset{< \sqrt{\varepsilon}}{\underbrace{\norm{f-g}_{L^2}}} + 
		\norm{g - g _{\varepsilon}}_{L^2} \\
		&\leq  \sqrt{\varepsilon} + \left( \int_{\pi - \varepsilon}^{\pi} \abs{g(t)-g _{\varepsilon}(t)} \,\mathrm{d}t  \right)^{\frac{1}{2}} \\
		&\leq \sqrt{\varepsilon} + \sqrt{\max_{x \in [-\pi - \varepsilon, \pi]} \abs{g- g _{\varepsilon}}\varepsilon} \\
		&= \sqrt{\varepsilon} + \sqrt{C} \sqrt{\varepsilon}.
	\end{align*}
	We conclude: any $f = L^2-$limit $g_n$ with $g_n \in C([-\pi,\pi])$ and $g_n(- \pi ) = g_n( \pi)$. Each $g_n = \norm{.}_\infty$-norm limit of an element in 
	$ \spn\set[\frac{1}{\sqrt{2 \pi}}e^{int}]{n \in \mathbb{Z}}$ as
	\[
		\norm{g-f}_{L^2} \leq \norm{g-f}^{\frac{1}{2}}_{\infty} (2 \pi)^{\frac{1}{2}}.
	\]
	Note that
	\[
		\left( \int_{- \pi}^{\pi} \abs{g(t)-f(t)}^2 \,\mathrm{d}t \right)^{\frac{1}{2}} \leq \max_{x \in [ - \pi, \pi]} \abs{g(t)- f(t)} \left( \int_{- \pi}^{\pi} \,\mathrm{d}t \right)^{\frac{1}{2}}.
	\]
	We get that each $g_n$ can be approximated in the $L^2$-norm by elements in $\spn\set[\frac{1}{\sqrt{2 \pi}}e^{int}]{n \in \mathbb{Z}}$ hence
	\[
		\spn\set[\frac{1}{\sqrt{2 \pi}}e^{int}]{n \in \mathbb{Z}} \subseteq L^2([- \pi,\pi]).
	\]
\end{beweis}

\subsection{Linear operators on Hilbert spaces} 
\label{sub:linear_operators_on_hilbert_spaces}
Set $(H_1,\skal{.}{.}_1)$ and $(H_2, \skal{.}{.}_2)$ Hilbert spaces. A bounded linear mapping $A: H_1 \to H_2$ is called bounded linear operator. \\
Bounded means in our case
\[
	\norm{Ax}_2 \leq C \norm{x}_1 \qquad \forall\, x \in H \text{ and some constant }C
\] 
\begin{beispiel}
	Set $H_1 = H_2 = L^2([0,1])$ and $K: [0,1] \times [0,1] \to \mathbb{C}$. Assume that $K$ is continuous. Consider
	\[
		(Af)(x) = \int_{0}^{1}K(x,y)f(y) \,\mathrm{d}y.
	\]
	$A$ is linear (trivial). Show that $A$ is bounded:
	\begin{align*}
		\norm{Af}_2 &\stackrel{\hphantom{\text{CS}}}{=} \int_{0}^{1} \abs{\int_{0}^{1}K(x,y)f(y) \,\mathrm{d}y}^2 \,\mathrm{d}x \\
		&\stackrel{\text{CS}}{\leq} \int_{0}^{1} \left( \int_{0}^{1}\abs{K(x,y)}^2 \,\mathrm{d}y  \cdot \int_{0}^{1}\abs{f(y)}^2 \,\mathrm{d}y \right)\,\mathrm{d}x \\
		&\stackrel{\hphantom{\text{CS}}}{=} \underset{< \infty}{\underbrace{\int_{0}^{1} \left( \int_{0}^{1} \abs{K(x,y)}^2 \,\mathrm{d}y \right) \,\mathrm{d}x}}
		\cdot \underset{= \norm{f}_2^2}{\underbrace{ \int_{0}^{1}\abs{f(y)}^2 \,\mathrm{d}y}}.
	\end{align*}
	Hence
	\[
		\norm{A} \leq \left( \int_{0}^{1} \int_{0}^{1} \abs{K(x,y)}^2 \,\mathrm{d}x \,\mathrm{d}y \right)^{\frac{1}{2}}.
	\]
\end{beispiel}
	Products $(A \cdot B)$ of operators $H \to H$ with $A: H \to H$ and $B: H \to H$ are defined by
	\[
		(A \cdot B)(f) := A(Bf).
	\]
\begin{satz}
	If $A$ and $B$ are bounded, then $A \cdot B$ is also bounded and 
	\[
		\norm{AB} \leq \norm{A}\norm{B}.
	\]
	In particular: for all $n \in \mathbb{N}$ $A^n$ is bounded and 
	\[
		\norm{A^n} \leq \norm{A}^n.
	\]
\end{satz}